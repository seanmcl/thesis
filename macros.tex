\usepackage{xcolor}
\usepackage{etoolbox}
\usepackage{fancyvrb}
\usepackage{times}
\usepackage{dashrule}
\usepackage{proof-dashed}
\usepackage[show]{ed}
\usepackage{framed}
\usepackage{fullpage}
\usepackage{float} % for the H in "\begin{figure}[H]"
\usepackage{graphicx}
\usepackage{tikz}
\usepackage{rotating}
\usetikzlibrary{arrows,decorations.pathmorphing,backgrounds,fit}
\usepackage{amsthm}
\usepackage{amsmath}
\usepackage{mathtools} % for \coloneqq
\usepackage{latexsym}
\usepackage{amssymb}            % for \multimap (-o)
\usepackage{stmaryrd}           % for \binampersand (&), \bindnasrepma (\paar)
\usepackage{wasysym}            % for \ocircle
\usepackage[numbers,sort]{natbib}
\usepackage[backref,pageanchor=true,plainpages=false, pdfpagelabels, bookmarks,bookmarksnumbered,
%pdfborder=0 0 0,  %removes outlines around hyper links in online display
]{hyperref}
\usepackage{doi}
\usepackage{subfigure}

% Approximately 1" margins, more space on binding side
%\usepackage[letterpaper,twoside,vscale=.8,hscale=.75,nomarginpar]{geometry}
%for general printing (not binding)
\usepackage[letterpaper,twoside,vscale=.8,hscale=.75,nomarginpar,hmarginratio=1:1]{geometry}

\hypersetup{colorlinks=true,citecolor=blue,urlcolor=blue,linkcolor=black}

% Theorems
\newtheorem{theorem}{Theorem}[chapter]
\newtheorem{lemma}[theorem]{Lemma}
\newtheorem{proposition}[theorem]{Proposition}
\newtheorem{definition}[theorem]{Definition}
\newtheorem{example}{Example}[section]
\newtheorem*{example*}{Example}

\DeclarePairedDelimiter{\ceil}{\lceil}{\rceil}
\DeclarePairedDelimiter{\floor}{\lfloor}{\rfloor}

\newcommand{\declarecommand}[1]{\providecommand{#1}{}\renewcommand{#1}}

\declarecommand{\ASet}[1]{\left<#1\right>}
\declarecommand{\All}{\forall}
\declarecommand{\AndN}{\wedge^-}
\declarecommand{\AndP}{\wedge^+}
\declarecommand{\And}{\wedge}
\declarecommand{\BSet}[1]{\left[#1\right]}
\declarecommand{\BSet}[1]{\left[#1\right]}
\declarecommand{\BV}{\mathsf{BV}}
\declarecommand{\Base}{\mathsf{Base}}
\declarecommand{\Bot}{\bot}
\declarecommand{\BvAnd}{\,\&\,}
\declarecommand{\CSeq}[3]{{#1}~|~{#2} \Longrightarrow {#3} \mathstrut}
\declarecommand{\C}{\ensuremath{\mathcal{C}}}
\declarecommand{\Down}{\downarrow}
\declarecommand{\D}{\ensuremath{\mathcal{D}}}
\declarecommand{\EEq}{\approx}
\declarecommand{\Eqdef}{\coloneqq}
\declarecommand{\Ex}{\exists}
\declarecommand{\E}{\ensuremath{\mathcal{E}}}
\declarecommand{\Iff}{\iff}
\declarecommand{\Imp}{\supset}
\declarecommand{\Inferd}[3][]{\infer-[$\scriptsize$ #1]{#2}{#3}}
\declarecommand{\Infere}[3][]{\infer=[$\scriptsize$ #1]{#2}{#3}}
\declarecommand{\Infers}[3][]{\infer*[$\scriptsize$ #1]{#2}{#3}}
\declarecommand{\Infer}[3][]{\infer[$\scriptsize$ #1]{#2}{#3}}
\declarecommand{\Inv}[1]{#1_{\mbox{\scriptsize Inv}}}
\declarecommand{\Ip}{\mathsf{Ip}}
\declarecommand{\LJF}{\Inv{\LJ}}
\declarecommand{\LJ}{\mathbf{G3i}}
\declarecommand{\LPF}{\mathbf{F}}
\declarecommand{\LP}{\mathbf{P}}
\declarecommand{\Mask}{\mathsf{Mask}}
\declarecommand{\Matches}{\mathsf{matches}}
\declarecommand{\Nat}{\mathbb{N}}
\declarecommand{\Not}{\neg}
\declarecommand{\Or}{\vee}
\declarecommand{\Sep}{\ |\ }
\declarecommand{\Seq}[2]{{#1} \Longrightarrow {#2}}
\declarecommand{\Set}[1]{\left\{#1\right\}}
\declarecommand{\Size}[1]{\left|#1\right|}
\declarecommand{\Subsumes}{\leq}
\declarecommand{\Top}{\top}
\declarecommand{\Up}{\uparrow}
\declarecommand{\figline}{\\[2em]}
\declarecommand{\Nat}{\mathbb{N}}
\declarecommand{\Set}[1]{\left\{#1\right\}}
\declarecommand{\ASet}[1]{\left<#1\right>}
\declarecommand{\BSet}[1]{\left[#1\right]}
\declarecommand{\Union}{\cup}
\declarecommand{\MultiUnion}{\uplus}
\declarecommand{\Inter}{\cap}
\declarecommand{\Diff}{\setminus}
\declarecommand{\Card}[1]{\left|#1\right|}
%declarewcommand{\Eqdef}{\stackrel{\mathrm{def}}{=}}
\declarecommand{\EqDef}{\mathrel{\mathop:}=}
\declarecommand{\Ueq}{\doteq}
\declarecommand{\sst}{\ |\ } % "set such that", e.g. \Set{x\sst x>5}
\declarecommand{\Atoms}{\cA}
\declarecommand{\SeqSeqs}{\SSet{\Seqs}}
\declarecommand{\SetSeqs}{\PSet{\Seqs}}
\declarecommand{\BMatch}{\mbox{bmatch}}
\declarecommand{\FMatch}{\mbox{fmatch}}
\declarecommand{\PSet}[1]{\cP(#1)}
\declarecommand{\SSet}[1]{\cS(#1)}
\declarecommand{\Subf}{\mbox{subf}}
\declarecommand{\Sgn}{\mbox{sgn}}
\declarecommand{\Subfs}{\mbox{subfs}}
\declarecommand{\Contracts}{\mbox{contracts}}
\declarecommand{\Paths}{\mbox{paths}}
\declarecommand{\LB}{\left[}
\declarecommand{\RB}{\right]}

% ------------------------------------------------------------------------------
%  Calligraphic
% ------------------------------------------------------------------------------

\declarecommand{\cA}{\mathcal{A}}
\declarecommand{\cB}{\mathcal{B}}
\declarecommand{\cC}{\mathcal{C}}
\declarecommand{\cD}{\mathcal{D}}
\declarecommand{\cE}{\mathcal{E}}
\declarecommand{\cF}{\mathcal{F}}
\declarecommand{\cG}{\mathcal{G}}
\declarecommand{\cH}{\mathcal{H}}
\declarecommand{\cI}{\mathcal{I}}
\declarecommand{\cJ}{\mathcal{J}}
\declarecommand{\cK}{\mathcal{K}}
\declarecommand{\cL}{\mathcal{L}}
\declarecommand{\cM}{\mathcal{M}}
\declarecommand{\cN}{\mathcal{N}}
\declarecommand{\cO}{\mathcal{O}}
\declarecommand{\cP}{\mathcal{P}}
\declarecommand{\cQ}{\mathcal{Q}}
\declarecommand{\cR}{\mathcal{R}}
\declarecommand{\cS}{\mathcal{S}}
\declarecommand{\cT}{\mathcal{T}}
\declarecommand{\cU}{\mathcal{U}}
\declarecommand{\cV}{\mathcal{V}}
\declarecommand{\cW}{\mathcal{W}}
\declarecommand{\cX}{\mathcal{X}}
\declarecommand{\cY}{\mathcal{Y}}
\declarecommand{\cZ}{\mathcal{Z}}

\declarecommand{\bA}{\mathbb{A}}
\declarecommand{\bB}{\mathbb{B}}
\declarecommand{\bC}{\mathbb{C}}
\declarecommand{\bD}{\mathbb{D}}
\declarecommand{\bE}{\mathbb{E}}
\declarecommand{\bF}{\mathbb{F}}
\declarecommand{\bG}{\mathbb{G}}
\declarecommand{\bH}{\mathbb{H}}
\declarecommand{\bI}{\mathbb{I}}
\declarecommand{\bJ}{\mathbb{J}}
\declarecommand{\bK}{\mathbb{K}}
\declarecommand{\bL}{\mathbb{L}}
\declarecommand{\bM}{\mathbb{M}}
\declarecommand{\bN}{\mathbb{N}}
\declarecommand{\bO}{\mathbb{O}}
\declarecommand{\bP}{\mathbb{P}}
\declarecommand{\bQ}{\mathbb{Q}}
\declarecommand{\bR}{\mathbb{R}}
\declarecommand{\bS}{\mathbb{S}}
\declarecommand{\bT}{\mathbb{T}}
\declarecommand{\bU}{\mathbb{U}}
\declarecommand{\bV}{\mathbb{V}}
\declarecommand{\bW}{\mathbb{W}}
\declarecommand{\bX}{\mathbb{X}}
\declarecommand{\bY}{\mathbb{Y}}
\declarecommand{\bZ}{\mathbb{Z}}

% ------------------------------------------------------------------------------
%  Proofs
% ------------------------------------------------------------------------------

% These definitions allow you to always use \redeclarecommand
\declarecommand{\Qa}{}
\declarecommand{\Qb}{}
\declarecommand{\Qc}{}
\declarecommand{\Qd}{}
\declarecommand{\Qe}{}
\declarecommand{\Qf}{}
\declarecommand{\Qg}{}
\declarecommand{\Qh}{}
\declarecommand{\Qi}{}
\declarecommand{\Qj}{}
\declarecommand{\Qk}{}
\declarecommand{\Ql}{}
\declarecommand{\Qm}{}
\declarecommand{\Qn}{}
\declarecommand{\Qo}{}
\declarecommand{\Qp}{}
\declarecommand{\Qq}{}
\declarecommand{\Qr}{}
\declarecommand{\Qs}{}
\declarecommand{\Qt}{}
\declarecommand{\Qu}{}
\declarecommand{\Qv}{}
\declarecommand{\Qw}{}
\declarecommand{\Qx}{}
\declarecommand{\Qy}{}
\declarecommand{\Qz}{}

\declarecommand{\Ra}{}
\declarecommand{\Rb}{}
\declarecommand{\Rc}{}
\declarecommand{\Rd}{}
\declarecommand{\Re}{}
\declarecommand{\Rf}{}
\declarecommand{\Rg}{}
\declarecommand{\Rh}{}
\declarecommand{\Ri}{}
\declarecommand{\Rj}{}
\declarecommand{\Rk}{}
\declarecommand{\Rl}{}
\declarecommand{\Rm}{}
\declarecommand{\Rn}{}
\declarecommand{\Ro}{}
\declarecommand{\Rp}{}
\declarecommand{\Rq}{}
\declarecommand{\Rr}{}
\declarecommand{\Rs}{}
\declarecommand{\Rt}{}
\declarecommand{\Ru}{}
\declarecommand{\Rv}{}
\declarecommand{\Rw}{}
\declarecommand{\Rx}{}
\declarecommand{\Ry}{}
\declarecommand{\Rz}{}

\declarecommand{\Sa}{}
\declarecommand{\Sb}{}
\declarecommand{\Sc}{}
\declarecommand{\Sd}{}
\declarecommand{\Se}{}
\declarecommand{\Sf}{}
\declarecommand{\Sg}{}
\declarecommand{\Sh}{}
\declarecommand{\Si}{}
\declarecommand{\Sj}{}
\declarecommand{\Sk}{}
\declarecommand{\Sl}{}
\declarecommand{\Sm}{}
\declarecommand{\Sn}{}
\declarecommand{\So}{}
\declarecommand{\Sp}{}
\declarecommand{\Sq}{}
\declarecommand{\Sr}{}
\declarecommand{\Ss}{}
\declarecommand{\St}{}
\declarecommand{\Su}{}
\declarecommand{\Sv}{}
\declarecommand{\Sw}{}
\declarecommand{\Sx}{}
\declarecommand{\Sy}{}
\declarecommand{\Sz}{}

%%% Local Variables:
%%% mode: latex
%%% TeX-master: "thesis"
%%% End:
