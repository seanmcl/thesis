\documentclass[]{article}

\usepackage[dvipsnames]{xcolor}
\usepackage[]{fullpage}
\usepackage[]{url}
\usepackage[]{amsthm}

\usepackage[
backref,
bookmarks,
bookmarksnumbered,
citecolor=Plum,
colorlinks=true,
filecolor=red,
linkcolor=blue,
pageanchor=true,
pdfpagelabels,
plainpages=false,
urlcolor=purple,
% pagebackref
]{hyperref}

\newcommand{\Atoms}{\mathcal{A}}
\newcommand{\And}{\wedge}
\newcommand{\Or}{\vee}
\newcommand{\Imp}{\rightarrow}
\newcommand{\Top}{\top}
\newcommand{\Bot}{\bot}
\newcommand{\Not}{\neg}
\newcommand{\Iff}{\Leftrightarrow}

\newcommand{\Hil}[2][]{#1 \vdash #2}

\newtheorem{theorem}{Theorem}[section]
\newtheorem*{theorem*}{Theorem}
\newtheorem{lemma}[theorem]{Lemma}
\newtheorem{corollary}[theorem]{Corollary}
\newtheorem{example}{Example}[section]
\newtheorem{definition}[theorem]{Definition}
\newtheorem*{remark}{Remark}


\begin{document}

\title{Proof search in intuitionistic propositional logic: a survey.}
\author{Sean McLaughlin\\ Google}
\maketitle

\begin{abstract}
  This report is a survey of the existing methods for automatically
  proving theorems in intuitionistic propositional logic.  It includes
  an extensive bibliography for those who wish to gain a deeper
  knowledge of the techniques summarized here.
\end{abstract}

\section{Introduction}
\label{sec:introduction}

Brouwer's intuitionistic logic was partly formalized by Kolmogorov,
Heyting and Glivenko between 1925 and 1930.  We call the propositional
fragment of their formalization intuitionistic propositional logic, or
IPL.  Proof search in IPL is more difficult than proof search in
propositional classical logic; the problem is PSPACE-complete.  This
document is a survey of the known proof search strategies for IPL.

\subsection{Language}
\label{sec:language}

The language of IPL is the same as that of classical propositional logic.
Assume an infinite set of symbols $\Atoms$, called \emph{atomic formulas}.
\emph{Formulas} of IPL are built from the following grammar:

\[
\mbox{Formulas F} ::= a ~|~ F \And F ~|~ \Top ~|~ F \Or F ~|~ \Bot ~|~
F \Imp F
\]

\noindent
where $a\in\Atoms$. Define $\Not A$ as $A \Imp \Bot$ and $A \Iff B$ as
$(A \Imp B) \And (B \Imp A)$.  The precedence order is, from strongest
to weakest, $\And, \Or, \Imp$. All binary connectives associate to the right.
For example, $A \And B \Or C \Or D \Imp E$ means
$((A \And B) \Or (C \Or D)) \Imp E$.

\subsection{Hilbert systems}
\label{sec:hilbert-systems}

The first formalizations were in the Hilbert style.  There are
many in the literature.  The following example comes from
\cite[Section 2.4]{Troelstra.2000.ProofTheory}.  The axioms are

\[
\begin{array}{l}
  A \Imp B \Imp A \\
  (A \Imp (B \Imp C)) \Imp ((A \Imp B) \Imp (A \Imp C)) \\
  A \Imp A \Or B \\
  B \Imp A \Or B \\
  (A \Imp C) \Imp (B \Imp C)  \Imp A \Or B \Imp C \\
  A \And B \Imp A\\
  A \And B \Imp B\\
  A \Imp B \Imp A \And B\\
  \Bot \Imp A\\
\end{array}
\]

Here $A,B,C$ are metavariables standing for any formula.
Deductions $\Hil[\Gamma]{A}$ are made under a set of assumptions $\Gamma$.
Proofs are achieved by reasoning from the axioms.

\begin{lemma}[Identity]
  For any formula $A$, $\Hil[A]{A}$.
\end{lemma}
\begin{proof}
  \begin{itemize}
  \item[]
  \item[1.]
    $\Hil[]{(A \Imp (B \Imp A) \Imp A) \Imp (A \Imp (B \Imp A)) \Imp
      (A \Imp A)}$
  \item[2.]
  \end{itemize}
\end{proof}

s k k

There are two rules of inference.

\[
\begin{array}{l}
  \Hil[\Gamma,A]{A} \\
  \mbox{If } \Hil[\Gamma]{A \Imp B} \mbox{ and } \Hil[\Gamma]{A}
  \mbox{ then } \Hil[\Gamma]{B}. \\
\end{array}
\]

Hilbert systems are useless as a means of proof search, as the reader
can readily verify by attempting a proof of even the simplest
of formulas.  (See Section~\ref{sec:ex-hilbert} for a proof
of $(A \Imp B) \Imp C \Or A \Imp B \Or C$.)  The problem is that
the formulas $A,B,C$ are unrestricted, what Pfenning calls
a non-analytic calculus~\cite{Pfenning.1984.CADE}.  This problem of
unrestricted formulas in proof systems led to Gentzen's development of
the sequent calculus in 1935.

\subsection{Sequent calculus}
\label{sec:sequent-calculus}



\section{Gentzen systems}
\label{sec:gentzen-systems}

~\cite{Kleene.1952.Metamathematics}.


abc~\cite{Dyckhoff.1992.JSL}

Gentzen Systems
\cite{Kleene.1952.Metamathematics}
\cite{Kleene.1952.Metamathematics}


\section{Hilbert systems}
\label{sec:hilbert-systems}


\section{Natural deduction}
\label{sec:natural-deduction}


\section{Loop detection}
\label{sec:loop-detection}

\section{Matrix methods}
\label{sec:matrix-methods}

\section{Model theory}
\label{sec:model-theory}

\section{Translations}
\label{sec:translations}

\section{Inverse methods}
\label{sec:inverse-method}

\section{Refutations}
\label{sec:refutations}


\section{Literature}
\label{sec:literature}

For more on the history of intuitionistic logic and its early
formalizations, see
\cite{
  Kleene.1952.Metamathematics,
  Troelstra.2000.ProofTheory,
  SEP.2013.DevelopmentOfIntuitionisticLogic}.

\section{Examples}
\label{sec:examples}

\subsection{Hilbert systems}
\label{sec:ex-hilbert}

\begin{lemma}
\label{lem:hilbert}
  \[
  (a \Imp b) \Imp (c \Or a) \Imp (b \Or c)
  \]
\end{lemma}

\begin{proof}

\begin{itemize}
\item[]
\item[1.] abc
\item[2.] abc
\end{itemize}

\end{proof}


\bibliographystyle{alpha}
\bibliography{review}

\end{document}

%%% Local Variables:
%%% mode: latex
%%% reftex-default-bibliography: ("review")
%%% TeX-master: t
%%% End:
