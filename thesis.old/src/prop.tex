
% \chapter{Propositional Logic}
% \label{chapter.prop}

% Intuitionistic propositional logic (IPL) is the starting point for our
% investigations into a uniform proof procedure for intuitionistic logics,
% and is a sublogic of all logics in later chapters.
% Though relatively simple compared with first-order logics,
% it poses some interesting challenges.  In particular, it exposes
% inefficiences in a naive implementation of the inverse method,
% the observation of which lead to important optimizations that
% apply to the other logics.

% We begin in Section~\ref{prop.sec.def}
% by defining two related sequent calculi for IPL, $\LJ$ and $\LJF$.
% $\LJ$ is a mostly standard presentation, equivalent to Gentzen's original
% LJ from 1934.  $\LJF$ is related to $\LJ$, but is more suitable
% to forward reasoning in the inverse method.  We conclude the section with
% a proof of soundness and completeness of $\LJF$ with respect to $\LJ$.
% Section~\ref{prop.sec.inverse} shows how to use the inverse method to
% turn $\LJF$ into a complete proof-search procedure.  In the process,
% we will observe a number of inefficiences.  Section~\ref{prop.sec.focus}
% defines focused calculi that correspond to $\LJ$.

% \section{Preliminaries}

% In this section we quickly review some background material and
% notations that will be necessary in the coming text.

% \paragraph{Multisets.}

% Let $\cU$ be a set. A \emph{(finite) multiset} of $\cU$
% is a function $f$ from
% $\cU$ to $\Nat$ where $f(x)=0$ for all but a finite number of elements of
% $\cU$.  The empty multiset, written $\cdot$ or $\emptyset$ is the
% function $\lambda x.\ 0$. If $f$ is a multiset then $f, x$ is a multiset
% where $(f, x) (x) = f(x)+1$ and $(f, x) (y)=f (y)$ when $x\neq y$.
% If $f$ and $g$ are multisets, then $f\Union g$ is a multiset where
% $(f\Union g) (x) = f(x)+g(x)$.  When we are writing both set and multiset union
% in the same part of the text, we will use $X \uplus Y$ for multiset union.
% % For a multiset $\Gamma$, write $\dedup(\Gamma)$ for the set obtained by deleting
% % all duplicates in $\Gamma$.
% The size of a set or multiset $\Gamma$ is written $\Card{\Gamma}$.

% \paragraph{Ordered sequences.}

% If $\cU$ is a set, a \emph{sequence} of $\cU$ is either empty (written $[]$)
% or has the form $x:S$ where $x\in \cU$ and $S$ is a sequence of $\cU$.
% We abbreviate $x_1 : x_2 : \cdots : x_n : []$ as $[x_1,\ldots,x_n]$.

% \paragraph{Notations.}

% If $\cU$ is a set, let $\PSet{\cU}$ be the powerset of $\cU$ and $\SSet{\cU}$ be
% the set of ordered sequences of elements of $\cU$.  $\SSet{\cU}_n$ is the
% set of sequences of $X$ with length $n$.

% \section{Sequent calculus}

% % \begin{quote}
% %   \textit{We define the backward calculus $\LJ$ for
% %     IPL, and remind the reader of some important properties.}
% % \end{quote}

% % \noindent

% \begin{definition}[Formulas]
%   Let $\Atoms$ be an infinite set of symbols, called \emph{atoms}.
%   Formulas of IPL are
%   \[
%   \mbox{Formulas } A ::= p \Sep A \And A \Sep \Top \Sep A \Or A
%   \Sep \Bot \Sep A \Imp A
%   \]
%   where $p\in\Atoms$ and $\And$, $\Or$, $\Imp$, $\Top$, $\Bot$
%   signify conjunction, disjunction, implication, truth, and falsehood
%   respectively.
%   Formulas of the first kind are called \emph{atomic}.
%   Derived connectives for negation and bi-implication are
%   \[
%   \begin{array}{rl}
%     \Not A &:= A \Imp \Bot \\
%     A \Iff B &:= (A \Imp B) \And (B \Imp A).
%   \end{array}
%   \]
% \end{definition}

% \noindent
% The semantics of formulas is typically given by the system of
% natural deduction due to Gentzen~\cite{Gentzen.1934.MZ}.  In natural deduction,
% the meaning of a formula is given by its introduction and elimination rules.
% An alternate semantics is given by the \emph{sequent calculus}, which is more
% appropriate for proof search.  Here the introduction and elimination rules
% are replaced by left and right rules, indicating how to use hypotheses
% and construct conclusions respectively.

% \begin{definition}[Sequents]
%   A \emph{sequent} is a pair $\ASet{\Gamma,\gamma}$
%   where $\Gamma$ is a multiset of formulas and
%   $\gamma$ is a set of formulas with at most one element\footnote{While it
%     is not necessary at the moment, we allow an empty consequent so that
%     we can use the same definition of sequents in the forward calculi that
%     occur later in this chapter.}.
%   $\Gamma$ is called the \emph{antecedents}
%   and $\gamma$ the \emph{consequent}.
%   We write $\ASet{\Gamma,\gamma}$ as $\PBSeq{\Gamma}{\gamma}$.
%   If $\gamma$ is empty, we write $\PBSeq{\Gamma}{\cdot}$.
%   If $\gamma$ is a singleton $\Set{A}$, we write $\PBSeq{\Gamma}{A}$.
%   If $\Gamma$ is empty, we write $\PBSeq{\cdot}{\gamma}$.
%   We denote the set of all sequents by $\Seqs$,
%   the set of all sequences of sequents by $\SeqSeqs$, and
%   the set of all sets of sequents by $\SetSeqs$.
% \end{definition}

% \begin{definition}[Inference rules]
%   An \emph{inference rule} (or simply \emph{rule})
%   is a pair $\ASet{\cC, \cH}$
%   where $\cC$ is a sequent, and $\cH$ is a sequence of sequents.
%   $\cC$ is called the \emph{conclusion} and
%   $\cH$ is called the \emph{premeses}.
%   If $\rho = \ASet{\cC, \cH}$ is a rule, define
%   $\Card{\rho} = \Card{\cH}$.
%   We write $\ASet{\cC, \ASet{\cH_1, \cH_2,\ldots,\cH_n}}$ as
%   \[
%   \Infer{\cC}{\cH_1 & \cH_2 & \ldots & \cH_n}
%   \]
%   A rule $\rho$ is called initial if $\Card{\rho}=0$.
% \end{definition}

% 
\newcommand{\LJInit}
{\Infer[$Init$]
  {\Seq{\Gamma,p}{p}}
  {}
}

\newcommand{\LJAndR}{
  \Infer[\And$-R$]
  {\Seq{\Gamma}{A_1 \And A_2}}
  {\Seq{\Gamma}{A_1} & \Seq{\Gamma}{A_2}}
}

\newcommand{\LJAndL}{
  \Infer[\And$-L$]
  {\Seq{\Gamma, A_1 \And A_2}{C}}
  {\Seq{\Gamma, A_1, A_2}{C}}
}

\newcommand{\LJImpR}{
  \Infer[\Imp$-R$]
  {\Seq{\Gamma}{A_1 \Imp A_2}}
  {\Seq{\Gamma, A_1}{A_2}}
}

\newcommand{\LJImpL}{
  \Infer[\Imp$-L$]
  {\Seq{\Gamma, A_1\Imp A_2}{C}}
  {\Seq{\Gamma, A_1\Imp A_2}{A_1} & \Seq{\Gamma, A_2}{C}}
}

\newcommand{\LJOrRL}{
  \Infer[\Or$-R$_1]
  {\Seq{\Gamma}{A_1 \Or A_2}}
  {\Seq{\Gamma}{A_1}}
}

\newcommand{\LJOrRR}{
  \infer[\Or$-R$_2]
  {\Seq{\Gamma}{A_1 \Or A_2}}
  {\Seq{\Gamma}{A_2}}
}

\newcommand{\LJOrL}{
  \infer[\Or$-L$]
  {\Seq{\Gamma, A_1 \Or A_2}{C}}
  {\Seq{\Gamma, A_1}{C} & \Seq{\Gamma, A_2}{C}}
}

\newcommand{\LJTopR}{
  \infer[\Top$-R$]
  {\Seq{\Gamma}{\Top}}
  {}
}

\newcommand{\LJTopL}{\mbox{No rule for $\Top$-L}}

\newcommand{\LJBotL}{
  \infer[\Bot$-L$]
  {\Seq{\Gamma, \Bot}{C}}
  {}
}

\newcommand{\LJBotR}{\mbox{No rule for $\Bot$-R}}

\begin{figure}[h]
  \begin{center}
    {\small
      \fbox{
        \begin{tabular}{c}
          \\
          \begin{tabular}{ccc}
            \LJInit
          \end{tabular}
          \\[2em]
          \begin{tabular}{cc}
            \LJAndR
            &
            \hspace{1cm}
            \LJAndL
          \end{tabular}
          \\[2em]
          \begin{tabular}{cc}
            \hspace{1cm}
            \LJTopR
            &
            \hspace{1cm}
            \LJTopL
          \end{tabular}
          \\[2em]
          \begin{tabular}{ccc}
            \LJOrRL
            &
            \hspace{10pt}
            \LJOrRR
            &
            \hspace{10pt}
            \LJOrL
          \end{tabular}
          \\[2em]
          \begin{tabular}{cc}
            \LJBotR
            &
            \hspace{20pt}
            \LJBotL
          \end{tabular}
          \\[2em]
          \begin{tabular}{cc}
            \LJImpR
            &
            \hspace{20pt}
            \LJImpL
          \end{tabular}
          \\[2em]
        \end{tabular}
      }
    }
  \end{center}
  \caption{$\LJ$}
  \label{fig.lj}
\end{figure}

%%% Local Variables:
%%% mode: latex
%%% TeX-master: "refute"
%%% End:

% 
\newcommand{\ab}{a \Imp b}
\newcommand{\ac}{a \Or c}
\newcommand{\abc}{(a\Imp b) \Imp c}
\newcommand{\bb}{\Bot \Or b}

\renewcommand{\Qa}{\PBSeq{a, b, \abc}{b}}
\renewcommand{\Ra}{\Infer[$Init$]{\Qa}{}}

\renewcommand{\Qb}{\PBSeq{b, \abc}{\ab}}
\renewcommand{\Rb}{\Infer[\Imp$-R$]{\Qb}{\Ra}}

\renewcommand{\Qc}{\PBSeq{b, c}{c}}
\renewcommand{\Rc}{\Infer[$Init$]{\Qc}{}}

\renewcommand{\Qd}{\PBSeq{b, c}{\ac}}
\renewcommand{\Rd}{\Infer[\Or$-L$]{\Qd}{\Rc}}

\renewcommand{\Qe}{\PBSeq{b, \abc}{\ac}}
\renewcommand{\Re}{\Infer[\Imp$-L$]{\Qe}{\Rd & \Rb}}

\renewcommand{\Qf}{\PBSeq{\Bot, \abc}{\ac}}
\renewcommand{\Rf}{\Infer[\Bot$-L$]{\Qf}{}}

\renewcommand{\Qg}{\PBSeq{\bb, \abc}{\ac}}
\renewcommand{\Rg}{\Infer[\Or$-L$]{\Qg}{\Rf & \Re}}

\renewcommand{\Qh}{\PBSeq{\bb}{(\abc) \Imp \ac}}
\renewcommand{\Rh}{\Infer[\Imp$-R$]{\Qh}{\Rg}}

\renewcommand{\Qj}{\PBSeq{\cdot}{\bb \Imp (\abc) \Imp \ac}}
\renewcommand{\Rj}{\Infer[\Imp$-R$]{\Qj}{\Rh}}

%\begin{sidewaysfigure}[angle=90]
\begin{figure}[H]
  \begin{center}
    \[
    \boxed{\Rj}
    \]
  \end{center}
\caption{A proof in $\LJ$}
\label{prop.lj-der}
%\end{sidewaysfigure}
\end{figure}


%%% Local Variables:
%%% mode: latex
%%% TeX-master: "../thesis"
%%% End:


% Figure~\ref{prop.lj} shows the inference rules of $\LJ$\footnote{
% $\LJ$ is equivalent to Gentzen's LJ~\cite{Gentzen.1934.MZ}.  There are a
% menagerie of sequent calculi equivlent to LJ but with slight technical
% differences.  See, for example, the many examples in \cite[Chapter
% 3]{Troelstra.2000.ProofTheory}.  LJ itself is not optimal for use in proof
% search due to its explicit structural rules.}, a standard
% sequent calculus for IPL.  Figure~\ref{prop.lj-der} shows an example
% proof in $\LJ$.  We remind the reader of some admissible rules of $\LJ$.

% \begin{lemma}[Weakening]
%   \label{prop.thm.weaken}
%   The rule
%   \[
%   \Infer[\mbox{Weaken}]
%   {\PBSeq{\Gamma,A}{\gamma}}
%   {\PBSeq{\Gamma}{\gamma}}
%   \]
%   is admissible in $\LJ$.
% \end{lemma}
% \begin{proof} Easy induction on the derivation. \end{proof}

% \begin{theorem}[Cut]
%   The rule
%   \[
%   \Infer[\mbox{Cut}]
%   {\PBSeq{\Gamma}{\gamma}}
%   {\PBSeq{\Gamma}{A} & \PBSeq{\Gamma,A}{\gamma}}
%   \]
%   is admissible in $\LJ$.
% \end{theorem}
% \begin{proof}
%   Due to Gentzen~\cite{Gentzen.1934.MZ}.  For a more modern proof,
%   see Troelstra~\cite{Troelstra.2000.ProofTheory}
% \end{proof}

% \begin{theorem}[Identity]
%   The rule
%   \[
%   \Infer[\mbox{Id}]
%   {\PBSeq{\Gamma, A}{A}}
%   {}
%   \]
%   is admissible in $\LJ$.
% \end{theorem}
% \begin{proof} Easy induction on $A$. \end{proof}

% \begin{theorem}[Contraction]
%   \label{prop.thm.contract}
%   The rule
%   \[
%   \Infer[\mbox{Contract}]
%   {\PBSeq{\Gamma, A}{\gamma}}
%   {\PBSeq{\Gamma, A, A}{\gamma}}
%   \]
%   is admissible in $\LJ$.
% \end{theorem}

% \begin{proof}
%   Induction on the derivation of $\PBSeq{\Gamma, A, A}{\gamma}$~\cite[Section 3.5.5]{Troelstra.2000.ProofTheory}.
% \end{proof}

% \begin{theorem}[Consistency]
%   \label{prop.thm.consistent}
%   $\PBSeq{\cdot}{\Bot}$ is not derivable.
% \end{theorem}
% \begin{proof} No inference rules apply. \end{proof}

% \section{Focusing}

% From a proof-search perspective, $\LJ$ is not optimal.  It admits too many
% proofs.  For example,

% \[
% \begin{array}{cc}
%   \Infer
%   {\PBSeq{a\And b}{b\And a}}
%   {
%     \Infer
%     {\PBSeq{a,b}{b\And a}}
%     {
%       \Infer
%       {\PBSeq{a,b}{b}}
%       {}
%       &
%       \Infer
%       {\PBSeq{a,b}{a}}
%       {}
%     }
%   }
%   &
%   \hspace{2em}
%   \Infer
%   {\PBSeq{a\And b}{b\And a}}
%   {
%     \Infer
%     {\PBSeq{a\And b}{b}}
%     {
%       \Infer
%       {\PBSeq{a,b}{b}}
%       {}
%     }
%     &
%     \Infer
%     {\PBSeq{a\And b}{a}}
%     {
%       \Infer
%       {\PBSeq{a,b}{a}}
%       {}
%     }
%     &
%   }
% \end{array}
% \]

% \noindent
% are both $\LJ$ derivations.  But the order in which we choose to decompose
% the formulas in the sequent is, in this case, irrelevant with respect to
% provability.  Call an inference rule \emph{invertible} if its conclusion
% is provable if and only if its premeses are.  A connective is called
% \emph{asynchronous} on the left (right) if its left (right) rule is invertible.
% Otherwise the connective is called \emph{synchronous} on the left (right).

% During proof search, invertible
% rules can be applied eagerly in some fixed order without affecting
% the provability of the sequent.  If we were to decree that we always
% apply invertible rules to the consequent before the antecedents, then
% the second proof would be a canonical representative of the proofs of
% $\PBSeq{a\And b}{b\And a}$.  The reason this is essential for proof search
% is that if a proof attempt were unsuccessful, say beginning with

% \[
% \Infer
% {\PBSeq{a\And b}{b\And c}}
% {
%   \Infer
%   {\PBSeq{a\And b}{b}}
%   {
%     \Infer
%     {\PBSeq{a,b}{b}}
%     {}
%   }
%   &
%   \Infer
%   {\PBSeq{a\And b}{c}}
%   {
%     ???
%   }
% }
% \]

% \noindent
% we shouldn't backtrack to search for a proof starting with

% \[
% \Infer
% {\PBSeq{a\And b}{b\And c}}
% {
%   \Infer
%   {\PBSeq{a,b}{b\And c}}
%   {
%     \Infer
%     {\PBSeq{a,b}{b}}
%     {}
%     &
%     \Infer
%     {\PBSeq{a,b}{c}}
%     {???}
%   }
% }.
% \]

% \noindent
% If the first attempt failed, so will the second.  This phenomenon is known as
% \emph{don't care nondeterminism}.  Choices that are not of this form are
% called \emph{don't know nondeterminism}.
% A more subtle redundancy occurs in non-invertible rules.  For example,

% \[
% \begin{array}{cc}
%   \Infer
%   {\PBSeq{d, d\Imp a}{(a\Or b)\Or c}}
%   {
%     \Infer
%     {\PBSeq{d, d\Imp a}{a\Or b}}
%     {
%       \Infer
%       {\PBSeq{d, d\Imp a}{a}}
%       {
%         \Infer
%         {\PBSeq{d, d\Imp a}{d}}
%         {}
%         &
%         \Infer
%         {\PBSeq{d, a}{a}}
%         {}
%       }
%     }
%   }
%   &
%   \hspace{2em}
%   \Infer
%   {\PBSeq{d, d\Imp a}{(a\Or b)\Or c}}
%   {
%     \Infer
%     {\PBSeq{d, d\Imp a}{a\Or b}}
%     {
%       \Infer
%       {\PBSeq{d, d\Imp a}{d}}
%       {}
%       &
%       \Infer
%       {\PBSeq{d, a}{a\Or b}}
%       {
%         \Infer
%         {\PBSeq{d, a}{a}}
%         {}
%       }
%     }
%   }
% \end{array}
% \]

% \noindent
% are two proofs of $\PBSeq{d, d\Imp a}{(a\Or b)\Or c}$.  Here the second is
% redundant because it does not obey the \emph{polarity} of the connectives;
% while decomposing $(a\Or b)\Or c$, we do not need to make the intermediate
% non-deterministic choice.  To make this precise, we refine the formulas
% of IPL.

% \subsection{Polarized formulas}

% Logical connectives have a \emph{polarity}, either positive or negative.  In
% linear logic, the polarity of each connective is uniquely determined.  Positive
% connectives are asynchronous on the left and synchronous on the right.  Negative
% connectives are synchronous on the left and asynchronous on the right.  This
% symmetry breaks down for intuitionistic logic where conjunction and truth are
% inherently ambiguous.  For example, both

% \[
% \Infer
% {\PBSeq{\Gamma, A_1 \And A_2}{C}}
% {\PBSeq{\Gamma, A_1, A_2}{C}}
% \]

% \noindent and the pair of rules

% \[
% \begin{array}{cc}
%   \Infer
%   {\PBSeq{\Gamma, A_1 \And A_2}{C}}
%   {\PBSeq{\Gamma, A_1}{C}}
%   &
%   \hspace{2em}
%   \Infer
%   {\PBSeq{\Gamma, A_1 \And A_2}{C}}
%   {\PBSeq{\Gamma, A_2}{C}}
% \end{array}
% \]

% \noindent
% are sound and complete.  In linear logic (see Section~\ref{chapter.linear}),
% there are two different conjunctions corresponding to the two possibilities:

% \[
% \begin{array}{ccc}
%   \Infer
%   {\PBSeq{\Gamma, A_1 \otimes A_2}{C}}
%   {\PBSeq{\Gamma, A_1, A_2}{C}}
%   &
%   \hspace{2em}
%   \Infer
%   {\PBSeq{\Gamma, A_1 \& A_2}{C}}
%   {\PBSeq{\Gamma, A_1}{C}}
%   &
%   \hspace{2em}
%   \Infer
%   {\PBSeq{\Gamma, A_1 \& A_2}{C}}
%   {\PBSeq{\Gamma, A_2}{C}}
% \end{array}
% \]

% \noindent
% In intuitionistic logic, since antecedents are combined and weakening is admissible,
% the distinction between the two forms of conjunction is lost.
% However, assigning polarities to the connectives is
% still a useful tool. In particular, we can take advantage of
% \emph{focusing}, described in the next section.  Thus, we will
% refine the structure of intuitionistic formulas by assigning polarities,
% yielding \emph{polarized formulas}~\cite{Lamarche.1995.LICS} where immediately
% nested formulas always have the same polarity, unless an explicit
% polarity-shifting connective $\Up$ or $\Down$ is encountered.  These coercions
% are called \emph{shifts}.  Implication has slightly special status, in that its
% left-hand side has opposite polarity from its right-hand side.  This is because
% in the sequent calculus for intuitionistic logic, the focusing behavior of
% connectives on the left-hand side is the opposite of their behavior on the
% right-hand side.

% \begin{align*}
%   \mbox{Positive formulas } A^+ &::= p^+ \Sep A^+ \Or A^+ \Sep \Bot \Sep A^+ \AndP A^+ \Sep \TopP \Sep \Down A^- \\
%   \mbox{Negative formulas } A^- &::= p^- \Sep A^+ \Imp A^- \Sep A^- ~ \AndN ~A^- \Sep \TopN \Sep \Up A^+
% \end{align*}

% \noindent The translation $A^-$ of an (unpolarized) formula $F$ in IPL
% is nondeterministic, subject only to the constraint that
% polarity is assigned to atomic formulas consistently (i.e., you can't have both
% $p^-$ and $p^+$ in the same translation), and $|A^-| = F$.  The arbitrary
% assignment of polarity to atomic formula is called a \emph{bias}.

% \[
% \begin{array}{lll}
%   |p^+| &=& p \\
%   |p^-| &=& p \\
%   |A^- \AndN B^-| &=& |A^-| \And |B^-| \\
%   |\TopP| &=& \Top \\
%   |A^+ \AndP B^+| &=& |A^+| \And |B^+|\\
%   |\TopN| &=& \Top \\
%   |A^+ \Or B^+| &=& |A^+| \Or |B^+| \\
%   |\Bot| &=& \Bot \\
%   |A^+ \Lolli B^-| &=& |A^+| \Imp |B^-| \\
%   |\Down A^-| &=& |A^-| \\
%   |\Up A^+| &=& |A^+| \\
% \end{array}
% \]

% \noindent
% For example, the formula $((a \Or c) \And (b \Imp c)) \Imp (a \Imp b) \Imp c$
% can be interpreted as any of the following polarized formulas (among others):

% \begin{align*}
%   &((\Down a^- \Or \Down c^-) \Tensor \Down(\Down b^- \Lolli c^-)) \Lolli (\Down (\Down a^- \Lolli b^-) \Lolli c^-) \\
%   &\Down\Up((\Down a^- \Or \Down c^-) \Tensor \Down(\Down b^- \Lolli c^-)) \Lolli (\Down\Up\Down (\Down a^- \Lolli b^-) \Lolli c^-) \\
%   &\Down(\Up (a^+ \Or c^+) \With (b^+ \Lolli \Up c^+)) \Lolli (\Down (a^+ \Lolli \Up b
%   ^+) \Lolli \Up c^+)
% \end{align*}

% \noindent
% (Shift operators have highest binding precedence in our presentation of the
% examples.)  As we will see, the choice of translation determines the search
% behavior on the resulting polarized formula.  Different choices can lead to
% search spaces with radically different structure~\cite{Chaudhuri.2006.IJCAR}.

% \subsection{Backward Focused Sequent Calculus}

% The backward focused calculus $\LPF$ is a refinement of $\LJ$ that eliminates
% don't-care nondeterministic choices, and manages don't-know
% nondeterminism by chaining such inferences in sequence.  Andreoli was
% the first to define this \emph{focusing} strategy and prove it
% complete~\cite{Andreoli.1992.JLC} for linear logic.  Similar proofs for other logics
% soon followed~\cite{Howe.1998.Thesis,Chaudhuri.2005.CSL,Chaudhuri.2006.IJCAR,Miller.2007.CSL,Zeilberger.2008.POPL},
% demonstrating that polarization and focusing can be applied to
% optimize search in a wide variety of logics.

% A difficulty with focused calculi is that they have a somewhat heavy notation.
% Past presentations define at least three classes of sequent, and numerous
% recursive judgements to define the system.  This makes the critical cut elimination
% theorem verbose, with dozens of different cases.  Here we follow
% Pfenning and Simmons' \emph{structural focalization} technique
% from~\cite{Simmons:2012:Thesis}.

% In structural focalization, there is only one sequent form, given by
% the following grammar

% \begin{align*}
%   \mbox{Antecedents } \ul{\Gamma} &::= \cdot\Sep A^-, \ul{\Gamma} \Sep A^+, \ul{\Gamma} \Sep [A^-],\ul{\Gamma} \Sep \Susp{A^+},\ul{\Gamma}\\
%   \mbox{Consequents } \ul{\gamma} &::= \cdot\Sep A^- \Sep A^+ \Sep [A^+] \Sep \Susp{A^-}\\
%   \mbox{Sequents } q &::= \PBSeq{\ul{\Gamma}}{\ul{\gamma}}
% \end{align*}


\chapter{Polarization and Focusing}

\section{Focusing}\label{prop.sec.focusing}
\subsection{Polarized formulas}\label{prop.sec.polar}

In linear logic, the polarity of each connective is uniquely
determined.  This is not true for intuitionistic logic where
conjunction and truth are inherently ambiguous.  We therefore
assign polarities to formulas in a preprocessing phase.  It is
convenient to represent the result as a \emph{polarized formula}~\cite{Lamarche.1995.LICS}
where immediately nested formulas always have the same polarity,
unless an explicit polarity-shifting connective $\Up$ or
$\Down$ is encountered.  These coercions are called \emph{shifts}.

Implication has slightly special status, in that its
left-hand side has opposite polarity from its right-hand side.
This is because in the sequent calculus for intuitionistic logic,
the focusing behavior of connectives on the left-hand side
is the opposite of their behavior on the right-hand side. (Here the
meta-variable $P$ ranges over atomic propositions.)

\begin{align*}
  \mbox{Positive formulas } A^+ &::= P^+ \Sep A^+ \Sum A^+ \Sep \Zero \Sep A^+ \Tensor A^+ \Sep \One \Sep \Down A^- \\
  \mbox{Negative formulas } A^- &::= P^- \Sep A^+ \Lolli A^- \Sep A^- ~ \With ~A^- \Sep \Top \Sep \Up A^+
\end{align*}

\noindent The translation $A^-$ of an (unpolarized) formula $F$ in IPL
is nondeterministic, subject only to the constraint that
the translation $|A^-| = F$.

\medskip
\begin{tabular}{c@{\hspace{15pt}}c@{\hspace{15pt}}c}
$|A^+ \Sum B^+| = |A^+| \Or |B^+|$ & $|\Zero| = \Bot$ & $|P^+| = P$ \\
$|A^+ \Tensor B^+| = |A^+| \And |B^+|$ & $|\One| = \top$ & $|\Down A^-| = |A^-|$ \\
$|A^- \With B^-| = |A^-| \And |B^-|$ & $|\Top| = \top$ & $|P^-| = P$ \\
$|A^+ \Lolli B^-| = |A^+| \Imp |B^-|$ & $|\Up A^+| = |A^+|$
\end{tabular}
\medskip

For example, the formula $((A \Or C) \And (B \Imp C)) \Imp (A \Imp B) \Imp C$
can be interpreted as any of the following polarized formulas (among others):

{\small
\begin{align*}
&((\Down A^- \Sum \Down C^-) \Tensor \Down(\Down B^- \Lolli C^-)) \Lolli (\Down (\Down A^- \Lolli B^-) \Lolli C^-) \\
&\Down\Up((\Down A^- \Sum \Down C^-) \Tensor \Down(\Down B^- \Lolli C^-)) \Lolli (\Down\Up\Down (\Down A^- \Lolli B^-) \Lolli C^-) \\
&\Down(\Up (A^+ \Sum C^+) \With (B^+ \Lolli \Up C^+)) \Lolli (\Down (A^+ \Lolli \Up B^+) \Lolli \Up C^+)
\end{align*}
}

Shift operators have highest binding precedence in our presentation of the
examples.  As we will see, the choice of translation determines the search
behavior on the resulting polarized formula.  Different choices can lead to
search spaces with radically different structure~\cite{Chaudhuri.2006.IJCAR}.

% A sequent of intuitionistic logic has the form $\Stable{\Gamma}{A}$,
% where $\Gamma$ is a set or multiset of formulas.  For
% purposes of Imogen it is convenient to always maintain $\Gamma$
% as a set, without duplicates.
% Since we can always eagerly decompose negative connectives
% on the right of a sequent and positive connectives on the left,
% the only sequents in our polarized calculus we need to consider
% have negative formulas on the left or positive formulas on the
% right, in addition to atoms which can appear with either polarity
% on either side.  The right-hand side could also be empty if we are
% deriving a contradiction.   We call such sequents \emph{stable}.
% \[\begin{array}{llcl}
% \mbox{Stable Hypotheses} & \Gamma &::= & \cdot \Sep \Gamma, A^- \Sep \Gamma, P^+ \\
% \mbox{Stable Conclusions} & \gamma &::= & A^+ \Sep P^- \Sep \cdot \\
% \mbox{Stable Sequents} & \multicolumn{3}{l}{\Stable{\Gamma}{\gamma}}
% \end{array}\]

\subsection{Derived rules}
\label{prop.sec.derived}

Thm: ===> A iff for any A', if |A'| = A then ===> A'

\section{The Polarized Inverse Method}\label{prop.sec.pinverse}

%%% Local Variables:
%%% mode: latex
%%% TeX-master: "thesis"
%%% End:


% \begin{theorem}[Weakening]
%   If $\PBSeq{\ul{\Gamma}}{\ul{\gamma}}$ then
%   $\PBSeq{\ul{\Gamma}, \ul{\Gamma'}}{\ul{\gamma}}$.
% \end{theorem}
% \begin{proof}
%   Straightforward induction on the derivation
%   $\PBSeq{\ul{\Gamma}}{\ul{\gamma}}$, adding the antecedents
%   $\ul{\Gamma'}$ at each step.
% \end{proof}

% \begin{theorem}[Substitution]
%   \begin{itemize}
%   \item []
%   \item
%     If $\PBSeq{\Gamma}{[A^+]}$ and $\PBSeq{\ul{\Gamma'},\Susp{A^+}}{\ul{\gamma}}$ then
%     $\PBSeq{\Gamma,\ul{\Gamma'}}{\ul{\gamma}}$.
%   \item
%     If $\PBSeq{\ul{\Gamma}}{\Susp{A^-}}$ and $\PBSeq{\Gamma',\Foc{A^-}}{\gamma}$ then
%     $\PBSeq{\ul{\Gamma},\Gamma'}{\gamma}$.
%   \end{itemize}
% \end{theorem}

% \begin{proof}
%   \begin{itemize}
%   \item[]
%   \item
%     Induction on the second derivation.  If the last rule is $\mbox{Id}^+$, weaken the first
%     derivation to get the result.
%   \item
%     Induction on the first derivation.  If the last rule is $\mbox{Id}^-$, weaken the second
%     derivation to get the result.
%   \end{itemize}
% \end{proof}

% \noindent
% Following Simmons, we use admissible rules in proofs by writing
% admissible rules applications with dotted lines to differentiate
% them from the primitive inference rules of the logic.  The above theorems
% give the following rules

% \[
% \begin{array}{c}
%   \Inferd[$Weaken$]
%   {\PBSeq{\ul{\Gamma}, \ul{\Gamma'}}{\ul{\gamma}}}
%   {\PBSeq{\ul{\Gamma}}{\ul{\gamma}}}
%   \\[2em]
%   \begin{array}{cc}
%     \Inferd[$Subst$^+]
%     {\PBSeq{\Gamma, \ul{\Gamma'}}{\ul{\gamma}}}
%     {
%       \PBSeq{\Gamma}{\Foc{A^+}}
%       &
%       \PBSeq{\ul{\Gamma'}, \Susp{A^+}}{\ul{\gamma}}
%     }
%     &
%     \hspace{2em}
%     \Inferd[$Subst$^-]
%     {\PBSeq{\ul{\Gamma}, \Gamma'}{\gamma}}
%     {
%       \PBSeq{\ul{\Gamma}}{\Susp{A^-}}
%       &
%       \PBSeq{\Gamma', \Foc{A^-}}{\gamma}
%     }
%   \end{array}
% \end{array}
% \]

% \begin{theorem}[Identity expansion]
%   \begin{itemize}
%   \item[]
%   \item If $\PBSeq{\Gamma}{\Susp{A^-}}$ then $\PBSeq{\Gamma}{A^-}$.
%   \item If $\PBSeq{\Gamma, \Susp{A^+}}{\gamma}$ then $\PBSeq{\Gamma, A^+}{\gamma}$.
%   \end{itemize}
% \end{theorem}

% \noindent As above, we write

% \[
% \begin{array}{cc}
%   \Inferd[$\eta$^-]
%   {\PBSeq{\Gamma, A^-}{\gamma}}
%   {\PBSeq{\Gamma, \Susp{A^-}}{A^-}}
%   &
%   \hspace{2em}
%   \Inferd[$\eta$^+]
%   {\PBSeq{\Gamma}{A^+}}
%   {\PBSeq{\Gamma}{\Susp{A^+}}}
%   {}
% \end{array}
% \]

% \begin{corollary}[Identity]
%   The rules

%   \[
%   \begin{array}{cc}
%     \Inferd[$Ident$^-]
%     {\PBSeq{\Gamma, A^-}{A^-}}
%     {}
%     &
%     \hspace{2em}
%     \Inferd[$Ident$^+]
%     {\PBSeq{\Gamma, A^+}{A^+}}
%     {}
%   \end{array}
%   \] are admissible.
% \end{corollary}

% \begin{proof}
%   \[
%   \begin{array}{cc}
%     \Inferd[$\eta$^-]
%     {\PBSeq{\Gamma, A^-}{A^-}}
%     {
%       \Infer[$Focus-L$]
%       {\PBSeq{\Gamma, A^-}{\Susp{A^-}}}
%       {
%         \Infer[$Id$^-]
%         {\PBSeq{\Gamma, \Foc{A^-}}{\Susp{A^-}}}
%         {}
%       }
%     }
%     &
%     \hspace{2em}
%     \Inferd[$\eta$^+]
%     {\PBSeq{\Gamma, A^+}{A^+}}
%     {
%       \Infer[$Focus-R$]
%       {\PBSeq{\Gamma, \Susp{A^+}}{A^+}}
%       {
%         \Infer[$Id$^+]
%         {\PBSeq{\Gamma, \Susp{A^+}}{\Foc{A^+}}}
%         {}
%       }
%     }
%   \end{array}
%   \]
% \end{proof}

% \begin{proof}
% Mutual induction on the formulas $A^+$, $A^-$.  Note that we will
% use the admissible rules Ident and $\eta$, but only on strictly
% smaller subformulas.

% 
\[
\begin{array}{c}
  
\renewcommand{\Qa}{\PBSeq{\Gamma, A^-}{A^-}}
\renewcommand{\Ra}{\Infer[$Ident$^-]{\Qa}{}}

\renewcommand{\Qb}{\PBSeq{\Gamma, A^-}{\Foc{\Down A^-}}}
\renewcommand{\Rb}{\Infer[\Down$-R$]{\Qb}{\Ra}}

\renewcommand{\Qc}{\PBSeq{\Gamma, \Susp{\Down A^-}}{\gamma}}
\renewcommand{\Rc}{\deduce{\Qc}{\cD}}

\renewcommand{\Qd}{\PBSeq{\Gamma, A^-}{\gamma}}
\renewcommand{\Rd}{\Infer[$Subst$^+]{\Qc}{\Rb & \Rc}}

\renewcommand{\Qe}{\PBSeq{\Gamma, \Down A^-}{\gamma}}
\renewcommand{\Re}{\Infer[\Down$-L$]{\Qe}{\Rd}}

\begin{array}{ccc}
  \Infer[\Exp^+]{\Qe}{\Rc}
  &
  \hspace{1em}
  \hookrightarrow
  &
  \hspace{-1em}
  \Re
\end{array}

%%% Local Variables:
%%% mode: latex
%%% TeX-master: "../../thesis"
%%% End:

  \\[6em]
  
\renewcommand{\Qa}{\PBSeq{\Gamma, \Susp{A^+}}{\Foc{A^+}}}
\renewcommand{\Ra}{\Infer[$Id$^+]{\Qa}{}}

\renewcommand{\Qb}{\PBSeq{\Gamma, \Susp{B^+}}{\Foc{B^+}}}
\renewcommand{\Rb}{\Infer[$Id$^+]{\Qb}{}}

\renewcommand{\Qc}{\PBSeq{\Gamma, \Susp{A^+},\Susp{B^+}}{\Foc{A^+\AndP B^+}}}
\renewcommand{\Rc}{\Infer[\AndP$-R$]{\Qc}{\Ra & \Rb}}

\renewcommand{\Qd}{\PBSeq{\Gamma, \Susp{A^+\AndP B^+}}{\gamma}}
\renewcommand{\Rd}{\deduce{\Qd}{\cD}}

\renewcommand{\Qe}{\PBSeq{\Gamma, \Susp{A^+},\Susp{B^+}}{\gamma}}
\renewcommand{\Re}{\Infer[$Subst$^+]{\Qe}{\Rc & \Rd}}

\renewcommand{\Qf}{\PBSeq{\Gamma, \Susp{A^+},B^+}{\gamma}}
\renewcommand{\Rf}{\Infer[\eta^+]{\Qf}{\Re}}

\renewcommand{\Qg}{\PBSeq{\Gamma, A^+,B^+}{\gamma}}
\renewcommand{\Rg}{\Infer[\eta^+]{\Qg}{\Rf}}

\renewcommand{\Qh}{\PBSeq{\Gamma, A^+\AndP B^+}{\gamma}}
\renewcommand{\Rh}{\Infer[\AndP$-L$]{\Qh}{\Rg}}

\begin{array}{ccc}
  \begin{array}{cc}
    \Infer[\eta^+]{\Qh}{\Rd}
    &
    \hspace{0em}
    \hookrightarrow
  \end{array}
  &
  \hspace{-8em}
  \Rh
\end{array}

%%% Local Variables:
%%% mode: latex
%%% TeX-master: "../../thesis"
%%% End:

\end{array}
\]

%%% Local Variables:
%%% mode: latex
%%% TeX-master: "../thesis"
%%% End:

% \end{proof}

% \begin{theorem}[Soundness]
%   \begin{itemize}
%   \item[]
%   \item If $\CSeq{\Gamma}{A}$ then $\PBSeq{\Gamma}{A}$
%   \end{itemize}
% \end{theorem}

% \begin{proof}
%   Routine induction of the derivations.
% \end{proof}

% \begin{theorem}[Completeness]
%   If $\PBSeq{\cdot}{A}$ then $\CSeq{\cdot}{A}$
% \end{theorem}
% \begin{proof}
%   There are a number of proofs in the literature for intuitionistic
% linear logic, e.g.~\cite{Chaudhuri.2006.Thesis} and via translation
% for intuitionistic logic~\cite{Miller.2007.Focusing,Liang.2009.TCS}.
% IPL is a subset of the logic in Chapter~\ref{chapter.constraints}, so
% this is an instance of Theorem~\ref{constr.thm.complete} as well.
% \end{proof}

% \subsection{Synthetic Connectives and Derived Rules}
% \label{prop.sec.derived}

% We have already observed that backward proofs have the
% property that the proof is broken into blocks, with stable sequents at
% the boundary.  The only rules applicable to stable sequents are the
% rules that select a formula on which to focus.  It is the formulas
% occurring in stable sequents that form the primary objects of our further
% inquiry.

% It helps to think of such formulas, abstracted over their free
% variables, as \emph{synthetic connectives}~\cite{Andreoli.2001.APAL}.
% Define the synthetic connectives of a formula $A$ as all subformulas
% of $A$ that could appear in stable sequents in a focused backward proof.
% In a change of perspective, we can consider each block of a proof as
% the application of a left or right rule for a synthetic connective.
% The rules operating on synthetic connectives are derived from the
% rules for its constituent formulas.  We can thus consider a
% backward proof as a proof using only these synthetic (derived) rules.
% Each derived rule then corresponds to a block of the original proof.

% Since we need only consider stable sequents and synthetic connectives,
% we can simplify notation, and ignore the (empty) positive left and
% negative right zones in the derived rules.
% Write $\PBSeq{\Gamma}{C}$ as
% \renewcommand{\PBSeq}[2]{\ensuremath{#1 \Longrightarrow #2}}
% $\PBSeq{\Gamma}{C}$.
% As a further simplification, we can give formulas a predicate label
% and abstract over its free variables.  This labeling technique is described in
% detail in~\cite{Voronkov.2001.Handbook}.  For the remainder, we assume
% this labeling has been carried out.  Define an \emph{atomic formula}
% as either a label or a predicate applied to a (possibly
% empty) list of terms.  After labeling, our sequents consist entirely
% of atomic formulas.

% \begin{example}
% Figure~\ref{prop.blocks} shows a focused proof.
% Here the inversion phases are colored blue and the focusing phases
% are purple.  The stable sequents, on the boundary, are colored red.
% While it looks rather complex for such a simple formula, keep in
% mind that the only choices are made at stable sequents.  Note
% that the bias of the atoms is essential to the shape of the proof.
% A different choice would lead to a proof with a much different structure.

% The synthetic connectives are $a \Imp b$
% and $(a \Imp b)\Imp c$.  There is a single derived rule for each
% synthetic connective (though this is not the case in general).  The
% atoms are assigned negative polarity.  We implicitly carry the
% principal formula of a left rule to all of its premises.

% \[
% \begin{array}{cc}
%   \Infer[r_1]{\PBSeq{a, a\Imp b}{b}}{}
%   &\hspace{2em}
%   \Infer[r_2]{\PBSeq{\Down (a\Imp b)\Imp c}{c}}{\PBSeq{a}{b}}
% \end{array}
% \]

% \noindent
% Using the derived rules, Figure~\ref{prop.blocks} can be compressed to the succinct

% \[
% \Infer[r_2]
% {\PBSeq{\Down (a\Imp b)\Imp c, a\Imp b}{c}}
% {\Infer[r_1]{\PBSeq{a, a\Imp b}{b}}{}}
% \]
% \end{example}


%%% Local Variables:
%%% mode: latex
%%% TeX-master: "thesis"
%%% End:
