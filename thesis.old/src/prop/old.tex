
\begin{theorem}[Completeness]
  \label{prop.thm.forward-complete}
  If $\Seq{\Gamma}{A}$ is derivable then there is a derivable
  sequent $\FSeq{\Gamma'}{\gamma'}$ where
  $\FB{\FSeq{\Gamma'}{\gamma'}}\Subsumes \Seq{\Gamma}{A}$.
\end{theorem}

\begin{proof}
  By induction on the derivation of $\Seq{\Gamma}{A}$.
  We show some representative rules.
  \begin{description}
  \item[Case:]
    \[ \PBInitRule \]
    \begin{tabbing}
      $\FSeq{p}{p}$ \` Init ($\gamma'=\Set{p}$, $\Gamma'=\Set{p}$) \\
    \end{tabbing}
  \item[Case:]
    \[ \PBImpRRule \]
    \begin{tabbing}
      $\FSeq{\Gamma_1}{\gamma}$ \` IH \\
      $\Gamma_1 \subseteq \Gamma, A_1$ \` IH \\
      $\gamma \subseteq \Set{A_2}$ \` IH \\
    \end{tabbing}
    \begin{description}
    \item[Case:]
      $A_1\in\Gamma_1$, $A_2\in\gamma$
      \begin{tabbing}
        $\Gamma_1=\Gamma',A_1$ \\
        $\gamma=\Set{A_2}$ \\
        $\Gamma' \subseteq \Gamma$ \\
        $\FSeq{\Gamma',A_1}{A_2}$ \` IH \\
        $\FSeq{\Gamma'}{A_1\Imp A_2}$ \` $\Imp$-R$_1$ \\
      \end{tabbing}
    \end{description}
    Other cases are similar, % with $\PFImpRb$ and $\PFImpRc$
  \item[Case:]
    \[ \PBImpLRule \]
    \begin{tabbing}
      $\FSeq{\Gamma_1}{\gamma_1}$ \` IH (1) \\
      $\Gamma_1\subseteq \Gamma, A_1\Imp A_2$ \` IH \\
      $\gamma_1\subseteq \Set{A_1}$ \` IH \\
      $\FSeq{\Gamma_2}{\gamma_2}$ \` IH (2) \\
      $\Gamma_2\subseteq \Gamma, A_2$ \` IH \\
      $\gamma_2\subseteq \Set{C}$ \` IH \\
    \end{tabbing}
    \begin{description}
    \item[Case:]
      $A_2\not\in\Gamma_2$
      Then setting $\Gamma'=\Gamma_2$ and $\gamma=\gamma_2$ will do.
    \item[Case:]
      $A_2\in\Gamma_2$ and $A_1\not\in\gamma_1$.
      Then setting $\Gamma'=\Gamma_1$ and $\gamma=\gamma_1$ will do.
    \item[Case:]
      $A_2\in\Gamma_2$ and $A_1\in\gamma_1$.
      Let $\Gamma_2=\Gamma_2',A_2$.  Then $\Gamma_2' \subseteq \Gamma$.
      Set $\gamma=\gamma_2$.
      Because $\Gamma_1 \subseteq \Gamma, A_1\Imp A_2$ and $\Gamma_2' \subseteq \Gamma$,
      some series of contractions
      of $\Gamma_1\Union\Gamma_2'$ will yield
      $\Gamma_3 \subseteq \Gamma, A_1\Imp A_2$.
      \[
      \Infere[$Contract$]
      { \FSeq{\Gamma_3,A_1\Imp A_2}{\gamma} }
      {
        \Infer[\Imp$-L$]
        {\FSeq{\Gamma_1\Union\Gamma_2',A_1\Imp A_2}{\gamma}}
        {\FSeq{\Gamma_1}{A_1} & \FSeq{\Gamma_2',A_2}{\gamma}}
      }
      \]
      If $A_1\Imp A_2\in\Gamma_3$, one more contraction will yield
      $\Gamma'$ such that $\Gamma' \subseteq \Gamma$.  Otherwise set $\Gamma'=\Gamma_3$.
    \end{description}
  \end{description}
\end{proof}

\begin{theorem}
  \label{prop.thm.subsume-goal}
  \begin{itemize}
  \item[]
  \item There is no derivation of $\FSeq{\cdot}{\cdot}$
  \item There is no derivable sequent that subsumes $\FSeq{\cdot}{\xi}$
  \end{itemize}
\end{theorem}

\begin{proof}
  \begin{itemize}
  \item[]
  \item By Theorem~\ref{prop.thm.forward-sound},
    a derivation of $\FSeq{\cdot}{\cdot}$ would yield a derivation of
    $\Seq{\cdot}{\Bot}$.  But this contradicts Theorem~\ref{prop.thm.consistent}.
  \item There is no sequent besides $\FSeq{\cdot}{\cdot}$ that subsumes
    $\FSeq{\cdot}{\xi}$.
  \end{itemize}
\end{proof}

\begin{remark}
Figure~\ref{prop.forward} shows a forward calculus $\LPF$ that is sound and
complete with respect to the backward rules.  There are some surprises.  Unlike
$\LPB$, contraction is not admissible, and is thus needed as an explicit rule.
The single rule $\Imp$-R becomes three rules %$\PFImpRa$, $\PFImpRb$, $\PFImpRc$, depending on whether the backward rule uses the antecedent
(consequent) or not.  (Consider, for example, trying to prove $\Bot\Imp A$.)
Figure~\ref{prop.backward-der} shows a derivation in $\LPB$ and
Figure~\ref{prop.forward-der} its corresponding derivation in $\LPF$.
\end{remark}

\subsection{Rule specialization}
\label{prop.sec.specialization}

$\LPF$ still does not satisfy the finite rule property.  There are still an
infinite number of instances of the \emph{Init} rule. The second part also fails.
For example, the subformula $A_2$
in \[\PBOrRLRule\] is unrestricted.  Here we are saved by the subformula
property. We can instantiate the rules with (the finite number of)
subformulas of the goal sequent.  Fix a formula $\xi$ to be checked for
provability for the remainder of this section.  Define $\LPFx$ to be $\LPF$
restricted to signed subformulas of $\xi$.  Then $\LPFx$ satisfies the finite
rule property.  For example, if $\xi=p_1\And p_2\Imp p_2\And p_1$
then

\[
\LPFx =
\begin{cases}
  \begin{array}{c@{\hspace{2em}}c@{\hspace{2em}}c}
    \Infer{\FSeq{p_1}{p_1}}{}
    &
    \Infer{\FSeq{p_2}{p_2}}{}
    &
    \Infer{\FSeq{\Gamma_1\Union\Gamma_2}{p_2\And p_1}}{\FSeq{\Gamma_1}{p_2} & \FSeq{\Gamma_2}{p_1}}
    \\[2em]
    \Infer{\FSeq{\Gamma,p_1\And p_2}{\gamma}}{\FSeq{\Gamma,p_1,p_2}{\gamma}}
    &
    \Infer{\FSeq{\Gamma}{\xi}}{\FSeq{\Gamma,p_1\And p_2}{p_2\And p_1}}
    &
    \Infer{\FSeq{\Gamma}{\xi}}{\FSeq{\Gamma}{p_2\And p_1}}
  \end{array}
\end{cases}
\]

\begin{theorem}[Soundness and completeness of $\LPFx$]
  {\ \\}
  \begin{quote}
    $\Seq{\cdot}{\xi}$ is derivable in $\LPB$
    iff $\FSeq{\cdot}{\xi}$ is derivable in $\LPFx$
  \end{quote}
\end{theorem}

\begin{proof}
  Theorems~\ref{prop.thm.forward-sound}, \ref{prop.thm.forward-complete},
  and \ref{prop.thm.subformula}.
\end{proof}

\section{The Inverse Method}

\begin{quote}
  \textit{
    We define the inverse method, a process for obtaining a forward
    proof search strategy from the forward calculus.
  }
\end{quote}

\subsection{Rule generation}

We now describe a systematic method for simplifying a set of
forward inference rules into

\begin{definition}

\end{definition}

\begin{definition}
  Let $A^s\in\subfs(\xi)$, $s\in\Set{+,-}$.  Define

  \[
  \rules(A^s) \EqDef
  \begin{cases}
    \Infer{\FSeq{p}{p}}{} &
    \mbox{$A=p$, $p$ is two-sided}
    \\[1em]
    \Infer
    {\FSeq{\cdot}{A_1\And A_2}}
    {\FSeq{\cdot}{A_1} & \FSeq{\cdot}{A_2}} &
    \mbox{$A=A_1\And A_2$, $s=+$}
    \\[1em]
    \Infer
    {\FSeq{A_1\And A_2}{\cdot}}
    {\FSeq{A_1, A_2}{\cdot}} &
    \mbox{$A=A_1\And A_2$, $s=-$}
    \\[1em]
    \Infer
    {\FSeq{\cdot}{\Top}}
    {} &
    \mbox{$A=\Top$, $s=+$}
    \\[1em]
    \Infer
    {\FSeq{\cdot}{A_1\Or A_2}}
    {\FSeq{\cdot}{A_1}}
    ,\hspace{1em}
    \Infer
    {\FSeq{\cdot}{A_1\Or A_2}}
    {\FSeq{\cdot}{A_2}} &
    \mbox{$A=A_1\Or A_2$, $s=+$}
    \\[1em]
    \Infer
    {\FSeq{A_1\Or A_2}{\cdot}}
    {\FSeq{A_1}{\cdot} & \FSeq{A_2}{\cdot}} &
    \mbox{$A=A_1\Or A_2$, $s=-$}
    \\[1em]
    \Infer
    {\FSeq{\Bot}{\cdot}}
    {} &
    \mbox{$A=\Bot$, $s=-$}
    \\[1em]
    \Infer
    {\FSeq{\cdot}{A_1\Imp A_2}}
    {\FSeq{A_1}{A_2}} &
    \mbox{$A=A_1\Imp A_2$, $s=+$}
    \\[1em]
    \Infer
    {\FSeq{A_1\Imp A_2}{\cdot}}
    {\FSeq{\cdot}{A_1} & \FSeq{A_2}{\cdot}} &
    \mbox{$A=A_1\Imp A_2$, $s=-$}
   \end{cases}
  \]
\end{definition}

\begin{definition}
  \[
  \LPFs = \bigcup_{A\in\subfs(\xi)} \rules(A)
  \]
\end{definition}

\noindent
Our running example becomes

\[
\LPFs =
\begin{cases}
  \begin{array}{ccc}
    \Infer{\FSeq{p_1}{p_1}}{}
    &
    \hspace{2em}
    \Infer{\FSeq{p_2}{p_2}}{}
    &
    \hspace{2em}
    \Infer{\FSeq{}{p_2\And p_1}}{\FSeq{\cdot}{p_2} & \FSeq{\cdot}{p_1}}
    \\[2em]
    \Infer{\FSeq{p_1\And p_2}{\cdot}}{\FSeq{p_1,p_2}{\cdot}}
    &
    \hspace{2em}
    \Infer{\FSeq{\cdot}{\xi}}{\FSeq{p_1\And p_2}{p_2\And p_1}}
  \end{array}
\end{cases}
\]

\subsection{Derivations and proof search}
\label{prop.sec.search}

\begin{definition}
  \begin{itemize}
  \item[]
  \item The \emph{goal} is $\FSeq{\cdot}{\xi}$.
  \item A \emph{database} is a set of sequents.
  \end{itemize}
\end{definition}

\noindent
Proof search in $\LPFx$ proceeds by
maintaining a database of provable sequents.  The database initially contains
only the initial sequents.  At each step, we use the matching
algorithm defined above to match hypotheses of the rules to sequents
in the existing database, thereby producing new sequents.  In time, we either generate
the goal, or no new sequents can be generated.  In the later case, $\cD$ is said
to be \emph{saturated}, and $\xi$ is unprovable.

We formalize this method of proof search following Voronkov~\cite[Section
7.4]{Voronkov.2001.Handbook}.  There are two operations on databases, adding
sequents via rule matching, and removing sequents by subsumption.

\begin{definition}
  We define a relation $\cD\steps\cD'$ between databases with the following
  properties.
  \begin{itemize}
  \item If $R$ is an inference rule,
    and $q_1, \ldots, q_n\in\cD$, and $q = \match(R, q_1,\ldots, q_n)$, then
    $\cD\steps\cD\Union{q}$.  Such an inference is called an
    \emph{inference step}.

  \item If$\Set{q_1,q_2}\subseteq\cD$ and  $q_1\Subsumes q_2$
    then $\cD\steps\cD\Diff\Set{q_2}$.
    Such an inference is called a \emph{subsumption step}.

  \item A \emph{step} is an inference step or a subsumption step.

  \item A \emph{derivation} is a sequence $\cD_0\steps\cD_1\steps\cdots$
    of databases, where $\cD_0$
    are the initial sequents and for each $i$, $\cD_i\steps\cD_{i+1}$.

  \item A derivation \emph{succeeds} if some $\cD_i$ contains the goal.
  \item A derivation \emph{fails} if it doesn't succeed.
  \end{itemize}
\end{definition}

\noindent
A technical problem arises due to the possibility of a derivation where
a sequent is repeatedly added and deleted, e.g. by an inference step followed
by a subsumption step.

\[
\cD\Union\Set{q_1,q_2}\steps\cD\Union\Set{q_1}\steps\cD\Union\Set{q_1,q_2}\steps\cdots
\]

\noindent We wish to rule out these sorts of derivations, which motivates the
notion of \emph{fair} derivations.

\begin{definition}
  The \emph{limit} of a derivation $\cD$ is
  \[
  \Dlimit = \bigcup_i\bigcap_{j\geq i}\cD_j
  \]

  \noindent
  A derivation is called \emph{fair} if for every inference rule $R$
  if $q_1$, \ldots, $q_n$, belong to the limit of the derivation, and
  $q = \match(R, q_1, \ldots, q_n)$ then there is some $i$ such that
  the inference $\cD_i\steps\cD_{i+1}$ where $\cD_{i+1}=\cD\Union\Set{q}$
  is a step in the derivation.
\end{definition}

\chapter{Propositional Logic}
\label{chapter.prop}

% In this chapter we describe the inverse method using intuitionistic
% propositional logic (IPL).  Its study elucidates many of the issues involved in
% efficient automated reasoning for the more complex logics that follow in later
% chapters.  Despite its relative simplicity, it has some practical applications.
% Mendler~\cite{Mendler.1996.Tableaux} uses IPL to model stabilization
% delays in physical electronic circuits, and Gurevich~\cite{Gurevich.1911.TCL}
% uses it as the foundation for Microsoft's authorization logic DKAL.

% The theorem-proving problem has received some attention in the literature.
% There are at least seven other theorem provers for IPL and a published set of
% benchmark problems.  Thus it is a good medium for comparing the inverse method
% with existing techniques, none of which uses forward search.

% We assume the reader is familiar with the basic aspects of sequent calculus for
% classical and intuitionistic logic.  A thorough reference is~\cite[Chapter
% 3]{Troelstra.2000.ProofTheory}.  After defining the syntax and proof theory of
% IPL in Section~\ref{prop.sec.def}, we show how to use the inverse method to
% obtain a forward proof search procedure.  The process takes us through 3
% additional logical systems:

% \[
% \LPB \longleftrightarrow
% \LPF \longleftrightarrow
% \LPFx \longleftrightarrow
% \LPFs
% \]

% \ednote{Fill in forward refereneces to other sections.}

% \subsection{Motivation}

% Resolution, a forward search strategy, is empirically the most effective method
% of proof search in classical first-order logic. Resolution theorem provers
% dominate the annual CADE system competition
% (CASC)~\cite{Sutcliffe.2002.AI,Sutcliffe.2006.AI}.  In our view, forward search
% has at least two significant advantages.  First, each sequent derived by forward
% search is a (miniature) theorem, independent of the state of the current search
% procedure.  By contrast, sequents arising during backward search are always dependent
% on other parts of the proof (though this will not be apparent until we
% study first-order logic in Chapter~\ref{chapter.fol}).
% Second, the database itself is amenable to optimizations,
% particularly strategies for reduncancy elimination, which dramatically improve
% the performance of classical first-order theorem provers~\cite{Voronkov.2001.Indexing}.
% Resolution, though, depends on normal forms specific to classical
% logic.  In particular, for each formula $f$ in classical logic, there is a
% quantifier-free formula $f'$ in disjunctive normal form that is
% equi-satisfiable.  There are no similar normal forms for the non-classical logics
% we study.  The inverse method is attractive because it shares the primary
% advantages of resolution, but does not require normal forms\footnote{There
% is a close connection between the inverse method and
% hyperresolution~\cite[Section 5]{Voronkov.2001.Handbook}.}.

% \ednote{Show an example of this, possibly at the end of the chapter}.

\section{Backward Sequent Calculus}
\label{prop.sec.def}

\begin{quote}
  \textit{We define the backward calculus $\LPB$ for
    IPL, and remind the reader of some important properties.}
\end{quote}

\noindent
Let $\cP$ be an infinite set of symbols, called \emph{labels}.
Propositional formulas are built from the following grammar:
\index{$\cP$}

\[
\mbox{Formulas } A ::= p \Sep A \And A \Sep \Top \Sep A \Or A
  \Sep \Bot \Sep A \Imp A
\]

\noindent where $p\in\mathcal{P}$ and $\And$, $\Or$, $\Imp$, $\Top$, $\Bot$
signify conjunction, disjunction, implication, truth, and falsehood
respectively.  Subformula of the first kind are called \emph{atomic}.

\begin{definition}
  Derived connectives for negation and bi-implication are

  \[
  \begin{array}{rl}
    \Not A &:= A \Imp \Bot \\
    A \Iff B &:= (A \Imp B) \And (B \Imp A)
  \end{array}
  \]
\end{definition}

\begin{definition}[Sequents]
  \begin{itemize}
  \item[]
  \item If $A$ is a formula, a \emph{signed formula} is
    $\Left{A}$ or $\Right{A}$.
  \item A signed formula of the form $\Left{A}$ is called a \emph{left formula}.
  \item A signed formula of the form $\Right{A}$ is called a \emph{right formula}.
  \item A \emph{sequent} is a set of signed formulas
    with at most one right formula.
  \end{itemize}
\end{definition}

\begin{remark}
  Most presentations use multisets of antecedents rather than sets.  Using sets
  is not ideal in the meta-theory of the sequent calculus.  For example, there
  should be two different proofs of $a\Imp a\Imp a$.  Using sets, there is only
  one.  Since we are concerned only with the existence of proofs and not their
  enumeration, using sets is perfectly acceptable.
\end{remark}

\begin{definition}[Inference rules]
  \begin{itemize}
    \item[]
    \item An \emph{inference rule} is a pair $\ASet{\cC, \cH}$
      where $\cC$ is
      a sequent, and $\cH$ is an ordered sequence of sequents.
    \item $\cC$ is called the \emph{conclusion}.
    \item $\cH$ are called the \emph{premeses}.
    \item We write $\ASet{\cC, \ASet{\cH_1, \cH_2,\ldots,\cH_n}}$ as
      \[
      \Infer{\cC}{\cH_1 & \cH_2 & \ldots & \cH_n}
      \]
    \item When $\cH=\emptyset$ we call the rule
      an \emph{initial sequent}
    \end{itemize}
\end{definition}

\begin{definition}
  A \emph{system} is a set of inference rules.
\end{definition}

\begin{definition}
  The system $\LPB$ is the set of all instances of
  the rule schemas of Figure~\ref{prop.backward}.
\end{definition}

\begin{definition}[Backward rule matching]
  \label{prop.def.backward-match}
  \begin{itemize}
  \item[]
  \item The function \emph{$\match$} takes an inference rule
    \[
    r=\infer
    {\Gamma}
    {\Gamma_1 & \cdots & \Gamma_n}
    \]
    \noindent
    and a sequent $\Delta$.  $\match(r, \Delta)$ is defined
    if $\Gamma \subseteq \Delta$.
    In that case, let $\Delta'=\Delta\Diff\Gamma$.  Then

    \[
    \match(r, q) = \ASet{\Gamma_1\Union\Delta', \ldots, \Gamma_n\Union\Delta'}
    \]
    \noindent
    subject to the constraint that the resulting sequents
    have at most one right formula.  If some resulting sequent
    has multiple right formulas, the
    match is not defined.
  \end{itemize}
\end{definition}

\begin{definition}
  \begin{itemize}
  \item[]
  \item A $\LPB$ \emph{derivation} in a system $\cS$ is a tree of sequents where the children of
    any node $q$ are the result of $\match(r, q)$ for some rule $r\in\cS$.
  \item A \emph{proof} is a derivation where the leaves are initial sequents.
  \item We write derivations as usual.  We use a two ruled line for a multiple
    step deduction.
  \end{itemize}
\end{definition}


\newcommand{\PBInit}{\Infer[$Init$]{\PBSeq{p}{p}}{}}

\newcommand{\PBAndR}{
  \Infer[\And$-R$]
  {\PBSeq{\cdot}{A_1 \And A_2}}
  {\PBSeq{\cdot}{A_1} & \PBSeq{\cdot}{A_2}}
}

\newcommand{\PBAndL}{
  \Infer[\And$-L$]
  {\PBSeq{A_1 \And A_2}{\cdot}}
  {\PBSeq{A_1, A_2}{\cdot}}
}

\newcommand{\PBImpR}{
  \Infer[\Imp$-R$]
  {\PBSeq{\cdot}{A_1 \Imp A_2}}
  {\PBSeq{A_1}{A_2}}
}

\newcommand{\PBImpL}{
  \Infer[\Imp$-L$]
  {\PBSeq{A_1\Imp A_2}{\cdot}}
  {\PBSeq{A_1\Imp A_2}{A_1} & \PBSeq{A_2}{\cdot}}
}

\newcommand{\PBOrRL}{
  \Infer[\Or$-R$_1]
  {\PBSeq{\cdot}{A_1 \Or A_2}}
  {\PBSeq{\cdot}{A_1}}
}

\newcommand{\PBOrRR}{
  \infer[\Or$-R$_2]
  {\PBSeq{\cdot}{A_1 \Or A_2}}
  {\PBSeq{\cdot}{A_2}}
}

\newcommand{\PBOrL}{
  \infer[\Or$-L$]
  {\PBSeq{A_1 \Or A_2}{\cdot}}
  {\PBSeq{A_1}{\cdot} & \PBSeq{\Gamma, A_2}{\cdot}}
}

\newcommand{\PBTopR}{
  \infer[\Top$-R$]
  {\PBSeq{\cdot}{\Top}}
  {}
}

\newcommand{\PBTopL}{\mbox{No rule for $\Top$-L}}

\newcommand{\PBBotL}{
  \infer[\Bot$-L$]
  {\PBSeq{\Bot}{\cdot}}
  {}
}

\newcommand{\PBBotR}{\mbox{No rule for $\Bot$-R}}

\begin{figure}
  \begin{center}
    {\small
      \fbox{
        \begin{tabular}{c}
          \begin{tabular}{ccc}
            \PBInit
            &
            \hspace{1cm}
            \PBTopR
            &
            \hspace{1cm}
            \PBTopL
          \end{tabular}
          \\[2em]
          \begin{tabular}{cc}
            \PBAndR
            &
            \hspace{1cm}
            \PBAndL
          \end{tabular}
          \\[2em]
          \begin{tabular}{ccc}
            \PBOrRL
            &
            \hspace{10pt}
            \PBOrRR
            &
            \hspace{10pt}
            \PBOrL
          \end{tabular}
          \\[2em]
          \begin{tabular}{cc}
            \PBBotR
            &
            \hspace{20pt}
            \PBBotL
          \end{tabular}
          \\[2em]
          \begin{tabular}{cc}
            \PBImpR
            &
            \hspace{20pt}
            \PBImpL
          \end{tabular}
        \end{tabular}
      }
    }
  \end{center}
  \caption{The backward calculus $\PB$}
  \label{prop.backward}
\end{figure}

%%% Local Variables:
%%% mode: latex
%%% TeX-master: "../thesis"
%%% End:


\noindent The backward inference rule schemas are shown in
Figure~\ref{prop.rules}.  They
are equivalent to Gentzen's LJ~\cite{Gentzen.1934.MZ}.

\begin{remark}
There are a menagerie of sequent calculi equivlent to LJ but with slight
technical differences.  LJ itself is not optimal for use in proof search due to
its explicit structural rules.  See, for example, the many examples in
\cite[Chapter 3]{Troelstra.2000.ProofTheory}.
\end{remark}

\begin{remark}
  The equivalence of $\LPB$ with LJ is easy enough that we do not include it
  here.  The main difference is that we use sets for antecedents, while LJ uses
  ordered sequences with explicit structural rules.
\end{remark}


\begin{remark}
  We are being purposely pedantic in our distinction between inference rules and
  rule schemas becauese our presentation is somewhat unusual.  Normally rule schemas are
  presented with meta-variables standing for the non-prinicipal formulas.  For
  example, the rule $\And$-T is normally written
  \[
  \Infer[\And$-L$]
  {\NSeq{\Gamma, \Left{A_1 \And A_2}}}
  {\NSeq{\Gamma, \Left{A_1}, \Left{A_2}}}
  \]
  \noindent
  Then the following is considered to be an instance of this schema,
  where $c\And d$ is the principal formula.
  \[
  \Infer[\And$-L$]
  {\NSeq{\Left{a}, \Left{b}, \Left{c \And d}}}
  {\NSeq{\Left{a}, \Left{b}, \Left{c}, \Left{d}}}
  \]
  \noindent
  This is not the case in our presentation.  The instance
  of the $\And$-T schema for  $c\And d$ is
  \[
  \Infer[\And$-L$]
  {\NSeq{\Left{c \And d}}}
  {\NSeq{\Left{c}, \Left{d}}}
  \]
  \noindent
  Our explicit rule matching makes
  \[
  \Infer[\And$-L$]
  {\NSeq{\Left{a}, \Left{b}, \Left{c \And d}}}
  {\NSeq{\Left{a}, \Left{b}, \Left{c}, \Left{d}}}
  \]
  \noindent
  a valid \emph{derivation}, but not a rule.  This distinction
  is especially significant in the focused calculus
  (Section~\ref{prop.sec.focus}).
\end{remark}

\begin{remark}
  It is easy to see by examining the rules and matching procedure
  that every sequent in
  a derivation of $\Set{\Right{A}}$ has exactly one right
  formula.  Because of this, it is common in other presentations
  of this material to write sequents
  as $\Seq{\Gamma}{A}$ where $\Gamma$ are the
  left formulas and $A$ is the right formula.
\end{remark}

\noindent
$\LPB$ with the rule matching of Def.~\ref{prop.def.backward-match} is suitable
for backward search.  Begin with the goal formula and repeatedly match the
inference rules, generating new sequents in the derivation.  Backtracking is
necessary when there is a choice of which rule to apply.
The only troubling rule in the system is $\Imp$-L.  Each of the other rules propegate only
subformulas of the principal formula.  In contrast, $\Imp$-L propegates the principal formula
as well as its subformulas to one of the hypotheses.  Backward search must thus
avoid looping behavior where $\Imp$-L is repeatedly applied to the same formula.
Various ways of addressing this issue have been considered
(Section~\ref{prop.sec.related}).  One benefit of forward search is that this
looping problem disappears.  Before continuing, we collect some relevant
properties of $\LPB$ that we will use later.

\subsection{The Subformula Property}

\begin{definition}[Signed subformulas]
  \label{prop.def.subf}
  \index{subformula!sign of}
  \begin{itemize}
  \item[]
  \item The \emph{immediate signed subformula} relation is shown in
     Figure~\ref{prop.subformulas}.
  \item The \emph{signed subformula} relation $\le$ is the reflexive transitive
    closure of $<$.
  \item Define $\subfs(\gamma) = \Set{\gamma'\sst \gamma'\le\gamma}$
  \item The signed subformulas of a set of signed formulas are the union of the
    signed subformulas of the elements of the set.
  \end{itemize}
\end{definition}


\begin{figure}
  \begin{center}
    \[
    \boxed{
      \begin{array}{lcr}
        \begin{aligned}
          \Right{A_1} &< \Right{(A_1 \And A_2)} \\
          \Right{A_2} &< \Right{(A_1 \And A_2)} \\
          \Left{A_1} &< \Left{(A_1 \And A_2)} \\
          \Left{A_2} &< \Left{(A_1 \And A_2)} \\
        \end{aligned}
        &
        \begin{aligned}
          \Right{A_1} &< \Right{(A_1 \Or A_2)} \\
          \Right{A_2} &< \Right{(A_1 \Or A_2)} \\
          \Left{A_1} &< \Left{(A_1 \Or A_2)} \\
          \Left{A_2} &< \Left{(A_1 \Or A_2)} \\
        \end{aligned}
        &
        \begin{aligned}
          \Left{A_1} &< \Right{(A_1 \Imp A_2)} \\
          \Right{A_2} &< \Right{(A_1 \Imp A_2)} \\
          \Right{A_1} &< \Left{(A_1 \Imp A_2)} \\
          \Left{A_2} &< \Left{(A_1 \Imp A_2)} \\
        \end{aligned}
      \end{array}
    }
    \]
  \end{center}
  \caption{Immediate signed subformulas}
  \label{prop.subformulas}
\end{figure}

%%% Local Variables:
%%% mode: latex
%%% TeX-master: "../thesis"
%%% End:


\begin{theorem}
\label{prop.thm.subformula}
If $\Delta$ is a sequent, any formula occurring in a sequent in a
derivation of $\Delta$ is a signed subformula of $\Delta$.
\end{theorem}

\begin{proof} Routine inspection of the inference rules. \end{proof}

\subsection{Admissible rules}

\begin{lemma}[Weakening]
  \label{prop.thm.weaken}
  The rule \[\Infer[\mbox{Weaken}]{\NSeq{\Left{A}}}{\NSeq{\cdot}}\]
  is admissible in $\LPB$.
\end{lemma}
\begin{proof} Easy induction on the derivation. \end{proof}

\begin{theorem}[Cut]
  The rule \[\Infer[\mbox{Cut}]{\NSeq{\cdot}}{\NSeq{\Right{A}} & \NSeq{\Left{A}}}\]
  is admissible in $\LPB$.
\end{theorem}
\begin{proof} Gentzen~\cite{Gentzen.1934.MZ}. \end{proof}

\begin{theorem}[Identity]
  The rule \[\Infer[\mbox{Id}]{\NSeq{\Left{A},\Right{A}}}{}\]
  is admissible in $\LPB$.
\end{theorem}
\begin{proof} Easy induction on $A$. \end{proof}

\begin{theorem}[Consistency]
  \label{prop.thm.consistent}
  $\NSeq{\Right{\Bot}}$ is not derivable.
\end{theorem}
\begin{proof} No inference rules apply. \end{proof}

\subsection{Subsumption}

\begin{definition}[Subsumption]
  \index{subsumes}
  \index{subsumption}
  \begin{itemize}
  \item[]
  \item $\Delta$ \emph{subsumes} $\Delta'$
    if $\Delta\subseteq\Delta'$
  \item Write $\Delta\Subsumes\Delta'$
  \item $\Delta$ \emph{properly subsumes} $\Delta'$ if
    $\Delta\Subsumes \Delta'$
    and $\Delta\neq \Delta'$.  Write $\Delta\SubsumesProp \Delta'$
  \item If $\cD$ is a set of sequents $\cD\Subsumes \Delta$ if
    there exists $\Delta'\in\cD$ such that $\Delta'\Subsumes \Delta$.
  \end{itemize}
\end{definition}

\noindent
Subsumption is a measure of the ``strength'' of a sequent
in the following sense.

\begin{theorem}
  \label{prop.thm.subs}
  If $\Delta\Subsumes \Delta'$ and there exists a
  derivation of $\Delta$ then there exists a derivation of
  $\Delta'$.
\end{theorem}
\begin{proof} Simply weaken the derivation of $\Delta$. \end{proof}

\begin{definition}
  Note that if $\Delta\Subsumes \Delta'$ and $\Delta'\Subsumes \Delta$ then
  $\Delta = \Delta'$.  Thus we can remove the subsumed elements from a set of
  sequents in any order and the result is unique.
  The \emph{subsumption closure} of a set
  $\cD$ is
  \[
  \Closure{\cD} = \Set{x\in\cD \sst \not\exists y\in\cD.\ y\neq x\And y\Subsumes x}
  \]
\end{definition}

\subsection{Atoms}

Atomic formulas play an important role in inverse method search.

\begin{definition}
  \index{$\atoms(A)$}
  For a formula $A$, let
  $\atoms(A)=\Set{p\sst p\in\subfs(A), p \mbox{ is atomic}}$.
\end{definition}

\begin{definition}
  Given a formula $A$,
  An atomic formula $p$ is called \emph{two-sided}
%  if $\Set{\Left(p},\Right{p}} \subseteq \subfs(A)$.
\end{definition}

\noindent For example, in $p \Imp q \Imp p$, $p$ is two-sided but $q$ is not.

\begin{theorem}
  We can restrict instances of Init to two-sided atoms.
\end{theorem}
\begin{proof}
  In every derivation in $\LPB$, the principal formula $p$ of the Init rule is
two-sided.
\end{proof}


\section{Forward Sequent Calculus}

\begin{quote}
  \textit{
    We derive a forward calculus $\LPF$ from $\LPB$, prove its soundness
    and completeness, and define $\LPFx$.
  }
\end{quote}

\subsection{The Forward calculus}

We've described rule matching as a function from the conclusion
of a rule to its premises.  We can also define matching in the
opposite direction, as a function of the premises to the conclusion.

\begin{definition}[Forward rule matching]
  \label{prop.def.match1}

  Given a sequence of sequents
  \[
  \cQ = \ASet{\Delta_1, \ldots, \Delta_n}
  \]

  \noindent and a rule
  \[
  r = \Infer
  {\Gamma}
  {\Gamma_1 & \cdots & \Gamma_n}
  \]

  \noindent let
  \[
  \Delta = \Gamma\Union (\Delta_1\Diff\Gamma_1) \Union \dots\Union(\Delta_n\Diff\Gamma_n).
  \]

  \noindent
  If $\Delta$ is a sequent (i.e. there is at most one
  right signed formula) then we write
  $\match(r, \cQ) = \Delta$.  Otherwise $\match(r, \cQ)$ is undefined.
  The sequents $\Delta_i$ are called the \emph{matched sequents} and the
  rule $r$ is called the \emph{matched rule}.  We can also lift matching
  to sets.
  Let $\cR$ be a set of rules and let $\cD$ be a set of sequents.
  Define
  \[
  \match(\cR, \cD) = \Set{q\sst q=\match(r, q_1, \ldots, q_n), r\in\cR, q_i\in\cD}
  \]
\end{definition}

\noindent
The forward calculus is shown in Figure~\ref{prop.forward}.

\newcommand{\PFInitRule}{
  \Infer[$Init$]
  {\VSeq{\Left{p},\Right{p}}}
  {}
}

\newcommand{\PFAndRRule}{
  \Infer[\And$-F$]
  {\VSeq{\Right{A_1 \And A_2}}}
  {\VSeq{\Right{A_1}} & \VSeq{\Right{A_2}}}
}

\newcommand{\PFAndLRule}{
  \Infer[\And$-T$]
  {\VSeq{\Left{A_1 \And A_2}}}
  {\VSeq{\Left{A_1}, \Left{A_2}}}
}

\newcommand{\PFImpRRule}{
  \Infer[\Imp$-F$]
  {\VSeq{\Right{A_1 \Imp A_2}}}
  {\VSeq{\Left{A_1},\Right{A_2}}}
}

\newcommand{\PFImpLRule}{
  \Infer[\Imp$-T$]
  {\VSeq{\Left{A_1\Imp A_2}}}
  {\VSeq{\Left{A_1\Imp A_2},\Right{A_1}} & \VSeq{\Left{A_2}}}
}

\newcommand{\PFOrRLRule}{
  \Infer[\Or$-F$_1]
  {\VSeq{\Right{A_1 \Or A_2}}}
  {\VSeq{\Right{A_1}}}
}

\newcommand{\PFOrRRRule}{
  \Infer[\Or$-F$_2]
  {\VSeq{\Right{A_1 \Or A_2}}}
  {\VSeq{\Right{A_2}}}
}

\newcommand{\PFOrLRule}{
  \Infer[\Or$-T$]
  {\VSeq{\Left{A_1 \Or A_2}}}
  {\VSeq{\Left{A_1}} & \VSeq{\Left{A_2}}}
}

\newcommand{\PFTopRRule}{
  \Infer[\Top$-F$]
  {\VSeq{\Right{\Top}}}
  {}
}

\newcommand{\PFBotLRule}{
  \Infer[\Bot$-T$]
  {\VSeq{\Left{\Bot}}}
  {}
}

\renewcommand{\figline}{\\[2em]}

\begin{figure}
  \begin{center}
    \fbox{
      \begin{tabular}{c}
        \begin{tabular}{ccc}
          \PFInitRule
          &
          \hspace{1cm}
          \PFTopRRule
          &
          \hspace{1cm}
          \mbox{No rule for $\Top$-T}
        \end{tabular}
        \figline
        \begin{tabular}{cc}
          \PFAndRRule
          &
          \hspace{1cm}
          \PFAndLRule
        \end{tabular}
        \figline
        \begin{tabular}{ccc}
          \PFOrRLRule
          &
          \hspace{10pt}
          \PFOrRRRule
          &
          \hspace{10pt}
          \PFOrLRule
        \end{tabular}
        \figline
        \begin{tabular}{cc}
          \mbox{No rule for $\Bot$-F}
          &
          \hspace{20pt}
          \PFBotLRule
        \end{tabular}
        \figline
        \begin{tabular}{cc}
          \PFImpRRule
          &
          \hspace{20pt}
          \PFImpLRule
        \end{tabular}
      \end{tabular}
    }
  \end{center}
  \caption{Forward inference rules}
  \label{prop.forward}
\end{figure}

%%% Local Variables:
%%% mode: latex
%%% TeX-master: "../thesis"
%%% End:



\newcommand{\Bl}{\Left{B}}
\newcommand{\Br}{\Right{B}}
\newcommand{\Al}{\Left{A}}
\newcommand{\Cl}{\Left{C}}
\newcommand{\Cr}{\Right{C}}
\newcommand{\Ab}{A \Imp B}
\newcommand{\Abr}{\Right{\Ab}}
\newcommand{\Ac}{A \Or C}
\newcommand{\Acr}{\Right{\Ac}}
\newcommand{\Abc}{(A\Imp B) \Imp C}
\newcommand{\Abcl}{\Left{\Abc}}
\newcommand{\Bb}{\Bot \Or B}
\newcommand{\Bbl}{\Left{\Bb}}

\renewcommand{\qa}{\NSeq{\Al, \Bl, \Br}}
\renewcommand{\ra}{\Infer[$Init$]{\qa}{}}

\renewcommand{\qb}{\NSeq{\Bl,\Abr}}
\renewcommand{\rb}{\Infer[\Imp$-F$]{\qb}{\ra}}

\renewcommand{\qc}{\NSeq{\Bl, \Cl, \Cr}}
\renewcommand{\rc}{\Infer[$Init$]{\qc}{}}

\renewcommand{\qd}{\NSeq{\Bl, \Cl, \Acr}}
\renewcommand{\rd}{\Infer[\Or$-T$]{\qd}{\rc}}

\renewcommand{\qe}{\NSeq{\Bl, \Abcl, \Acr}}
\renewcommand{\re}{\Infer[\Imp$-T$]{\qe}{\rd & \rb}}

\renewcommand{\qf}{\NSeq{\Left{\Bot}, \Abcl, \Acr}}
\renewcommand{\rf}{\Infer[\Bot$-T$]{\qf}{}}

\renewcommand{\qg}{\NSeq{\Bbl, \Abcl, \Acr}}
\renewcommand{\rg}{\Infer[\Or$-T$]{\qg}{\rf & \re}}

\renewcommand{\qh}{\NSeq{\Bbl, \Right{(\Abc) \Imp \Ac}}}
\renewcommand{\rh}{\Infer[\Imp$-F$]{\qh}{\rg}}

\renewcommand{\qj}{\NSeq{\Right{\Bb \Imp (\Abc) \Imp \Ac}}}
\renewcommand{\rj}{\Infer[\Imp$-F$]{\qj}{\rh}}

\begin{figure}
  %\begin{sidewaysfigure}[angle=90]
    \begin{center}
      \[
      \boxed{\rj}
      \]
    \end{center}
    \caption{A derivation in $\LPB$}
    \label{prop.backward-der}
    %\end{sidewaysfigure}
\end{figure}


%%% Local Variables:
%%% mode: latex
%%% TeX-master: "../thesis"
%%% End:

%
\renewcommand{\ab}{a \Imp b}
\renewcommand{\ac}{a \Or c}
\renewcommand{\abc}{(a\Imp b) \Imp c}
\renewcommand{\bb}{\Bot \Or b}

\renewcommand{\Qa}{\FSeq{b}{b}}
\renewcommand{\Ra}{\Infer[$Init$]{\Qa}{}}

\renewcommand{\Qb}{\FSeq{b}{\ab}}
\renewcommand{\Rb}{\Infer[\Imp$-R$]{\Qb}{\Ra}}

\renewcommand{\Qc}{\FSeq{c}{c}}
\renewcommand{\Rc}{\Infer[$Init$]{\Qc}{}}

\renewcommand{\Qd}{\FSeq{c}{\ac}}
\renewcommand{\Rd}{\Infer[\Or$-R$_2]{\Qd}{\Rc}}

\renewcommand{\Qe}{\FSeq{b, \abc}{\ac}}
\renewcommand{\Re}{\Infer[\Imp$-L$]{\Qe}{\Rd & \Rb}}

\renewcommand{\Qf}{\FSeq{\Bot}{\cdot}}
\renewcommand{\Rf}{\Infer[\Bot$-L$]{\Qf}{}}

\renewcommand{\Qg}{\FSeq{\bb, \abc}{\ac}}
\renewcommand{\Rg}{\Infer[\Or$-L$]{\Qg}{\Rf & \Re}}

\renewcommand{\Qh}{\FSeq{\bb}{(\abc) \Imp \ac}}
\renewcommand{\Rh}{\Infer[\Imp$-R$]{\Qh}{\Rg}}

\renewcommand{\Qj}{\FSeq{\cdot}{\bb \Imp (\abc) \Imp \ac}}
\renewcommand{\Rj}{\Infer[\Imp$-R$]{\Qj}{\Rh}}

\begin{figure}
  \begin{center}
    \fbox{
      \Rj
    }
  \end{center}
  \caption{A derivation in $\PF$}
  \label{prop.forward-der}
\end{figure}


%%% Local Variables:
%%% mode: latex
%%% TeX-master: "../thesis"
%%% End:


\subsection{Soundness and Completeness}

The soundness and completeness theorems justify using forward sequents for proof
search.  The theorems are not totally straightforward, as they must account for
an empty consequent and missing parametric antecedents.  For example,
$\NSeq{\Left{p},\Left{q},\Right{p}}$ is derivable in $\LPB$ but
$\FSeq{p, q}{p}$ is not derivable in $\LPF$.
(Weakening is not an admissible rule.)  Instead we show we can always
prove a stronger sequent.

\begin{theorem}[Soundness]
  \label{prop.thm.forward-sound}
  If $\NSeq{\Delta}$ is derivable in $\LPF$ then
  $\VSeq{\Delta}$ is derivable in $\LPB$.
\end{theorem}

\begin{proof}
  By induction on the derivation of $\VSeq{\Delta}$.
  Let $\Delta = \match(r, \cQ)$ where $r$ is the last
  rule application.  We show some representative rules.
  \begin{description}
  \item[Case:]
    $r = \mbox{Init}$.
    Then $\Delta$ is an instance of $Init$
    \[ \Infer[Init]{\VSeq{\Left{p},\Right{p}}}{} \]
    and we have the identical derivation
    \[ \Infer[Init]{\NSeq{\Left{p},\Right{p}}}{} \]
  \item[Case:]
    \Infer[\Imp$-R$]
    {\VSeq{\Right{A_1 \Imp A_2}}}
    {\VSeq{\Left{A_1},\Right{A_2}}}
    By IH we have $\NSeq{\Delta}$.  There are a few cases, depending
    on whether $\Left{A_1}$
    If

  \item[Case:]
    \[ \PFImpLRule \]
    \begin{tabbing}
      $\FB{\FSeq{\Gamma_1, A_1\Imp A_2}{A_1}}$ \` IH \\
      $\Seq{\Gamma_1, A_1\Imp A_2}{A_1}$ \` Def. \\
      $\Seq{\Gamma_1, \Gamma_2, A_1\Imp A_2}{A_1}$ \` Weaken \\
      $\FB{\FSeq{\Gamma_2, A_2}{\gamma}}$ \` IH \\
      $\Seq{\Gamma_2, A_2}{\FB{\gamma}}$ \` Def. \\
      $\Seq{\Gamma_1, \Gamma_2, A_2}{\FB{\gamma}}$ \` Weaken \\
      $\Seq{\Gamma_1, \Gamma_2, A_1\Imp A_2}{\FB{\gamma}}$ \` $\Imp$-L\\
      $\FB{\FSeq{\Gamma_1, \Gamma_2, A_1\Imp A_2}{\gamma}}$
      \` Def., possibly with contractions (Lemma~\ref{prop.thm.contract})\\
    \end{tabbing}
  \end{description}
\end{proof}

\begin{theorem}[Completeness]
  \label{prop.thm.forward-complete}
  If $\Seq{\Gamma}{A}$ is derivable then there is a derivable
  sequent $\FSeq{\Gamma'}{\gamma'}$ where
  $\FB{\FSeq{\Gamma'}{\gamma'}}\Subsumes \Seq{\Gamma}{A}$.
\end{theorem}

\begin{proof}
  By induction on the derivation of $\Seq{\Gamma}{A}$.
  We show some representative rules.
  \begin{description}
  \item[Case:]
    \[ \PBInitRule \]
    \begin{tabbing}
      $\FSeq{p}{p}$ \` Init ($\gamma'=\Set{p}$, $\Gamma'=\Set{p}$) \\
    \end{tabbing}
  \item[Case:]
    \[ \PBImpRRule \]
    \begin{tabbing}
      $\FSeq{\Gamma_1}{\gamma}$ \` IH \\
      $\Gamma_1 \subseteq \Gamma, A_1$ \` IH \\
      $\gamma \subseteq \Set{A_2}$ \` IH \\
    \end{tabbing}
    \begin{description}
    \item[Case:]
      $A_1\in\Gamma_1$, $A_2\in\gamma$
      \begin{tabbing}
        $\Gamma_1=\Gamma',A_1$ \\
        $\gamma=\Set{A_2}$ \\
        $\Gamma' \subseteq \Gamma$ \\
        $\FSeq{\Gamma',A_1}{A_2}$ \` IH \\
        $\FSeq{\Gamma'}{A_1\Imp A_2}$ \` $\Imp$-R$_1$ \\
      \end{tabbing}
    \end{description}
    Other cases are similar, % with $\PFImpRb$ and $\PFImpRc$
  \item[Case:]
    \[ \PBImpLRule \]
    \begin{tabbing}
      $\FSeq{\Gamma_1}{\gamma_1}$ \` IH (1) \\
      $\Gamma_1\subseteq \Gamma, A_1\Imp A_2$ \` IH \\
      $\gamma_1\subseteq \Set{A_1}$ \` IH \\
      $\FSeq{\Gamma_2}{\gamma_2}$ \` IH (2) \\
      $\Gamma_2\subseteq \Gamma, A_2$ \` IH \\
      $\gamma_2\subseteq \Set{C}$ \` IH \\
    \end{tabbing}
    \begin{description}
    \item[Case:]
      $A_2\not\in\Gamma_2$
      Then setting $\Gamma'=\Gamma_2$ and $\gamma=\gamma_2$ will do.
    \item[Case:]
      $A_2\in\Gamma_2$ and $A_1\not\in\gamma_1$.
      Then setting $\Gamma'=\Gamma_1$ and $\gamma=\gamma_1$ will do.
    \item[Case:]
      $A_2\in\Gamma_2$ and $A_1\in\gamma_1$.
      Let $\Gamma_2=\Gamma_2',A_2$.  Then $\Gamma_2' \subseteq \Gamma$.
      Set $\gamma=\gamma_2$.
      Because $\Gamma_1 \subseteq \Gamma, A_1\Imp A_2$ and $\Gamma_2' \subseteq \Gamma$,
      some series of contractions
      of $\Gamma_1\Union\Gamma_2'$ will yield
      $\Gamma_3 \subseteq \Gamma, A_1\Imp A_2$.
      \[
      \Infere[$Contract$]
      { \FSeq{\Gamma_3,A_1\Imp A_2}{\gamma} }
      {
        \Infer[\Imp$-L$]
        {\FSeq{\Gamma_1\Union\Gamma_2',A_1\Imp A_2}{\gamma}}
        {\FSeq{\Gamma_1}{A_1} & \FSeq{\Gamma_2',A_2}{\gamma}}
      }
      \]
      If $A_1\Imp A_2\in\Gamma_3$, one more contraction will yield
      $\Gamma'$ such that $\Gamma' \subseteq \Gamma$.  Otherwise set $\Gamma'=\Gamma_3$.
    \end{description}
  \end{description}
\end{proof}

\begin{theorem}
  \label{prop.thm.subsume-goal}
  \begin{itemize}
  \item[]
  \item There is no derivation of $\FSeq{\cdot}{\cdot}$
  \item There is no derivable sequent that subsumes $\FSeq{\cdot}{\xi}$
  \end{itemize}
\end{theorem}

\begin{proof}
  \begin{itemize}
  \item[]
  \item By Theorem~\ref{prop.thm.forward-sound},
    a derivation of $\FSeq{\cdot}{\cdot}$ would yield a derivation of
    $\Seq{\cdot}{\Bot}$.  But this contradicts Theorem~\ref{prop.thm.consistent}.
  \item There is no sequent besides $\FSeq{\cdot}{\cdot}$ that subsumes
    $\FSeq{\cdot}{\xi}$.
  \end{itemize}
\end{proof}

\begin{remark}
Figure~\ref{prop.forward} shows a forward calculus $\LPF$ that is sound and
complete with respect to the backward rules.  There are some surprises.  Unlike
$\LPB$, contraction is not admissible, and is thus needed as an explicit rule.
The single rule $\Imp$-R becomes three rules %$\PFImpRa$, $\PFImpRb$, $\PFImpRc$, depending on whether the backward rule uses the antecedent
(consequent) or not.  (Consider, for example, trying to prove $\Bot\Imp A$.)
Figure~\ref{prop.backward-der} shows a derivation in $\LPB$ and
Figure~\ref{prop.forward-der} its corresponding derivation in $\LPF$.
\end{remark}

\subsection{Rule specialization}
\label{prop.sec.specialization}

$\LPF$ still does not satisfy the finite rule property.  There are still an
infinite number of instances of the \emph{Init} rule. The second part also fails.
For example, the subformula $A_2$
in \[\PBOrRLRule\] is unrestricted.  Here we are saved by the subformula
property. We can instantiate the rules with (the finite number of)
subformulas of the goal sequent.  Fix a formula $\xi$ to be checked for
provability for the remainder of this section.  Define $\LPFx$ to be $\LPF$
restricted to signed subformulas of $\xi$.  Then $\LPFx$ satisfies the finite
rule property.  For example, if $\xi=p_1\And p_2\Imp p_2\And p_1$
then

\[
\LPFx =
\begin{cases}
  \begin{array}{c@{\hspace{2em}}c@{\hspace{2em}}c}
    \Infer{\FSeq{p_1}{p_1}}{}
    &
    \Infer{\FSeq{p_2}{p_2}}{}
    &
    \Infer{\FSeq{\Gamma_1\Union\Gamma_2}{p_2\And p_1}}{\FSeq{\Gamma_1}{p_2} & \FSeq{\Gamma_2}{p_1}}
    \\[2em]
    \Infer{\FSeq{\Gamma,p_1\And p_2}{\gamma}}{\FSeq{\Gamma,p_1,p_2}{\gamma}}
    &
    \Infer{\FSeq{\Gamma}{\xi}}{\FSeq{\Gamma,p_1\And p_2}{p_2\And p_1}}
    &
    \Infer{\FSeq{\Gamma}{\xi}}{\FSeq{\Gamma}{p_2\And p_1}}
  \end{array}
\end{cases}
\]

\begin{theorem}[Soundness and completeness of $\LPFx$]
  {\ \\}
  \begin{quote}
    $\Seq{\cdot}{\xi}$ is derivable in $\LPB$
    iff $\FSeq{\cdot}{\xi}$ is derivable in $\LPFx$
  \end{quote}
\end{theorem}

\begin{proof}
  Theorems~\ref{prop.thm.forward-sound}, \ref{prop.thm.forward-complete},
  and \ref{prop.thm.subformula}.
\end{proof}

\section{The Inverse Method}

\begin{quote}
  \textit{
    We define the inverse method, a process for obtaining a forward
    proof search strategy from the forward calculus.
  }
\end{quote}

\subsection{Rule generation}

We now describe a systematic method for simplifying a set of
forward inference rules into

\begin{definition}

\end{definition}

\begin{definition}
  Let $A^s\in\subfs(\xi)$, $s\in\Set{+,-}$.  Define

  \[
  \rules(A^s) \EqDef
  \begin{cases}
    \Infer{\FSeq{p}{p}}{} &
    \mbox{$A=p$, $p$ is two-sided}
    \\[1em]
    \Infer
    {\FSeq{\cdot}{A_1\And A_2}}
    {\FSeq{\cdot}{A_1} & \FSeq{\cdot}{A_2}} &
    \mbox{$A=A_1\And A_2$, $s=+$}
    \\[1em]
    \Infer
    {\FSeq{A_1\And A_2}{\cdot}}
    {\FSeq{A_1, A_2}{\cdot}} &
    \mbox{$A=A_1\And A_2$, $s=-$}
    \\[1em]
    \Infer
    {\FSeq{\cdot}{\Top}}
    {} &
    \mbox{$A=\Top$, $s=+$}
    \\[1em]
    \Infer
    {\FSeq{\cdot}{A_1\Or A_2}}
    {\FSeq{\cdot}{A_1}}
    ,\hspace{1em}
    \Infer
    {\FSeq{\cdot}{A_1\Or A_2}}
    {\FSeq{\cdot}{A_2}} &
    \mbox{$A=A_1\Or A_2$, $s=+$}
    \\[1em]
    \Infer
    {\FSeq{A_1\Or A_2}{\cdot}}
    {\FSeq{A_1}{\cdot} & \FSeq{A_2}{\cdot}} &
    \mbox{$A=A_1\Or A_2$, $s=-$}
    \\[1em]
    \Infer
    {\FSeq{\Bot}{\cdot}}
    {} &
    \mbox{$A=\Bot$, $s=-$}
    \\[1em]
    \Infer
    {\FSeq{\cdot}{A_1\Imp A_2}}
    {\FSeq{A_1}{A_2}} &
    \mbox{$A=A_1\Imp A_2$, $s=+$}
    \\[1em]
    \Infer
    {\FSeq{A_1\Imp A_2}{\cdot}}
    {\FSeq{\cdot}{A_1} & \FSeq{A_2}{\cdot}} &
    \mbox{$A=A_1\Imp A_2$, $s=-$}
   \end{cases}
  \]
\end{definition}

\begin{definition}
  \[
  \LPFs = \bigcup_{A\in\subfs(\xi)} \rules(A)
  \]
\end{definition}

\noindent
Our running example becomes

\[
\LPFs =
\begin{cases}
  \begin{array}{ccc}
    \Infer{\FSeq{p_1}{p_1}}{}
    &
    \hspace{2em}
    \Infer{\FSeq{p_2}{p_2}}{}
    &
    \hspace{2em}
    \Infer{\FSeq{}{p_2\And p_1}}{\FSeq{\cdot}{p_2} & \FSeq{\cdot}{p_1}}
    \\[2em]
    \Infer{\FSeq{p_1\And p_2}{\cdot}}{\FSeq{p_1,p_2}{\cdot}}
    &
    \hspace{2em}
    \Infer{\FSeq{\cdot}{\xi}}{\FSeq{p_1\And p_2}{p_2\And p_1}}
  \end{array}
\end{cases}
\]

% \subsection{Rule matching}

% To simplify the calculus $\LPFx$, we can define rule matching
% in such a way that parametric antecedents are absent from the
% rules, and the three rules $\PFImpRa$, $\PFImpRb$, $\PFImpRc$
% can be handled by the single rule.

% \[
% \Infer[\Imp\mbox{-R}]
% {\FSeq{\cdot}{A_1 \Imp A_2}}
% {\FSeq{A_1}{A_2}}
% \]

% \noindent
% Because we will use the following definition in later
% sections when inference rules can have more than two hypotheses
% (cf. Section~\ref{prop.sec.derived}), we define matching in a
% more general way than is currently required.

% \begin{definition}[Rule Matching]
%   \label{prop.def.match1}

%   Sequents
%   \[
%   \begin{array}{c}
%     q_1 = \FSeq{\Delta_1}{\delta_1} \\
%     \vdots \\
%     q_n = \FSeq{\Delta_n}{\delta_n} \\
%   \end{array}
%   \]
%   \emph{match} rule

%   \[
%   R = \Infer
%   {\FSeq{\Gamma}{A}}
%   {\FSeq{\Gamma_1}{A_1} & \cdots & \FSeq{\Gamma_n}{A_n}}
%   \]

%   \noindent if the following condition holds for all $1\leq i\leq n$:
%   either $\delta_i = A_i$ or $\delta_i = \cdot$.  In that case, the resulting sequent is

%   \[
%   q=\FSeq{\Gamma\Union (\Delta_1\Diff\Gamma_1)\Union \ldots \Union (\Delta_n\Diff\Gamma_n)}{A}
%   \]

%   \noindent If there is a premise with an empty succedent in the rule, then the
%   conclusion also has an empty succedent.  In this case, we can rearrange the
%   premises so that the first $k$ premises have an empty antecedent.  Then we can
%   use the following definition of matching. Sequents
%   \[
%   \begin{array}{c}
%     q_1 = \FSeq{\Delta_1}{\delta_1} \\
%     \vdots \\
%     q_n = \FSeq{\Delta_n}{\delta_n} \\
%   \end{array}
%   \]
%   \emph{match} rule

%   \[
%   R = \Infer
%   {\FSeq{\Gamma}{\cdot}}
%   {\FSeq{\Gamma_1}{\cdot} & \cdots & \FSeq{\Gamma_k}{\cdot} & \FSeq{\Gamma_{k+1}}{\gamma_{k+1}} & \cdots & \FSeq{\Gamma_n}{\gamma_n}}
%   \]

%   \noindent if one of the following conditions holds:

%   \begin{enumerate}
%   \item
%     \begin{itemize}
%     \item For all $1\leq i\leq k$, $\delta_i = \cdot$.
%     \item For all $k+1\leq i\leq n$, $\delta_i = \cdot$ or $\delta_i = \gamma_i$.
%     \end{itemize}
%     In this case the resulting sequent is
%     \[
%     q = \FSeq{\Gamma\Union (\Delta_1\Diff\Gamma_1)\Union \ldots \Union (\Delta_n\Diff\Gamma_n)}{\cdot}
%     \]
%   \item
%     \begin{itemize}
%     \item There exists $1\leq i\leq k$, $\delta_i = A$.
%     \item For all $1\leq i\leq k$, $\delta_i = \cdot$ or $\delta_i = A$.
%     \item For all $k+1\leq i\leq n$, $\delta_i = \cdot$ or $\delta_i = A_i$.
%     \end{itemize}
%     In this case the resulting sequent is
%     \[
%     q = \FSeq{\Gamma\Union (\Delta_1\Diff\Gamma_1)\Union \ldots \Union (\Delta_n\Diff\Gamma_n)}{A}
%     \]
%   \end{enumerate}

%   In the second case, we say the empty consequent has been \emph{instantiated}.
%   In both cases, we write $\match(R, q_1,\ldots,q_n) = q$.
%   The sequents $q_i$ are called the \emph{matched sequents} and the
%   rule $R$ is called the \emph{matched rule}.  We can also lift matching to sets.
%   Let $\cR$ be a set of rules and let $\cD$ be a set of sequents.
%   Define
%   \[
%   \match(\cR, \cD) = \Set{q\sst q=\match(R, q_1, \ldots, q_n), R\in\cR, q_i\in\cD}
%   \]
% \end{definition}

% \noindent
% To see how this eliminates the distinction between the three implication
% right rules, consider matching the three sequents

% \[
% \begin{array}{ccc}
%   \FSeq{a, c}{b}
%   &
%   \hspace{2em}
%   \FSeq{a, c}{\cdot}
%   &
%   \hspace{2em}
%   \FSeq{c}{b}
% \end{array}
% \]

% \noindent to the rule

% \[
% \Infer
% {\FSeq{\cdot}{a\Imp b}}
% {\FSeq{a}{b}}
% \]
% \noindent
% All resulting sequents are $\FSeq{c}{a\Imp b}$ as expected.

% \begin{remark}
% The inference systems $\LPFx$ and $\LPFs$ are very similar. Sequents
% have identical structure, and the rules of $\LPFs$ are a subset of those of
% $\LPFx$ with parametric formulas removed.  We will write
% $\SSeq{\Gamma}{\gamma}$ for sequents of $\LPFs$ in the soundness
% and completeness proofs below to distinguish the two kinds of
% sequents.  After those proofs are complete we will use
% $\FSeq{\Gamma}{\gamma}$ since $\LPFx$ is of no further interest in
% the chapter.
% \end{remark}

\subsection{Derivations and proof search}
\label{prop.sec.search}

\begin{definition}
  \begin{itemize}
  \item[]
  \item The \emph{goal} is $\FSeq{\cdot}{\xi}$.
  \item A \emph{database} is a set of sequents.
  \end{itemize}
\end{definition}

\noindent
Proof search in $\LPFx$ proceeds by
maintaining a database of provable sequents.  The database initially contains
only the initial sequents.  At each step, we use the matching
algorithm defined above to match hypotheses of the rules to sequents
in the existing database, thereby producing new sequents.  In time, we either generate
the goal, or no new sequents can be generated.  In the later case, $\cD$ is said
to be \emph{saturated}, and $\xi$ is unprovable.

We formalize this method of proof search following Voronkov~\cite[Section
7.4]{Voronkov.2001.Handbook}.  There are two operations on databases, adding
sequents via rule matching, and removing sequents by subsumption.

\begin{definition}
  We define a relation $\cD\steps\cD'$ between databases with the following
  properties.
  \begin{itemize}
  \item If $R$ is an inference rule,
    and $q_1, \ldots, q_n\in\cD$, and $q = \match(R, q_1,\ldots, q_n)$, then
    $\cD\steps\cD\Union{q}$.  Such an inference is called an
    \emph{inference step}.

  \item If$\Set{q_1,q_2}\subseteq\cD$ and  $q_1\Subsumes q_2$
    then $\cD\steps\cD\Diff\Set{q_2}$.
    Such an inference is called a \emph{subsumption step}.

  \item A \emph{step} is an inference step or a subsumption step.

  \item A \emph{derivation} is a sequence $\cD_0\steps\cD_1\steps\cdots$
    of databases, where $\cD_0$
    are the initial sequents and for each $i$, $\cD_i\steps\cD_{i+1}$.

  \item A derivation \emph{succeeds} if some $\cD_i$ contains the goal.
  \item A derivation \emph{fails} if it doesn't succeed.
  \end{itemize}
\end{definition}

\noindent
A technical problem arises due to the possibility of a derivation where
a sequent is repeatedly added and deleted, e.g. by an inference step followed
by a subsumption step.

\[
\cD\Union\Set{q_1,q_2}\steps\cD\Union\Set{q_1}\steps\cD\Union\Set{q_1,q_2}\steps\cdots
\]

\noindent We wish to rule out these sorts of derivations, which motivates the
notion of \emph{fair} derivations.

\begin{definition}
  The \emph{limit} of a derivation $\cD$ is
  \[
  \Dlimit = \bigcup_i\bigcap_{j\geq i}\cD_j
  \]

  \noindent
  A derivation is called \emph{fair} if for every inference rule $R$
  if $q_1$, \ldots, $q_n$, belong to the limit of the derivation, and
  $q = \match(R, q_1, \ldots, q_n)$ then there is some $i$ such that
  the inference $\cD_i\steps\cD_{i+1}$ where $\cD_{i+1}=\cD\Union\Set{q}$
  is a step in the derivation.
\end{definition}

\subsection{Soundness and completeness}

\begin{lemma}
  \label{prop.thm.initial-same}
  \begin{itemize}
  \item[]
  \item $q$ is initial in $\LPFs$ iff $\FB{q}$ is initial in $\LPFx$.
  \item $q$ is initial in $\LPFx$ iff $\FB{q}$ is initial in $\LPF$.
  \end{itemize}
\end{lemma}

\begin{proof} Inspection of the rules.
  Each antecedent is either a singleton or empty, thus a set does not
  differ from a multiset.
\end{proof}

% \begin{theorem}
%   \label{prop.thm.irule-sound}
%   Let $\cD_0 \steps\cD_1\steps\cdots$ be a derivation.
%   If $\FSeq{\Gamma}{\gamma}\in\cD_i$ then there exists
%   $\FSeq{\Gamma'}{\gamma'}\Subsumes \FSeq{\Gamma}{\gamma}$
%   and a derivation of $\FSeq{\Gamma'}{\gamma'}$ in $\LPFx$.
% \end{theorem}

% \begin{proof}
%   Since $\FSeq{\Gamma}{\gamma}\in\cD_i$, there exists a least $n$ such that
%   $q\in\cD_n$.  We proceed by induction on $n$.
%   If $n=0$ then $q$ is an initial sequent and $\FB{q}$ is thus derivable
%   in $\LPFx$ by Lemma~\ref{prop.thm.initial-same}.
%   Otherwise, $\FSeq{\Gamma}{\gamma} = \match(R, q_1,\ldots,q_n)$ where $q_i\in\cD_{n-1}$.
%   We show some representative cases.
%   \begin{description}
%   \item[Case:]
%     \[
%     R=\Infer
%     {\FSeq{\cdot}{A_1\And A_2}}
%     {\FSeq{\cdot}{A_1} & \FSeq{\cdot}{A_2}}
%     \]
%     Let $\FSeq{\Gamma_1}{\gamma_1}, \FSeq{\Gamma_2}{\gamma_2}$ be the
%     matched sequents.  Then there are $\Gamma_1',\gamma_1',\Gamma_2',\gamma_2'$
%     where
%     \begin{tabbing}
%       $\gamma=A_1\And A_2$ \` Case. \\
%       $\Gamma=\Gamma_1\Union\Gamma_2$ \` Case. \\
%       $\FSeq{\Gamma_1'}{\gamma_1'}$ \` IH\\
%       $\Gamma_1' \subseteq \Gamma_1$ \` IH\\
%       $\gamma_1' \subseteq \gamma_1$ \` IH\\
%       $\FSeq{\Gamma_2'}{\gamma_2'}$ \` IH\\
%       $\Gamma_2' \subseteq \Gamma_2$ \` IH\\
%       $\gamma_2' \subseteq \gamma_2$ \` IH\\
%     \end{tabbing}
%     If $\gamma_1' =\cdot$, then set $\gamma'=\gamma_1'$, $\Gamma'=\Gamma_1'$.
%     If $\gamma_2' =\cdot$, then set $\gamma'=\gamma_2'$, $\Gamma'=\Gamma_2'$.
%     In either case we have $\FSeq{\Gamma'}{\gamma'}\Subsumes \FSeq{\Gamma}{\gamma}$
%     as required.    Otherwise set $\Gamma'=\Gamma_1'\Union\Gamma_2'$.
%     \begin{tabbing}
%       $\FSeq{\Gamma_1'}{A_1}$ \` $\gamma_1=\Set{A_1}$\\
%       $\FSeq{\Gamma_2'}{A_2}$ \` $\gamma_2=\Set{A_2}$\\
%       $\FSeq{\Gamma_1'\MultiUnion\Gamma_2'}{A_1\And A_2}$ \` $\PFAndR$ \\
%       $\FSeq{\Gamma'}{A_1\And A_2}$ \` Sequence of contractions \\
%     \end{tabbing}
%     and we have $\FSeq{\Gamma'}{q}\Subsumes \FSeq{\Gamma}{\gamma}$ as required.
%   \item[Case:]
%     \[
%     R=\Infer
%     {\FSeq{\cdot}{A_1\Imp A_2}}
%     {\FSeq{A_1}{A_2}}
%     \]
%     Let $\FSeq{\Gamma_1}{\gamma_1}$ be the matched sequent.
%     Then $\gamma=A_1\Imp A_2$, $\Gamma=\Gamma_1\Diff\Set{A}$, and
%     $\gamma_1 \subseteq \Set{A_2}$.
%     By the induction hypothesis we have $\FSeq{\Gamma_1'}{\gamma_1'}$ with
%     $\Gamma_1' \subseteq \Gamma_1$ and $\gamma_1' \subseteq \gamma$.
%     \begin{description}
%     \item[Case:]
%       $\gamma'=\cdot$ and $A_1\not\in \Gamma'$.
%       Then $\FSeq{\Gamma'}{\gamma'}\Subsumes \FSeq{\Gamma}{\gamma} $.
%     \item[Case:]
%       $\gamma'= \Set{A_2}$ and $A_1\not\in\Gamma_1'$.
%       Apply $\PFImpRb$ to get
%       $\FSeq{\Gamma_1'}{A_1\Imp A_2}\Subsumes\FSeq{\Gamma}{\gamma}$.
%     \item[Case:]
%       $\gamma'=\cdot$ and $\Gamma_1'=\Gamma_1'', A_1$.
%       Apply $\PFImpRc$ to get
%       Then $\FSeq{\Gamma_1''}{\gamma'}\Subsumes\FSeq{\Gamma}{\gamma}$.
%     \item[Case:]
%       $\gamma'= \Set{A_2}$ and $A_1\in\Gamma_1'$.
%       We can use $\PFImpRa$ to get a derivation of
%       $\FSeq{\Gamma_1'\Diff\Set{A_1}}{A_1\Imp A_2}$
%       which has the required properties.
%     \end{description}
%   \end{description}
% \end{proof}

\begin{corollary}[Soundness]
  \label{prop.thm.lpfs-sound}
  If a derivation is successful, then $\xi$ is provable in $\LPFx$.
\end{corollary}

\begin{proof}
  In a successful derivation, $\FSeq{\cdot}{\xi}\in\cD_i$ for some $i$.
  So there is some derivable sequent $q$ such that
  $q\Subsumes\FSeq{\cdot}{\xi}$.
  By Theorem~\ref{prop.thm.subsume-goal}
  $q = \FSeq{\cdot}{\xi}$.
\end{proof}

\begin{lemma}
  \label{prop.thm.limit1}
  In a derivation, for any $i$, if $q\in\cD_i$ then there is a sequent
  $q'\in\Dlimit$ such that $q'\Subsumes q$.
\end{lemma}
\begin{proof}
  If $q\not\in\Dlimit$ then it was removed by a subsumption step in
  step $j>i$.  Thus there is a sequent $q'\in\cD_{j-1}$ such that $q'\Subsumes q$.
  Since there is no infinite sequence $q_0\Subsumed q_1\Subsumed \ldots$ there
  is some $q_i\Subsumes q\in\Dlimit$
\end{proof}

\begin{lemma}
  \label{prop.thm.limit}
  $\Set{q_1,\ldots,q_n}\subseteq\Dlimit$, $R\in\LPFs$, and
  $\match(R, q_1,\ldots, q_n)=q$, then there is a $q'\in\Dlimit$
  such that $q'\Subsumes q$.
\end{lemma}

\begin{proof}
  Lemma~\ref{prop.thm.limit1} and the definition of limit.
\end{proof}

\begin{theorem}[Completeness]
  \label{prop.thm.irule-complete}
  If $q$ is derivable in $\LPFx$, then
  the limit of every fair derivation includes a sequent $q'$
  such that $q'\Subsumes \FB{q}$.
\end{theorem}

\begin{proof}
  Induction on the derivation of $q$.  Initial rules are elements of $\cR_0$.
  If $q$ is initial then $\FB{q}\in\cD_0$ by Lemma~\ref{prop.thm.initial-same}.
  Otherwise $q$ results from a rule application.  We show some representative cases.
  \begin{description}
  \item[Case:]
    \[
    \Infer[\PFAndR]
    {\FSeq{\Gamma_1\MultiUnion\Gamma_2}{A_1 \And A_2}}
    {\FSeq{\Gamma_1}{A_1} & \FSeq{\Gamma_2}{A_2}}
    \]
    So $q=\FSeq{\Gamma_1\MultiUnion\Gamma_2}{A_1\And A_2}$.
    By IH, there are sequents
    $q_1=\FSeq{\Gamma_1'}{\gamma_1'}$ and $q_2=\FSeq{\Gamma_2'}{\gamma_2'}$ such
    \begin{itemize}
    \item $q_1,q_2\in \Dlimit$
    \item $q_1\Subsumes \FB{\FSeq{\Gamma_1}{A_1}}$
    \item $q_2\Subsumes \FB{\FSeq{\Gamma_2}{A_2}}$
    \end{itemize}

    Therefore

    \begin{itemize}
    \item $\Gamma_1' \subseteq \dedup(\Gamma_1)$
    \item $\Gamma_2' \subseteq \dedup(\Gamma_2)$
    \end{itemize}

    Since $(A_1\And A_2)^r$ is a subformula of $\xi$, the rule
    \[
    \begin{array}{ccc}
      R= &
      \Infer
      {\FSeq{\cdot}{A_1\And A_2}}
      {\FSeq{\cdot}{A_1} & \FSeq{\cdot}{A_2}}
      & \in \LPFs
    \end{array}
    \]
    Let $q_0=\match(R, q_1, q_2)=\FSeq{\Gamma_1'\Union\Gamma_2'}{A_1\And A_2}$.
    Clearly $q_0\Subsumes\FB{q}$.
    By Lemma~\ref{prop.thm.limit},
    there is a sequent $q'\in\Dlimit$ such that
    $q'\Subsumes q_0\Subsumes \FB{q}$ as required.
  \item[Case:]
    \[
    \Infer[\PFImpRa]
    {\FSeq{\Gamma}{A_1 \Imp A_2}}
    {\FSeq{\Gamma, A_1}{A_2}}
    \]
    By IH, there is a sequent
    $q_0=\FSeq{\Gamma_0}{\gamma_0}$ such that
    \begin{itemize}
    \item $q_0\in \Dlimit$
    \item $q_0\Subsumes \FB{\FSeq{\Gamma, A_1}{A_2}}$
    \end{itemize}
    Therefore $\Gamma_0 \subseteq \dedup(\Gamma, A_1)$.
    Since $(A_1\Imp A_2)^r$ is a subformula of $\xi$, the rule

    \[
    \begin{array}{ccc}
      R= &
      \Infer
      {\FSeq{\cdot}{A_1\Imp A_2}}
      {\FSeq{A_1}{A_2}}
      & \in \LPFs
    \end{array}
    \]

    Let $q_1=\match(R, q_0)=\FSeq{\Gamma_0\Diff\Set{A_1}}{A_1\Imp A_2}$.
    Clearly $q_1\Subsumes\FB{q}$.
    By Lemma~\ref{prop.thm.limit},
    there is a sequent $q'\in\Dlimit$ such that
    $q'\Subsumes q_1\Subsumes \FB{q}$ as required.

  \item[Case:] The cases for $\PFImpRb$ and $\PFImpRc$ are similar.
  \item[Case:]
    \[
    \Infer[\PFOrL]
    {\FSeq{\Gamma_1\MultiUnion\Gamma_2, A_1 \Or A_2}{\gamma_1 \Union \gamma_2}}
    {\FSeq{\Gamma_1, A_1}{\gamma_1} & \FSeq{\Gamma_2, A_2}{\gamma_2}}
    \]
    where $\gamma_1\Union\gamma_2$ has cardinality at most 1.

    By IH, there are sequents
    $q_1=\FSeq{\Gamma_1'}{\gamma_1'}$ and $q_2=\FSeq{\Gamma_2'}{\gamma_2'}$ such
    \begin{itemize}
    \item $q_1,q_2\in \Dlimit$
    \item $q_1\Subsumes \FB{\FSeq{\Gamma_1, A_1}{\gamma_1}}$
    \item $q_2\Subsumes \FB{\FSeq{\Gamma_2, A_2}{\gamma_2}}$
    \end{itemize}

    Therefore

    \begin{itemize}
    \item $\Gamma_1' \subseteq \dedup(\Gamma_1,A_1)$
    \item $\gamma_1' \subseteq \gamma_1$
    \item $\Gamma_2' \subseteq \dedup(\Gamma_2,A_2)$
    \item $\gamma_2' \subseteq \gamma_2$
    \end{itemize}

    Since $(A_1\Or A_2)^l$ is a subformula of $\xi$, the rule
    \[
    \begin{array}{ccc}
      R= &
      \Infer
      {\FSeq{\cdot}{A_1\And A_2}}
      {\FSeq{\cdot}{A_1} & \FSeq{\cdot}{A_2}}
      & \in \LPFs
    \end{array}
    \]
    Let $q_0=\match(R, q_1, q_2)=\FSeq{\Gamma_1'\Diff\Set{A_1}\Union\Gamma_2'\Diff\Set{A_2}}{\gamma_1'\Union\gamma_2'}$.
    Clearly $q_0\Subsumes\FB{q}$.
    By Lemma~\ref{prop.thm.limit},
    there is a sequent $q'\in\Dlimit$ such that
    $q'\Subsumes q_0\Subsumes \FB{q}$ as required.
  \end{description}
\end{proof}

\begin{corollary}
  \label{prop.thm.inverse-complete}
  If any fair derivation fails, $\xi$ is not provable in $\LPFx$.
\end{corollary}

\begin{proof}
  If $\xi$ is provable, then there is some sequent
  $q\Subsumes\FB{\FSeq{\cdot}{\xi}}=\FSeq{\cdot}{\xi}$
  in the limit of every fair derivation.
  By Theorem~\ref{prop.thm.subsume-goal}, $q=\FSeq{\cdot}{\xi}$.
  So the derivation is successful.
\end{proof}

\subsection{Paths}

Using $\LPFs$ to do forward search is possible.  As a practical matter though,
working directly with subformulas is clumsy.  In order to match an inference
rule to a sequent we need to compare formulas for equality.  This is
inefficient.  Instead, we invent a short alias, called a label, for each
subformula and work with labels rather than subformulas.  Before we define
labels, we use \emph{paths} to uniquely specify subformula occurrances.

\begin{definition}[Paths]
  \begin{itemize}
  \item[]
  \item A \emph{path component} is one of the symbols $L$ or $R$.
  \item A \emph{path} is a finite ordered sequence of path components.
  \item The \emph{subformula of $A$ at path $\pi$},
    written $\subf(A, \pi)$ is defined as follows.
    \[
    \subf(A, \pi) \EqDef
    \begin{cases}
      A & \pi = []\\
      \subf(A_1, \pi') & \pi = L : \pi', A\in\Set{A_1\And A_2, A_2\Or A_2, A_1\Imp A_2}\\
      \subf(A_2, \pi') & \pi = R : \pi', A\in\Set{A_1\And A_2, A_2\Or A_2, A_1\Imp A_2}\\
      \mbox{undefined} & \emph{otherwise} \\
    \end{cases}
    \]
  \item For a fixed formula $\xi$ and set of paths $\Pi$,
    $\subf(\xi, \Pi) \EqDef \Set{A\sst\subf(\xi, \pi) = A, \pi\in\Pi}$
  \end{itemize}
\end{definition}

\noindent
For example, the paths and corresponding subformulas of $p \Imp q\Imp p$ are

\[
\begin{array}{c|c}
  \mbox{Path} & \mbox{Subformula} \\
  \hline \\
  \left[\right]  & p\Imp q\Imp p \\
  \left[L\right] & p \\
  \left[R\right] & q\Imp p \\
  \left[L,R\right] & q \\
  \left[R,R\right] & p \\
\end{array}
\]

\noindent
Note that $p$ has two distinct corresponding paths.  We thus need to
distinguish between a subformula and a subformula occurrance.

\begin{definition}
  \begin{itemize}
  \item[]
  \item A \emph{subformula occurrance} in $A$ is a path $\pi$ such that $\subf(A, \pi)$ is
    defined.
  \item The \emph{sign} of a subformula occurrance is the sign of the corresponding
    subformula.
  \item A subformula occurrance $\pi$ is \emph{atomic} if $\subf(A, \pi)$ is
    atomic.
  \item Let $\occs(\xi)$ be the set of subformula occurrances in $\xi$.
  \end{itemize}
\end{definition}

\subsection{Labels}

Now we are in a position to define mappings between subformula occurrances
and short labels.

\begin{definition}[Labeling]
  A function $\lab : \occs(\xi)\to\cP$ is called a \emph{labeling function} if
  it has the following properties
  \begin{itemize}
  \item $\lab(\pi) = \subf(\xi, \pi) = p$ if $\pi$ is atomic.
  \item For $\pi,\pi'$, if $\subf(\xi, \pi) = \subf(\xi, \pi') = \Top$
    then $\lab(\pi) = \lab(\pi')$.
  \item For $\pi,\pi'$, if $\subf(\xi, \pi) = \subf(\xi, \pi') = \Bot$
    then $\lab(\pi) = \lab(\pi')$.
  \item for any non-atomic subformula occurrance $\pi$, where
    $\subf(\xi, \pi)\not\in \Set{\Top,\Bot}$ and any subformula occurrance $\pi'$,
    $\lab(\pi) \neq \lab(\pi')$.
  \end{itemize}
  Since $\cP$ is infinite, labeling functions exist for any fixed $\xi$.  Fix
  a labeling function.
  \begin{itemize}
  \item A label $p$ is called \emph{atomic} if $p\in\atoms(\xi)$.
  \item $\lab(A)$ is called the \emph{label} of $A$.
  \item In a slight abuse of notation, we write $\lab(\Top)$ and $\lab(\Bot)$,
    since each subformula occurrance of $\Top$ and $\Bot$ gets the same label.
  \item For a set of formulas $X$, $\labs(X) = \Set{l\sst \lab(A) = l, A\in X}$
  \item $\cL = \labs(\subfs(\xi))$.
  \end{itemize}
\end{definition}

\noindent
Note that $\lab$ is not injective, as it may map different occurrances of
atomic formulas, $\Top$ and $\Bot$ to the same label.  Thus there is
no inverse function to $\lab$.  However, restricted to labels of non-atomic
subformulas not equal to $\Top$ or $\Bot$, the function is injective, so
has an inverse.  Call this inverse $\Path : \cL\to\occs(\xi)$.
Now we can define a function from labels to subformulas:

\begin{definition}[Unlabeling]
  \begin{itemize}
  \item[]
  \item Define
    \[
    \unlab(p) \EqDef
    \begin{cases}
      p & p \mbox{ is atomic } \\
      \Top & p = \lab(\Top) \\
      \Bot & p = \lab(\Bot) \\
      \subf(\xi, \Path(p)) & \mbox{otherwise}
    \end{cases}
    \]
  \item For a set of labels $X$, $\unlab(X) \EqDef \Set{A\sst \unlab(p) = A, p\in X}$
  \end{itemize}
\end{definition}

\noindent
Now we can revisit the previous example using labels.
Again, let $\xi=A_1\And A_2\Imp A_2\And A_1$, and define the following
labeling function:

\[
\begin{array}{c|l|c}
  \mbox{label} & \mbox{path} & \mbox{subformula} \\
  \hline
  l_1 & [] & p_1\And p_2\Imp p_2\And p_1\\
  l_2 & [L] & p_1\And p_2\\
  p_1 & [L,L] & p_1\\
  p_2 & [R,L] & p_2\\
  l_3 & [R] & p_2\And p_1\\
  p_2 & [L,R] & p_2\\
  p_1 & [R,R] & p_1\\
\end{array}
\]

\noindent
The inference rules become\footnote{Because the correspondence between
subformulas and labels is so simple, we do not invent a new name for the labeled
calculus.}

\[
\LPFs =
\begin{cases}
  \begin{array}{c@{\hspace{2em}}c@{\hspace{2em}}c}
    \Infer{\FSeq{p_1}{p_1}}{}
    &
    \Infer{\FSeq{p_2}{p_2}}{}
    &
    \Infer[r_1]{\FSeq{\cdot}{l_3}}{\FSeq{\cdot}{p_2} & \FSeq{\cdot}{p_1}}
    \\[2em]
    \Infer[r_2]{\FSeq{l_2}{\gamma}}{\FSeq{p_1,p_2}{\gamma}}
    &
    \Infer[r_3]{\FSeq{\cdot}{l_1}}{\FSeq{l_2}{l_3}}
  \end{array}
\end{cases}
\]

\noindent
leading to an easy derivation:

\[
\begin{array}{lr@{\hspace{2em}}l}
  1. & \FSeq{p_1}{p_1}& \mbox{Init} \\
  2. & \FSeq{p_2}{p_2}& \mbox{Init}\\
  3. & \FSeq{p_1,p_2}{l_3}& \mbox{$r_1$ with 1,2}\\
  4. & \FSeq{l_2}{l_3}&\mbox{$r_2$ with 3}\\
  5. & \FSeq{}{l_1}&\mbox{$r_3$ with 4}\\
\end{array}
\]

\paragraph{A note on subformulas vs. subformula occurrances.}

Suppose $\xi$ has a subformula of the form $A\And A$, where $A$ is large.
The paths corresponding to the subformulas of the two copies of $A$
differ.  We
made a choice to generate rules for both occurrances.  It is easy to imagine
generating rules for a single occurrance and then modifying the other rules
accordingly, yielding many fewer inference rules during proof search.  It seems
clear that on, say, the formula $A\Imp A\And A$, a set of rules so generated
would be more efficient than the ones we generate using the path/label method;
with our choice we need to prove $A$ twice with totally different sets of
inference rules.  The main justification for our choice is that merging the
inference rules for identical subformulas would prevent a number of the
optimizations we discuss in Section~\ref{prop.sec.optimize}.  While it is an
empirical question we have not tested, we hypothesize that the performance
improvements due to the optimizations generally outweigh redundant
work due to the loss of potential sharing in the inference rules.

\subsection{The Inverse Method}
\label{prop.sec.inverse}

\begin{definition} A set of sequents $X$ is called \emph{minimal} if there do
not exist $q,q'\in X$ such that $q\Subsumes q'$.  It is clear any finite set $X$
can be made minimal by removing subsumed sequents. Fix some procedure for doing
so, called $\clear(X)$.
\end{definition}

With the rules and axioms of $\LPFs$ in hand, we can develop a simple proof
search algorithm as follows.  Let $\cD_0$ be the axioms of $\LPFs$, and let
$\cR$ be the rules.  Let $\cD_{i+1}=\clear(\cD_i\Union\match(\cR, \cD_i))$.
This corresponds to a derivation.  The move from $\cD_i$ to $\cD_{i+1}$ is
equivalent to some sequence of inference steps followed by some sequence of
subsumption steps.  The derivation is obviously fair.  Since there are only a
finite number of search sequents consisting of subformulas of $\xi$, and the
databases don't contain duplicates, this process terminates.  This procedure is
therefore decision procedure for IPL.

We have described a process called the inverse method whereby
a sequent calculus for backward search is transformed into a
calculus for forward search.  The main steps are

\begin{enumerate}
\item Find a set of forward rules that are sound and complete with
  respect to the backward rules.
\item Specialize the rules to subformulas of a goal formula.
\item Label subformulas so matching is efficient.
\item\label{prop.enum.rules}
  Use the rules to create a derivation for $\xi$.  $\xi$ is
  provable iff the sequent occurs in some database in the derivation.
\end{enumerate}

\subsection{Example}

We give an example of an inverse method proof of

\[
\xi = (\Bot \Or b) \Imp ((a \Imp b) \Imp c) \Imp (a \Or c)
\]

\paragraph{Labeling.}

First we label the signed subformulas of $\xi$.

\[
\begin{array}{c@{\hspace{2em}}c@{\hspace{2em}}c}
p_1^r = a^r \Or c^r & p_2^r = a^l \Imp b^r & p_3^l = p_2^r \Imp c^l \\
p_4^r = p_3^l \Imp p_1^r & p_5^l = \Bot^l \Or b^l & p_6^r = p_5^l \Imp p_4^r
\end{array}
\]

\paragraph{Rule Specialization.}

Now we specialize the rules for the labels $p_1, \ldots, p_6$.

\[
\begin{array}{c@{\hspace{2em}}c@{\hspace{2em}}c@{\hspace{2em}}c}
\Infer[r_1]{\FSeq{\cdot}{p_1}}{\FSeq{\cdot}{a}} &
\Infer[r_2]{\FSeq{\cdot}{p_1}}{\FSeq{\cdot}{c}} &
\Infer[r_3]{\FSeq{\cdot}{p_2}}{\FSeq{a}{b}} &
\Infer[r_4]{\FSeq{p_3}{\cdot}}{\FSeq{\cdot}{p_2} & \FSeq{c}{\cdot}}
\\\\
\multicolumn{4}{c}{
  \begin{array}{c@{\hspace{2em}}c@{\hspace{2em}}c}
    \Infer[r_5]{\FSeq{\cdot}{p_4}}{\FSeq{p_3}{p_1}} &
    \Infer[r_6]{\FSeq{p_5}{\cdot}}{\FSeq{\Bot}{\cdot} & \FSeq{b}{\cdot}} &
    \Infer[r_7]{\FSeq{\cdot}{p_6}}{\FSeq{p_5}{p_4}}
  \end{array}
}
\end{array}
\]

\noindent
The initial axioms are

\[
\begin{array}{c@{\hspace{2em}}c@{\hspace{2em}}c@{\hspace{2em}}c}
\FSeq{a}{a} & \FSeq{b}{b} & \FSeq{c}{c} & \FSeq{\Bot}{\cdot}
\end{array}
\]


\newcommand{\FSubsumed}{\sqsubset}
\newcommand{\BSubsumed}{\sqsupset}
\newcommand{\MarkSub}[2]{\color{gray}#1 \` \color{gray} #2}

\begin{figure}
  \begin{center}
    \fbox{
      \begin{tabular}{c@{\hspace{2cm}}c}
        \begin{minipage}[b]{0.4\linewidth}
          \begin{tabbing}
            \textbf{Goal} \\
            \\
            \FSeq{\cdot}{p_6} \\
            \\
            \textbf{Stage 0} \\
            \\
            $q_0$. \FSeq{a}{a} \\
            $q_1$. \FSeq{b}{b} \\
            $q_2$. \FSeq{c}{c} \\
            $q_3$. \FSeq{\Bot}{\cdot} \\
            \\
            \textbf{Stage 1} \\
            \\
            $q_4$. \FSeq{a}{p_1} \` ($r_1$, $q_0$) \\
            $q_5$. \FSeq{b}{p_2} \` ($r_3$, $q_1$) \\
            $q_6$. \FSeq{c}{p_1} \` ($r_2$, $q_2$) \\
            $q_7$. \FSeq{p_5}{b} \` ($r_6$, $q_3, q_1$)) \\
            \\
            \textbf{Stage 2} \\
            \\
            $q_8$. \FSeq{p_5}{p_2} \` ($r_3$, $q_7$) \\
            $q_9$. \FSeq{b, p_3}{c} \` ($r_4$, $q_5, q_2$) \\
            $q_{10}$. \FSeq{b, p_3}{p_1} \` ($r_4$, $q_5, q_6$) \\
            $q_{11}$. \FSeq{a}{p_4} \` ($r_5$, $q_4$) \\
            $q_{12}$. \FSeq{c}{p_4} \` ($r_5$, $q_6$) \\
            \MarkSub{\FSeq{p_5}{p_2}}{$(r_6, q_3, q_5) \BSubsumed q_8$} \\
          \end{tabbing}
        \end{minipage}
        &
        \begin{minipage}[b]{0.4\linewidth}
          \begin{tabbing}
            \textbf{Stage 3} \\
            \\
            \MarkSub{\FSeq{b, p_3}{p_1}}{$(r_2, q_9) \BSubsumed q_{10}$} \\
            $q_{13}$. \FSeq{p_5, p_3}{p_4} \` ($r_4$, $q_8, q_{12}$) \\
            $q_{14}$. \FSeq{p_5, p_3}{p_1} \` ($r_4$, $q_8, q_{6}$) \\
            $q_{15}$. \FSeq{p_5, p_3}{c} \` ($r_4$, $q_8, q_{2}$) \\
            \MarkSub{$q_{16}$.{\FSeq{b, p_3}{p_4}}}{($r_4$, $q_5, q_{12}) \FSubsumed q_{17}$} \\
            $q_{17}$. \FSeq{b}{p_4} \` ($r_5$, $q_{14}$) \\
            \MarkSub{{\FSeq{p_5, p_3}{c}}}{$r_6$, $q_3, q_{9}. \FSubsumed q_{15}$} \\
            \MarkSub{{\FSeq{p_5, p_3}{p_1}}}{$r_6$, $q_3, q_{10}. \FSubsumed q_{14}$} \\
            $q_{18}$. \FSeq{a}{p_6} \` ($r_7$, $q_{11}$) \\
            $q_{19}$. \FSeq{c}{p_6} \` ($r_7$, $q_{12}$) \\
            \\
            \textbf{Stage 4} \\
            \\
            \MarkSub{\FSeq{p_5, p_3}{p_1}}{$(r_2, q_{15}) \BSubsumed q_{14}$} \\
            \MarkSub{$q_{20}$. \FSeq{p_5, p_3}{p_6}}{$(r_4, q_8, q_{19}) \FSubsumed q_{22}$} \\
            $q_{21}$. \FSeq{p_5}{p_4} \` ($r_4$, $q_{14}$) \\
            \MarkSub{\FSeq{p_5}{p_4}}{$(r_6, q_{17}) \BSubsumed q_{21}$} \\
            $q_{22}$. \FSeq{p_3}{p_6} \` ($r_7$, $q_{13}$)) \\
            $q_{23}$. \FSeq{b}{p_6} \` ($r_7$, $q_{17}$)) \\
            \\
            \textbf{Stage 5} \\
            \\
            $q_{24}$. \FSeq{p_5}{p_6} \` ($r_6$, $q_{23}$)) \\
            $q_{25}$. \FSeq{\cdot}{p_6} \` ($r_7$, $q_{21}$)) \\
          \end{tabbing}
        \end{minipage}
      \end{tabular}
    }
  \end{center}
  \caption{Example}
  \label{prop.trace}
\end{figure}

%%% Local Variables:
%%% mode: latex
%%% TeX-master: "../thesis"
%%% End:


We show a run of the algorithm in Figure~\ref{prop.trace}.
Each sequent in the database is numbered.  The non-axioms have the
rule and sequents from which they are derived in the right column.
The grayed-out sequents are either forward or backward subsumed.
Forward subsumed sequents are marked by $\FSubsumed$ and
backward by $\BSubsumed$.  (Also note, a forward subsumed sequent
is not assigned a number since it never enters the database.)

% \section{Proof terms}
% \label{prop.sec.proofterms}

% Soundness and
% completeness of $\LPB$ with respect to natural deduction are due
% to Gentzen, and the proof for our variation is effectively identical.  We
% will discuss natural deduction in section~\ref{prop.sec.proofterms} when we
% discuss proof terms.

% Theorem provers are complicated programs, and like any other
% program they are rife with bugs.  One way to gain more confidence
% in the result is for the program to produce a \emph{proof term},
% and efficiently verifiable certificate of the correctness of a
% derivation.  Our implementation traces deduction steps and
% upon success, builds a proof term in natural deduction from the
% trace.

% \subsection{Natural Deduction}

% Gentzen's natural deduction calculus is shown in Figure~\ref{prop.ndp}.

% Given a proof term, we can easily normalize it and
% check that it is a valid proof for a
% formula (E.g., Pfenning~\cite{Pfenning.2004.TheoremProvingLectureNotes}).

% 
\newcommand{\MNBoxI}{
  \Infer[\Box-$I$^a]
  {\MNd{\Gamma}{\MLam l:\MLe{w}{a}.~M : \MAt{\Box A}{w}}}
  {\MNd[\Delta,a:\MWorld]{\Gamma, l : \MLe{w}{a}}{M : \MAt{A}{a}}}
}

\newcommand{\MNBoxE}{
  \Infer[\Box-$E$]
  {\MNd{\Gamma}{M~[\MLe{w}{w'}]} : \MAt{A}{w'}}
  {
    \MNd{\Gamma}{M : \MAt{\Box A}{w}}
    &
    \Delta~;~\Gamma\Entails\MLe{w}{w'}
  }
}

\newcommand{\MNDiaI}{
  \Infer[\Dia-$I$]
  {\MNd{\Gamma}{\ASet{\MLe{w}{w'}} M} : \MAt{\Dia A}{w}}
  {
    \MNd{\Gamma}{M : \MAt{A}{w'}}
    &
    \Delta~;~\Gamma\Entails\MLe{w}{w'}
  }
}

\newcommand{\MNDiaE}{
  \Infer[\Dia-$E$^a]
  {\MNd{\Gamma}{\Let\ASet{l:\MLe{w'}{a}}x = M~\In~N : \MAt{C}{w}}}
  {
    \MNd{\Gamma}{M : \MAt{\Dia A}{w'}}
    &
    \MNd[\Delta,a:\MWorld]{\Gamma,l:\MLe{w'}{a},x:\MAt{A}{a}}{N : \MAt{C}{w}}
  }
}

\begin{figure}[H]
  \begin{center}
    {\footnotesize
      \fbox{
        \begin{tabular}{c}
          \begin{tabular}{cc}
            \MNBoxI
            &
            \hspace{2em}
            \MNBoxE
          \end{tabular}
          \\[2em]
          \begin{tabular}{c}
            \MNDiaI
          \end{tabular}
          \\[2em]
          \begin{tabular}{c}
            \MNDiaE
          \end{tabular}
        \end{tabular}
      }
    }
  \end{center}
  \caption{Natural deduction.}
  \label{modal.nd}
\end{figure}


%%% Local Variables:
%%% mode: latex
%%% TeX-master: "../thesis"
%%% End:


% \subsection{Proof Trace}

% Proof traces are maintained with the database of sequents.

% Define
% \[
% \Let{x}{t_1}{t_2} = (\lambda x.\ t_2) t_1
% \]

% 
\newcommand{\SCInit}{\ensuremath{\mbox{Init}}}
\newcommand{\SCInitRule}{\infer[$\RuleNameSize$ \SCInit]{\Seq{\Gamma, p}{p}}{}}

\newcommand{\SCAndR}{\ensuremath{\And\mbox{-R}}}
\newcommand{\SCAndRRule}{
  \infer[$\RuleNameSize$ \SCAndR]
  {\Seq{\Gamma}{A_1 \And A_2}}
  {\Seq{\Gamma}{A_1} & \Seq{\Gamma}{A_2}}
}

\newcommand{\SCAndL}{\ensuremath{\And\mbox{-L}}}
\newcommand{\SCAndLRule}{
  \infer[$\RuleNameSize$ \SCAndL]
  {\Seq{\Gamma, A_1 \And A_2}{C}}
  {\Seq{\Gamma, A_1, A_2}{C}}
}

\newcommand{\SCImpR}{\ensuremath{\Imp\mbox{-R}}}
\newcommand{\SCImpRRule}{
  \infer[$\RuleNameSize$ \SCImpR]
  {\Seq{\Gamma}{A_1 \Imp A_2}}
  {\Seq{\Gamma, A_1}{A_2}}
}

\newcommand{\SCImpL}{\ensuremath{\Imp\mbox{-L}}}
\newcommand{\SCImpLRule}{
  \infer[$\RuleNameSize$ \SCImpL]
  {\Seq{\Gamma, A_1\Imp A_2}{C}}
  {\Seq{\Gamma, A_1\Imp A_2}{A_1} & \Seq{\Gamma, A_2}{C}}
}

\newcommand{\SCOrRL}{\ensuremath{\Or\mbox{-R}_1}}
\newcommand{\SCOrRLRule}{
  \infer[$\RuleNameSize$ \SCOrRL]
  {\Seq{\Gamma}{A_1 \Or A_2}}
  {\Seq{\Gamma}{A_1}}
}

\newcommand{\SCOrRR}{\ensuremath{\Or\mbox{-R}_2}}
\newcommand{\SCOrRRRule}{
  \infer[$\RuleNameSize$ \SCOrRR]
  {\Seq{\Gamma}{A_1 \Or A_2}}
  {\Seq{\Gamma}{A_2}}
}

\newcommand{\SCOrL}{\ensuremath{\Or\mbox{-L}}}
\newcommand{\SCOrLRule}{
  \infer[$\RuleNameSize$ \SCOrL]
  {\Seq{\Gamma, A_1 \Or A_2}{C}}
  {\Seq{\Gamma, A_1}{C} & \Seq{\Gamma, A_2}{C}}
}

\newcommand{\SCTopR}{\ensuremath{\Top\mbox{-R}}}
\newcommand{\SCTopRRule}{
  \infer[$\RuleNameSize$ \SCTopR]
  {\Seq{\Gamma}{\Top}}
  {}
}

\newcommand{\SCBotL}{\ensuremath{\Bot\mbox{-L}}}
\newcommand{\SCBotLRule}{
  \infer[$\RuleNameSize$ \SCBotL]
  {\Seq{\Gamma, \Bot}{C}}
  {}
}

\renewcommand{\figline}{\\[2em]}

\begin{figure}
  \begin{center}
    \fbox{
      \begin{tabular}{cccccc}
        \SCInitRule
        &
        \SCBotLRule
        &
        \SCTopRRule
        \figline
        \SCOrRLRule
        &
        \SCOrRRRule
        &
        \SCOrLRule
        \figline
        \multicolumn{3}{c}{
          \begin{tabular}{cc}
            \SCImpRRule
            &
            \SCImpLRule
            \figline
            \SCAndRRule
            &
            \SCAndLRule
            \figline
            \mbox{No rule for $\Bot$-R}
            &
            \mbox{No rule for $\Top$-L}
          \end{tabular}
        }
      \end{tabular}
    }
  \end{center}
  \caption{Sequent Calculus}
  \label{rules.sc}
  \index{$\Seq{\Gamma}{A}$}
\end{figure}

%%% Local Variables:
%%% mode: latex
%%% TeX-master: "../thesis"
%%% End:


% \section{Focused proofs}
% \label{prop.sec.polar}

% \begin{quote}
%   \textit{
%     We revisit the backward calculus $\LPB$ for IPL
%     and demonstrate some redundancies.  A subset of $\LPB$ proofs called
%     \emph{focused proofs} are shown to be complete.}
% \end{quote}

% \subsection{Polarized formulas}
% \ednote{Note that polariteis and subformula signs are unrelated}
% Logical connectives have a \emph{polarity}, either positive or negative.  In
% linear logic, the polarity of each connective is uniquely determined.  Positive
% connectives are asynchronous on the left and synchronous on the right.  Negative
% connectives are synchronous on the left and asynchronous on the right.  This
% symmetry breaks down for intuitionistic logic where conjunction and truth are
% inherently ambiguous.  For example, both

% \[
% \Infer
% {\Seq{\Gamma, A_1 \And A_2}{C}}
% {\Seq{\Gamma, A_1, A_2}{C}}
% \]

% \noindent and the pair of rules

% \[
% \begin{array}{cc}
%   \Infer
%   {\Seq{\Gamma, A_1 \And A_2}{C}}
%   {\Seq{\Gamma, A_1}{C}}
%   &
%   \hspace{2em}
%   \Infer
%   {\Seq{\Gamma, A_1 \And A_2}{C}}
%   {\Seq{\Gamma, A_2}{C}}
% \end{array}
% \]

% \noindent
% are sound and complete.  In linear logic (see Section~\ref{chapter.linear}),
% there are two different conjunctions corresponding to the two possibilities:

% \[
% \begin{array}{ccc}
%   \Infer
%   {\Seq{\Gamma, A_1 \Tensor A_2}{C}}
%   {\Seq{\Gamma, A_1, A_2}{C}}
%   &
%   \hspace{2em}
%   \Infer
%   {\Seq{\Gamma, A_1 \With A_2}{C}}
%   {\Seq{\Gamma, A_1}{C}}
%   &
%   \hspace{2em}
%   \Infer
%   {\Seq{\Gamma, A_1 \With A_2}{C}}
%   {\Seq{\Gamma, A_2}{C}}
% \end{array}
% \]

% \noindent
% Since antecedents are combined and weakening is admissible, the distinction
% between the two forms of conjunction is lost.
% However, assigning polarities to the connectives is
% still a useful tool. In particular, we can take advantage of
% \emph{focusing}, described in the next section.  Thus, we will
% refine the structure of intuitionistic formulas by assigning polarities,
% yielding \emph{polarized formula}~\cite{Lamarche.1995.LICS} where immediately
% nested formulas always have the same polarity, unless an explicit
% polarity-shifting connective $\Up$ or $\Down$ is encountered.  These coercions
% are called \emph{shifts}.  Implication has slightly special status, in that its
% left-hand side has opposite polarity from its right-hand side.  This is because
% in the sequent calculus for intuitionistic logic, the focusing behavior of
% connectives on the left-hand side is the opposite of their behavior on the
% right-hand side. (Here the meta-variable $P$ ranges over atomic propositions.)

% \begin{align*}
%   \mbox{Positive formulas } A^+ &::= p^+ \Sep A^+ \Sum A^+ \Sep \Zero \Sep A^+ \Tensor A^+ \Sep \One \Sep \Down A^- \\
%   \mbox{Negative formulas } A^- &::= p^- \Sep A^+ \Lolli A^- \Sep A^- ~ \With ~A^- \Sep \Top \Sep \Up A^+
% \end{align*}

% \noindent The translation $A^-$ of an (unpolarized) formula $F$ in IPL
% is nondeterministic, subject only to the constraints that
% polarity is assigned to atomic formulas consistently (i.e., you can't have both
% $p^-$ and $p^+$ in the same translation), and $|A^-| = F$.  The arbitrary
% assignment of polarity to atomic formula is called a \emph{bias}.

% \[
% \begin{aligned}
%   |p^+| &= p \\
%   |p^-| &= p \\
%   |A^- \With B^-| &= |A^-| \And |B^-| \\
%   |\Top| &= \top \\
%   |A^+ \Tensor B^+| &= |A^+| \And |B^+|\\
%   |\One| &= \top \\
%   |A^+ \Sum B^+| &= |A^+| \Or |B^+| \\
%   |\Zero| &= \Bot \\
%   |A^+ \Lolli B^-| &= |A^+| \Imp |B^-| \\
%   |\Down A^-| &= |A^-| \\
%   |\Up A^+| &= |A^+| \\
% \end{aligned}
% \]

% \noindent
% For example, the formula $((a \Or c) \And (b \Imp c)) \Imp (a \Imp b) \Imp c$
% can be interpreted as any of the following polarized formulas (among others):

% \begin{align*}
%   &((\Down a^- \Sum \Down c^-) \Tensor \Down(\Down b^- \Lolli c^-)) \Lolli (\Down (\Down a^- \Lolli b^-) \Lolli c^-) \\
%   &\Down\Up((\Down a^- \Sum \Down c^-) \Tensor \Down(\Down b^- \Lolli c^-)) \Lolli (\Down\Up\Down (\Down a^- \Lolli b^-) \Lolli c^-) \\
%   &\Down(\Up (a^+ \Sum c^+) \With (b^+ \Lolli \Up c^+)) \Lolli (\Down (a^+ \Lolli \Up b
%   ^+) \Lolli \Up c^+)
% \end{align*}

% \noindent
% Shift operators have highest binding precedence in our presentation of the
% examples.  As we will see, the choice of translation determines the search
% behavior on the resulting polarized formula.  Different choices can lead to
% search spaces with radically different structure~\cite{Chaudhuri.2006.IJCAR}.

% \subsection{Backward Focused Sequent Calculus}

% The backward calculus is a refinement of $\LPB$ that eliminates
% don't-care nondeterministic choices, and manages don't-know
% nondeterminism by chaining such inferences in sequence.  Andreoli was
% the first to define this \emph{focusing} strategy and prove it
% complete~\cite{Andreoli.1992.JLC} for linear logic.  Similar proofs for other logics
% soon followed~\cite{Howe.1998.Thesis,Chaudhuri.2005.CSL,Chaudhuri.2006.IJCAR,Miller.2007.CSL,Zeilberger.2008.POPL},
% demonstrating that polarization and focusing can be applied to
% optimize search in a wide variety of logics.

% The polarized calculus is defined via four mutually recursive
% judgments.  In the judgments, we separate the antecedents into positive
% and negative zones.  We write $\Gamma$ for an unordered collection
% of negative formulas or positive atoms.  Dually, $C$ stands for
% a positive formula or a negative atom.

% The first two judgments, shown in Figure~\ref{prop.invert}
% concern formulas with invertible rules on the right and
% left.  Together, the two judgments form the \emph{inversion phase} of focusing.
% The context $\Delta^+$ is consists entirely of positive formulas and is ordered
% so that inference rules can only be applied to the rightmost formula,
% eliminating don't-care nondeterminism.

% 
\newcommand{\CCFocusR}{
  \Infer[$focus-R$]
  {\CCSeq{\Psi}{\Gamma}{A^+}}
  {\CCSeq{\Psi}{\Gamma}{[A^+]}}
}

\newcommand{\CCFocusL}{
  \Infer[$focus-L$]
  {\CCSeq{\Psi}{\Gamma,A^-}{\gamma}}
  {\CCSeq{\Psi}{\Gamma, [A^-]}{\gamma}}
}

\newcommand{\CCEtaP}{
  \Infer[\eta^+]
  {\CCSeq{\Psi}{\Gamma,p^+}{\gamma}}
  {\CCSeq{\Psi}{\Gamma, \ASet{p^+}}{\gamma}}
}

\newcommand{\CCEtaN}{
  \Infer[\eta^-]
  {\CCSeq{\Psi}{\Gamma}{p^-}}
  {\CCSeq{\Psi}{\Gamma}{\ASet{p^-}}}
}

\newcommand{\CCIdP}{
  \Infer[$id$^+]
  {\CCSeq{\Psi}{\Gamma,\ASet{A_1^+}}{[A_2^+]}}
  {\Delta;\Psi\Entails A_1\Ueq A_2}
}

\newcommand{\CCIdN}{
  \Infer[$id$^+]
  {\CCSeq{\Psi}{\Gamma,[A_1^-]}{\ASet{A_2^-}}}
  {\Delta;\Psi\Entails A_1\Ueq A_2}
}

\newcommand{\CCUpR}{
  \Infer[\Up$-R$]
  {\CCSeq{\Psi}{\Gamma}{\Up A^+}}
  {\CCSeq{\Psi}{\Gamma}{A^+}}
}

\newcommand{\CCUpL}{
  \Infer[\Up$-L$]
  {\CCSeq{\Psi}{\Gamma,[\Up A^-]}{\gamma}}
  {\CCSeq{\Psi}{\Gamma, A^-}{\gamma}}
}

\newcommand{\CCDownR}{
  \Infer[\Down$-R$]
  {\CCSeq{\Psi}{\Gamma}{[\Down A^-]}}
  {\CCSeq{\Psi}{\Gamma}{A^-}}
}

\newcommand{\CCDownL}{
  \Infer[\Down$-L$]
  {\CCSeq{\Psi}{\Gamma, \Down A^-}{\gamma}}
  {\CCSeq{\Psi}{\Gamma, A^-}{\gamma}}
}

\newcommand{\CCTopPR}{
  \Infer[\TopP$-R$]
  {\CCSeq{\Psi}{\Gamma}{[\TopP]}}
  {}
}

% \newcommand{\CCTopPL}{
%   \Infer[\TopP$-L$]
%   {\CCSeq{\Psi}{\Gamma,\TopP}{\gamma}}
%   {\CCSeq{\Psi}{\Gamma}{\gamma}}
% }

\newcommand{\CCTopNR}{
  \Infer[\TopN$-R$]
  {\CCSeq{\Psi}{\Gamma}{\TopN}}
  {}
}

% \newcommand{\CCTopNL}{
%   \Infer[\TopN$-L$]
%   {\CCSeq{\Psi}{\Gamma,[\TopN]}{\gamma}}
%   {}
% }

\newcommand{\CCAndPR}{
  \Infer[\AndP$-R$]
  {\CCSeq{\Psi}{\Gamma}{[A_1^+\AndP A_2^+]}}
  {\CCSeq{\Psi}{\Gamma}{[A_1^+]} & \CCSeq{\Psi}{\Gamma}{[A_2^+]}}
}

\newcommand{\CCAndPL}{
  \Infer[\AndP$-L$]
  {\CCSeq{\Psi}{\Gamma,A_1^+\AndP A_2^+}{\gamma}}
  {\CCSeq{\Psi}{\Gamma,A_1^+,A_2^+}{\gamma}}
}

\newcommand{\CCAndNR}{
  \Infer[\AndN$-R$]
  {\CCSeq{\Psi}{\Gamma}{A_1^-\AndN A_2^-}}
  {\CCSeq{\Psi}{\Gamma}{A_1^-} & \CCSeq{\Psi}{\Gamma}{A_2^-}}
}

\newcommand{\CCAndNL}{
  \Infer[\AndN$-L$_i]
  {\CCSeq{\Psi}{\Gamma,[A_1^-\AndN A_2^-]}{\gamma}}
  {\CCSeq{\Psi}{\Gamma,[A_i^-]}{\gamma}}
}

\newcommand{\CCOrR}{
  \Infer[\Or$-R$_i]
  {\CCSeq{\Psi}{\Gamma}{[A_1^+\Or A_2^+]}}
  {\CCSeq{\Psi}{\Gamma}{A_i^+}}
}

\newcommand{\CCOrL}{
  \Infer[\Or$-L$]
  {\CCSeq{\Psi}{\Gamma,A_1^+\Or A_2^+}{\gamma}}
  {\CCSeq{\Psi}{\Gamma,A_1}{\gamma} & \CCSeq{\Psi}{\Gamma,A_2}{\gamma}}
}

\newcommand{\CCImpR}{
  \Infer[\Imp$-R$]
  {\CCSeq{\Psi}{\Gamma}{A_1^+\Imp A_2^-}}
  {\CCSeq{\Psi}{\Gamma,A_1^+}{A_2^-}}
}

\newcommand{\CCImpL}{
  \Infer[\Imp$-L$]
  {\CCSeq{\Psi}{\Gamma,[A_1^+\Imp A_2^-]}{\gamma}}
  {\CCSeq{\Psi}{\Gamma,A_1^+\Imp A_2^-}{[A_1^+]} & \CCSeq{\Psi}{\Gamma,[A_2^-]}{\gamma}}
}

\newcommand{\CCAllR}{
  \Infer[\All$-R$^a]
  {\CCSeq{\Psi}{\Gamma}{\All x:\sigma.~A^-(x)}}
  {\CCSeq[\Delta,a:\sigma]{\Psi}{\Gamma}{A^-(a)}}
}

\newcommand{\CCAllL}{
  \Infer[\All$-L$]
  {\CCSeq{\Psi}{\Gamma, [\All x:\sigma.~A^-(x)]}{\gamma}}
  {\CCSeq{\Psi}{\Gamma, [A^-(t)]}{\gamma} & \Delta\Entails t:\sigma}
}

\newcommand{\CCExR}{
  \Infer[\Ex$-R$]
  {\CCSeq{\Psi}{\Gamma}{[\Ex x:\sigma.~A^+(x)]}}
  {\CCSeq{\Psi}{\Gamma}{[A^+(t)]} & \Delta\Entails t:\sigma}
}

\newcommand{\CCExL}{
  \Infer[\Ex$-L$^a]
  {\CCSeq{\Psi}{\Gamma, \Ex x:\sigma.~A^+(x)}{\gamma}}
  {\CCSeq[\Delta,a:\sigma]{\Psi}{\Gamma, A^+(a)}{\gamma}}
}

\newcommand{\CCDomR}{
  \Infer[$dom-R$]
  {\CCSeq{\Psi}{\Gamma}{[e^+]}}
  {\Psi\Entails e^+}
}

\newcommand{\CCDomL}{
  \Infer[$dom-L$]
  {\CCSeq{\Psi}{\Gamma, e^+}{\gamma}}
  {\CCSeq{\Psi, e^+}{\Gamma}{\gamma}}
}

\begin{figure}
  \begin{center}
    {\small
      \begin{tabular}{c}
        \\[1em]
        \begin{tabular}{ccc}
          \CCFocusR
          &\hsp
          \CCFocusL
          &\hsp
          \CCEtaP
        \end{tabular}
        \\[2em]
        \begin{tabular}{ccc}
          \CCEtaN
          &\hsp
          \CCIdP
          &\hsp
          \CCIdN
        \end{tabular}
        \\[2em]
        \begin{tabular}{ccc}
          \CCUpR
          &\hsp
          \CCUpL
          &\hsp
          \CCDownR
        \end{tabular}
        \\[2em]
        \begin{tabular}{ccc}
          \CCDownL
          &\hsp
          \CCDomR
          &\hsp
          \CCDomL
        \end{tabular}
        \\[2em]
        \begin{tabular}{cc}
          \CCAllR
          &\hsp
          \CCAllL
        \end{tabular}
        \\[2em]
        \begin{tabular}{cc}
          \CCExR
          &\hsp
          \CCExL
        \end{tabular}
        \\[2em]
        \begin{tabular}{cc}
          \CCTopPR
          &\hsp
          \CCTopNR
        \end{tabular}
        \\[2em]
        \begin{tabular}{cc}
          \CCAndPR
          &\hsp
          \CCAndPL
        \end{tabular}
        \\[2em]
        \begin{tabular}{cc}
          \CCAndNR
          &\hsp
          \CCAndNL
        \end{tabular}
        \\[2em]
        \begin{tabular}{cc}
          \CCOrR
          &\hsp
          \CCOrL
        \end{tabular}
        \\[2em]
        \begin{tabular}{cc}
          \CCImpR
          &\hsp
          \CCImpL
        \end{tabular}
      \end{tabular}
    }
  \end{center}
  \caption{Focused sequent calculus}
  \label{constr.focus}
\end{figure}

%%% Local Variables:
%%% mode: latex
%%% TeX-master: "../thesis"
%%% End:


% The next two judgments, shown in Figure~\ref{prop.focus}
% are concerned with non-invertible rules.  These
% two judgments make up the \emph{focusing phase}.
% Backward search for a proof of $A^-$ starts with an inversion from
% $\CSeq{\cdot}{A^-}$.  The proof then alternates between focusing and
% inversion phases.  The
% sequents at the boundary of the inversion and focusing phases
% have the form $\Seq{\Gamma}{C}$.  We call such
% sequents \emph{stable}.  There are two rules, shown in Figure~\ref{prop.neutral}
% that control the phase changes at stable sequents (the block boundaries) by
% selecting which subformula gains focus.

% \begin{theorem}[Soundness]
%   \begin{itemize}
%   \item[]
%   \item If $\CSeq{\Gamma}{A}$ then $\Seq{\Gamma}{A}$
%   \end{itemize}
% \end{theorem}

% \begin{proof}
%   Routine induction on the derivations.
% \end{proof}

% \begin{theorem}[Completeness]
%   If $\Seq{\cdot}{A}$ then $\CSeq{\cdot}{A}$
% \end{theorem}
% \begin{proof}
%   There are a number of proofs in the literature for intuitionistic
% linear logic, e.g.~\cite{Chaudhuri.2006.Thesis} and via translation
% for intuitionistic logic~\cite{Miller.2007.Focusing,Liang.2009.TCS}.
% IPL is a subset of the logic in Chapter~\ref{chapter.constraints}, so
% this is an instance of Theorem~\ref{constr.thm.complete} as well.
% \end{proof}

% \subsection{Synthetic Connectives and Derived Rules}
% \label{prop.sec.derived}

% We have already observed that backward proofs have the
% property that the proof is broken into blocks, with stable sequents at
% the boundary.  The only rules applicable to stable sequents are the
% rules that select a formula on which to focus.  It is the formulas
% occurring in stable sequents that form the primary objects of our further
% inquiry.

% It helps to think of such formulas, abstracted over their free
% variables, as \emph{synthetic connectives}~\cite{Andreoli.2001.APAL}.
% Define the synthetic connectives of a formula $A$ as all subformulas
% of $A$ that could appear in stable sequents in a focused backward proof.
% In a change of perspective, we can consider each block of a proof as
% the application of a left or right rule for a synthetic connective.
% The rules operating on synthetic connectives are derived from the
% rules for its constituent formulas.  We can thus consider a
% backward proof as a proof using only these synthetic (derived) rules.
% Each derived rule then corresponds to a block of the original proof.

% Since we need only consider stable sequents and synthetic connectives,
% we can simplify notation, and ignore the (empty) positive left and
% negative right zones in the derived rules.
% Write $\Seq{\Gamma}{C}$ as
% \renewcommand{\Seq}[2]{\ensuremath{#1 \Longrightarrow #2}}
% $\Seq{\Gamma}{C}$.
% As a further simplification, we can give formulas a predicate label
% and abstract over its free variables.  This labeling technique is described in
% detail in~\cite{Voronkov.2001.Handbook}.  For the remainder, we assume
% this labeling has been carried out.  Define an \emph{atomic formula}
% as either a label or a predicate applied to a (possibly
% empty) list of terms.  After labeling, our sequents consist entirely
% of atomic formulas.

% \begin{example}
% Figure~\ref{prop.blocks} shows a focused proof.
% Here the inversion phases are colored blue and the focusing phases
% are purple.  The stable sequents, on the boundary, are colored red.
% While it looks rather complex for such a simple formula, keep in
% mind that the only choices are made at stable sequents.  Note
% that the bias of the atoms is essential to the shape of the proof.
% A different choice would lead to a proof with a much different structure.

% The synthetic connectives are $a \Imp b$
% and $(a \Imp b)\Imp c$.  There is a single derived rule for each
% synthetic connective (though this is not the case in general).  The
% atoms are assigned negative polarity.  We implicitly carry the
% principal formula of a left rule to all of its premises.

% \[
% \begin{array}{cc}
%   \Infer[r_1]{\Seq{a, a\Imp b}{b}}{}
%   &\hspace{2em}
%   \Infer[r_2]{\Seq{\Down (a\Imp b)\Imp c}{c}}{\Seq{a}{b}}
% \end{array}
% \]

% \noindent
% Using the derived rules, Figure~\ref{prop.blocks} can be compressed to the succinct

% \[
% \Infer[r_2]
% {\Seq{\Down (a\Imp b)\Imp c, a\Imp b}{c}}
% {\Infer[r_1]{\Seq{a, a\Imp b}{b}}{}}
% \]

% %

\newcommand{\nseq}[3]{{\color{blue}{\Nseq{#1}{#2}{#3}}}}
\newcommand{\pseq}[3]{{\color{blue}{\Pseq{#1}{#2}{#3}}}}
\newcommand{\rfoc}[2]{{\color{Fuchsia}{\Rfoc{#1}{#2}}}}
\newcommand{\lfoc}[3]{{\color{Fuchsia}{\Lfoc{#1}{#2}{#3}}}}
\newcommand{\stable}[2]{{\color{red}{\Seq{#1}{#2}}}}

\renewcommand{\ab}{a \Lolli b}
\renewcommand{\abc}{\Down (\ab) \Lolli c}

\renewcommand{\qr}{\rfoc{\abc, \ab, a}{a}}
\renewcommand{\rr}{\Infer[$Init-R$]{\qr}{}}

\renewcommand{\qq}{\lfoc{\abc, \ab, a}{b}{b}}
\renewcommand{\rq}{\Infer[$Init-L$]{\qq}{}}

\renewcommand{\qp}{\lfoc{\abc, \ab, a}{\ab}{b}}
\renewcommand{\rp}{\Infer[\Imp$-L$]{\qp}{\rq & \rr}}

\renewcommand{\qo}{\stable{\abc, \ab, a}{b}}
\renewcommand{\ro}{\Infer[$Focus-L$]{\qo}{\rp}}

\renewcommand{\qn}{\pseq{\abc, \ab, a}{\cdot}{b}}
\renewcommand{\rn}{\Infer[$Stable$]{\qn}{\ro}}

\renewcommand{\qm}{\pseq{\abc, \ab}{a}{b}}
\renewcommand{\rm}{\Infer[$Atom-L$]{\qm}{\rn}}

\renewcommand{\ql}{\nseq{\abc, \ab}{a}{b}}
\renewcommand{\rl}{\Infer[$Atom-R$]{\ql}{\rm}}

\renewcommand{\qk}{\nseq{\abc, \ab}{\cdot}{\ab}}
\renewcommand{\rk}{\Infer[\Imp$-R$]{\qk}{\rl}}

\renewcommand{\qj}{\rfoc{\abc, \ab}{\Down (\ab)}}
\renewcommand{\rj}{\Infer[\Down$-R$]{\qj}{\rk}}

\renewcommand{\qi}{\lfoc{\abc, \ab}{c}{c}}
\renewcommand{\ri}{\Infer[$Init-L$]{\qi}{}}

\renewcommand{\qh}{\lfoc{\abc, \ab}{\abc}{c}}
\renewcommand{\rh}{\Infer[\Imp$-L$]{\qh}{\ri & \rj}}

\renewcommand{\qg}{\stable{\abc, \ab}{c}}
\renewcommand{\rg}{\Infer[$Focus-L$]{\qg}{\rh}}

\renewcommand{\qf}{\pseq{\abc, \ab}{\cdot}{c}}
\renewcommand{\rf}{\Infer[$Stable$]{\qf}{\rg}}

\renewcommand{\qe}{\pseq{\ab}{\Down(\abc)}{c}}
\renewcommand{\re}{\Infer[\Down$-L$]{\qe}{\rf}}

\renewcommand{\qd}{\pseq{\cdot}{\Down(\abc), \Down(\ab)}{c}}
\renewcommand{\rd}{\Infer[\Down$-L$]{\qd}{\re}}

\renewcommand{\qc}{\nseq{\cdot}{\Down(\abc), \Down(\ab)}{c}}
\renewcommand{\rc}{\Infer[$Atom-R$]{\qc}{\rd}}

\renewcommand{\qb}{\nseq{\cdot}{\Down(\abc)}{\Down(\ab) \Lolli c}}
\renewcommand{\rb}{\Infer[\Lolli$-R$]{\qb}{\rc}}

\renewcommand{\qa}{\nseq{\cdot}{\cdot}{\Down(\abc) \Lolli \Down(\ab) \Lolli c}}
\renewcommand{\ra}{\Infer[\Lolli$-R$]{\qa}{\rb}}

\begin{sidewaysfigure}
{\small
\[ \ra \]
}
\caption{Backward proof with $a^+$, $b^-$, $c^-$}
\label{prop.blocks}
\end{sidewaysfigure}

%%% Local Variables:
%%% mode: latex
%%% TeX-master: "../thesis"
%%% End:


% \end{example}

% \section{Implementation}



% \section{Derivation strategies}

% Let $\cR$ be the set of derived rules for a goal sequent.
% Step~\ref{prop.enum.rules} of the inverse method
% (Section~\ref{prop.sec.inverse}) is to generate a derivation from $\cR$.
% Any derivation will find the goal
% (Corollary~\ref{prop.thm.inverse-complete}) so we have considerable flexibility
% in choosing a strategy for applying the rules.  We need only assure that
% the derivation we devise is fair.

% \subsection{A Generic Otter Loop}

% We maintain two separate
% databases of facts, $\cD_i^k$ and $\cD_i^a$ where
% $\cD_i=\cD_i^a\Union\cD_i^k$.  $\cD^k$ is called the \emph{kept}
% database and $\cD^a$ is called the \emph{active} database.
% We maintain the invariant that $\match(\cR, \cD_i^a)\subseteq \cD_i$.
% If $\cD_i^k=\emptyset$ then by the invariant we have
% $\match(\cR, \cD_i)\subseteq \cD_i$, and the database is saturated.

% \begin{align*}
%   \cD_0^a &= \emptyset\\
%   \cD_0^k &= \mbox{the initial sequents of $\cR$}\\
%   \mu(i) &= \mbox{some element of $\cD_i^k$}\\
%   \cD_{i+1}^a &= \Closure{\cD_i^a\Union\Set{\mu(i)}}\\
%   \cD_{i+1}^k &= \Closure{(\cD_i^k\Diff\Set{\mu(i)})\Union\match(\cR, \cD_{i+1}^a)}\\
%   \cD_i &= \cD_i^k\Union \cD_i^a\\
% \end{align*}

% \noindent $cD$ can be made a derivation by arbitrarily sequentializing the
% set operations into inference and subsumption steps.  However, we still are not
% guaranteed the derivation is fair.

% \begin{definition}
%   Given a derivation $\cD$, a function $\mu$ is called
% a \emph{selection function} if for all $i$, $f(i)\in \cD_i^k$.
% $\mu$ is \emph{fair} if for any $x\in\Dlimit$ there exists an $i$ such
% that $\mu(i)=x$.
% \end{definition}

% \begin{theorem}
%   If $\mu$ is fair then the derivation $\cD$ defined above is fair.
% \end{theorem}
% \begin{proof}
%   Let $q_1,\ldots,q_n\in\Dlimit$ and $q\in\match(R,q_1,\ldots,q_n)$.
% For each $i$, there is an $f(i)$ such
% that $q_i = \mu(f(i))$.  Let $k=\max(f(i))$.  Then some inference step in
% the sequentialization from $\cD_k$ to $\cD_{k+1}$ will add $q$ to the derivation (though
% it may be subsumed by other sequents in $\cD_{k+1}$ and thus not appear
% there).
% \end{proof}

% \noindent The general strategy of maintaining a kept and active database
% and selecting elements from kept by some fair process is called
% the \emph{Otter loop} after McCune's
% Otter\footnote{Otter is the successor of Prover9~\cite{McCune:2010:Prover9}}
% theorem prover described in~\cite{McCune.1990.CADE}.

% \subsection{Partial-application}

% In partial-application, instead of being fixed, the set of inference rules grows
% during rule application.  We maintain not only kept and active databases of
% sequents, as described

% \subsubsection{Cascading}

% vs. partial application:

% 1) easier to subsume rules
% 2) Many fewer matches.
% 3) partial application forces an ordering on the premises.
%    Cascading does not.

% e.g. H1 H2
%      -----
%        H

% Match r with q1, q2, q3

% H2
% __
%  H1

% H2
% __
%  H2

% H2
% __
%  H3

% Now you must match H2 at least 3 times with some sequent q.


% \subsection{Optimizations}
% \label{prop.sec.optimize}

% \paragraph{Globalization.}

% Proof search in the inverse method begins with an inversion phase which produces
% stable sequents.  These sequents are proved independently.  Fix a stable sequent
% $\Seq{\Gamma}{A}$.  The inverse method will continually test whether new
% facts subsume the goal.  I.e., for $\FSeq{\Gamma'}{\gamma'}$ to subsume the
% goal, we must have $\Gamma' \subseteq \Gamma$.  This is equivalent to $\Gamma'
% \Diff \Gamma = \emptyset$.  Thus, it is sufficient to remove $\Gamma$ from the
% antecedents of any generated sequent.  It is justified by modifying the
% invariant of a database $\cD$. If $\FSeq{\Gamma'}{\gamma}\in\cD$ and $\Set{A}
% \subseteq \gamma$ then $\Seq{\Gamma\Union\Gamma'}{A}$, is provable in $\LPB$.
% This optimization saves space and reduces the complexity of matching and
% subsumption during proof search.

% \paragraph{Bi-implication lengthening.}

% Bi-implication, as a derived connective, derives its focusing behavior from that
% of its constituent connectives.  The right derived rule is

% {\small
%   \noindent
%   \[
%   \begin{array}{lr}
%     \\[2em]
%     \Infer[\With$-L$]
%     {\Nseq{\Delta}{\Gamma}{(\Down A_1^-\Imp A_2^-)\With(\Down A_2^-\Imp A_1^-)}}
%     {
%       \Infer[\Imp$-R$]
%       {\Nseq{\Delta}{\Gamma}{\Down A_1^-\Imp A_2^-}}
%       {
%         \Infer[\Down$-L$*]
%         {\Nseq{\Delta}{\Gamma, \Down A_1^-}{A_2^-}}
%         {\Nseq{\Delta, A_1^-}{\Gamma}{A_2^-}}
%       }
%       &
%       \Infer[\Imp$-R$]
%       {\Nseq{\Delta}{\Gamma}{\Down A_2^-\Imp A_1^-}}
%       {
%         \Infer[\Down$-L$*]
%         {\Nseq{\Delta}{\Gamma, \Down A_2^-}{A_1^-}}
%         {\Nseq{\Delta, A_2^-}{\Gamma}{A_1^-}}
%       }
%     }
%     &
%     \hspace{2em}
%     \Infer[\Iff$-R$]
%     {\Nseq{\Delta}{\Gamma}{A_1^-\Iff A_2^-}}
%     {\Nseq{\Delta, A_1^-}{\Gamma}{A_2^-} & \Nseq{\Delta, A_2^-}{\Gamma}{A_1^-}}
%     \\[2em]
%   \end{array}
%   \]
% }

% \noindent
% The rules marked * are not part of the given inversion system because
% in a focused proof we would continue decomposing on the right until we
% reached a positive formula or a negative atom, but since the order
% is irrelevant for completeness this is a valid derived rule.
% The left rules are

% {\small
%   \noindent
%   \[
%   \begin{array}{lr}
%     \\[2em]
%     \Infer[\With$-L$_1]
%     {\Lfoc{\Delta}{(\Down A_1^-\Imp A_2^-)\With(\Down A_2^-\Imp A_1^-)}{C}}
%     {
%       \Infer[\Imp$-L$]
%       {\Lfoc{\Delta}{\Down A_1^-\Imp A_2^-}{C}}
%       {
%         \Lfoc{\Delta}{A_2^-}{C}
%         &
%         \Infer[\Down$-R$]
%         {\Rfoc{\Delta}{\Down A_1^-}}
%         {\Nseq{\Delta}{\cdot}{A_1^-}}
%       }
%     }
%     &
%     \hspace{2em}
%     \Infer[\Iff$-L$_1]
%     {\Lfoc{\Delta}{A_1^-\Iff A_2^-}{C}}
%     {\Lfoc{\Delta}{A_2^-}{C} & \Nseq{\Delta}{\cdot}{A_1^-}}
%     \\[2em]
%     \Infer[\With$-L$_2]
%     {\Lfoc{\Delta}{(\Down A_1^-\Imp A_2^-)\With(\Down A_2^-\Imp A_1^-)}{C}}
%     {
%       \Infer[\Imp$-L$]
%       {\Lfoc{\Delta}{\Down A_2^-\Imp A_1^-}{C}}
%       {
%         \Lfoc{\Delta}{A_1^-}{C}
%         &
%         \Infer[\Down$-R$]
%         {\Rfoc{\Delta}{\Down A_2^-}}
%         {\Nseq{\Delta}{\cdot}{A_2^-}}
%       }
%     }
%     &
%     \hspace{2em}
%     \Infer[\Iff$-L$_1]
%     {\Lfoc{\Delta}{A_1^-\Iff A_2^-}{C}}
%     {\Lfoc{\Delta}{A_1^-}{C} & \Nseq{\Delta}{\cdot}{A_2^-}}
%   \end{array}
%   \]
% }

% \noindent
% We generally want focusing phases to be as long as possible;
% we can always shorten focusing phases with explicit shifts, but cannot
% extend them. The loss of focus on the right premise can be avoided when
% $A_2^-= \Up A^+$, e.g.

% {\small
%   \[
%   \Infer[\Iff$-L$_1]
%   {\Lfoc{\Delta}{A_1^-\Iff \Up A_2^+}{C}}
%   {\Lfoc{\Delta}{A_1^-}{C} & \Rfoc{\Delta}{A_2^+}}
%   \]
% }

% \noindent
% Optimizing bi-implication is important in the ILTP benchmark, as
% over 66\% of the problems involve the connective.

% \paragraph{Rule subsumption.}

% \paragraph{Exact matching.}

% \paragraph{Branch disabling.}

% \section{Experimental Results}

% \subsection{Simulations}

% Weak focusing, weak inversion, single step

% \section{Notes}

% A brief history of the inverse method occurs in~\cite{Voronkov.2001.Handbook}.
% It was first described (without labeling) by Gentzen~\cite{Gentzen.1934.MZ}, who
% noticed it forms a decision procedure for IPL.  The term ``inverse method'' was
% coined by Maslov~\cite{Maslov.1964.Inverse}.

% \cite{Howe.1998.Thesis}

% Focusing

% \subsection{Related Work}
% \label{prop.sec.related}

% In this section we discuss other techniques for theorem proving in
% IPL.  Work that prove theorems in first-order intuitionistic
% logic (and thus to IPL as well) is reviewed in Chapter~\ref{chapter.fol}.

% \paragraph{Contraction-free calculi.}

% There are a number of different methods for theorem proving in IPL in
% the literature.  Most early work
% revolves around the following observation: In the
% rules of Figure~\ref{prop.backward}, the only rule that can confound
% a bounded backward search is the rule $\PBImpL$:
% \[
% \PBImpLRule
% \]
% \noindent Indeed, the backtracking implicit in this rule can not be
% avoided; IPL is PSPACE-complete~\cite{Statman.1979.TCS}.
% Some attempts at attenuate the problem are~\cite{Franzen.1987.SICS,Franzen.1989.SICS}.
% Dyckhoff~\cite{Dyckhoff.1992.JSL}, rediscovering earlier work of Vobor'ev~\cite{Vorobev.1952.Doklady},
% discards $\PBImpL$ in favor of the following four rules

% \[
% \begin{array}{cc}
%   \Infer[\Imp$-L$_1]
%   {\Seq{\Gamma,p\Imp A, p}{C}}
%   {\Seq{\Gamma, A, p}{C}}
%   &
%   \hspace{2em}
%   \Infer[\Imp$-L$_2]
%   {\Seq{\Gamma,(A_1\And A_2)\Imp A}{C}}
%   {\Seq{\Gamma, A_1\Imp (A_2\Imp A)}{C}}
%   \figline
%   \Infer[\Imp$-L$_3]
%   {\Seq{\Gamma,(A_1\Or A_2)\Imp A}{C}}
%   {\Seq{\Gamma, A_1\Imp A, A_2\Imp A}{C}}
%   &
%   \hspace{2em}
%   \Infer[\Imp$-L$_4]
%   {\Seq{\Gamma,(A_1\Imp A_2)\Imp A}{C}}
%   {\Seq{\Gamma, A_2\Imp A}{A_1\Imp A_2} & \Seq{\Gamma, A}{C}}
% \end{array}
% \]
% \noindent
% He shows that this ``contraction-free'' calculus is terminating, and does
% not require loop checking.  (Note that it does not satisfy the subformula property,
% and is thus not naievly amenable to implementation via the inverse method.)
% Implementations of the calculus LJT are~\cite{Dyckhoff.1997.LJT} and
% \emph{Porgi}~\cite{Stoughton.1996.CADE}.

% Hudelmaier~\cite{Hudelmaier.1993.JLC} noticed that because of the duplications
% of formulae in the rules $\Imp$-L$_3$, $\Imp$-L$_4$, the procedure uses space
% exponential in the size of the input formula.  He improves on
% Dyckhoff by developing a calculus that uses at most $N\log(N)$ space.
% A space-efficient implementation is STRIP~\cite{LarcheyWendling.2001.STRIP}.

% Probably the most mature implementation of a theorem prover for IPL is
% PITP~\cite{Avellone.2004.Goedel,Avellone.2007.PITP,Avellone.1996.Tableaux,Avellone.2008.TCS}.
% It essentially follows the ideas of Hudelmaier, with a number of optimizations
% based on the underlying Kripke semantics.  (IPTP~\cite{Avellone.2003.IPTP} is an
% earlier incarnation.)  PITP generates neither proofs nor countermodels, though
% their method is is capable of producing either~\cite{Dyckhoff.1995.SG}.

% \paragraph{A model-theoretic approach.}

% A totally different approach is taken in~\cite{Gore.2012.IJCAR}.
% Because the IPL with Kripke semantics satisfies the finite model
% property, it is theoretically possible to construct all potential
% models of a formula and simply check whether they satisfy the formula.
% If they all do, the formula is a theorem.  Otherwise you have a
% concrete countermodel.  As a practical technique this seems unwieldy,
% as the number of models is exponential in the size of the formula.
% The authors demonstrate it is possible to use BDDs to concisely
% represent the models and check satisfiability efficiently.


% \section{Future Work}

% \paragraph{Countermodels.}

% One weakness of our implementation is the inability to produce
% countermodels, Kripkean or otherwise.  It seems clear that a saturated
% database is evidence of the non-provability of a formula;  if the
% database subsumes the initial sequents, no further (non-subsumed)
% inferences are possible, and the database does not subsume the goal
% formula, the formula is not provable.  This is not entirely satisfying.
% For example, it relies on correctly implementing subsumption, which is
% easy in the propositional case but is increasingly complicated in
% later chapters.  Software bugs can (and have) led Imogen to report false
% negatives.

% A more satisfying solution would be to generate countermodels, such
% as those generated by STRIP et al.  It is an open question whether
% such a model can be extracted from a saturated database.  Even if that
% is not possible, could the failed state of an inverse method proof
% guide counterexample generation in a less direct way?

% \paragraph{Polarization heuristics.}

% As we have seen, there are infinitely many choices of a polarization for
% an input formula.  We found some that worked particularly well for the
% ILTP problems, but do not yet have a good general theory of how polarization
% affects the search space.  While some steps have been taken in this direction,
% e.g. associating SLD-resolution and hyperresolution with the assignment of
% atoms as positive or negative respectively, a more thorough understanding is
% called for.

% \paragraph{Optimizations.}


% \subsection{Notes}

% \begin{remark}
%   Most presentations of sequent calculus use multisets rather
%   than sets.  Indeed, using sets is not ideal in the meta-theory
%   of sequent calculus.  For example,
%   there should be two different proofs of $a\Imp a\Imp a$.
%   Using sets, there is only one.
% \end{remark}

% \begin{theorem}
%   If
% \end{theorem}
% \begin{proof}

%  However, using sets is sound and
%   complete with respect to the multiset system in the sense that
%   given a sequent in the multiset system, you can always derive
%   a sequent that subsumes it in the set system.
%   (Since contraction is an admissible rule in the multiset system,
%   you can prove this trivially by induction, where after ever inference
%   you use the contraction rule as many times as possible.)
%   Using sets eases the notational burdon of our many transitions between
%   forward and backward, focused and unfocused calculi.
% \end{proof}

%%% Local Variables:
%%% mode: latex
%%% TeX-master: "thesis"
%%% End:
