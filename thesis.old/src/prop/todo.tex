
\chapter{Focusing}

In this chapter we revisit the backward calculus $\LPB$ for IPL
and demonstrate some redundancy.  A subset of $\LPB$ proofs called
\emph{focused proofs} are shown to be complete.  The inverse method
is applied to focused proofs to yield a new calculus $\FPFs$.
We describe our theorem prover \emph{Imogen}, an implementation of $\FPFs$.

% A sequent of intuitionistic logic has the form $\Stable{\Gamma}{A}$,
% where $\Gamma$ is a set or multiset of formulas.  For
% purposes of Imogen it is convenient to always maintain $\Gamma$
% as a set, without duplicates.
% Since we can always eagerly decompose negative connectives
% on the right of a sequent and positive connectives on the left,
% the only sequents in our polarized calculus we need to consider
% have negative formulas on the left or positive formulas on the
% right, in addition to atoms which can appear with either polarity
% on either side.  The right-hand side could also be empty if we are
% deriving a contradiction.   We call such sequents \emph{stable}.
% \[\begin{array}{llcl}
% \mbox{Stable Hypotheses} & \Gamma &::= & \cdot \Sep \Gamma, A^- \Sep \Gamma, P^+ \\
% \mbox{Stable Conclusions} & \gamma &::= & A^+ \Sep P^- \Sep \cdot \\
% \mbox{Stable Sequents} & \multicolumn{3}{l}{\Stable{\Gamma}{\gamma}}
% \end{array}\]


\section{Proof search}

In this section we describe an abstract machiene for proof search in $\LPFs$ and describe
some optimizations.

\subsection{An Otter loop}

\begin{definition}
  \begin{itemize}
  \item[]
  \item A \emph{state} is a pair of databases $\ASet{\cD^a,\cD^k}$.
  \item $\cD^a$ is called the \emph{active database}.
  \item $\cD^k$ is called the \emph{kept database}.
  \item A state is \emph{saturated} if $\cD^k=\Set{}$.
  \end{itemize}
\end{definition}

\begin{definition}
  A set of sequents $X$ is called \emph{minimal} if there do not exist $Q,Q'\in
  X$ such that $Q\Subsumes Q'$.  It is clear any finite set $X$ can be made
  minimal by removing subsumed sequents. Fix some procedure
  for doing so, called $\clear(X)$.
\end{definition}


\begin{algorithm}
  \begin{algorithmic}[1]
    \STATE Input a formula $\xi$
    \STATE $\NS \Gets \Set{}$
    \STATE $\KS \Gets $ the axioms of $\LPFs$
    \STATE $\RU \Gets $ the rules of $\LPFs$
    \WHILE{$\KS \neq \Set{}$} \label{ruleloop:while}
    \FOR{$R \in \RU$}
    \STATE $\NS \Gets \match(R, \KS, \AS) \Union \NS$
    \ENDFOR
    \IF{$\exists s\in \NS.\ s\Subsumes \FSeq{\cdot}{\xi}$}
    \PRINT \Quote{Proof found. $\xi$ is true.}
    \RETURN
    \ENDIF
    \STATE $\AS \Gets \KS \Union \AS$
    \STATE $\KS \Gets \NS$
    \STATE $\NS \Gets \emptyset$
    \ENDWHILE
    \PRINT \Quote{The database is saturated.  $\xi$ is false.}
    \RETURN
  \end{algorithmic}
  \caption{The Imogen Fixed-Rule Loop}
  \label{inverse:algorithm:fixed}
\end{algorithm}

%%% Local Variables:
%%% mode: latex
%%% TeX-master: "../../../main"
%%% End:


\subsection{Subsumption}

In Section~\label{prop.sec.search} we defined subsumption steps generally as
the removal of a subsumed sequent from the database via a subsumption step.
In this section we define three concrete  is used in our implementation.
The first two are standard in the literature.  The last one, to our knowledge,
is new.

\paragraph{Forward Subsumption}

\begin{definition}
  Define $\nextdb(\cR, \cD) = \cD\Union \match(\cR, \cD)$.
  Given a set of rules $\cR$, define $\cD_0 =$ the initial sequents
  of $\cR$.  Define $\cD_n =\nextdb(\cR, \cD_{n-1})$.
  A sequent $Q$ is \emph{$\cR$-provable} if there exists an
  $n$ and a sequent $Q'$ such that $Q'\in\cD_n$ and $Q'\Subsumes Q$.
  If $\nextdb(\cR, \cD) = \cD$ we say that $\cD$ is \emph{$\cR$-saturated}.
\end{definition}

\paragraph{Backward Subsumption}


\paragraph{Rule Subsumption}

\begin{definition}
  Let $\cR$ be a set of inference rules, and let $\cD$ be a database.
  Define
  \[
  \match(R, \cD) = \Set{Q | Q_1,\ldots,Q_n\in\cD} \mbox{ and } \match(R, Q_1,\ldots,Q_n)=Q
  \]
\end{definition}

\begin{definition}
  Define $\nextdb(\cR, \cD) = \cD\Union \match(\cR, \cD)$.
  Given a set of rules $\cR$, define $\cD_0 =$ the initial sequents
  of $\cR$.  Define $\cD_n =\nextdb(\cR, \cD_{n-1})$.
  A sequent $Q$ is \emph{$\cR$-provable} if there exists an
  $n$ and a sequent $Q'$ such that $Q'\in\cD_n$ and $Q'\Subsumes Q$.
  If $\nextdb(\cR, \cD) = \cD$ we say that $\cD$ is \emph{$\cR$-saturated}.
\end{definition}

%%% Local Variables:
%%% mode: latex
%%% TeX-master: "thesis"
%%% End:
