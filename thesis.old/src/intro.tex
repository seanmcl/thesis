
\chapter*{Introduction}\label{chapter.intro}

\begin{quote} Building a theorem prover is a cross between proving a
theorem and driving a sports car.  -- Tobias Nipkow\end{quote}

This thesis describes the theory and implementation of a theorem
prover called Imogen which proves theorems in a number of
intuitionistic logics.  Imogen is interesting for two reasons.  The
first reason is performance.  On the propositional fragment, its
performance is comparable to the best existing theorem provers, of
which there are many.  On full first-order logic, it performs
significantly better than any other existing theorem provers on
standard benchmarks.  The second reason is its generality.  With
relatively minor changes to the code that handles first-order logic,
Imogen is able to prove theorems in less well-known logics, such as
linear logic, two fundamentally different types of modal logic, and
ordered logic.  The way Imogen accomplishes these two feats is the
subject matter of this thesis.

Before we begin, it's worth considering why we should care about
theorem proving in intuitionistic logics.  While no one would doubt
its theoretical and historical interest, any practical use that would
require theorem proving may indeed require some justification.
This is especially true due to the fact that the theorem proving
problem is, practically speaking, much harder than the classical
problem.  I'll give two concrete examples to justify this statement,
one well-known and one less so.  First, the decision problem
for the propositional fragment is PSPACE-complete for intuitionistic
logic, while the classical problem is merely NP-complete.  Second,
a standard unification problem that arises during proof search
in first-order logic with equality is undecidable



%%% Local Variables:
%%% mode: latex
%%% TeX-master: "thesis"
%%% End:
