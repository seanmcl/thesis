
\clearpage

\vspace*{2in}
\begin{quote}
  \begin{center}
    ``O Imogen, Safe mayst thou wander, safe return again!'' \\
    \hspace{3em}  --- William Shakespeare, (\textit{Cymbeline}, Act III, Scene 5)
  \end{center}
\end{quote}
\vspace*{\fill}

"I don't dislike him, I just don't like him. Which is quite
 different." (Episode 2.06) -- Violet Crawley, Dowager Countess of Grantham






There is an important non-analytic rule that is sometimes added to
tableau systems, the cut rule. It says, at any point in a tableau
construction we can split the end of a branch, labeling the two new
branch nodes with T X and F X, for an arbitrary formula X. Since X can
be any formula, obviously analyticity is violated. There is a more
restricted version of the rule, in which X is required to be a
subformula of the formula being proved—this is called analytic cut.

...

Why consider analytic cut? For one thing, it can shorten proofs, and
does not violate the subformula principle, so proof search procedures
can incorporate it in a reasonable way. It has sometimes been included
in tableau implementations for this reason. The reader cannot fail to
have noticed that while nine representative modal logics were
intro-duced in Table 1, tableau systems were given for only six of
them. Tableau systems for the other three are missing, though if
analytic cut is allowed, destructive tableau systems can be created,
[33]

Propositional dynamic logic is, perhaps, the original example, see
[35] for a thorough treatment. In this multi-modal logic, modal
operators correspond to computer programs. The semantical treatment of
the while operator requires a fixpoint construction. From early on
there was a tableau system for the logic, [51]. It is, however, of a
specialized nature that so far has not lent itself well to treatment
by the general methodologies of this chapter. Propositional dynamic
logic was extended to the propositional μ-calculus in [41], with modal
operators in the language corresponding directly to fixpoint
constructions. I do not know how to bring tableau methods to bear.


section .5.5 handbook of modal logic

-----------

. . there is no one fundamental logical no-tion of necessity, nor
consequently of possi-bility. If this conclusion is valid, the subject
of modality ought to be banished from logic, since propositions are
simply true or false . . . [Russell, 1905]

, 1905] Bertrand Russell. Necessity and possibility, 1905. Paper read
to the Oxford Philosophical Society on 22 October 1905. Published in
The Collected Papers of Bertrand Russell, Volume 4, pages 507–520,
Routledge, 1994.


-----------------

The scenario for managing permissions in these cases varies by
service. Here are some examples of how permissions are handled for
different services:

In Auto Scaling, users must have permission to use Auto Scaling, but
don't need to be explicitly granted permission to manage Amazon EC2
instances.

In AWS Data Pipeline, an IAM role determines what a pipeline can do;
users need permission to assume the role. (For details, see Granting
Permissions to Pipelines with IAM in the AWS Data Pipeline Developer
Guide.)

For details about how to configure permissions properly so that an AWS
service is able to accomplish the tasks you intend, refer to the
documentation for the service you are calling.

-- http://docs.aws.amazon.com/IAM/latest/UserGuide/policies_permissions.html

%%% Local Variables:
%%% mode: latex
%%% TeX-master: "thesis"
%%% End:
