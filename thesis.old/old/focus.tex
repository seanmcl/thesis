
\chapter{Polarization and Focusing}

\section{Focusing}\label{prop.sec.focusing}
\subsection{Polarized formulas}\label{prop.sec.polar}

In linear logic, the polarity of each connective is uniquely
determined.  This is not true for intuitionistic logic where
conjunction and truth are inherently ambiguous.  We therefore
assign polarities to formulas in a preprocessing phase.  It is
convenient to represent the result as a \emph{polarized formula}~\cite{Lamarche.1995.LICS}
where immediately nested formulas always have the same polarity,
unless an explicit polarity-shifting connective $\Up$ or
$\Down$ is encountered.  These coercions are called \emph{shifts}.

Implication has slightly special status, in that its
left-hand side has opposite polarity from its right-hand side.
This is because in the sequent calculus for intuitionistic logic,
the focusing behavior of connectives on the left-hand side
is the opposite of their behavior on the right-hand side. (Here the
meta-variable $P$ ranges over atomic propositions.)

\begin{align*}
  \mbox{Positive formulas } A^+ &::= P^+ \Sep A^+ \Sum A^+ \Sep \Zero \Sep A^+ \Tensor A^+ \Sep \One \Sep \Down A^- \\
  \mbox{Negative formulas } A^- &::= P^- \Sep A^+ \Lolli A^- \Sep A^- ~ \With ~A^- \Sep \Top \Sep \Up A^+
\end{align*}

\noindent The translation $A^-$ of an (unpolarized) formula $F$ in IPL
is nondeterministic, subject only to the constraint that
the translation $|A^-| = F$.

\medskip
\begin{tabular}{c@{\hspace{15pt}}c@{\hspace{15pt}}c}
$|A^+ \Sum B^+| = |A^+| \Or |B^+|$ & $|\Zero| = \Bot$ & $|P^+| = P$ \\
$|A^+ \Tensor B^+| = |A^+| \And |B^+|$ & $|\One| = \top$ & $|\Down A^-| = |A^-|$ \\
$|A^- \With B^-| = |A^-| \And |B^-|$ & $|\Top| = \top$ & $|P^-| = P$ \\
$|A^+ \Lolli B^-| = |A^+| \Imp |B^-|$ & $|\Up A^+| = |A^+|$
\end{tabular}
\medskip

For example, the formula $((A \Or C) \And (B \Imp C)) \Imp (A \Imp B) \Imp C$
can be interpreted as any of the following polarized formulas (among others):

{\small
\begin{align*}
&((\Down A^- \Sum \Down C^-) \Tensor \Down(\Down B^- \Lolli C^-)) \Lolli (\Down (\Down A^- \Lolli B^-) \Lolli C^-) \\
&\Down\Up((\Down A^- \Sum \Down C^-) \Tensor \Down(\Down B^- \Lolli C^-)) \Lolli (\Down\Up\Down (\Down A^- \Lolli B^-) \Lolli C^-) \\
&\Down(\Up (A^+ \Sum C^+) \With (B^+ \Lolli \Up C^+)) \Lolli (\Down (A^+ \Lolli \Up B^+) \Lolli \Up C^+)
\end{align*}
}

Shift operators have highest binding precedence in our presentation of the
examples.  As we will see, the choice of translation determines the search
behavior on the resulting polarized formula.  Different choices can lead to
search spaces with radically different structure~\cite{Chaudhuri.2006.IJCAR}.

% A sequent of intuitionistic logic has the form $\Stable{\Gamma}{A}$,
% where $\Gamma$ is a set or multiset of formulas.  For
% purposes of Imogen it is convenient to always maintain $\Gamma$
% as a set, without duplicates.
% Since we can always eagerly decompose negative connectives
% on the right of a sequent and positive connectives on the left,
% the only sequents in our polarized calculus we need to consider
% have negative formulas on the left or positive formulas on the
% right, in addition to atoms which can appear with either polarity
% on either side.  The right-hand side could also be empty if we are
% deriving a contradiction.   We call such sequents \emph{stable}.
% \[\begin{array}{llcl}
% \mbox{Stable Hypotheses} & \Gamma &::= & \cdot \Sep \Gamma, A^- \Sep \Gamma, P^+ \\
% \mbox{Stable Conclusions} & \gamma &::= & A^+ \Sep P^- \Sep \cdot \\
% \mbox{Stable Sequents} & \multicolumn{3}{l}{\Stable{\Gamma}{\gamma}}
% \end{array}\]

\subsection{Derived rules}
\label{prop.sec.derived}

Thm: ===> A iff for any A', if |A'| = A then ===> A'

\section{The Polarized Inverse Method}\label{prop.sec.pinverse}

%%% Local Variables:
%%% mode: latex
%%% TeX-master: "thesis"
%%% End:
