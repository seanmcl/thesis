
\chapter{Modal Logic}\label{chapter.modal}


A logic is called \emph{modal} if it augments the usual connectives
of classical and intuitionistic logic with connectives intended to capture
\emph{modes} of reasoning.  For example, rather than asserting that proposition
$A$ is \emph{categorically} true, one might assert that $A$ is \emph{usually}
true, \emph{sometimes} true, \emph{necessarily} true,
someone knows or believes that $A$ is true, $A$ is true \emph{after} some
point in time, or after a program has terminated.
Modal logics are widely used in computer science.  Classical
temporal logic, a modal logic capturing a notion of time, is the foundation of
model checking, an important technique for hardware and software
verification~\cite{Clarke.2001.Handbook}.  It has been used to model distributed
computation~\cite{Murphy.2004.LICS}, and
authorization~\cite{Gurevich.2009.DKAL2,Gurevich.2008.DKAL,Garg.2009.Thesis}.
Dynamic logic and Hennessy-Milner logic are other examples of modal logics with
applications to computer science.

In this chapter we will focus on normal intuitionistic modal logics (IML).  IML
extends intuitionistic logic with the connectives $\Box$ (necessity) and $\Dia$
(possibility).  We follow Simpson~\cite{Simpson.1994.Thesis} in defining the
semantics of IML using a Kripke semantics.  Propositions are true in the
context of a \emph{world}.  An \emph{accessibility relation} $\cR$ oversees the
possibility of navigating between the different worlds.  If $w$ and $w'$ are
worlds and $w\cR w'$, we say $w'$ is \emph{accessible} or \emph{visible} from
$w$.

\section{Syntax and Semantics}
\label{modal.sec.def}

\begin{definition}
  \begin{itemize}
  \item[]
  \item Formulas are built from first-order formulas with the additional
    branches
    \[
    \mbox{Formulas } A ::= \ldots \Sep \Box A \Sep \Dia A
    \]
  \item Let $\UWorld$ be an infinite set of symbols, called \emph{worlds}.
  \item A \emph{labeled formula} is a pair $\ASet{A,w}$ written $\MAt{A}{w}$.
  \item A \emph{modal sequent} is a tuple $\ASet{\Delta,\Psi,\Gamma,\gamma}$
    written $\MSeq{\Psi}{\Gamma}{\gamma}$ where $\Delta$ is a typing context,
    $\Psi$ is a set of

  \end{itemize}
\end{definition}


\newcommand{\MNBoxI}{
  \Infer[\Box-$I$^a]
  {\MNd{\Gamma}{\MLam l:\MLe{w}{a}.~M : \MAt{\Box A}{w}}}
  {\MNd[\Delta,a:\MWorld]{\Gamma, l : \MLe{w}{a}}{M : \MAt{A}{a}}}
}

\newcommand{\MNBoxE}{
  \Infer[\Box-$E$]
  {\MNd{\Gamma}{M~[\MLe{w}{w'}]} : \MAt{A}{w'}}
  {
    \MNd{\Gamma}{M : \MAt{\Box A}{w}}
    &
    \Delta~;~\Gamma\Entails\MLe{w}{w'}
  }
}

\newcommand{\MNDiaI}{
  \Infer[\Dia-$I$]
  {\MNd{\Gamma}{\ASet{\MLe{w}{w'}} M} : \MAt{\Dia A}{w}}
  {
    \MNd{\Gamma}{M : \MAt{A}{w'}}
    &
    \Delta~;~\Gamma\Entails\MLe{w}{w'}
  }
}

\newcommand{\MNDiaE}{
  \Infer[\Dia-$E$^a]
  {\MNd{\Gamma}{\Let\ASet{l:\MLe{w'}{a}}x = M~\In~N : \MAt{C}{w}}}
  {
    \MNd{\Gamma}{M : \MAt{\Dia A}{w'}}
    &
    \MNd[\Delta,a:\MWorld]{\Gamma,l:\MLe{w'}{a},x:\MAt{A}{a}}{N : \MAt{C}{w}}
  }
}

\begin{figure}[H]
  \begin{center}
    {\footnotesize
      \fbox{
        \begin{tabular}{c}
          \begin{tabular}{cc}
            \MNBoxI
            &
            \hspace{2em}
            \MNBoxE
          \end{tabular}
          \\[2em]
          \begin{tabular}{c}
            \MNDiaI
          \end{tabular}
          \\[2em]
          \begin{tabular}{c}
            \MNDiaE
          \end{tabular}
        \end{tabular}
      }
    }
  \end{center}
  \caption{Natural deduction.}
  \label{modal.nd}
\end{figure}


%%% Local Variables:
%%% mode: latex
%%% TeX-master: "../thesis"
%%% End:


\newcommand{\PBInit}{\Infer[$Init$]{\PBSeq{p}{p}}{}}

\newcommand{\PBAndR}{
  \Infer[\And$-R$]
  {\PBSeq{\cdot}{A_1 \And A_2}}
  {\PBSeq{\cdot}{A_1} & \PBSeq{\cdot}{A_2}}
}

\newcommand{\PBAndL}{
  \Infer[\And$-L$]
  {\PBSeq{A_1 \And A_2}{\cdot}}
  {\PBSeq{A_1, A_2}{\cdot}}
}

\newcommand{\PBImpR}{
  \Infer[\Imp$-R$]
  {\PBSeq{\cdot}{A_1 \Imp A_2}}
  {\PBSeq{A_1}{A_2}}
}

\newcommand{\PBImpL}{
  \Infer[\Imp$-L$]
  {\PBSeq{A_1\Imp A_2}{\cdot}}
  {\PBSeq{A_1\Imp A_2}{A_1} & \PBSeq{A_2}{\cdot}}
}

\newcommand{\PBOrRL}{
  \Infer[\Or$-R$_1]
  {\PBSeq{\cdot}{A_1 \Or A_2}}
  {\PBSeq{\cdot}{A_1}}
}

\newcommand{\PBOrRR}{
  \infer[\Or$-R$_2]
  {\PBSeq{\cdot}{A_1 \Or A_2}}
  {\PBSeq{\cdot}{A_2}}
}

\newcommand{\PBOrL}{
  \infer[\Or$-L$]
  {\PBSeq{A_1 \Or A_2}{\cdot}}
  {\PBSeq{A_1}{\cdot} & \PBSeq{\Gamma, A_2}{\cdot}}
}

\newcommand{\PBTopR}{
  \infer[\Top$-R$]
  {\PBSeq{\cdot}{\Top}}
  {}
}

\newcommand{\PBTopL}{\mbox{No rule for $\Top$-L}}

\newcommand{\PBBotL}{
  \infer[\Bot$-L$]
  {\PBSeq{\Bot}{\cdot}}
  {}
}

\newcommand{\PBBotR}{\mbox{No rule for $\Bot$-R}}

\begin{figure}
  \begin{center}
    {\small
      \fbox{
        \begin{tabular}{c}
          \begin{tabular}{ccc}
            \PBInit
            &
            \hspace{1cm}
            \PBTopR
            &
            \hspace{1cm}
            \PBTopL
          \end{tabular}
          \\[2em]
          \begin{tabular}{cc}
            \PBAndR
            &
            \hspace{1cm}
            \PBAndL
          \end{tabular}
          \\[2em]
          \begin{tabular}{ccc}
            \PBOrRL
            &
            \hspace{10pt}
            \PBOrRR
            &
            \hspace{10pt}
            \PBOrL
          \end{tabular}
          \\[2em]
          \begin{tabular}{cc}
            \PBBotR
            &
            \hspace{20pt}
            \PBBotL
          \end{tabular}
          \\[2em]
          \begin{tabular}{cc}
            \PBImpR
            &
            \hspace{20pt}
            \PBImpL
          \end{tabular}
        \end{tabular}
      }
    }
  \end{center}
  \caption{The backward calculus $\PB$}
  \label{prop.backward}
\end{figure}

%%% Local Variables:
%%% mode: latex
%%% TeX-master: "../thesis"
%%% End:


\begin{example}

  \renewcommand{\Sa}{\MSeq{\MAt{p}{w_2}}{\MAt{p}{w_2}}}
  \renewcommand{\Qa}{\Infer[$Init$]{\Sa}{}}

  \renewcommand{\Sb}{\MSeq{\Psi,\MAt{\Box p}{w_0}}{\MAt{p}{w_2}}}
  \renewcommand{\Qb}{\Infer[\Box$-L$]{\Sb}{\Qa & \Delta~;~\Psi\Entails \MLe{w_0}{w_2}}}

  \renewcommand{\Sc}{\MSeq[w_1:\MWorld]{\MLe{w_0}{w_1},\MAt{\Box p}{w_0}}{\MAt{\Box p}{w_1}}}
  \renewcommand{\Qc}{\Infer[\Box$-R$^{w_2}]{\Sc}{\Qb}}

  \renewcommand{\Sd}{\MSeq[\cdot]{\MAt{\Box p}{w_0}}{\MAt{\Box\Box p}{w_0}}}
  \renewcommand{\Qd}{\Infer[\Box$-R$^{w_1}]{\Sd}{\Qc}}

  \renewcommand{\Se}{\MSeq[\cdot]{\cdot}{\MAt{\Box p\Imp \Box\Box p}{w_0}}}
  \renewcommand{\Qe}{\Infer[\Imp$-R$]{\Se}{\Qd}}

  \[
  \begin{array}{c}
    \begin{array}{rl}
      \Delta&=w_1:\MWorld,w_2:\MWorld\\
      \Psi&=\MLe{w_0}{w_1}\And\MLe{w_1}{w_2}\\
    \end{array}
    \\[2em]
    \Qe
  \end{array}
  \]
\end{example}

\section{Encoding IML in $\CB$}

While it is possible to design a focused inverse method theorem prover based
on $\MB$, it is not necessary.
We can define an adequate translation of IML formulas to $\CB$ formulas,
where there is a bijection between focused proofs in $\MB$ and focused
proofs of the translated formulas.  It is also possible to recover single-step
$\MB$ proofs using a translation that adds more shifts.

\subsection{Translating world formulas}

Recall from Section~\ref{fol.sec.label}
that $\Free(A)$ is an ordered list of the free variables and parameters
occurring in $A$, and that
We extend the $\Free$ function to work on world
formulas by reserving the first position for
the world $w$.  In particular, (abusing notation slightly)
$\Free(\MAt{A}{w}) = \ASet{w, \Free(A)}$.  For example,
$\Free(\MAt{p(x, y) \And q(a)}{w})=\ASet{w,x,y,a}$.  Using the labeling
function of Section~\ref{fol.sec.label} (extended to generate labels for
the modal operators), define
the translation function $\MTrans{\MAt{A}{w}} = l_A(\Free(\MAt{A}{w}))$.
Thus $\Arity(\MAt{A}{w})=\Arity(A)+1$.
\ednote{Make sure you define $\Free$ when you talk about labeling in FOL.}

\[
\MTrans{\MAt{C}{w}} =
\begin{cases}
  C=\Box A & \All w':\MWorld.~\MLe{w}{w'}\Imp l_A(w', \Free(A)) \\
  C=\Dia A & \Ex w':\MWorld.~\MLe{w}{w'}\And l_A(w', \Free(A)) \\
  \mbox{otherwise} & l_{C}(w, \Free(C))
\end{cases}
\]




% \newcommand{\MTInit}{
%   \Infer[$Init$]
%   {\MSeq{\Gamma,\MAt{p}{w}}{\MAt{p}{w}}}
%   {\Delta\Entails \Gamma,\MAt{p}{w}~\Ok}
% }


% \newcommand{\MTInit}{
%   \Infer[$Init$]
%   {\MSeq{\Gamma,\MAt{p}{w}}{\MAt{p}{w}}}
%   {\Delta\Entails \Gamma,\MAt{p}{w}~\Ok}
% }

% \begin{figure}[H]
%   \begin{center}
%     {\footnotesize
%       \fbox{
%         \begin{tabular}{c}
%           \begin{tabular}{cc}
%             \MNBoxI
%             &
%             \hspace{2em}
%             \MNBoxE
%           \end{tabular}
%           \\[2em]
%           \begin{tabular}{c}
%             \MNDiaI
%           \end{tabular}
%           \\[2em]
%           \begin{tabular}{c}
%             \MNDiaE
%           \end{tabular}
%         \end{tabular}
%       }
%     }
%   \end{center}
%   \caption{Translation.}
%   \label{modal.trans}
% \end{figure}


%%% Local Variables:
%%% mode: latex
%%% TeX-master: "../thesis"
%%% End:


\renewcommand{\Si}{\CSeq{\Psi}{\MAt{\Box p}{w_0},\MLe{w_0}{w_2}\Imp p(w_2)}{\MLe{w_0}{w_2}}}
\renewcommand{\Qi}{\Infer{\Si}{\Psi\Entails\MLe{w_0}{w_2}}}

\renewcommand{\Sh}{\CSeq{\Psi}{\MAt{\Box p}{w_0}, p(w_2)}{p(w_2)}}
\renewcommand{\Qh}{\Infer{\Sh}{}}% \Qi & \Qj}}

\renewcommand{\Sg}{\CSeq{\Psi}{\MAt{\Box p}{w_0},\MLe{w_0}{w_2}\Imp p(w_2)}{p(w_2)}}
\renewcommand{\Qg}{\Infer[\Imp$-L$]{\Sg}{\Qh & \Qi}}

\renewcommand{\Sf}{\CSeq{\Psi}{\MAt{\Box p}{w_0}}{p(w_2)}}
\renewcommand{\Qf}{\Infer[\All$-L$]{\Sf}{\Qg & \Delta\Entails w_2:\MWorld}}

\renewcommand{\Se}{\CSeq[w_1:\MWorld]{\MLe{w_0}{w_1}}{\MAt{\Box p}{w_0}}{\MAt{\Box p}{w_1}}}
\renewcommand{\Qe}{\Infere{\Se}{\Qf}}

\renewcommand{\Sd}{\CSeq[w_1:\MWorld]{\cdot}{\MLe{w_0}{w_1},\MAt{\Box p}{w_0}}{\MAt{\Box p}{w_1}}}
\renewcommand{\Qd}{\Infer[$E-L$]{\Sd}{\Qe}}

\renewcommand{\Sc}{\CSeq[w_1:\MWorld]{\cdot}{\MAt{\Box p}{w_0}}{\MLe{w_0}{w_1}\Imp\MAt{\Box p}{w_1}}}
\renewcommand{\Qc}{\Infer[\Imp$-R$^{w_1}]{\Sc}{\Qd}}

\renewcommand{\Sb}{\CSeq[\cdot]{\cdot}{\MAt{\Box p}{w_0}}{\MAt{\Box\Box p}{w_0}}}
\renewcommand{\Qb}{\Infer[\All$-R$]{\Sb}{\Qc}}

\renewcommand{\Sa}{\CSeq[\cdot]{\cdot}{\cdot}{\MAt{\Box p\Imp \Box\Box p}{w_0}}}
\renewcommand{\Qa}{\Infer[\Imp$-R$]{\Sa}{\Qb}}

% \begin{figure}
%   \begin{center}
% %    {\footnotesize
%       \fbox{
%         \begin{array}{c}
%           \begin{array}{rl}
%             \Delta&=w_1:\MWorld,w_2:\MWorld\\
%             \Psi&=\MLe{w_0}{w_1}\And\MLe{w_1}{w_2}\\
%           \end{array}
%           \\[2em]
%           \Qa
%         \end{array}
%       }
% %    }
%   \end{center}
%   \caption{Translation example.}
%   \label{modal.trans-ex}
% \end{figure}

%%% Local Variables:
%%% mode: latex
%%% TeX-master: "../thesis"
%%% End:


\subsection{Multi-modal IML}

\section{Related Work}

\cite{Alechina.2006.JAL}
\cite{Amati.1994.SL}
\cite{Pfenning.2010.ModalLogic}

\section{XXX Notes}

\begin{itemize}
\item Adequacy wrt focused proofs
\item Simulate single-step Pfenning modal logic with shifts.
\item Change translation to get non-constant domain logics.
  For example by changing the sort of the translation of box to
  $|\Box A @ w| = \All w':wld.~w\le w'\Imp |A|_w'$ where all quantifiers
  in $A$ are translated in the sort $\sigma_{w'}$.
\item Pfenning/Davies via translation
\item Lax logic via (double) translation
\item Write down focused Pfenning-Davies.  Don't know if this has been done.
\end{itemize}

%%% Local Variables:
%%% mode: latex
%%% TeX-master: "thesis"
%%% End:
