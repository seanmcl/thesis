
From a proof search perspective, $\LJ$ is a \emph{backward calculus};
the rules are intended to be read from bottom to top.
This is evident
in the rules $\And$-R where the same meta-variable
$\Gamma$ occurs in both premises of the rule, and in Init where
$\Gamma$ could be anything.  (In a search from the leaves, how could you
instantiate $\Gamma$?)
A calculus corresponding to $\LJ$ oriented toward forward search is given in
Figure~\ref{prop.ljf}, and an example derivation in
Figure~\ref{prop.forward-der}.  (Note that in the rule $\Or$-L,
we must have $\Card{\gamma_1\Union\gamma_2} \leq 1$ to obey the
definiton of sequents.)
Relating the forward and backward system requires the following
important way to compare the relative strength of sequents.


\newcommand{\LJFInit}{
  \Infer[$Init$]{\PFSeq{p}{p}}{}
}

\newcommand{\LJFAndR}{
  \Infer[\And$-R$]
  {\PFSeq{\Gamma_1,\Gamma_2}{A_1 \And A_2}}
  {\PFSeq{\Gamma_1}{A_1} & \PFSeq{\Gamma_2}{A_2}}
}

\newcommand{\LJFAndLa}{
  \Infer[\And$-L$_1]
  {\PFSeq{\Gamma,A_1 \And A_2}{\gamma}}
  {\PFSeq{\Gamma,A_1}{\gamma}}
}

\newcommand{\LJFAndLb}{
  \Infer[\And$-L$_2]
  {\PFSeq{\Gamma,A_1 \And A_2}{\gamma}}
  {\PFSeq{\Gamma,A_2}{\gamma}}
}

\newcommand{\LJFImpRa}{
  \Infer[\Imp$-R$_1]
  {\PFSeq{\Gamma}{A_1 \Imp A_2}}
  {\PFSeq{\Gamma,A_1}{A_2}}
}

\newcommand{\LJFImpRb}{
  \Infer[\Imp$-R$_2]
  {\PFSeq{\Gamma}{A_1 \Imp A_2}}
  {\PFSeq{\Gamma}{A_2}}
}

\newcommand{\LJFImpRc}{
  \Infer[\Imp$-R$_3]
  {\PFSeq{\Gamma}{A_1 \Imp A_2}}
  {\PFSeq{\Gamma,A_1}{\cdot}}
}

\newcommand{\LJFImpL}{
  \Infer[\Imp$-L$]
  {\PFSeq{\Gamma_1,\Gamma_2, A_1\Imp A_2}{\gamma}}
  {\PFSeq{\Gamma_1}{A_1} & \PFSeq{\Gamma_2, A_2}{\gamma}}
}

\newcommand{\LJFOrRa}{
  \Infer[\Or$-R$_1]
  {\PFSeq{\Gamma}{A_1 \Or A_2}}
  {\PFSeq{\Gamma}{A_1}}
}

\newcommand{\LJFOrRb}{
  \Infer[\Or$-R$_2]
  {\PFSeq{\Gamma}{A_1 \Or A_2}}
  {\PFSeq{\Gamma}{A_2}}
}

\newcommand{\LJFOrL}{
  \Infer[\Or$-L$]
  {\PFSeq{\Gamma_1, \Gamma_2, A_1 \Or A_2}{\gamma_1\Union\gamma_2}}
  {\PFSeq{\Gamma_1, A_1}{\gamma_1} & \PFSeq{\Gamma_2, A_2}{\gamma_2}}
}

\newcommand{\LJFTopR}{
  \Infer[\Top$-R$]
  {\PFSeq{\cdot}{\Top}}
  {}
}

\newcommand{\LJFBotL}{
  \Infer[\Bot$-L$]
  {\PFSeq{\Bot}{\cdot}}
  {}
}

\newcommand{\LJFContract}{
  \Infer[$Contract$]
  {\PFSeq{\Gamma, A}{\gamma}}
  {\PFSeq{\Gamma, A, A}{\gamma}}
}

\begin{figure}
  \begin{center}
    {\small
      \fbox{
        \begin{tabular}{c}
          \begin{tabular}{cccc}
            \LJFInit
            &
            \hspace{2em}
            \LJFTopR
            &
            \hspace{2em}
            \LJFBotL
            &
            \hspace{2em}
            No rule for $\Top$-L
          \end{tabular}
          \\[2em]
          \begin{tabular}{ccc}
            \LJFAndR
            &
            \hspace{2em}
            \LJFAndLa
            &
            \hspace{2em}
            \LJFAndLb
          \end{tabular}
          \\[2em]
          \begin{tabular}{ccc}
            \LJFImpL
            &
            \hspace{2em}
            \LJFContract
            &
            \hspace{2em}
            No rule for $\Bot$-R
          \end{tabular}
          \\[2em]
          \begin{tabular}{ccc}
            \LJFImpRa
            &
            \hspace{2em}
            \LJFImpRb
            &
            \hspace{2em}
            \LJFImpRc
          \end{tabular}
          \\[2em]
          \begin{tabular}{ccc}
            \LJFOrRa
            &
            \hspace{2em}
            \LJFOrRb
            &
            \hspace{2em}
            \LJFOrL
          \end{tabular}
        \end{tabular}
      }
    }
  \end{center}
  \caption{The forward calculus $\LJF$}
  \label{prop.ljf}
\end{figure}

%%% Local Variables:
%%% mode: latex
%%% TeX-master: "../thesis"
%%% End:


\renewcommand{\ab}{a \Imp b}
\renewcommand{\ac}{a \Or c}
\renewcommand{\abc}{(a\Imp b) \Imp c}
\renewcommand{\bb}{\Bot \Or b}

\renewcommand{\Qa}{\PFSeq{b}{b}}
\renewcommand{\Ra}{\Infer[$Init$]{\Qa}{}}

\renewcommand{\Qb}{\PFSeq{b}{\ab}}
\renewcommand{\Rb}{\Infer[\Imp$-R$]{\Qb}{\Ra}}

\renewcommand{\Qc}{\PFSeq{c}{c}}
\renewcommand{\Rc}{\Infer[$Init$]{\Qc}{}}

\renewcommand{\Qd}{\PFSeq{c}{\ac}}
\renewcommand{\Rd}{\Infer[\Or$-R$_2]{\Qd}{\Rc}}

\renewcommand{\Qe}{\PFSeq{b, \abc}{\ac}}
\renewcommand{\Re}{\Infer[\Imp$-L$]{\Qe}{\Rd & \Rb}}

\renewcommand{\Qf}{\PFSeq{\Bot}{\cdot}}
\renewcommand{\Rf}{\Infer[\Bot$-L$]{\Qf}{}}

\renewcommand{\Qg}{\PFSeq{\bb, \abc}{\ac}}
\renewcommand{\Rg}{\Infer[\Or$-L$]{\Qg}{\Rf & \Re}}

\renewcommand{\Qh}{\PFSeq{\bb}{(\abc) \Imp \ac}}
\renewcommand{\Rh}{\Infer[\Imp$-R$]{\Qh}{\Rg}}

\renewcommand{\Qj}{\PFSeq{\cdot}{\bb \Imp (\abc) \Imp \ac}}
\renewcommand{\Rj}{\Infer[\Imp$-R$]{\Qj}{\Rh}}

\begin{figure}
  \begin{center}
    \fbox{
      \Rj
    }
  \end{center}
  \caption{A derivation in $\LJF$}
  \label{prop.forward-der}
\end{figure}


%%% Local Variables:
%%% mode: latex
%%% TeX-master: "../thesis"
%%% End:


\begin{definition}[Subsumption]
A sequent $\PBSeq{\Gamma}{\gamma}$ \emph{subsumes}
$\PBSeq{\Gamma'}{\gamma'}$ if $\Gamma\subseteq\Gamma'$ and
$\gamma\subseteq\gamma'$.  We write
$\PBSeq{\Gamma}{\gamma}\Subsumes\PBSeq{\Gamma'}{\gamma'}$.
\end{definition}

\noindent
To see why such a measure is necessary, consider the sequent
$\PBSeq{p, q}{p}$, initial in $\LJ$.  This sequent is not provable
in $\LJF$.  Instead, we can prove (trivially) $\PFSeq{p}{p}$ which
is stronger, in that it uses fewer assumptions to show the same
consequent.

\begin{theorem}
If $q$ is derivable in $\LJ$ and $q\Subsumes q'$ then $q'$ is derivable
in $\LJ$.
\end{theorem}

\begin{proof}
  Easy induction on the derivation of $q$.  We can add any extra hypotheses
in $q'$ to every sequent in the derivation.
\end{proof}

\begin{theorem}
  \begin{itemize}
  \item []
  \item If $q$ is derivable in $\LJF$ then $q$ is derivable in $\LJ$.
  \item If $q$ is derivable in $\LJ$ then there exists $q'$ such
    that $q'\Subsumes q$ and $q'$ is derivable in $\LJF$.
  \end{itemize}
\end{theorem}

\begin{proof}
  \cite{Voronkov.2001.Handbook}
\end{proof}

The differences between $\LJ$ and $\LJF$ are mostly cosmetic.  The main
substantial difference is the existence of the contraction rule.
This rule is admissible in $\LJ$ but is needed as an explicit rule in
$\LJF$.  Also, the single rule $\Imp$-R of $\LJ$ has three versions
in $\LJF$, corresponding to whether either of the sides of the implication
are present.  (For example, you couldn't prove $a\Imp b\Imp a$ with only
$\Imp$-R$_1$.)  We can reconcile the differences by defining a single
inference system, where the \emph{matching} of sequents to inference
rule schemas differs between the forward and backward variants.  The
reasons we want to do this will not become clear until later in the
chapter, when we discuss the focused calculus.  In anticipation of
that section, having a single inference system that is interpreted in two
different ways will make the proofs that follow more straightforward.

A combined calculus is shown in Figure~\ref{prop.p}.  It is interpreted
in the context of forward and backward search using forward
and backward \emph{systems}.


\newcommand{\PInit}{\Infer[$Init$]{\PFSeq{p}{p}}{}}

\newcommand{\PAndR}{
  \Infer[\And$-R$]
  {\PFSeq{\cdot}{A_1 \And A_2}}
  {\PFSeq{\cdot}{A_1} & \PFSeq{\cdot}{A_2}}
}

\newcommand{\PAndL}{
  \Infer[\And$-L$]
  {\PFSeq{A_1 \And A_2}{\cdot}}
  {\PFSeq{A_1, A_2}{\cdot}}
}

\newcommand{\PImpR}{
  \Infer[\Imp$-R$]
  {\PFSeq{\cdot}{A_1 \Imp A_2}}
  {\PFSeq{A_1}{A_2}}
}

\newcommand{\PImpL}{
  \Infer[\Imp$-L$]
  {\PFSeq{A_1\Imp A_2}{\cdot}}
  {\PFSeq{A_1\Imp A_2}{A_1} & \PFSeq{A_2}{\cdot}}
}

\newcommand{\POrRL}{
  \Infer[\Or$-R$_1]
  {\PFSeq{\cdot}{A_1 \Or A_2}}
  {\PFSeq{\cdot}{A_1}}
}

\newcommand{\POrRR}{
  \infer[\Or$-R$_2]
  {\PFSeq{\cdot}{A_1 \Or A_2}}
  {\PFSeq{\cdot}{A_2}}
}

\newcommand{\POrL}{
  \infer[\Or$-L$]
  {\PFSeq{A_1 \Or A_2}{\cdot}}
  {\PFSeq{A_1}{\cdot} & \PFSeq{\Gamma, A_2}{\cdot}}
}

\newcommand{\PTopR}{
  \infer[\Top$-R$]
  {\PFSeq{\cdot}{\Top}}
  {}
}

\newcommand{\PTopL}{\mbox{No rule for $\Top$-L}}

\newcommand{\PBotL}{
  \infer[\Bot$-L$]
  {\PFSeq{\Bot}{\cdot}}
  {}
}

\newcommand{\PBotR}{\mbox{No rule for $\Bot$-R}}

\begin{figure}
  \begin{center}
    {\small
      \fbox{
        \begin{tabular}{c}
          \begin{tabular}{ccc}
            \PInit
            &
            \hspace{1cm}
            \PTopR
            &
            \hspace{1cm}
            \PTopL
          \end{tabular}
          \\[2em]
          \begin{tabular}{cc}
            \PAndR
            &
            \hspace{1cm}
            \PAndL
          \end{tabular}
          \\[2em]
          \begin{tabular}{ccc}
            \POrRL
            &
            \hspace{10pt}
            \POrRR
            &
            \hspace{10pt}
            \POrL
          \end{tabular}
          \\[2em]
          \begin{tabular}{cc}
            \PBotR
            &
            \hspace{20pt}
            \PBotL
          \end{tabular}
          \\[2em]
          \begin{tabular}{cc}
            \PImpR
            &
            \hspace{20pt}
            \PImpL
          \end{tabular}
        \end{tabular}
      }
    }
  \end{center}
  \caption{The combined calculus $\LP$}
  \label{prop.p}
\end{figure}

%%% Local Variables:
%%% mode: latex
%%% TeX-master: "../thesis"
%%% End:


\begin{definition}[Backward systems]
  \begin{itemize}
  \item[]
  \item A \emph{backward system} is a pair $\cB=\ASet{\cR, \BMatch}$ where
    $\cR$ is a set of inference rules and
    $\BMatch\in\cR\times \Seqs\to\SeqSeqs$.
    If $\BMatch(\rho, q)$ is defined, we say that $q$ \emph{matches}
    $\rho$.
  \item If $q$ is a sequent, a \emph{$\cB$-derivation} of $q$
    is a tree of sequents with root $q$ where the children of
    any node $q'$ are the elements of $\BMatch(\rho, q')$ for some
    rule $\rho\in\cR$.
  \item A sequent $q$ is called \emph{initial} if there exists
    an initial rule $\rho\in\cR$ such that $q$ matches $\rho$.
  \item An \emph{$\cB$-proof} is an $\cB$-derivation
    where the leaves are initial sequents.
  \end{itemize}
\end{definition}

\begin{definition}[Forward systems]
  \begin{itemize}
  \item[]
  \item A \emph{forward system} is a pair $\cF=\ASet{\cR, \FMatch}$ where
    $\cR$ is a set of inference rules and
    $\FMatch\in\cR\times \SeqSeqs\to\SetSeqs$.
    (Matching sequents to a rule in forward search may produce multiple
    results.)
  \item If $q$ is a sequent, an \emph{$\cF$-derivation} of $q$
    is a tree of sequents
    with root $q$ where if $\cQ$ are the parents of a
    node $q'$, then $q'\in \FMatch(\rho, \cQ)$ for some
    rule $\rho\in\cR$.
  \item An \emph{$\cF$-proof} is an $\cF$-derivation where the
    leaves are the conclusions of initial rules.
  \end{itemize}
\end{definition}

\paragraph{Notation.}
  We systematically use the following (possibly subscripted)
  symbols to represent the different syntactic entities.
  \begin{center}
    \begin{tabular}{l@{\hspace{3em}}l}
      $p$ & atomic formulas\\
      $A,B$ & formulas\\
      $\Gamma$ & antecedents of rule sequents\\
      $\gamma$ & consequents of rule sequents\\
      $\Delta$ & antecedents of non-rule sequents\\
      $\delta$ & consequents of non-rule sequents\\
      $\rho$ & inference rules\\
      $q$ & sequents\\
      $\cD$ & sets of sequents\\
      $\cQ$ & sequences of sequents\\
      $\cR$ & sets of inference rules \\
    \end{tabular}
  \end{center}
  We sometimes abuse notation and write, e.g. $\cQ \subseteq \cD$.

\begin{definition}[Backward rule matching]
  \label{prop.def.backward-match}
  Let $q=\PBSeq{\Delta}{\delta}$ be a sequent and
  \[
  \rho=
  \begin{array}{c}
    \infer
    {\PBSeq{\Gamma}{\gamma}}
    {\PBSeq{\Gamma_1}{\gamma_1} & \cdots & \PBSeq{\Gamma_n}{\gamma_n}}
  \end{array}
  \]

  \noindent
  be a rule.  $\BMatch(\rho, q)$ is defined as follows:

  \begin{align*}
    \Delta' &= \Delta\Diff\Gamma\\
    \delta' &= \delta\Diff\gamma\\
    \Gamma_i' &= \Gamma_i\Union\Delta'\\
    \gamma_i' &= \gamma_i\Union\delta'\\
    \BMatch(\rho, q) &=
    \ASet{
      \PBSeq{\Delta_1'}{\delta_1'},
      \ldots,
      \PBSeq{\Delta_n'}{\delta_n'}}
  \end{align*}

  \noindent
  subject to the constraint that each $\delta_i$
  has at most one formula.  Otherwise $\BMatch(\rho,q)$ is not
  defined.
\end{definition}

\begin{definition}
  A \emph{rule schema} is an instance of a rule with
  $\PB$ is the backward system consiting of of all instances of
  the rule schemas of Figure~\ref{prop.backward} and the
  matching function $\BMatch$ defined above.
  A backward derivation is shown in Figure~\ref{prop.backward-der}.
\end{definition}

\noindent

We are being intentionally pedantic in our distinction between inference rules and
rule schemas because our presentation is somewhat unusual.  Normally rule schemas are
presented with meta-variables standing for the non-prinicipal formulas.  For
example, the rule $\And$-L is normally written
\[
\Infer[\And$-L$]
{\PBSeq{\Gamma, A_1 \And A_2}{C}}
{\PBSeq{\Gamma, A_1, A_2}{C}}
\]
\noindent
where the symbols $\Gamma$ and $C$ are meta-variables
With this formulation, the following is considered to be an
instance of the schema
\[
\Infer[\And$-L$]
{\PBSeq{b, c \And d}{a}}
{\PBSeq{b, c, d}{a}}
\]
\noindent
This is not the case in our presentation.  The instance
of the $\And$-T schema for  $c\And d$ is
\[
\Infer[\And$-L$]
{\PBSeq{c \And d}{\cdot}}
{\PBSeq{c, d}{\cdot}}
\]
\noindent
The definition of \textit{$\BMatch$} makes
\[
\Infer[\And$-L$]
{\PBSeq{b, c \And d}{a}}
{\PBSeq{b, c, d}{a}}
\]
\noindent
a valid \emph{derivation}, but not an instance of the rule.
This distinction will be
important when we create focused inference rules in
Section~\ref{prop.sec.focus}.

% 
\newcommand{\PBInit}{\Infer[$Init$]{\PBSeq{p}{p}}{}}

\newcommand{\PBAndR}{
  \Infer[\And$-R$]
  {\PBSeq{\cdot}{A_1 \And A_2}}
  {\PBSeq{\cdot}{A_1} & \PBSeq{\cdot}{A_2}}
}

\newcommand{\PBAndL}{
  \Infer[\And$-L$]
  {\PBSeq{A_1 \And A_2}{\cdot}}
  {\PBSeq{A_1, A_2}{\cdot}}
}

\newcommand{\PBImpR}{
  \Infer[\Imp$-R$]
  {\PBSeq{\cdot}{A_1 \Imp A_2}}
  {\PBSeq{A_1}{A_2}}
}

\newcommand{\PBImpL}{
  \Infer[\Imp$-L$]
  {\PBSeq{A_1\Imp A_2}{\cdot}}
  {\PBSeq{A_1\Imp A_2}{A_1} & \PBSeq{A_2}{\cdot}}
}

\newcommand{\PBOrRL}{
  \Infer[\Or$-R$_1]
  {\PBSeq{\cdot}{A_1 \Or A_2}}
  {\PBSeq{\cdot}{A_1}}
}

\newcommand{\PBOrRR}{
  \infer[\Or$-R$_2]
  {\PBSeq{\cdot}{A_1 \Or A_2}}
  {\PBSeq{\cdot}{A_2}}
}

\newcommand{\PBOrL}{
  \infer[\Or$-L$]
  {\PBSeq{A_1 \Or A_2}{\cdot}}
  {\PBSeq{A_1}{\cdot} & \PBSeq{\Gamma, A_2}{\cdot}}
}

\newcommand{\PBTopR}{
  \infer[\Top$-R$]
  {\PBSeq{\cdot}{\Top}}
  {}
}

\newcommand{\PBTopL}{\mbox{No rule for $\Top$-L}}

\newcommand{\PBBotL}{
  \infer[\Bot$-L$]
  {\PBSeq{\Bot}{\cdot}}
  {}
}

\newcommand{\PBBotR}{\mbox{No rule for $\Bot$-R}}

\begin{figure}
  \begin{center}
    {\small
      \fbox{
        \begin{tabular}{c}
          \begin{tabular}{ccc}
            \PBInit
            &
            \hspace{1cm}
            \PBTopR
            &
            \hspace{1cm}
            \PBTopL
          \end{tabular}
          \\[2em]
          \begin{tabular}{cc}
            \PBAndR
            &
            \hspace{1cm}
            \PBAndL
          \end{tabular}
          \\[2em]
          \begin{tabular}{ccc}
            \PBOrRL
            &
            \hspace{10pt}
            \PBOrRR
            &
            \hspace{10pt}
            \PBOrL
          \end{tabular}
          \\[2em]
          \begin{tabular}{cc}
            \PBBotR
            &
            \hspace{20pt}
            \PBBotL
          \end{tabular}
          \\[2em]
          \begin{tabular}{cc}
            \PBImpR
            &
            \hspace{20pt}
            \PBImpL
          \end{tabular}
        \end{tabular}
      }
    }
  \end{center}
  \caption{The backward calculus $\PB$}
  \label{prop.backward}
\end{figure}

%%% Local Variables:
%%% mode: latex
%%% TeX-master: "../thesis"
%%% End:

% 
\newcommand{\Bl}{\Left{B}}
\newcommand{\Br}{\Right{B}}
\newcommand{\Al}{\Left{A}}
\newcommand{\Cl}{\Left{C}}
\newcommand{\Cr}{\Right{C}}
\newcommand{\Ab}{A \Imp B}
\newcommand{\Abr}{\Right{\Ab}}
\newcommand{\Ac}{A \Or C}
\newcommand{\Acr}{\Right{\Ac}}
\newcommand{\Abc}{(A\Imp B) \Imp C}
\newcommand{\Abcl}{\Left{\Abc}}
\newcommand{\Bb}{\Bot \Or B}
\newcommand{\Bbl}{\Left{\Bb}}

\renewcommand{\qa}{\NSeq{\Al, \Bl, \Br}}
\renewcommand{\ra}{\Infer[$Init$]{\qa}{}}

\renewcommand{\qb}{\NSeq{\Bl,\Abr}}
\renewcommand{\rb}{\Infer[\Imp$-F$]{\qb}{\ra}}

\renewcommand{\qc}{\NSeq{\Bl, \Cl, \Cr}}
\renewcommand{\rc}{\Infer[$Init$]{\qc}{}}

\renewcommand{\qd}{\NSeq{\Bl, \Cl, \Acr}}
\renewcommand{\rd}{\Infer[\Or$-T$]{\qd}{\rc}}

\renewcommand{\qe}{\NSeq{\Bl, \Abcl, \Acr}}
\renewcommand{\re}{\Infer[\Imp$-T$]{\qe}{\rd & \rb}}

\renewcommand{\qf}{\NSeq{\Left{\Bot}, \Abcl, \Acr}}
\renewcommand{\rf}{\Infer[\Bot$-T$]{\qf}{}}

\renewcommand{\qg}{\NSeq{\Bbl, \Abcl, \Acr}}
\renewcommand{\rg}{\Infer[\Or$-T$]{\qg}{\rf & \re}}

\renewcommand{\qh}{\NSeq{\Bbl, \Right{(\Abc) \Imp \Ac}}}
\renewcommand{\rh}{\Infer[\Imp$-F$]{\qh}{\rg}}

\renewcommand{\qj}{\NSeq{\Right{\Bb \Imp (\Abc) \Imp \Ac}}}
\renewcommand{\rj}{\Infer[\Imp$-F$]{\qj}{\rh}}

\begin{figure}
  %\begin{sidewaysfigure}[angle=90]
    \begin{center}
      \[
      \boxed{\rj}
      \]
    \end{center}
    \caption{A derivation in $\LPB$}
    \label{prop.backward-der}
    %\end{sidewaysfigure}
\end{figure}


%%% Local Variables:
%%% mode: latex
%%% TeX-master: "../thesis"
%%% End:


$\PB$ is a foundation for backward search.  Begin with the goal formula and
repeatedly match the inference rules, generating new sequents in the derivation.
Backtracking is necessary when there is a choice of which rule to apply.  The
only troubling rule in the system is $\Imp$-L.  Each of the other rules
propegate only subformulas of the principal formula, thus decreasing the
size of the sequent by some measure.  In contrast, $\Imp$-L
propegates the principal formula as well as its subformulas to one of the
hypotheses.  Backward search must thus avoid looping behavior where $\Imp$-L is
repeatedly applied to the same formula.  Various ways of addressing this issue
have been considered (Section~\ref{prop.sec.related}).  One benefit of forward
search is that loop detection is not requried.

Before continuing, we collect some relevant
properties of $\PB$ that we will use later.

\subsection{The Subformula Property}

All the calculi we will consider in this thesis satisfy \emph{the subformula
property};  any sequent in a proof will consist entirely of signed subformulas
of the goal sequent.  In fact, we require a slightly stronger property than is normally
understood by the subformula property.
  Namely, we want to distinguish
different \emph{occurrances} of subformulas that occur in different parts of the goal
formula.  So for instance, while there are only two distinct subformulas of
$p\And p$, there are three subformula occurrances
becauese of the two distinct occurrances of $p$.

We also distinguish subformulas by their \emph{sign}.  The sign is either
positive or negative; a subformula with positive sign will only occur
as a consequent, while a negative subformula will only occur as an
antecedent.

\begin{definition}[Signed subformulas]
  \label{prop.def.subf}
  \index{subformula!sign of}
  \begin{itemize}
  \item[]
  \item If $A$ is a formula, a \emph{signed formula} is
    $\Left{A}$ or $\Right{A}$.
  \item The \emph{immediate signed subformula} relation is shown in
    Figure~\ref{prop.subformulas}.
  \item The \emph{signed subformula} relation $\le$ is the reflexive transitive
    closure of $<$.
  \item If $x$ is a signed formula,
    define
    \[ \Subfs(x) = \Set{x'\sst x'\le x} \]
  \item If $X$ is a multiset of signed formulas, define
    \[
    \Subfs(X) = \bigcup_{x\in X}\Subfs(x)
    \]
  \item If $\Gamma$ is a multiset of formulas, define
    \[
    \begin{array}{cc}
      \Left{\Gamma} =
      \begin{cases}
        \emptyset & \Gamma = \emptyset\\
        \Left{A}, \Left{\Gamma'} & \Gamma = A, \Gamma'
      \end{cases}
      &
      \hspace{2em}
      \Right{\Gamma} =
      \begin{cases}
        \emptyset & \Gamma = \emptyset\\
        \Right{A}, \Right{\Gamma} & \Gamma = A, \Gamma'
      \end{cases}
    \end{array}
    \]
  \item $\Subfs(\PBSeq{\Gamma}{\gamma}) = \Subfs(\Left{\Gamma}\Union\Right{\gamma})$
  \end{itemize}
\end{definition}

\begin{theorem}[The subformula property]
\label{prop.thm.subformula}
If $q'$ is a sequent in a derivation of $q$ then $\Subfs(q')
\subseteq \Subfs(q)$.
\end{theorem}

\begin{proof} Routine inspection of the inference rules. \end{proof}

In fact, a stronger subformula property holds as well.

%\subsection{Admissible rules}

\begin{lemma}[Weakening]
  \label{prop.thm.weaken}
  The rule
  \[
  \Infer[\mbox{Weaken}]
  {\PBSeq{A}{\cdot}}
  {\PBSeq{\cdot}{\cdot}}
  \]
  is admissible in $\PB$.
\end{lemma}
\begin{proof} Easy induction on the derivation. \end{proof}

\begin{theorem}[Cut]
  The rule
  \[
  \Infer[\mbox{Cut}]
  {\PBSeq{\cdot}{\cdot}}
  {\PBSeq{\cdot}{A} & \PBSeq{A}{\cdot}}
  \]
  is admissible in $\PB$.
\end{theorem}
\begin{proof} Due to Gentzen~\cite{Gentzen.1934.MZ}. \end{proof}

\begin{theorem}[Identity]
  The rule
  \[
  \Infer[\mbox{Id}]
  {\PBSeq{A}{A}}
  {}
  \]
  is admissible in $\PB$.
\end{theorem}
\begin{proof} Easy induction on $A$. \end{proof}

\begin{theorem}[Contraction]
  \label{prop.thm.contract}
  The rule
  \[
  \Infer[\mbox{Contracts}]
  {\PBSeq{A}{\cdot}}
  {\PBSeq{A, A}{\cdot}}
  \]
  is admissible in $\PB$.
\end{theorem}

\begin{proof}
  Induction on the derivation of $\PBSeq{A, A}{\cdot}$~\cite[Section 3.5.5]{Troelstra.2000.ProofTheory}.
\end{proof}

\begin{theorem}[Consistency]
  \label{prop.thm.consistent}
  $\PBSeq{\cdot}{\Bot}$ is not derivable.
\end{theorem}
\begin{proof} No inference rules apply. \end{proof}

%\subsection{Subsumption}

\begin{definition}[Subsumption]
  \index{subsumes}
  \index{subsumption}
  \begin{itemize}
  \item[]
  \item $\PBSeq{\Gamma_1}{\gamma_1}$ \emph{subsumes}
    $\PBSeq{\Gamma_2}{\gamma_2}$ if $\Gamma_1\subseteq\Gamma_2$
    and $\gamma_1\subseteq\gamma_2$.
  \item Write $q_1\Subsumes q_2$ if $q_1$ subsumes $q_2$.
  \item $q_1$ \emph{properly subsumes} $q_2$ if $q_1\Subsumes q_2$
    and $q_1\neq q_2$.  Write $q_1\ProperlySubsumes q_2$
  \item If $\cD$ is a set of sequents $\cD\Subsumes q$ if
    there exists $q'\in\cD$ such that $q'\Subsumes q$.
  \end{itemize}
\end{definition}

\noindent
(These defintions holds for forward sequents $\PFSeq{\Delta}{\delta}$ as well.
Recall that they are identical objects, written differently to
emphasize the context in which they appear.)
Subsumption is a measure of the strength of a sequent
in the following sense.

\begin{theorem}
  \label{prop.thm.subs}
  If $q\Subsumes q'$ and there is a
  $\PB$-derivation of $q$ then there is a $\PB$-derivation of $q'$.
\end{theorem}
\begin{proof}  Weaken the derivation of $q$. \end{proof}

%\subsection{Atoms}

Atomic formulas play an important role in forward search.

\begin{definition}
  \index{$\Atoms(A)$}
  For a formula $A$, let
  $\Atoms(A)=\Set{p\sst \Left{p}\in\Subfs(A) \mbox{ or } \Right{p}\in\Subfs(A)}$.
\end{definition}

\begin{definition}
  An atomic formula $p$ is called \emph{two-sided in $A$}
  if $\Set{\Left{p},\Right{p}} \subseteq \Subfs(A)$.
\end{definition}

\noindent For example, $p$ is two-sided in $p \Imp q \Imp p$
but $q$ is not.

\begin{theorem}
  In every derivation of $\PBSeq{\cdot}{A}$, the atoms of all
  occurrences of \emph{Init} are two-sided in $A$.
\end{theorem}
\begin{proof} Simple induction on the derivation. \end{proof}

\section{Forward Sequent Calculus}

\begin{quote}
  \textit{
    We derive a forward calculus $\PF$ from $\PB$, prove its soundness
    and completeness, and define $\PFx$.
  }
\end{quote}

%\subsection{Contraction}
\noindent
In intuitionistic logic, unlike linear logic, we can use the same antecedent an unlimited
number of times.  (Using multisets as antecedents is primarily a technical device.)
The admissibility of contraction (Theorem~\ref{prop.thm.contraction})
demonstrates this fact.  In forward calculi, contraction is normally
included as an explicit rule.  We handle contraction differently, by including
it as part of the rule matching process.

\begin{definition}[Contracts]
  \begin{itemize}
  \item[]
  \item $q$ is a \emph{contract} of $q'$ if there is a (backward)
    derivation of $q$ from $q'$ using only the contraction rule.
    (Theorem~\ref{prop.thm.contract})
  \item The set of contracts of a sequent $q$ is
    written $\Contracts(q)$.
  \item $\ASet{q_1', \ldots, q_n'}$ is a \emph{contract} of a
    sequence $\ASet{q_1, \ldots, q_n}$ if $q_i'\in\Contracts(q)$
    for all $i$.
  \item The set of contracts of a sequence of sequents $\cQ$ is
    written $\Contracts(\cQ)$.
  \end{itemize}
\end{definition}

%\subsection{Forward rule matching}
\noindent
Previously we described rule matching as a function from the conclusion of a
rule to its premises.  We can also define matching in the opposite direction, as
a function of the premises to the conclusion.  Note that we match all contracts
of the matching sequents to the hypotheses of the rule.  The need for this use
of contracts will become clear in Chapter~\ref{fol.chapter}.

\begin{definition}[Forward rule matching]
  \begin{itemize}
  \item[]
  \item
    Let
    \[
    \rho = \Infer
    {\PFSeq{\Gamma}{\gamma}}
    {\PFSeq{\Gamma_1}{\gamma_1} & \cdots & \PFSeq{\Gamma_n}{\gamma_n}}
    \]

    \noindent
    be an inference rule and let
    \[
    \cQ = \ASet{\PFSeq{\Delta_1}{\delta_1}, \ldots, \PFSeq{\Delta_n}{\delta_n}}
    \]

    \noindent be a sequence of sequents.  Let
    \begin{align*}
      \Delta &= \Gamma\Union (\Delta_1\Diff\Gamma_1) \Union \dots\Union(\Delta_n\Diff\Gamma_n)\\
      \delta &= \gamma\Union (\delta_1\Diff\gamma_1) \Union \dots\Union(\delta_n\Diff\gamma_n)\\
    \end{align*}

    \noindent
    If $\Card{\delta}\leq 1$ then define
    \[
    \FMatch_1(\rho, \cQ)=\PFSeq{\Delta}{\delta}.
    \]
  \item Define
    \[
    \FMatch(\rho, \cQ)= \Set{\FMatch_1(\rho, \cQ') \sst \cQ'\in\Contracts(\cQ)}
    \]

  \item We can also lift matching to sets.
    Let $\cR$ be a set of rules and let $\cD$ be a set of sequents.
    Define
    \[
    \FMatch(\cR, \cD) =
    \bigcup_{\rho\in\cR, \cQ \in \ASet{\cD}} \FMatch(\rho, \cQ)
    \]
  \end{itemize}
\end{definition}

\begin{definition}
  $\PF$ is the forward system consiting of of all instances of
  the rule schemas of Figure~\ref{prop.forward} and the
  matching function $\FMatch$ defined above.
  A forward derivation is shown in Figure~\ref{prop.forward-der}.
\end{definition}

Note that the rule schemas of $\PF$ are identical (as mathematical objects)
to those of $\PB$.  We write them again to emphasize the forward nature
of the rules.  The identity of forward and backward rules will not hold
when defining the constraint calculi~(Chapter~\ref{chapter.constr}).

% 
\newcommand{\PFInitRule}{
  \Infer[$Init$]
  {\VSeq{\Left{p},\Right{p}}}
  {}
}

\newcommand{\PFAndRRule}{
  \Infer[\And$-F$]
  {\VSeq{\Right{A_1 \And A_2}}}
  {\VSeq{\Right{A_1}} & \VSeq{\Right{A_2}}}
}

\newcommand{\PFAndLRule}{
  \Infer[\And$-T$]
  {\VSeq{\Left{A_1 \And A_2}}}
  {\VSeq{\Left{A_1}, \Left{A_2}}}
}

\newcommand{\PFImpRRule}{
  \Infer[\Imp$-F$]
  {\VSeq{\Right{A_1 \Imp A_2}}}
  {\VSeq{\Left{A_1},\Right{A_2}}}
}

\newcommand{\PFImpLRule}{
  \Infer[\Imp$-T$]
  {\VSeq{\Left{A_1\Imp A_2}}}
  {\VSeq{\Left{A_1\Imp A_2},\Right{A_1}} & \VSeq{\Left{A_2}}}
}

\newcommand{\PFOrRLRule}{
  \Infer[\Or$-F$_1]
  {\VSeq{\Right{A_1 \Or A_2}}}
  {\VSeq{\Right{A_1}}}
}

\newcommand{\PFOrRRRule}{
  \Infer[\Or$-F$_2]
  {\VSeq{\Right{A_1 \Or A_2}}}
  {\VSeq{\Right{A_2}}}
}

\newcommand{\PFOrLRule}{
  \Infer[\Or$-T$]
  {\VSeq{\Left{A_1 \Or A_2}}}
  {\VSeq{\Left{A_1}} & \VSeq{\Left{A_2}}}
}

\newcommand{\PFTopRRule}{
  \Infer[\Top$-F$]
  {\VSeq{\Right{\Top}}}
  {}
}

\newcommand{\PFBotLRule}{
  \Infer[\Bot$-T$]
  {\VSeq{\Left{\Bot}}}
  {}
}

\renewcommand{\figline}{\\[2em]}

\begin{figure}
  \begin{center}
    \fbox{
      \begin{tabular}{c}
        \begin{tabular}{ccc}
          \PFInitRule
          &
          \hspace{1cm}
          \PFTopRRule
          &
          \hspace{1cm}
          \mbox{No rule for $\Top$-T}
        \end{tabular}
        \figline
        \begin{tabular}{cc}
          \PFAndRRule
          &
          \hspace{1cm}
          \PFAndLRule
        \end{tabular}
        \figline
        \begin{tabular}{ccc}
          \PFOrRLRule
          &
          \hspace{10pt}
          \PFOrRRRule
          &
          \hspace{10pt}
          \PFOrLRule
        \end{tabular}
        \figline
        \begin{tabular}{cc}
          \mbox{No rule for $\Bot$-F}
          &
          \hspace{20pt}
          \PFBotLRule
        \end{tabular}
        \figline
        \begin{tabular}{cc}
          \PFImpRRule
          &
          \hspace{20pt}
          \PFImpLRule
        \end{tabular}
      \end{tabular}
    }
  \end{center}
  \caption{Forward inference rules}
  \label{prop.forward}
\end{figure}

%%% Local Variables:
%%% mode: latex
%%% TeX-master: "../thesis"
%%% End:


\begin{remark}
  Weakening is not admissible in $\PF$.  In $\PB$, weakening a
  proof of $\PBSeq{\Delta}{A}$ by $B$ corresponds to adding $B$ to
  the antecedents of every sequent in the derivation.  The augmented
  initial sequents remain initial.  In $\PF$ the initial sequents
  are fixed, $\PFSeq{p}{p}$, so this argument does not hold.
\end{remark}

\subsection{Soundness and Completeness}

The soundness and completeness theorems justify using forward sequents for proof
search.
\begin{theorem}[Soundness]
  \label{prop.thm.forward-sound}
  If $\PFSeq{\Delta}{\delta}$ is derivable in $\PF$ then
  $\PBSeq{\Delta}{\delta}$ is derivable in $\PB$.
\end{theorem}

\begin{proof}
  By induction on the derivation of $\PFSeq{\Delta}{\delta}$.
  Let $\PFSeq{\Delta}{\delta} = \FMatch(r, \cQ)$ where $r$ is the last
  rule application.  We show some representative rules.
  \begin{description}
  \item[Case:]
    $r = \mbox{Init}$.
    Then $\Delta$ is an instance of $Init$
    \[ \Infer[Init]{\VSeq{\Left{p},\Right{p}}}{} \]
    and we have the identical derivation
    \[ \Infer[Init]{\NSeq{\Left{p},\Right{p}}}{} \]
  \item[Case:]
    \Infer[\Imp$-R$]
    {\VSeq{\Right{A_1 \Imp A_2}}}
    {\VSeq{\Left{A_1},\Right{A_2}}}
    By IH we have $\NSeq{\Delta}$.  There are a few cases, depending
    on whether $\Left{A_1}$
    If

  \item[Case:]
    \[ \PFImpL \]
    \begin{tabbing}
      $\FB{\PFSeq{\Gamma_1, A_1\Imp A_2}{A_1}}$ \` IH \\
      $\PBSeq{\Gamma_1, A_1\Imp A_2}{A_1}$ \` Def. \\
      $\PBSeq{\Gamma_1, \Gamma_2, A_1\Imp A_2}{A_1}$ \` Weaken \\
      $\FB{\PFSeq{\Gamma_2, A_2}{\gamma}}$ \` IH \\
      $\PBSeq{\Gamma_2, A_2}{\FB{\gamma}}$ \` Def. \\
      $\PBSeq{\Gamma_1, \Gamma_2, A_2}{\FB{\gamma}}$ \` Weaken \\
      $\PBSeq{\Gamma_1, \Gamma_2, A_1\Imp A_2}{\FB{\gamma}}$ \` $\Imp$-L\\
      $\FB{\PFSeq{\Gamma_1, \Gamma_2, A_1\Imp A_2}{\gamma}}$
      \` Def., possibly with contractions (Lemma~\ref{prop.thm.contract})\\
    \end{tabbing}
  \end{description}
\end{proof}

The completeness theorem is not totally straightforward, as the provable sequents
are not identical.  For example, due to the inadmissibility of weakening,
$\PBSeq{p, q}{p}$ is derivable in $\PB$ but
$\PFSeq{p, q}{p}$ is not derivable in $\PF$.
Instead we show we can always prove a stronger sequent.

%
\renewcommand{\ab}{a \Imp b}
\renewcommand{\ac}{a \Or c}
\renewcommand{\abc}{(a\Imp b) \Imp c}
\renewcommand{\bb}{\Bot \Or b}

\renewcommand{\Qa}{\FSeq{b}{b}}
\renewcommand{\Ra}{\Infer[$Init$]{\Qa}{}}

\renewcommand{\Qb}{\FSeq{b}{\ab}}
\renewcommand{\Rb}{\Infer[\Imp$-R$]{\Qb}{\Ra}}

\renewcommand{\Qc}{\FSeq{c}{c}}
\renewcommand{\Rc}{\Infer[$Init$]{\Qc}{}}

\renewcommand{\Qd}{\FSeq{c}{\ac}}
\renewcommand{\Rd}{\Infer[\Or$-R$_2]{\Qd}{\Rc}}

\renewcommand{\Qe}{\FSeq{b, \abc}{\ac}}
\renewcommand{\Re}{\Infer[\Imp$-L$]{\Qe}{\Rd & \Rb}}

\renewcommand{\Qf}{\FSeq{\Bot}{\cdot}}
\renewcommand{\Rf}{\Infer[\Bot$-L$]{\Qf}{}}

\renewcommand{\Qg}{\FSeq{\bb, \abc}{\ac}}
\renewcommand{\Rg}{\Infer[\Or$-L$]{\Qg}{\Rf & \Re}}

\renewcommand{\Qh}{\FSeq{\bb}{(\abc) \Imp \ac}}
\renewcommand{\Rh}{\Infer[\Imp$-R$]{\Qh}{\Rg}}

\renewcommand{\Qj}{\FSeq{\cdot}{\bb \Imp (\abc) \Imp \ac}}
\renewcommand{\Rj}{\Infer[\Imp$-R$]{\Qj}{\Rh}}

\begin{figure}
  \begin{center}
    \fbox{
      \Rj
    }
  \end{center}
  \caption{A derivation in $\PF$}
  \label{prop.forward-der}
\end{figure}


%%% Local Variables:
%%% mode: latex
%%% TeX-master: "../thesis"
%%% End:


\section{Toward an efficient implementation}

\begin{definition}[Paths]
  A \emph{path} is a sequence of the set $\Set{l,r}$.
  Define $\Paths^+(A)$, and $\Paths^-(A)$ as follows:
  \[
  \begin{array}{l}
    \Paths^+(\Top) = \Paths^+(\Bot) = \Paths^+(p) = \Set{[]}\\
    \Paths^-(\Top) = \Paths^-(\Bot) = \Paths^-(p) = \Set{[]}\\
    \Paths^+(A\And B) = \Paths^+(A\Or B) =
    \Set{l : \pi\ |\ \pi\in\Paths^+(A)} \Union
    \Set{r : \pi\ |\ \pi\in\Paths^+(B)} \Union \Set{[]}\\
    \Paths^-(A\And B) = \Paths^-(A\Or B) =
    \Set{l : \pi\ |\ \pi\in\Paths^-(A)} \Union
    \Set{r : \pi\ |\ \pi\in\Paths^-(B)} \Union \Set{[]}\\
    \Paths^+(A\Imp B) =
    \Set{l : \pi\ |\ \pi\in\Paths^-(A)} \Union
    \Set{r : \pi\ |\ \pi\in\Paths^+(B)} \Union \Set{[]}\\
    \Paths^-(A\Imp B) =
    \Set{l : \pi\ |\ \pi\in\Paths^+(A)} \Union
    \Set{r : \pi\ |\ \pi\in\Paths^-(B)} \Union \Set{[]}\\
  \end{array}
  \]
\end{definition}

\noindent
If $\pi\in\Paths^\pm(A)$ then some subformula of $A$ corresponds to
$\pi$.  Define

\[
\Subf(\pi, A) =
\begin{cases}
  A & \pi = []\\
  \Subf(\pi', A_1) & \pi = l : \pi', A = A_1 * A_2\\
  \Subf(\pi', A_2) & \pi = r : \pi', A = A_1 * A_2\\
\end{cases}
\]
\noindent
where $* \in \Set{\And,\Or,\Imp}$.
If $\pi\in\Paths^+(A)$ we say that $\Subf(\pi)$ is
a \emph{positive} subformula of $A$, and
if $\pi\in\Paths^-(A)$ we say that $\Subf(\pi)$ is
a \emph{negative} subformula of $A$.  We write either
$\Sgn(\pi)=+$ or $\Sgn(\pi)=-$ respectively.
If $\Subf(\pi)$ is atomic, we say that $\pi$ is an atomic path.

\begin{example}
  The paths of $(a\Imp b) \Imp ((a\Imp b) \Imp c) \Imp c$ are

  \[
  \begin{array}{lll}
    \pi & \Subf(\pi) & \Sgn(\pi) \\
    \hline \\
    \LB\RB & (a\Imp b) \Imp ((a\Imp b) \Imp c) \Imp c & + \\
    \LB l\RB & a\Imp b & - \\
    \LB l,l\RB & a & + \\
    \LB r,l\RB & b & - \\
    \LB r\RB & ((a\Imp b) \Imp c) \Imp c & + \\
    \LB l,r\RB & (a\Imp b) \Imp c & - \\
    \LB r,r\RB & c & + \\
    \LB l,l,r\RB & a\Imp b & + \\
    \LB r,l,r\RB & c & - \\
    \LB l,l,l,r\RB & a & - \\
    \LB r,l,l,r\RB & b & + \\
  \end{array}
  \]
\end{example}

In all that follows, we assume a fixed goal formula $\xi$.
Define

\[
\begin{array}{c}
  \Paths^+ = \Paths^+(\xi)\\
  \Paths^- = \Paths^-(\xi)\\
  \Subf(\pi) = \Subf(\pi,\xi)\\
\end{array}
\]

\begin{definition}[Labels]
  A \emph{labeling function} $f$ is a surjective function from paths to atoms
  that maps atomic paths to their corresponding atoms, and non-atomic paths
  to fresh, distinct atoms.  More formally,

  \begin{itemize}
  \item If $\pi$ is atomic, then $f(\pi) = \Subf(\pi)$.
  \item If $\pi$ and $\pi'$ are non-atomic, then $f(\pi) \neq f(\pi')$
    and $f(\pi)$ is not an atom of $\xi$.
  \end{itemize}

  \noindent
  It is easy to see that labeling functions exist.  Fix such a function $f$.  Then
  we call $f(\pi)$ the \emph{label} of $\pi$.  If $\Subf(\pi) = A$ we sometimes
  call $f(\pi)$ \emph{the} label of $A$, though this is actually ambiguous since there
  could be many paths $\pi_i$ with $\Subf(\pi_i) = A$.  If any ambiguity could
  arise in such a word choice, we will use paths rather than formulas.
\end{definition}

\section{Path calculus}

\begin{definition}[Sequents]
  A \emph{sequent} is a pair $\ASet{\Gamma,\gamma}$
  where $\Gamma\subseteq\Paths^-$ (a multiset) and $\gamma\subseteq \Paths^+$
  is a set of formulas with at most one element.
  $\Gamma$ is called the \emph{antecedents}
  and $\gamma$ the \emph{consequent}.
  We write $\ASet{\Gamma,\gamma}$ as either $\PBSeq{\Gamma}{\gamma}$
  or $\PFSeq{\Gamma}{\gamma}$.  The former is used when we describe
  backward calculi, and the later for forward calculi.  Keep in mind
  though that behind the notation the objects are identical, and we freely
  convert between them in proofs that relate forward and backward systems.
  If $\gamma$ is empty, we write $\PBSeq{\Gamma}{\cdot}$.
  If $\gamma$ is a singleton $\Set{A}$, we write $\PBSeq{\Gamma}{A}$.
  If $\Gamma$ is empty, we write $\PBSeq{\cdot}{\gamma}$.
  We denote the set of all sequents by $\Seqs$,
  the set of all sequences of sequents by $\SeqSeqs$, and
  the set of all sets of sequents by $\SetSeqs$.
  %  If $\cD$ is a set of sequents, then $\ASet{\cD}$ is the set
  %   of sequences of sequents with elements in $\cD$.
  Note that we will almost always abuse notation and write a sequent
  as a list of formulas rather than paths.  When we write
  $\PBSeq{A}{B}$ we mean a sequent $\PBSeq{\pi_A}{\pi_B}$ where
  $\Subf(\pi_A) = A$ and $\Subf(\pi_B) = B$.
\end{definition}

\section{XXX Notes}
or $\PFSeq{\Gamma}{\gamma}$.  The former is used when we describe
  backward calculi, and the later for forward calculi.  Keep in mind
  though that behind the notation the objects are identical, and we freely
  convert between them in proofs that relate forward and backward systems.

\begin{itemize}
\item Note that polarities and subformula signs are unrelated.
\item paragraph about positive and negative AND
\item COMPACT proof terms
\item (The restriction of $\gamma$ to size at most one distinguishes
    intuitionistic from classical logic.)
\end{itemize}

%%% Local Variables:
%%% mode: latex
%%% TeX-master: "thesis"
%%% End:
