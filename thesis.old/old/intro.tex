
\begin{quote}
The aim has been to lay the groundwork for a theory of automated deduction in
arbitrary symbolic logics by providing empirical evidence for the existence of
a \emph{uniform} approach to efficient proof search in both truth-functional and
non truth-functional logics.  Whilst the formulation of such a comprehensive
theory is at the time of writing beyond the author's reach, it is intended that
the form such a theory might take, and the ideas on which it might be based, be
suggested by the methods brought to bear in the text.~\cite{Wallen.1990.Book}
---\textit{Lincoln Wallen} (1990)
\end{quote}

\vspace{1cm}

This thesis continues Wallen's efforts toward uniform methods for
efficient proof search in non-classical logics.  While Wallen's framework
was founded on matrix methods, we use a dual approach based on the
inverse method.  Like resolution in classical
logic, the inverse method is a bottom-up, saturation-based proof procedure.
Unlike resolution, which relies upon strong clause normal forms for its
effectiveness, the inverse method is based on the sequent calculus.
Thus, it can handle logics where no clausal normal forms exist such as
intuitionistic logic.


Resolution  |  Inverse method |  Tableaux |
---------------------------------------------



 While automated deduction in
classical logic is a mature field, with dozens of theorem provers and an
annual competition, deduction in nonclassical logics has been
comparitively neglected\footnote{A notable exception is model checking in temporal
logic}.

Why use non-classical logic?

\cite{Garg.10.SSP}
\cite{Gurevich.08.DKAL}
\cite{Gurevich.11.TCL}

Gurevich writes
\begin{quote}
Eventually we switched from algebra to logic that treats $x ^ y$ and $x ! y$ as
conjunction and implication respectively. And something unexpected happened,
a little miracle. The logic of infons happened to be a natural and conservative
extension of disjunction-free intuitionistic logic.~\citep{Gurevich.09.DKAL2}
\end{quote}


Coq, Agda

Modal

We have only begun to explore the applications of linear logic.  Ordered
logic is still mostly unknown.


We represent fine-grained control of the search behavior
by \emph{polarizing} the input formula.
In manipulating the polarity of atoms and subformulas, we can
often improve the search time by several orders of magnitude.


\section{Summary}

Chapter~\ref{chapter.fol} introduces the polarized inverse method for
intuitionistic first-order logic in detail.  Chapter~\ref{chapter.constraints}
extends first-order sequents with constraints, and extends focusing and
the inverse method with generic constraint domains.  Chapter~\ref{chapter.modal}
uses the constraint system to develop proof procedures for some intuitionistic
modal logics.  Chapter~\ref{chapter.substruct} applies the constraint system
to substructural logics.  By varying the rules of the constraint domain we
handle linear and ordered logic.  Chapter~\ref{chapter.implementation} describes
the details of our implementation \emph{Imogen}, and shows some experiments.
Chapter~\ref{chapter.conclusion} concludes, surveys related work, and suggests
directions for future research.
