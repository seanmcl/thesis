
\chapter{Constraints}\label{chapter.constraints}

\section{Motivation}

\subsection{Equality}

In Chapter~\ref{chapter.fol}, terms are only considered equal if they
are syntactically equal.  The question arises,
how can we reason in axiomatic theories. The most natural example
is first-order logic with equality, axiomitized by the following
equations

\[
\begin{array}{l}
  \All x.~x = x
  \\
  \All x~y.~x = y \Imp y = x
  \\
  \All x~y~z.~x = y \Imp y = z\Imp x = z
  \\
  \All \vec{x}~\vec{y}.~
  x_1=y_1'\And \ldots\And x_n=y_n'\Imp f(x_1,\ldots,x_n) = f(y_1,\ldots,y_n)
\end{array}
\]

\noindent
One approach is to add the axioms as premeses to a sequent, and prove the
sequent under the additional assumptions.  This approach, however, may not lead
to a very efficient theorem prover.  In the classical literature, many
specialized reasoning techniques, such as paramodulation, have been engineered
to be far more effective than adding axioms.  Like resolution, paramodulation is
fundamentally classical, and investigating other strategies is of some interest.
Voronkov showed~\cite{Voronkov.1996.CADE} that intuitionistic logic with
equality can be reduced to simultaneous rigid E-unification, which Degtyarev and
Voronkov showed~\cite{Degtyarev.1996.TCS} undecidable.

\subsection{Unification}

The reduction from intuitionistic logic with equality to simultaneous rigid
E-unification depends on equalities being explicit atomic formulas, in
particular allowing negated equalities to occur as premeses.  We
call interpreted atomic formulas \emph{extrinsic} constraints.
In one sense, equality is already a part of backward proof search,
in the form of unification.
This view is detailed, e.g., in Pfenning's notes on automated
deduction~\cite[Section 4.3-4.4]{Pfenning.2004.TheoremProvingLectureNotes}.
The backward rules are modified to account for unification variables.

\[
\begin{tabular}{cc}
  \textbf{Ground} & \textbf{Unification}
  \\
  \Infer[$Init$]
  {\CSeq{\Gamma, p}{p}}
  {}
  &
  \hspace{2em}
  \Infer[$Init$]
  {\CSeq{\Phi}{\Gamma, p}{q}}
  {\Entails p\Ueq q}
  \\[2em]
  \Infer[\All$-R$_a]
  {\CSeq{\Gamma}{\All x.~A(x)}}
  {\CSeq{\Gamma}{A(a)}}
  &
  \hspace{2em}
  \Infer[\All$-R$_a]
  {\CSeq{\Phi}{\Gamma}{\All x.~A(x)}}
  {\CSeq{\Phi,a}{\Gamma}{A(a)}}
  \\[2em]
  \Infer[\All$-L$]
  {\CSeq{\Gamma, \All x.~A(x)}{C}}
  {\CSeq{\Gamma,A(t)}{C}}
  &
  \hspace{2em}
  \Infer[\All$-L$]
  {\CSeq{\Phi}{\Gamma, \All x.~A(x)}{C}}
  {\CSeq{\Phi}{\Gamma,A(X_{\Phi})}{C}}
\end{tabular}
\]

\noindent
where in the $\All$-L rule, the variable $X_\Phi$ is a unification
variable that is allowed to depend on the parameters in $\Phi$.
A proof, then, is a derivation where the side-conditions on all
the initial sequents hold simultaneously.  While formalizing
backward search in this way is beyond the scope of this thesis,
there are two lessons relevant for our work.
The first is that it is possible to defer computations
(here the choice of existential instantiations).  The second is
that finding the solutions can take place in a totally different
logic than the one in which we are searching for proofs
(here encoded in the $\Entails$ relation).

\subsection{Most-general unifiers}

Backward proof search mainly works because the unification problem in the free
term algebra is particularly simple.  In particular, the existence of most
general unifiers means that we never have to reconsider a choice of unifiers;
there is always a best choice!  Adding interpreted function symbols to the term
language will in general break this property.  We don't need to look far for
examples.  Reasoning about a symmetric operator $\oplus$ with the axiom $\All
x~y.~x\oplus y=y\oplus x$ already lacks mgus;  $\Set{x\to s,y\to
t}$ and $\Set{x\to t, y\to s}$ both unify $x\oplus y$ and $s\oplus t$ (assuming
$x,y\not\in\Vars s,t)$, but
neither is more general than the other.  It is not necessary to allow equality
as an atomic predicate; we should certainly be able to prove the sequent
$\CSeq{p(c\oplus d)}{p(d\oplus c)}$.

\section{Natural deduction}


\newcommand{\MNBoxI}{
  \Infer[\Box-$I$^a]
  {\MNd{\Gamma}{\MLam l:\MLe{w}{a}.~M : \MAt{\Box A}{w}}}
  {\MNd[\Delta,a:\MWorld]{\Gamma, l : \MLe{w}{a}}{M : \MAt{A}{a}}}
}

\newcommand{\MNBoxE}{
  \Infer[\Box-$E$]
  {\MNd{\Gamma}{M~[\MLe{w}{w'}]} : \MAt{A}{w'}}
  {
    \MNd{\Gamma}{M : \MAt{\Box A}{w}}
    &
    \Delta~;~\Gamma\Entails\MLe{w}{w'}
  }
}

\newcommand{\MNDiaI}{
  \Infer[\Dia-$I$]
  {\MNd{\Gamma}{\ASet{\MLe{w}{w'}} M} : \MAt{\Dia A}{w}}
  {
    \MNd{\Gamma}{M : \MAt{A}{w'}}
    &
    \Delta~;~\Gamma\Entails\MLe{w}{w'}
  }
}

\newcommand{\MNDiaE}{
  \Infer[\Dia-$E$^a]
  {\MNd{\Gamma}{\Let\ASet{l:\MLe{w'}{a}}x = M~\In~N : \MAt{C}{w}}}
  {
    \MNd{\Gamma}{M : \MAt{\Dia A}{w'}}
    &
    \MNd[\Delta,a:\MWorld]{\Gamma,l:\MLe{w'}{a},x:\MAt{A}{a}}{N : \MAt{C}{w}}
  }
}

\begin{figure}[H]
  \begin{center}
    {\footnotesize
      \fbox{
        \begin{tabular}{c}
          \begin{tabular}{cc}
            \MNBoxI
            &
            \hspace{2em}
            \MNBoxE
          \end{tabular}
          \\[2em]
          \begin{tabular}{c}
            \MNDiaI
          \end{tabular}
          \\[2em]
          \begin{tabular}{c}
            \MNDiaE
          \end{tabular}
        \end{tabular}
      }
    }
  \end{center}
  \caption{Natural deduction.}
  \label{modal.nd}
\end{figure}


%%% Local Variables:
%%% mode: latex
%%% TeX-master: "../thesis"
%%% End:


\begin{example}
  Let $\xi=(\All x.~p(x)\Imp p(f(x)))\Imp c=d\Imp p(c)\Imp p(f(d))$.
  The following are both proof terms for $\xi$:
  \[
  \begin{array}{cc}
    \Lam H_1:(\All x.~p(x)\Imp p(f(x))).~
    \Lam H_2:(c=d).~
    \Lam H_3:(p(c)).~
    H_1~c~H_3 : p(f(d))
    \\
    \Lam H_1:(\All x.~p(x)\Imp p(f(x))).~
    \Lam H_2:(c=d).~
    \Lam H_3:(p(c)).~
    H_1~d~(H_3 : p(d))
  \end{array}
  \]
\end{example}

\section{Constraint Domains}

\begin{definition}[Constraint domain]
  A \emph{constraint domain} is a tuple
  $\ConDom = \ASet{\UFunc_{\ConDom},\UAtom_{\ConDom},\Entails_{\ConDom}}$
  where $\UFunc_{\ConDom}\subseteq\UFunc$ is a set of function symbols and
  $\UAtom_{\ConDom}\subseteq\UAtom$ is a subset of atomic formulas.
  Constraint formulas are built from the following grammar.
  \[
  \mbox{Constraints } \Psi ::= t_1\Ueq t_2 \Sep p \Sep A \And A \Sep \Top \Sep A \Imp A \Sep \Bot
  \]
  where $\Funcs(t_i)\subseteq\UFunc_{\ConDom}, p\in\cA_\ConDom$.
  Finally, $\Psi_1\Entails_{\ConDom}\Psi_2$ is a relation where
  $\Psi_1$ and $\Psi_2$ are constraint formulas.  If $p\not\in\UAtom_{\ConDom}$ then
  the expression
  $p(t_1,\ldots,t_n)\Ueq p(s_1,\ldots,s_n)$ is shorthand for the formula
  Write $\Psi\CEquiv\Psi'$ if $\Psi\Entails\Psi'$ and $\Psi'\Entails\Psi$.
  The entailment relation must satisfy the following properties.
  \begin{enumerate}
  \item
    $\Psi\Entails\Psi$
  \item
    $\Psi\Entails\Top$
  \item
    If $\Psi_1\Entails\Psi_2$ and $\Psi_2\Entails\Psi_3$ then
    $\Psi_1\Entails\Psi_3$.
  \item
    If $\Psi\Entails \Psi_1\And\Psi_2$ then $\Psi\Entails \Psi_1$ and $\Psi\Entails \Psi_2$.
  \item
    If $\Psi\Entails \Psi_1$ and $\Psi\Entails\Psi_2$ then
    $\Psi\Entails \Psi_1\And\Psi_2$
  \item
    If $\Psi\Entails \All x.~\Psi'(x)$ then for any $t$, $\Psi\Entails\Psi'(t)$.
  \item
    If $\Psi\Entails \Ex x.~\Psi'(x)$ then for some $t$, $\Psi\Entails\Psi'(t)$.
  \item
    If $\Psi\Entails \Psi(x)$ and $x\not\in\Psi$ then $\Psi\Entails\All x.~\Psi(x)$.
  \item
    If $\Psi_1\Entails \Psi_1'$ and $\Psi_2\Entails \Psi_2'$
    then $\Psi_1'\Imp\Psi_2\Entails \Psi_1\Imp\Psi_2'$.
  \item\label{constr.def.equiv} $x = t\And \Psi \CEquiv \Psi\cdot\Set{x\to t}$
  \end{enumerate}

  \noindent
  Rule~\ref{constr.def.equiv} ensures that entailment respects substitution.
  We use the phrase ``by entailment reasoning'' to indicate reasoning from these
  rules.
\end{definition}

\begin{theorem}
  Some easy consequences of the entailment relaton are:
  \begin{itemize}
  \item[]
  \item $\Psi_1\And\Psi_2 \CEquiv \Psi_2\And\Psi_1$
  \item $(\Psi_1\And\Psi_2)\And\Psi_3 \CEquiv \Psi_1\And(\Psi_2\And\Psi_3)$
  \end{itemize}
\end{theorem}

\section{Backward calculus}

\begin{definition}[Constraint sequents]
  A sequent is a pair $\ASet{\Psi, \sigma}$
  where $\Psi$ is a constraint formula and $\sigma$ is a first-order sequent.
  We write backward sequents as $\CSeq{\Psi}{\Delta}{\delta}$ and
  forward sequents as $\CFSeq{\Psi}{\Delta}{\delta}$.
  $\Psi$ is called the \emph{constraint context}.
\end{definition}

\begin{definition}[Subsumption]
  $\CSeq{\Psi}{\Delta}{\delta}$ \emph{subsumes}
  $\CSeq{\Psi'}{\Delta'}{\delta'}$ if
  $\Delta \subseteq \Delta'$, $\delta \subseteq \delta'$, and $\Psi'\Entails\Psi$.
\end{definition}

\noindent
The constraints on the subsuming sequent must be weaker than that of the
subsumed sequent.  This is natural, since a sequent with $\Psi=\Bot$ is
trivial, so subsuming it should be easy.

\begin{definition}[Backward inference rules]
  An inference rule is a tuple $\ASet{\psi, \cH, \cC}$ where
  $\cH = \ASet{H_1,\ldots,H_n}$, the hypotheses, are first-order sequents,
  $\cC$ is a first-order sequent, and $\psi~:~\cF\to\ASet{\cF}_n$.
  We write
  % \[
  % \Infer{\cC}{\}
  % \]
\end{definition}


\begin{definition}[Unification equations]
  For formulas $A,B$, define
  \[
  A\Ueq B=
  \begin{cases}
    p\Ueq q & A=p,~B=q,~p,q\in\cA_{\ConDom}\\
    t_1\Ueq t_1'\And\cdots\And t_n\Ueq t_n' & A = p(t_1,\ldots,t_n),~B=p(t_1',\ldots,t_n'),~ p\in\cA\\
    (A_1\Ueq B_1) \And (A_2\Ueq B_2) & A=A_1*A_2,~B=B_1*B_2,~*\in\Set{\And,\Or,\Imp}\\
    A'(c)\Ueq B'(c) & A=\All x.~A'(x),~B=\All y.~B'(y) \\
    A'(c)\Ueq B'(c) & A=\Ex x.~A'(x),~B=\Ex y.~B'(y) \\
    \Bot & \mbox{otherwise}\\
  \end{cases}
  \]
  The first case is to allow unification of atomic formulas from the constraint
  domain to be determined by the domain.  For example, it should be possible to
  unify $a < b$ and $b > a$ in an arithmetic theory, even though the predicates
  are not syntactically equal.
  The unification equations of quantified formulas should not be affected by
  the bound variable.  Using a constant symbol in place of the variable captures
  this independence.
\end{definition}

\begin{theorem}[Strengthening]
  \label{constr.thm.strengthen}
  \begin{itemize}
  \item[]
  \item If $\CSeq{\Psi}{\Delta}{\delta}$ and $\Psi'\Entails\Psi$ then
    $\CSeq{\Psi'}{\Delta}{\delta}$.
  \item If $\CSeq{\Psi,a=b}{\Delta}{\delta}$ and $a\not\in\Delta,\delta$ then
    $\CSeq{\Psi}{\Delta}{\delta}$
  \end{itemize}
\end{theorem}

\begin{proof}
  TODO
\end{proof}

\begin{theorem}[Substitution]
  \label{constr.thm.subst}
  \begin{itemize}
  \item[]
  \item
    If there is a derivation of $\CSeq{\Psi}{\Gamma}{\gamma}$ and $\theta$
    is a valid substitution then
    there is a derivation of $(\CSeq{\Psi}{\Gamma}{\gamma})\theta$.
  \item
    If there is a derivation of $\CFSeq{\Psi}{\Gamma}{\gamma}$ and $\theta$
    is a valid substitution then
    there is a derivation of $(\CFSeq{\Psi}{\Gamma}{\gamma})\theta$.
  \end{itemize}
\end{theorem}

\begin{proof}
  Induction on the derivation, using the validity of the
  substitution to avoid variable capture.
\end{proof}

\begin{theorem}[Weakening]
  \begin{itemize}
  \item[]
  \item
    If $\CSeq{\Psi}{\Delta}{\delta}$ then $\CSeq{\Psi}{\Delta,A}{\delta}$.
  \item
    If $\CSeq{\Psi}{\Delta}{\cdot}$ then $\CSeq{\Psi}{\Delta}{A}$.
  \end{itemize}
\end{theorem}
\begin{proof} Induction. \end{proof}

\begin{proof} The leaves of any deduction are valid initial sequents. \end{proof}

\begin{definition}[Forward inference rules]
  A forward inference rule is a triple $\ASet{\psi,\Delta,\delta}$
  where $\psi~:~\ASet{\cF}_n\to\cF$.
\end{definition}

\begin{definition}[Backward matching]

\end{definition}

\begin{definition}[Forward matching]

\end{definition}


\newcommand{\PBInit}{\Infer[$Init$]{\PBSeq{p}{p}}{}}

\newcommand{\PBAndR}{
  \Infer[\And$-R$]
  {\PBSeq{\cdot}{A_1 \And A_2}}
  {\PBSeq{\cdot}{A_1} & \PBSeq{\cdot}{A_2}}
}

\newcommand{\PBAndL}{
  \Infer[\And$-L$]
  {\PBSeq{A_1 \And A_2}{\cdot}}
  {\PBSeq{A_1, A_2}{\cdot}}
}

\newcommand{\PBImpR}{
  \Infer[\Imp$-R$]
  {\PBSeq{\cdot}{A_1 \Imp A_2}}
  {\PBSeq{A_1}{A_2}}
}

\newcommand{\PBImpL}{
  \Infer[\Imp$-L$]
  {\PBSeq{A_1\Imp A_2}{\cdot}}
  {\PBSeq{A_1\Imp A_2}{A_1} & \PBSeq{A_2}{\cdot}}
}

\newcommand{\PBOrRL}{
  \Infer[\Or$-R$_1]
  {\PBSeq{\cdot}{A_1 \Or A_2}}
  {\PBSeq{\cdot}{A_1}}
}

\newcommand{\PBOrRR}{
  \infer[\Or$-R$_2]
  {\PBSeq{\cdot}{A_1 \Or A_2}}
  {\PBSeq{\cdot}{A_2}}
}

\newcommand{\PBOrL}{
  \infer[\Or$-L$]
  {\PBSeq{A_1 \Or A_2}{\cdot}}
  {\PBSeq{A_1}{\cdot} & \PBSeq{\Gamma, A_2}{\cdot}}
}

\newcommand{\PBTopR}{
  \infer[\Top$-R$]
  {\PBSeq{\cdot}{\Top}}
  {}
}

\newcommand{\PBTopL}{\mbox{No rule for $\Top$-L}}

\newcommand{\PBBotL}{
  \infer[\Bot$-L$]
  {\PBSeq{\Bot}{\cdot}}
  {}
}

\newcommand{\PBBotR}{\mbox{No rule for $\Bot$-R}}

\begin{figure}
  \begin{center}
    {\small
      \fbox{
        \begin{tabular}{c}
          \begin{tabular}{ccc}
            \PBInit
            &
            \hspace{1cm}
            \PBTopR
            &
            \hspace{1cm}
            \PBTopL
          \end{tabular}
          \\[2em]
          \begin{tabular}{cc}
            \PBAndR
            &
            \hspace{1cm}
            \PBAndL
          \end{tabular}
          \\[2em]
          \begin{tabular}{ccc}
            \PBOrRL
            &
            \hspace{10pt}
            \PBOrRR
            &
            \hspace{10pt}
            \PBOrL
          \end{tabular}
          \\[2em]
          \begin{tabular}{cc}
            \PBBotR
            &
            \hspace{20pt}
            \PBBotL
          \end{tabular}
          \\[2em]
          \begin{tabular}{cc}
            \PBImpR
            &
            \hspace{20pt}
            \PBImpL
          \end{tabular}
        \end{tabular}
      }
    }
  \end{center}
  \caption{The backward calculus $\PB$}
  \label{prop.backward}
\end{figure}

%%% Local Variables:
%%% mode: latex
%%% TeX-master: "../thesis"
%%% End:


\begin{theorem}[Soundness]
  If $\CSeq{\Psi}{\Gamma}{\gamma}$ then $\CNd{\Psi,\Gamma}{\gamma}$.
\end{theorem}
\begin{proof}

\end{proof}

\begin{theorem}[Completeness]
  If $\CNd{\Gamma}{\gamma}$ then $\CSeq{\Psi}{\Gamma}{\gamma}$
\end{theorem}
\begin{proof}

\end{proof}

\section{Forward calculus}


\newcommand{\PFInitRule}{
  \Infer[$Init$]
  {\VSeq{\Left{p},\Right{p}}}
  {}
}

\newcommand{\PFAndRRule}{
  \Infer[\And$-F$]
  {\VSeq{\Right{A_1 \And A_2}}}
  {\VSeq{\Right{A_1}} & \VSeq{\Right{A_2}}}
}

\newcommand{\PFAndLRule}{
  \Infer[\And$-T$]
  {\VSeq{\Left{A_1 \And A_2}}}
  {\VSeq{\Left{A_1}, \Left{A_2}}}
}

\newcommand{\PFImpRRule}{
  \Infer[\Imp$-F$]
  {\VSeq{\Right{A_1 \Imp A_2}}}
  {\VSeq{\Left{A_1},\Right{A_2}}}
}

\newcommand{\PFImpLRule}{
  \Infer[\Imp$-T$]
  {\VSeq{\Left{A_1\Imp A_2}}}
  {\VSeq{\Left{A_1\Imp A_2},\Right{A_1}} & \VSeq{\Left{A_2}}}
}

\newcommand{\PFOrRLRule}{
  \Infer[\Or$-F$_1]
  {\VSeq{\Right{A_1 \Or A_2}}}
  {\VSeq{\Right{A_1}}}
}

\newcommand{\PFOrRRRule}{
  \Infer[\Or$-F$_2]
  {\VSeq{\Right{A_1 \Or A_2}}}
  {\VSeq{\Right{A_2}}}
}

\newcommand{\PFOrLRule}{
  \Infer[\Or$-T$]
  {\VSeq{\Left{A_1 \Or A_2}}}
  {\VSeq{\Left{A_1}} & \VSeq{\Left{A_2}}}
}

\newcommand{\PFTopRRule}{
  \Infer[\Top$-F$]
  {\VSeq{\Right{\Top}}}
  {}
}

\newcommand{\PFBotLRule}{
  \Infer[\Bot$-T$]
  {\VSeq{\Left{\Bot}}}
  {}
}

\renewcommand{\figline}{\\[2em]}

\begin{figure}
  \begin{center}
    \fbox{
      \begin{tabular}{c}
        \begin{tabular}{ccc}
          \PFInitRule
          &
          \hspace{1cm}
          \PFTopRRule
          &
          \hspace{1cm}
          \mbox{No rule for $\Top$-T}
        \end{tabular}
        \figline
        \begin{tabular}{cc}
          \PFAndRRule
          &
          \hspace{1cm}
          \PFAndLRule
        \end{tabular}
        \figline
        \begin{tabular}{ccc}
          \PFOrRLRule
          &
          \hspace{10pt}
          \PFOrRRRule
          &
          \hspace{10pt}
          \PFOrLRule
        \end{tabular}
        \figline
        \begin{tabular}{cc}
          \mbox{No rule for $\Bot$-F}
          &
          \hspace{20pt}
          \PFBotLRule
        \end{tabular}
        \figline
        \begin{tabular}{cc}
          \PFImpRRule
          &
          \hspace{20pt}
          \PFImpLRule
        \end{tabular}
      \end{tabular}
    }
  \end{center}
  \caption{Forward inference rules}
  \label{prop.forward}
\end{figure}

%%% Local Variables:
%%% mode: latex
%%% TeX-master: "../thesis"
%%% End:


\subsection{Soundness and completeness}

\begin{lemma}[Inversion lemma]
  \begin{itemize}
  \item[]
  \item If $\CSeq{\Psi}{\Delta,A\And B}{\delta}$ then
    $\CSeq{\Psi}{\Delta,A, B}{\delta}$.
  \item If $\CSeq{\Psi}{\Delta,\Ex x.\ A(x)}{\delta}$ then
    $\CSeq{\Psi}{\Delta,A(a)}{\delta}$ where $a\not\in\Psi,\Delta,\delta$.
  \end{itemize}
\end{lemma}

\begin{proof} Easy inductions. \end{proof}

\begin{theorem}[Contraction]
  \label{constr.thm.contract}
  The rule
  \[
  \Infer[$Contract$]
  {\CSeq{\Psi}{\Gamma,A}{\gamma}}
  {\CSeq{\Psi}{\Gamma,A,A'}{\gamma} & \Psi\Entails A\Ueq A'}
  \]
  \noindent
  is admissible in $\CB$.
\end{theorem}

\begin{definition}
  We call a contraction step \emph{syntactic} if the formulas are
  syntactically identical.  In that case, the rule
  \[
  \Infer[$Contract$]
  {\CFSeq{\Psi}{\Gamma, A}{\gamma}}
  {\CFSeq{\Psi}{\Gamma, A, A}{\gamma}}
  \]
  holds, since there are no nontrivial equations arising from
  $A\Ueq A$.
\end{definition}


\begin{proof}
  Induction on the derivation of $\CSeq{\Psi}{\Gamma,A,A'}{\gamma}$.
  If $A$ is not principal in the last rule applied, then use the
  induction hypotheses and the same rule.  Otherwise, $A$ is
  principal.  We show some representative cases.
  \begin{description}
  \item[Case:]
    \[
    \Infer
    {\CSeq{\Psi}{\Gamma,A_1\And A_2,A_1'\And A_2'}{\gamma}}
    {\CSeq{\Psi}{\Gamma,A_1, A_2,A_1'\And A_2'}{\gamma}}
    \]
    where $\Psi\Entails A_1\And A_2\Ueq A_1'\And A_2'$.
    By the definition of an entailment relation, we have that
    $\Psi\Entails A_1\And A_2\Ueq A_1'\And A_2'$
    By the inversion lemma we have a derivation of
    $\CSeq{\Psi}{\Gamma,A_1, A_2,A_1', A_2'}{\gamma}$.
    Then use the induction hypothesis (twice).  The side-conditions
    are valid since if $\Psi\Entails A_1\And A_2$ then $\Psi\Entails A_i$
    and the conjuncts of $A_1\Ueq A_1'$ are a subset of the conjuncts
    of $A_1\And A_2\Ueq A_1'\And A_2'$.
  \item[Case:]
    \[
    \Infer[\Ex$-L$_a]
    {\CSeq{\Psi}{\Gamma,\Ex x.~A(x), B}{\gamma}}
    {\CSeq{\Psi}{\Gamma, A(a), B}{\gamma}}
    \]
    where $\Psi\Entails\Ex x~.A(x)\Ueq B$ and $a\not\in B$.  If $B$ is not an
    existential, then $\Psi\Entails\Bot$ so $\Psi\Entails A(a)\Ueq B$
    and we can use the IH directly, giving a derivation of
    $\CSeq{\Psi}{\Gamma, A(a)}{\gamma}$ and we can use $\Ex$-L.
    Otherwise $B=\Ex x.~B'(x)$ and we can use the inversion lemma for a
    derivation of ${\CSeq{\Psi}{\Gamma, A(a), B(b)}{\gamma}}$ where
    $a\neq b$, $b\not\in\Psi,\Gamma,A(a),\gamma$.  Then we have
    $\Psi\And a=b\Entails A(a)\Ueq B(b)$ and we have a derivation of
    ${\CSeq{\Psi\And a=b}{\Gamma, A(a)}{\gamma}}$ from which we can again use
    $\Ex$-L to derive ${\CSeq{\Psi\And a=b}{\Gamma, \Ex x.~A(x)}{\gamma}}$.
    Theorem \ref{constr.thm.strengthen} gives us a derivation of
    ${\CSeq{\Psi}{\Gamma, \Ex x.~A(x)}{\gamma}}$ since $a$ and $b$ do not
    occur in $\Gamma, \Ex x.~A(x), \gamma$.
  \end{description}
\end{proof}

\begin{lemma}
  \label{constr.lem.exists}
  If $\CSeq{\Psi(x)}{\Gamma}{\gamma}$ and $x\not\in\Gamma,\gamma$ then
  $\CSeq{\Ex x.~\Psi(x)}{\Gamma}{\gamma}$.
\end{lemma}
\begin{proof}
  This is an odd lemma, since in the rules currently given
  for the backward calculus, no formulas ever enter the constraints,
  and the proof is trivial.  We will of course need to revisit this
  proof when we introduce rules that add formulas to the constraint context.
\end{proof}

\begin{theorem}[Soundness]
  If there is a derivation of $\CFSeq{\Psi}{\Delta}{\delta}$ then there is
  a derivation of any $\CSeq{\Psi'}{\Delta'}{\delta'}$ where
  $(\CFSeq{\Psi}{\Delta}{\delta})\Subsumes (\CSeq{\Psi'}{\Delta'}{\delta'})$
\end{theorem}

\begin{proof}
  We have $\Psi'\Entails\Psi$, $\Delta \subseteq\Delta'$ and $\delta \subseteq\delta'$.
  We proceed by induction on the derivation.   We show some representative cases.
  \begin{description}
  \item[Case:]
    \[
    \Infer
    {\CFSeq{p\Ueq p'}{p}{p'}}
    {}
    \]
    Then $\Delta'=\Delta'', p$, $\delta=\delta'=p'$ and
    $\Psi'\Entails p\Ueq p'$.  Then the Init rule gives a derivation
    of $\CSeq{\Psi'}{\Delta'', p}{p'}$ as required.
  \item[Case:]
    \[
    \Infer[\And$-R$]
    {\CFSeq{\Psi_1\And\Psi_2}{\Gamma_1\Union\Gamma_2}{A_1\And A_2}}
    { \CFSeq{\Psi_1}{\Gamma_1}{A_1} & \CSeq{\Psi_2}{\Gamma_2}{A_2} }
    \]
    We have that $\Gamma_1\Union\Gamma_2 \subseteq \Gamma$,
    $\gamma_1\Union\gamma_2 \subseteq \gamma$ and
    $\Phi\Entails\Phi_1\And\Phi_2$.  Using multiset and entailment
    reasoning, we can use the IH to derive
    $\CSeq{\Psi}{\Gamma}{A_1}$ and $\CSeq{\Psi}{\Gamma}{A_2}$.  Then
    $\And$-R yields $\CSeq{\Psi}{\Gamma}{A_1\And A_2}$.
  \item[Case:]
    \[
    \Infer[\All$-R$^x]
    {\CFSeq{\All x.~\Psi(x)}{\Delta}{\All x.~A(x)}}
    {\CFSeq{\Psi(x)}{\Delta}{A(x)}}
    \]
    Since $\Psi'\Entails\All x.~\Psi(x)$, $\Psi'\Entails\Psi(x)$,
    so we can use the IH
    to obtain $\CSeq{\Psi'}{\Delta}{A(x)}$.  Then

    \[
    \Infere[$Weaken$]
    {\CSeq{\Psi'}{\Delta'}{\All x.~A(x)}}
    {
      \Infer[\All$-R$]
      {\CSeq{\Psi'}{\Delta}{\All x.~A(x)}}
      {\CSeq{\Psi'}{\Delta}{A(x)}}
    }
    \]
  \item[Case:]
    \[
    \Infer[\All$-L$]
    {\CFSeq{\Psi}{\Delta, \All x.~A(x)}{\delta}}
    {\CFSeq{\Psi}{\Delta, A(t)}{\delta}}
    \]
    By IH we have a derivation of $\CSeq{\Psi'}{\Delta, A(t)}{\delta}$.
    If $\All x.~A(x)\in\Delta$ then we can use
    $\All$-L directly.  Otherwise weaken the derivation and use $\All$-L.
  \item[Case:]
    \[
    \Infer[$Contract$]
    {\CFSeq{\Psi \And A \doteq A'}{\Delta, A}{\delta}}
    {\CFSeq{\Psi}{\Delta, A, A'}{\delta}}
    \]
    Since $\Psi'\Entails \Psi\And A\Ueq A'$, $\Psi'\Entails\Psi$.  We can thus use the IH
    to derive $\CSeq{\Psi'}{\Delta, A, A'}{\delta}$.
    Since $\Psi'\Entails A\Ueq A'$ and contraction is admissible
    (Theorem~\ref{constr.thm.contract})
    we have a derivation of $\CSeq{\Psi'}{\Delta, A}{\delta}$.  Weaken this
    derivation to get $\CSeq{\Psi'}{\Delta', A}{\delta}$
  \item[Case:]
    \[
    \Infer[\Ex$-C$]
    {\CFSeq{\Ex x.~\Psi(x)}{\Delta}{\delta}}
    {\CFSeq{\Psi(x)}{\Delta}{\delta} & x\not\in\Vars(\Delta,\delta)}
    \]
    Since $\Psi'\Entails \Ex x.~\Psi(x)$, there is some $t$ such that
    $\Psi'\Entails \Psi(t)$.
    By Theorem~\ref{constr.thm.subst} we have a derivation of
    $(\CFSeq{\Psi(x)}{\Delta}{\delta})\Set{x\to t}$.  Since
    $x\not\in\Vars(\Delta,\delta)$ this is also a derivation of
    $\CFSeq{\Psi(t)}{\Delta}{\delta}$.  By IH, we have
    $\CSeq{\Psi'}{\Delta}{\delta}$.  Weakening gives the final result.
  \end{description}
\end{proof}

\begin{theorem}[Completeness]
  If $\CSeq{\Psi}{\Delta}{\delta}$ then there exists $\Psi',\Delta',\delta'$ such
  that $(\CFSeq{\Psi'}{\Delta'}{\delta'})\Subsumes (\CSeq{\Psi}{\Delta}{\delta})$.
\end{theorem}

\begin{proof}
  Induction on the derivation.
  \begin{description}
  \item[Case:]
    \[
    \Infer[$Init$]
    {\CSeq{\Psi}{p}{p'}}
    {\Psi \models p \doteq p'}
    \]
    $(\CFSeq{p\Ueq p'}{p}{p'})\Subsumes (\CSeq{\Psi}{\Delta, p}{p'})$ since
    $\Psi \models p \doteq p'$.
  \item[Case:]
    \[
    \Infer[\And$-R$]
    {\CSeq{\Psi}{\Gamma}{A_1\And A_2}}
    { \CSeq{\Psi}{\Gamma}{A_1} & \CSeq{\Psi}{\Gamma}{A_2} }
    \]
    By IH, we have $q_1=\CFSeq{\Psi_1}{\Gamma_1}{\gamma_1}$ and
    $q_2=\CFSeq{\Psi_2}{\Gamma_2}{\gamma_2}$ where $\Psi\Entails \Psi_1$ and
    $\Psi\Entails \Psi_2$.  If $A_i\not\in\gamma_i$ then $q_i$ will do.  Otherwise
    we use $\And$-R to get
    $\CFSeq{\Psi_1\And\Psi_2}{\Gamma_1\Union\Gamma_2}{A_1\And A_2}$.  By entailment
    reasoning, $\Psi\Entails\Psi_1\And\Psi_2$.  We may need to follow this
    derivation by some number of syntatic contractions, yielding the result.
  \item[Case:]
    \[
    \Infer[\All$-R$^a]
    {\CSeq{\Psi}{\Gamma}{\All x.~A(x)}}
    {\CSeq{\Psi}{\Gamma}{A(a)}}
    \]
    By IH, there is $q=(\CFSeq{\Psi'}{\Delta}{\delta})\Subsumes(\CSeq{\Psi}{\Gamma}{A(a)})$.
    If $\delta=\emptyset$ then $q\Subsumes\CSeq{\Psi}{\Gamma}{\All x.~A(x)}$.
    Otherwise we can use $\All$-R to yield
    $\CFSeq{\All x.~\Psi'(x)}{\Delta}{\All x.~A(x)}$.  Since
    $\Psi\Entails\Psi'(a)$,
    $\Psi\Entails\All x.~\Psi'(x)$ as required.
  \item[Case:]
    \[
    \Infer[\All$-L$]
    {\CSeq{\Psi}{\Gamma, \All x.~A(x)}{\gamma}}
    {\CSeq{\Psi}{\Gamma, \All x.~A(x), A(t)}{\gamma}}
    \]
    By IH, there is $q=\CFSeq{\Psi'}{\Gamma'}{\gamma'}$ such that
    $q\Subsumes(\CSeq{\Psi}{\Gamma, \All x.~A(x), A(t)}{\gamma})$.
    If $A(t)\not\in\Gamma'$, then $q$ will do.  Otherwise, use $\All$-L to
    derive $\CFSeq{\Psi'}{\Gamma',\All x.~A(x)}{\gamma'}$.  If the number of
    occurrances of $\All x.~A(x)$ in $\Gamma'$ is the same as that in
    $\Gamma$, we need a syntactic contraction step.
  \end{description}
\end{proof}

\section{Free-variable calculus}


\newcommand{\CEInit}{
  \Infer[$Init$]
  {\CFSeq{p \doteq p'}{p}{p'}}
  {}
}

\newcommand{\CEAllR}{
  \Infer[\All$-R$^x]
  {\CFSeq{\All x.~\Psi(x)}{\Gamma}{\All x.~A(x)}}
  {\CFSeq{\Psi(x)}{\Gamma}{A(x)}}
}

\newcommand{\CEAllL}{
  \Infer[\All$-L$]
  {\CFSeq{\Psi}{\Gamma, \All x.~A(x)}{\gamma}}
  {\CFSeq{\Psi}{\Gamma, A(x)}{\gamma}}
}

\newcommand{\CECon}{
  \Infer[$Contract$]
  {\CFSeq{\Psi \And A \doteq A'}{\Gamma, A}{\gamma}}
  {\CFSeq{\Psi}{\Gamma, A, A'}{\gamma}}
}

\newcommand{\CEExC}{
  \Infer[\Ex$-C$]
  {\CFSeq{\Ex x.~\Psi(x)}{\Gamma}{\gamma}}
  {\CFSeq{\Psi(x)}{\Gamma}{\gamma} & x\not\in\Vars(\Gamma,\gamma)}
}

\newcommand{\CEExR}{
  \Infer[\Ex$-R$]
  {\CFSeq{\Psi}{\Gamma}{\Ex x.~A(x)}}
  {\CFSeq{\Psi}{\Gamma}{A(x)}}
}

\newcommand{\CEExL}{
  \Infer[\Ex$-L$^a]
  {\CFSeq{\Psi}{\Gamma, \Ex x.~A(x)}{\gamma}}
  {\CFSeq{\Psi}{\Gamma, A(x)}{\gamma}}
}

\newcommand{\CEAndR}{
  \Infer[\And$-R$]
  {\CFSeq{\Psi_1\And\Psi_2}{\Gamma_1\Union\Gamma_2}{A_1\And A_2}}
  { \CFSeq{\Psi_1}{\Gamma_1}{A_1} & \CFSeq{\Psi_2}{\Gamma_2}{A_2} }
}

\begin{figure}

  In Init, the terms occurring in $p$ and $p'$ are distinct variables.
  \begin{center}
    {\small
      \fbox{
        \begin{tabular}{c}
          \begin{tabular}{ccc}
            \CEInit
            &
            \hspace{2em}
            \CEExR
            &
            \hspace{2em}
            \CEExL
          \end{tabular}
          \\[2em]
          \begin{tabular}{cc}
            \CEAllR
            &
            \hspace{2em}
            \CEAllL
            \\[2em]
            \CECon
            &
            \hspace{2em}
            \CEExC
          \end{tabular}
          \\[2em]
          \begin{tabular}{c}
            \CEAndR
          \end{tabular}
        \end{tabular}
      }
    }
  \end{center}
  \caption{The free-variable calculus.}
  \label{constr.free}
\end{figure}

%%% Local Variables:
%%% mode: latex
%%% TeX-master: "../thesis"
%%% End:


\begin{theorem}[Simplify]
  If $\CFSeq{\Psi}{\Gamma}{\gamma}$ and $\Psi'\Entails\Psi$ then
  $\CFSeq{\Psi'}{\Gamma}{\gamma}$.
\end{theorem}

\begin{example}

  Let $\xi=(\All x.~p(x)\Imp p(f(x)))\Imp c=d\Imp p(c)\Imp p(f(d))$.

  \begin{itemize}
  \item[]
  \item Label the subformulas
    \[
    \begin{array}{l@{\hspace{2em}}l@{\hspace{2em}}l}
      l_1^-(x) : p(x)\Imp p(f(x))
      &
      l_2^- : \All x.~l_1(x)
      &
      l_3^+ : p(c)\Imp p(f(d))
      \\
      l_4^+ : c=d\Imp l_3
      &
      l_5^+ : l_2\Imp l_4
    \end{array}
    \]
  \item Calculate the initial rules
    \[
    \begin{array}{c}
      \begin{array}{ccc}
        \Infer[r_1]
        { \CFSeq{\Psi_1\And\Psi_2}{l_1(x)}{\cdot} }
        { \CFSeq{\Psi_1}{p(f(x))}{\cdot} & \CFSeq{\Psi_2}{\cdot}{p(x)} }
        &
        \hspace{2em}
        \Infer[r_2]
        { \CFSeq{\Psi}{l_2}{\cdot} }
        { \CFSeq{\Psi}{ l_1(x) }{\cdot} }
        &
        \hspace{2em}
        \Infer[r_3]
        { \CFSeq{\Psi}{\cdot}{l_3} }
        { \CFSeq{\Psi}{ p(c) }{p(f(d))} }
      \end{array}
      \\[2em]
      \begin{array}{ccc}
        \Infer[r_4]
        { \CFSeq{\Psi}{\cdot}{l_4} }
        { \CFSeq{\Psi}{ c=d }{l_3} }
        &
        \hspace{2em}
        \Infer[r_5]
        { \CFSeq{\Psi}{\cdot}{l_5} }
        { \CFSeq{\Psi}{ l_2 }{l_4} }
        &
        \hspace{2em}
        \Infer[r_6]
        { \CFSeq{\Psi\And c=d}{ c=d }{\cdot} }
        { \CFSeq{\Psi}{\cdot}{\cdot} }
      \end{array}
    \end{array}
    \]

  \item The goal is $\CFSeq{\Top}{\cdot}{l_5}$
  \item Saturate the database

    \renewcommand{\Sa}{x_1\Ueq c\And x_2\Ueq c}
    \renewcommand{\Qa}{\infer{\FSeq{p(x_1)}{p(x_2)}}{\Sa}}

    \renewcommand{\Sb}{\Ex x_5.~x_3\Ueq f(x_5)\And x_4\Ueq f(x_5)}
    \renewcommand{\Qb}{\infer{\FSeq{p(x_3)}{p(x_4)}}{\Sb}}

    \renewcommand{\Sc}{x_3\Ueq f(y_1)\And x_2 \Ueq  y_1\And \Sa\And (\Sb)}
    \renewcommand{\Qc}{\infer{\FSeq{p(x_1),l_1(y_1)}{p(x_4)}}{\Sc}}

    \renewcommand{\Sd}{x_1\Ueq c\And y_1\Ueq c \And (\Ex x_5.~f(y_1)\Ueq f(x_5)\And x_4\Ueq f(x_5))}
    \renewcommand{\Qd}{\Infer{\FSeq{p(x_1),l_1(y_1)}{p(x_4)}}{\Sd}}

    \renewcommand{\Se}{x_1\Ueq c\And y_1\Ueq c \And (\Ex x_5.~f(c)\Ueq f(x_5)\And x_4\Ueq f(x_5))}
    \renewcommand{\Qe}{\Infer{\FSeq{p(x_1),l_2}{p(x_4)}}{\Se}}

    \renewcommand{\Sf}{\Ex y_1.~x_1\Ueq c\And y_1\Ueq c \And (\Ex x_5.~f(c)\Ueq f(x_5)\And x_4\Ueq f(x_5))}
    \renewcommand{\Qf}{\Infer{\FSeq{p(x_1),l_2}{p(x_4)}}{\Sf}}

    \renewcommand{\Sg}{x_1\Ueq c\And (\Ex x_5.~f(c)\Ueq f(x_5)\And x_4\Ueq f(x_5))}
    \renewcommand{\Qg}{\Infer{\FSeq{p(x_1),l_2}{p(x_4)}}{\Sg}}

    \renewcommand{\Sh}{x_1\Ueq c\And x_4\Ueq f(d)\And x_1\Ueq c\And (\Ex x_5.~f(c)\Ueq f(x_5)\And x_4\Ueq f(x_5))}
    \renewcommand{\Qh}{\Infer{\FSeq{l_2}{l_3}}{\Sh}}

    \renewcommand{\Si}{\Ex x_5.~f(c)\Ueq f(x_5)\And f(d)\Ueq f(x_5)}
    \renewcommand{\Qi}{\Infer{\FSeq{l_2}{l_3}}{\Si}}

    \renewcommand{\Sj}{c=d\And(\Ex x_5.~f(c)\Ueq f(x_5)\And f(d)\Ueq f(x_5))}
    \renewcommand{\Qj}{\Infer{\FSeq{l_2,c=d}{l_3}}{\Sj}}

    \renewcommand{\Sk}{c=d\And(\Ex x_5.~f(c)\Ueq f(x_5)\And f(d)\Ueq f(x_5))}
    \renewcommand{\Qk}{\Infer{\FSeq{l_2}{l_4}}{\Sk}}

    \renewcommand{\Sl}{c=d\And(\Ex x_5.~f(c)\Ueq f(x_5)\And f(d)\Ueq f(x_5))}
    \renewcommand{\Ql}{\Infer{\FSeq{\cdot}{l_5}}{\Sl}}

    \begin{tabbing}
    $\Qa$ \` 1. Init\\[1em]
    $\Qb$ \` 2. Init\\[1em]
    $\Qc$ \` 3. $r_1[1, 2]$\\[1em]
    $\Qd$ \` 4. $\Set{x_3\to f(y_1),x_2\to y_1}$\\[1em]
    $\Qe$ \` 5. $r_2[4], \Set{y_1\to c}$\\[1em]
    $\Qf$ \` 6. $\Ex$-C$[5]$\\[1em]
    $\Qg$ \` 7. Simp.\\[1em]
    $\Qh$ \` 8. $r_3[7]$\\[1em]
    $\Qi$ \` 9. $\Set{x_1\to c,x_4\to f(d)}$\\[1em]
    $\Qj$ \` 10. $r_6[9]$\\[1em]
    $\Qk$ \` 11. $r_4[10]$\\[1em]
    $\Ql$ \` 12. $r_5[11]$\\[1em]
    \end{tabbing}
  \end{itemize}

  \noindent
  The final sequent subsumes the goal because
  $\Top\Entails c=d\And(\Ex x_5.~f(c)\Ueq f(x_5)\And f(d)\Ueq f(x_5))$
  with $x_5\to d$.  (If $f$ is uninterpreted by the domain, we
  could have simplified the constraint of sequent 4 to
  $x_1\Ueq c\And y_1\Ueq c \And x_4\Ueq c$ making the remaining derivation
  simpler.)

\end{example}

\section{Focused derivations}

\begin{definition}[Formulas]
  \[
  \begin{array}{rrl}
    \mbox{Sorts} &\sigma & ::= \SortI \Sep \ldots \\
    \mbox{Domain Atoms} & e & ::= x < y \Sep x = y \Sep \ldots \\
    \mbox{Positive Formulas} & A^+ & ::= e^+ \Sep p^+ \Sep A^+\AndP A^+\Sep \TopP \Sep A^+\Or A^+ \Sep \Down A^- \Sep \Ex x:\sigma.\ A^+ \\
    \mbox{Negative Formulas} & A^- & ::= p^- \Sep A^-\AndN A^- \Sep \TopN \Sep A^+\Imp A^- \Sep \All x:\sigma.\ A^- \\
  \end{array}
  \]
\end{definition}

\begin{definition}[Erasure]
  \[
  \begin{array}{cc}
    \begin{array}{rl}
      \Erase{e^+}&=e^+\\
      \Erase{p^+}&=p^+\\
      \Erase{p^-}&=p^-\\
      \Erase{A_1^+\AndP A_2^+} &= \Erase{A_1^+}\AndP\Erase{A_2^+} \\
      \Erase{\TopP} &= \Top \\
      \Erase{A_1^+\Or A_2^+} &= \Erase{A_1^+}\Or\Erase{A_2^+} \\
      \Erase{\Ex x:\sigma.~A^+} &= \Ex x:\sigma.\Erase{A^+}\\
      \Erase{A_1^-\AndN A_2^-} &= \Erase{A_1^-}\AndN\Erase{A_2^-} \\
      \Erase{\TopN} &= \Top \\
      \Erase{A_1^+\Imp A_2^-} &= \Erase{A_1^+}\Imp\Erase{A_2^-} \\
      \Erase{\All x:\sigma.~A^-} &= \All x:\sigma.~\Erase{A^-}\\
    \end{array}
    &
    \hspace{2em}
    \begin{array}{rl}
      \Erase{\CCSeq{\Psi}{\Gamma}{\gamma}}&=\CCSeq{\Psi}{\Erase{\Gamma}}{\Erase{\gamma}} \\
      \Erase{\cdot}&=\cdot\\
      \Erase{\Gamma,\gamma}&=\Erase{\Gamma},\Erase{\gamma}\\
      \Erase{[A]}&=\Erase{A}\\
      \Erase{\ASet{A}}&=\Erase{A}\\
    \end{array}
  \end{array}
  \]
\end{definition}


\newcommand{\CCFocusR}{
  \Infer[$focus-R$]
  {\CCSeq{\Psi}{\Gamma}{A^+}}
  {\CCSeq{\Psi}{\Gamma}{[A^+]}}
}

\newcommand{\CCFocusL}{
  \Infer[$focus-L$]
  {\CCSeq{\Psi}{\Gamma,A^-}{\gamma}}
  {\CCSeq{\Psi}{\Gamma, [A^-]}{\gamma}}
}

\newcommand{\CCEtaP}{
  \Infer[\eta^+]
  {\CCSeq{\Psi}{\Gamma,p^+}{\gamma}}
  {\CCSeq{\Psi}{\Gamma, \ASet{p^+}}{\gamma}}
}

\newcommand{\CCEtaN}{
  \Infer[\eta^-]
  {\CCSeq{\Psi}{\Gamma}{p^-}}
  {\CCSeq{\Psi}{\Gamma}{\ASet{p^-}}}
}

\newcommand{\CCIdP}{
  \Infer[$id$^+]
  {\CCSeq{\Psi}{\Gamma,\ASet{A_1^+}}{[A_2^+]}}
  {\Delta;\Psi\Entails A_1\Ueq A_2}
}

\newcommand{\CCIdN}{
  \Infer[$id$^+]
  {\CCSeq{\Psi}{\Gamma,[A_1^-]}{\ASet{A_2^-}}}
  {\Delta;\Psi\Entails A_1\Ueq A_2}
}

\newcommand{\CCUpR}{
  \Infer[\Up$-R$]
  {\CCSeq{\Psi}{\Gamma}{\Up A^+}}
  {\CCSeq{\Psi}{\Gamma}{A^+}}
}

\newcommand{\CCUpL}{
  \Infer[\Up$-L$]
  {\CCSeq{\Psi}{\Gamma,[\Up A^-]}{\gamma}}
  {\CCSeq{\Psi}{\Gamma, A^-}{\gamma}}
}

\newcommand{\CCDownR}{
  \Infer[\Down$-R$]
  {\CCSeq{\Psi}{\Gamma}{[\Down A^-]}}
  {\CCSeq{\Psi}{\Gamma}{A^-}}
}

\newcommand{\CCDownL}{
  \Infer[\Down$-L$]
  {\CCSeq{\Psi}{\Gamma, \Down A^-}{\gamma}}
  {\CCSeq{\Psi}{\Gamma, A^-}{\gamma}}
}

\newcommand{\CCTopPR}{
  \Infer[\TopP$-R$]
  {\CCSeq{\Psi}{\Gamma}{[\TopP]}}
  {}
}

% \newcommand{\CCTopPL}{
%   \Infer[\TopP$-L$]
%   {\CCSeq{\Psi}{\Gamma,\TopP}{\gamma}}
%   {\CCSeq{\Psi}{\Gamma}{\gamma}}
% }

\newcommand{\CCTopNR}{
  \Infer[\TopN$-R$]
  {\CCSeq{\Psi}{\Gamma}{\TopN}}
  {}
}

% \newcommand{\CCTopNL}{
%   \Infer[\TopN$-L$]
%   {\CCSeq{\Psi}{\Gamma,[\TopN]}{\gamma}}
%   {}
% }

\newcommand{\CCAndPR}{
  \Infer[\AndP$-R$]
  {\CCSeq{\Psi}{\Gamma}{[A_1^+\AndP A_2^+]}}
  {\CCSeq{\Psi}{\Gamma}{[A_1^+]} & \CCSeq{\Psi}{\Gamma}{[A_2^+]}}
}

\newcommand{\CCAndPL}{
  \Infer[\AndP$-L$]
  {\CCSeq{\Psi}{\Gamma,A_1^+\AndP A_2^+}{\gamma}}
  {\CCSeq{\Psi}{\Gamma,A_1^+,A_2^+}{\gamma}}
}

\newcommand{\CCAndNR}{
  \Infer[\AndN$-R$]
  {\CCSeq{\Psi}{\Gamma}{A_1^-\AndN A_2^-}}
  {\CCSeq{\Psi}{\Gamma}{A_1^-} & \CCSeq{\Psi}{\Gamma}{A_2^-}}
}

\newcommand{\CCAndNL}{
  \Infer[\AndN$-L$_i]
  {\CCSeq{\Psi}{\Gamma,[A_1^-\AndN A_2^-]}{\gamma}}
  {\CCSeq{\Psi}{\Gamma,[A_i^-]}{\gamma}}
}

\newcommand{\CCOrR}{
  \Infer[\Or$-R$_i]
  {\CCSeq{\Psi}{\Gamma}{[A_1^+\Or A_2^+]}}
  {\CCSeq{\Psi}{\Gamma}{A_i^+}}
}

\newcommand{\CCOrL}{
  \Infer[\Or$-L$]
  {\CCSeq{\Psi}{\Gamma,A_1^+\Or A_2^+}{\gamma}}
  {\CCSeq{\Psi}{\Gamma,A_1}{\gamma} & \CCSeq{\Psi}{\Gamma,A_2}{\gamma}}
}

\newcommand{\CCImpR}{
  \Infer[\Imp$-R$]
  {\CCSeq{\Psi}{\Gamma}{A_1^+\Imp A_2^-}}
  {\CCSeq{\Psi}{\Gamma,A_1^+}{A_2^-}}
}

\newcommand{\CCImpL}{
  \Infer[\Imp$-L$]
  {\CCSeq{\Psi}{\Gamma,[A_1^+\Imp A_2^-]}{\gamma}}
  {\CCSeq{\Psi}{\Gamma,A_1^+\Imp A_2^-}{[A_1^+]} & \CCSeq{\Psi}{\Gamma,[A_2^-]}{\gamma}}
}

\newcommand{\CCAllR}{
  \Infer[\All$-R$^a]
  {\CCSeq{\Psi}{\Gamma}{\All x:\sigma.~A^-(x)}}
  {\CCSeq[\Delta,a:\sigma]{\Psi}{\Gamma}{A^-(a)}}
}

\newcommand{\CCAllL}{
  \Infer[\All$-L$]
  {\CCSeq{\Psi}{\Gamma, [\All x:\sigma.~A^-(x)]}{\gamma}}
  {\CCSeq{\Psi}{\Gamma, [A^-(t)]}{\gamma} & \Delta\Entails t:\sigma}
}

\newcommand{\CCExR}{
  \Infer[\Ex$-R$]
  {\CCSeq{\Psi}{\Gamma}{[\Ex x:\sigma.~A^+(x)]}}
  {\CCSeq{\Psi}{\Gamma}{[A^+(t)]} & \Delta\Entails t:\sigma}
}

\newcommand{\CCExL}{
  \Infer[\Ex$-L$^a]
  {\CCSeq{\Psi}{\Gamma, \Ex x:\sigma.~A^+(x)}{\gamma}}
  {\CCSeq[\Delta,a:\sigma]{\Psi}{\Gamma, A^+(a)}{\gamma}}
}

\newcommand{\CCDomR}{
  \Infer[$dom-R$]
  {\CCSeq{\Psi}{\Gamma}{[e^+]}}
  {\Psi\Entails e^+}
}

\newcommand{\CCDomL}{
  \Infer[$dom-L$]
  {\CCSeq{\Psi}{\Gamma, e^+}{\gamma}}
  {\CCSeq{\Psi, e^+}{\Gamma}{\gamma}}
}

\begin{figure}
  \begin{center}
    {\small
      \begin{tabular}{c}
        \\[1em]
        \begin{tabular}{ccc}
          \CCFocusR
          &\hsp
          \CCFocusL
          &\hsp
          \CCEtaP
        \end{tabular}
        \\[2em]
        \begin{tabular}{ccc}
          \CCEtaN
          &\hsp
          \CCIdP
          &\hsp
          \CCIdN
        \end{tabular}
        \\[2em]
        \begin{tabular}{ccc}
          \CCUpR
          &\hsp
          \CCUpL
          &\hsp
          \CCDownR
        \end{tabular}
        \\[2em]
        \begin{tabular}{ccc}
          \CCDownL
          &\hsp
          \CCDomR
          &\hsp
          \CCDomL
        \end{tabular}
        \\[2em]
        \begin{tabular}{cc}
          \CCAllR
          &\hsp
          \CCAllL
        \end{tabular}
        \\[2em]
        \begin{tabular}{cc}
          \CCExR
          &\hsp
          \CCExL
        \end{tabular}
        \\[2em]
        \begin{tabular}{cc}
          \CCTopPR
          &\hsp
          \CCTopNR
        \end{tabular}
        \\[2em]
        \begin{tabular}{cc}
          \CCAndPR
          &\hsp
          \CCAndPL
        \end{tabular}
        \\[2em]
        \begin{tabular}{cc}
          \CCAndNR
          &\hsp
          \CCAndNL
        \end{tabular}
        \\[2em]
        \begin{tabular}{cc}
          \CCOrR
          &\hsp
          \CCOrL
        \end{tabular}
        \\[2em]
        \begin{tabular}{cc}
          \CCImpR
          &\hsp
          \CCImpL
        \end{tabular}
      \end{tabular}
    }
  \end{center}
  \caption{Focused sequent calculus}
  \label{constr.focus}
\end{figure}

%%% Local Variables:
%%% mode: latex
%%% TeX-master: "../thesis"
%%% End:


\begin{theorem}[Focal substitution]
  \begin{itemize}
  \item[]
  \item If $\CCSeq{\Psi_1}{\Gamma_1}{[A^+]}$ and $\CCSeq{\Psi_2}{\Gamma_2,\ASet{A^+}}{\gamma}$
    then $\CCSeq{\Psi_1\And\Psi_2}{\Gamma_1,\Gamma_2}{\gamma}$
  \item If $\CCSeq{\Psi_1}{\Gamma_1}{\ASet{A^-}}$ and $\CCSeq{\Psi_2}{\Gamma_2,[A^-]}{\gamma}$
    then $\CCSeq{\Psi_1\And\Psi_2}{\Gamma_1,\Gamma_2}{\gamma}$
  \end{itemize}
\end{theorem}

\begin{proof}
TODO
\end{proof}

\begin{theorem}[Cut admissibility]
  \newcommand{\GammaF}{\PossFocus{\Gamma}}
  \newcommand{\GammaI}{\PossInvert{\Gamma}}
  \newcommand{\gammaF}{\PossFocus{\gamma}}
  \newcommand{\gammaI}{\PossInvert{\gamma}}
  \begin{enumerate}
  \item[]
  \item If $\CCSeq{\Psi}{\Gamma
    }{[A^+]}$ and $\CCSeq{\Psi}{\GammaI,A^+}{\gammaI}$
    then $\CCSeq{\Psi}{\Gamma}{\gamma}$.
  \item If $\CCSeq{\Psi}{\Gamma}{A^-}$ and $\CCSeq{\Psi}{\Gamma,[A^-]}{\gamma}$
    then $\CCSeq{\Psi}{\Gamma}{\gamma}$.
  \item If $\CCSeq{\Psi}{\Gamma}{A^+}$ and $\CCSeq{\Psi}{\Gamma,A^+}{\gamma}$
    then $\CCSeq{\Psi}{\Gamma}{\gamma}$.
  \item If $\CCSeq{\Psi}{\Gamma}{A^-}$ and $\CCSeq{\Psi}{\Gamma,A^-}{\gamma}$
    then $\CCSeq{\Psi}{\Gamma}{\gamma}$.
  \end{enumerate}
\end{theorem}

\begin{theorem}[Identity expansion]
  \begin{itemize}
  \item[]
  \item If $\CCSeq{\Psi}{\Gamma}{\ASet{A^-}}$ then $\CCSeq{\Psi_1}{\Gamma}{A^-}$.
  \item If $\CCSeq{\Psi}{\Gamma, \ASet{A^+}}{\gamma}$ then
    $\CCSeq{\Psi}{\Gamma,A^+}{\gamma}$
  \end{itemize}
\end{theorem}

\begin{proof}
TODO
\end{proof}

\section{Related Work}

\section{Future Work}


\subsection{Non-convex domains}

DLO example



\newcommand{\CBOrC}{
  \Infer[\Or$-C$]
  {\CSeq{\Psi}{\Gamma}{C}}
  {\Psi \models \Psi_1 \Or \Psi_2 &
    \begin{array}[b]{l}
      \CSeq{\Psi_1}{\Gamma}{C} \\
      \CSeq{\Psi_2}{\Gamma}{C}
    \end{array}}
}

\newcommand{\CBExC}{
  \Infer[\Ex$-C$]
  {\CSeq{\Psi}{\Gamma}{C}}
  {\Psi \models \Ex x.~\Psi_1(x) & \CSeq{\Psi_1(x)}{\Gamma}{C}}
}

\newcommand{\CFOrC}{
  \Infer[\Or$-C$]
  {\CFSeq{\Psi_1 \Or \Psi_2}{\Gamma_1 \Union \Gamma_2}{\gamma_1 \Union \gamma_2}}
  {\CFSeq{\Psi_1}{\Gamma_1}{\gamma_1} & \CFSeq{\Psi_2}{\Gamma_2}{\gamma_2}}
}

\newcommand{\CFBotC}{
  \Infer[\Bot$-C$]
  {\CFSeq{\Bot}{\cdot}{\cdot}}
  {}
}

\begin{figure}
  \begin{center}
    {\small
      \fbox{
        \begin{tabular}{cc}
          \textbf{Backward rules} & \textbf{Forward rules}
          \\[2em]
          \CBOrC
          &
          \hspace{2em}
          \CFOrC
          \\[2em]
          \CBExC
          &
          \hspace{2em}
          \CFExC
        \end{tabular}
      }
    }
  \end{center}
  \caption{Non-convex rules}
  \label{constr.nonconvex}
\end{figure}


%%% Local Variables:
%%% mode: latex
%%% TeX-master: "../thesis"
%%% End:


\section{XXX Notes XXX}

\begin{itemize}
\item When we add rules that muck with $\Psi$ (e.g. in modal logic)
  we must revisit all of the proofs to add the cases for the new
  rules.
\item Example: contracting $\CFSeq{\Top}{p(x), p(y)}{p(y)}$
  gives $\CFSeq{\Top\And x=y}{p(y)}{p(y)}$ from which you
  need $\CFSeq{\Ex x.~\Top\And x=y}{p(y)}{p(y)}$.
\item Faced with a rule like
  \Infer
  {\CFSeq{\Psi}{x = y}{\cdot}}
  {\CFSeq{x=y\Imp\Psi}{\cdot}{\cdot}}

  we are in the position of repeatedly applying this rule.
  It may be necessary of course (the situation is similar to
  $\All$-L and $\Box$-L), but we should \emph{never
    contract} two hypotheses arising this way.
\end{itemize}

\section{CURRENT}

\subsection{The unfocused rules are admissible.}


\begin{theorem}[Soundness of the focusing system]
  If $\CCSeq{\Psi}{\Gamma}{\gamma}$ then
  $\CSeq{\Psi}{\Erase{\Gamma}}{\Erase{\gamma}}$.
\end{theorem}

\begin{proof}
  Easy induction.
\end{proof}

\begin{definition}[Stability]
  TODO
\end{definition}

\begin{lemma}
  If $\Gamma$ and $\gamma$ are stable, the following rules are derivable:
  \[
  \begin{array}{cc}
    \Infer[$susp-R$]
    {\CCSeq{\Psi}{\Gamma}{\Down A^-}}
    {\CCSeq{\Psi}{\Gamma}{\Susp{A^-}}}
    &
    \hspace{2em}
    \Infer[$susp-L$]
    {\CCSeq{\Psi}{\Gamma,\Up A^+}{\gamma}}
    {\CCSeq{\Psi}{\Gamma,\Susp{A^+}}{\gamma}}
    \\[2em]
    \Infer[\DU$-R$]
    {\CCSeq{\Psi}{\Gamma}{\DU A^+}}
    {\CCSeq{\Psi}{\Gamma}{A^+}}
    &
    \hspace{2em}
    \Infer[\UD$-L$]
    {\CCSeq{\Psi}{\Gamma,\UD A^-}{\gamma}}
    {\CCSeq{\Psi}{\Gamma,A^-}{\gamma}}
  \end{array}
  \]
\end{lemma}
\begin{proof}
  \[
  \begin{array}{cc}
    \infer
    {\CCSeq{\Psi}{\Gamma}{\Down A^-}}
    { \infer
      {\CCSeq{\Psi}{\Gamma}{[\Down A^-]}}
      { \infer
        {\CCSeq{\Psi}{\Gamma}{A^-}}
        {\CCSeq{\Psi}{\Gamma}{\Susp{A^-}}}
      }
    }
    &
    \hspace{2em}
    \infer
    {\CCSeq{\Psi}{\Gamma,\Up A^+}{\gamma}}
    { \infer
      {\CCSeq{\Psi}{\Gamma,[\Up A^+]}{\gamma}}
      { \infer
        {\CCSeq{\Psi}{\Gamma,A^+}{\gamma}}
        {\CCSeq{\Psi}{\Gamma,\Susp{A^+}}{\gamma}}
      }
    }
    \\[2em]
    \infer
    {\CCSeq{\Psi}{\Gamma}{\DU A^+}}
    { \infer
      {\CCSeq{\Psi}{\Gamma}{[\DU A^+]}}
      { \infer
        {\CCSeq{\Psi}{\Gamma}{\Down A^+}}
        {\CCSeq{\Psi}{\Gamma}{A^+}}
      }
    }
    &
    \hspace{2em}
    \infer
    {\CCSeq{\Psi}{\Gamma,\UD A^-}{\gamma}}
    { \infer
      {\CCSeq{\Psi}{\Gamma,[\UD A^-]}{\gamma}}
      { \infer
        {\CCSeq{\Psi}{\Gamma,\Up A^-}{\gamma}}
        {\CCSeq{\Psi}{\Gamma,A^-}{\gamma}}
      }
    }
  \end{array}
  \]
\end{proof}

\begin{theorem}[Completeness of the focusing system]
  If $\CSeq{\Psi}{\Erase{\Gamma}}{\Erase{\gamma}}$ and
  $\Gamma$ and $\gamma$ are stable then
  $\CCSeq{\Psi}{\Gamma}{\gamma}$.
\end{theorem}

\begin{corollary}

\end{corollary}

\begin{proof}
  \begin{description}
  \item[Case:]
    \[
    \Infer[\And$-R$]
    {\CSeq{\Psi}{\Erase{\Gamma}}{A_1\And A_2}}
    { \CSeq{\Psi}{\Erase{\Gamma}}{A_1} & \CSeq{\Psi}{\Erase{\Gamma}}{A_2} }
    \]

    \noindent
    Here there are two cases, depending on whether the pre-erased conjunction
    is positive or negative.
    \begin{description}
    \item[Case:]
      There exist $A_1^+$, $A_2^+$ such that
      $\gamma^+=(\Down\Up)^n(A_1^+\AndP A_2^+)$ where $A_1=\Erase{A_1^+}$ and
      $A_2=\Erase{A_2^+}$.  By IH, we have derivations
      of $\CCSeq{\Psi}{\Gamma}{A_i^+}$.  Then

      \renewcommand{\Sq}{\CCSeq{\Psi}{\Gamma}{A_2^+}}

      \renewcommand{\Sp}{\CCSeq{\Psi}{\Gamma}{\Up A_2}}
      \renewcommand{\Qp}{\Infer[\Up$-R$]{\Sp}{\Sq}}

      \renewcommand{\So}{\CCSeq{\Psi}{\Gamma}{[\DU A_2]}}
      \renewcommand{\Qo}{\Infer[\Down$-R$]{\So}{\Qp}}

      \renewcommand{\Sn}{\CCSeq{\Psi}{\Gamma}{A_1^+}}

      \renewcommand{\Sm}{\CCSeq{\Psi}{\Gamma}{\Up A_1}}
      \renewcommand{\Qm}{\Infer[\Up$-R$]{\Sm}{\Sn}}

      \renewcommand{\Sl}{\CCSeq{\Psi}{\Gamma}{[\DU A_1]}}
      \renewcommand{\Ql}{\Infer[\Down$-R$]{\Sl}{\Qm}}

      \renewcommand{\Sk}{\CCSeq{\Psi}{\Gamma}{[\DU A_1\AndP \DU A_2]}}
      \renewcommand{\Qk}{\Infer[\AndP$-R$]{\Sk}{\Ql & \Qo}}

      \renewcommand{\Sj}{\CCSeq{\Psi}{\Gamma}{\DU A_1\AndP \DU A_2}}
      \renewcommand{\Qj}{\Infer[$foc-R$]{\Sj}{\Qk}}

      \renewcommand{\Si}{\CCSeq{\Psi}{\Gamma,\Susp{\Up A_1}, \Susp{\Up A_2}}{[A_2^+]}}
      \renewcommand{\Qi}{\Infer[$id$]{\Si}{}}

      \renewcommand{\Sh}{\CCSeq{\Psi}{\Gamma,\Susp{\Up A_1}, \Susp{\Up A_2}}{[A_1^+]}}
      \renewcommand{\Qh}{\Infer[$id$]{\Sh}{}}

      \renewcommand{\Sg}{\CCSeq{\Psi}{\Gamma,\Susp{\Up A_1}, \Susp{\Up A_2}}{[A_1^+\AndP A_2^+]}}
      \renewcommand{\Qg}{\Infer[\AndP$-R$]{\Sg}{\Qh & \Qi}}

      \renewcommand{\Sf}{\CCSeq{\Psi}{\Gamma,\Susp{\Up A_1}, \Susp{\Up A_2}}{A_1^+\AndP A_2^+}}
      \renewcommand{\Qf}{\Infer[$foc-R$]{\Sf}{\Qg}}

      \renewcommand{\Se}{\CCSeq{\Psi}{\Gamma,\Up A_1, \Up A_2}{A_1^+\AndP A_2^+}}
      \renewcommand{\Qe}{\Infere[$susp-L$]{\Se}{\Qf}}

      \renewcommand{\Sd}{\CCSeq{\Psi}{\Gamma,\DU A_1, \DU A_2}{A_1^+\AndP A_2^+}}
      \renewcommand{\Qd}{\Infere[\Down$-L$]{\Sd}{\Qe}}

      \renewcommand{\Sc}{\CCSeq{\Psi}{\Gamma,\DU A_1\AndP \DU A_2}{A_1^+\AndP A_2^+}}
      \renewcommand{\Qc}{\Infer[\AndP$-L$]{\Sc}{\Qd}}

      \renewcommand{\Sb}{\CCSeq{\Psi}{\Gamma}{A_1^+\AndP A_2^+}}
      \renewcommand{\Qb}{\Infer[$cut$]{\Sb}{\Qc & \Qk}}

      \renewcommand{\Sa}{\CCSeq{\Psi}{\Gamma}{\DU^n(A_1^+\AndP A_2^+)}}
      \renewcommand{\Qa}{\Infere[\DU$-L$]{\Sa}{\Qb}}

      {\tiny
        \[
        \Qa
        \]
      }

    \item[Case:]
      There exist $A_1'^-$, $A_2'^-$ such that
      $C^+=A_1\AndN A_2$ where $A_1=\Down\Up\ldots A_1'^-$,
      $A_2=\Down\Up\ldots A_2'^-$.
    \end{description}

  \end{description}
\end{proof}

%%% Local Variables:
%%% mode: latex
%%% TeX-master: "../thesis"
%%% End:



%%% Local Variables:
%%% mode: latex
%%% TeX-master: "thesis"
%%% End:
