
\chapter{First-order Logic}\label{chapter.fol}

\begin{remark}\label{fol.remark.multiset}
Using sets would yield the same free variable calculus, since
unlike in the propositional case contraction is still necessary.
Moreover, it complicates the completeness proof XXX more.
\end{remark}

Intuitionistic logic (IL) was conceived in the beginning of the 20th century as a
response to Brouwer's rejection~\cite{Brouwer.1907.Thesis} of the reasoning
principles used in mathematics at the time.  In particular, he rejected the
tautological status of the law of the excluded middle $P\Or\Not P$.  The next
three decades saw increasing interest in the subject, with advances made by
Glivenko, Heyting, and Kolmogorov among others.  The most important early
development for our purposes was made in 1934 by Gentzen~\cite{Gentzen.1934.MZ},
where he defined a \emph{sequent calculus}.

\section{Gentzen's LJ}

It is Gentzen's sequent calculus $\LJ$ that forms the basis for proof search
in IL.  Formulas of IL are built from variables $x$, function symbols $f$,
and predicate symbols $p$ according to the following grammar:

\[
\begin{array}{lll}
  &\mbox{Terms } T &::= x \Sep f(\vec{T})\\
  &\mbox{Formulas } A &::= p(\vec{T}) \Sep A \And A \Sep \Top \Sep A \Or A
  \Sep \Bot \Sep A \Imp A \Sep \All x.\ A \Sep \Ex x.\ A\\
\end{array}
\]


\begin{figure}
  \begin{center}
    \[
    \boxed{
      \begin{array}{lcr}
        \begin{aligned}
          \Right{A_1} &< \Right{(A_1 \And A_2)} \\
          \Right{A_2} &< \Right{(A_1 \And A_2)} \\
          \Left{A_1} &< \Left{(A_1 \And A_2)} \\
          \Left{A_2} &< \Left{(A_1 \And A_2)} \\
        \end{aligned}
        &
        \begin{aligned}
          \Right{A_1} &< \Right{(A_1 \Or A_2)} \\
          \Right{A_2} &< \Right{(A_1 \Or A_2)} \\
          \Left{A_1} &< \Left{(A_1 \Or A_2)} \\
          \Left{A_2} &< \Left{(A_1 \Or A_2)} \\
        \end{aligned}
        &
        \begin{aligned}
          \Left{A_1} &< \Right{(A_1 \Imp A_2)} \\
          \Right{A_2} &< \Right{(A_1 \Imp A_2)} \\
          \Right{A_1} &< \Left{(A_1 \Imp A_2)} \\
          \Left{A_2} &< \Left{(A_1 \Imp A_2)} \\
        \end{aligned}
      \end{array}
    }
    \]
  \end{center}
  \caption{Immediate signed subformulas}
  \label{prop.subformulas}
\end{figure}

%%% Local Variables:
%%% mode: latex
%%% TeX-master: "../thesis"
%%% End:


\newcommand{\LJInit}
{\Infer[$Init$]
  {\PBSeq{\Gamma,p}{p}}
  {}
}

\newcommand{\LJAndR}{
  \Infer[\And$-R$]
  {\PBSeq{\Gamma}{A_1 \And A_2}}
  {\PBSeq{\Gamma}{A_1} & \PBSeq{\Gamma}{A_2}}
}

\newcommand{\LJAndL}{
  \Infer[\And$-L$]
  {\PBSeq{\Gamma, A_1 \And A_2}{C}}
  {\PBSeq{\Gamma, A_1, A_2}{C}}
}

\newcommand{\LJImpR}{
  \Infer[\Imp$-R$]
  {\PBSeq{\Gamma}{A_1 \Imp A_2}}
  {\PBSeq{\Gamma, A_1}{A_2}}
}

\newcommand{\LJImpL}{
  \Infer[\Imp$-L$]
  {\PBSeq{\Gamma, A_1\Imp A_2}{C}}
  {\PBSeq{\Gamma, A_1\Imp A_2}{A_1} & \PBSeq{\Gamma, A_2}{C}}
}

\newcommand{\LJOrRL}{
  \Infer[\Or$-R$_1]
  {\PBSeq{\Gamma}{A_1 \Or A_2}}
  {\PBSeq{\Gamma}{A_1}}
}

\newcommand{\LJOrRR}{
  \infer[\Or$-R$_2]
  {\PBSeq{\Gamma}{A_1 \Or A_2}}
  {\PBSeq{\Gamma}{A_2}}
}

\newcommand{\LJOrL}{
  \infer[\Or$-L$]
  {\PBSeq{\Gamma, A_1 \Or A_2}{C}}
  {\PBSeq{\Gamma, A_1}{C} & \PBSeq{\Gamma, A_2}{C}}
}

\newcommand{\LJTopR}{
  \infer[\Top$-R$]
  {\PBSeq{\Gamma}{\Top}}
  {}
}

\newcommand{\LJTopL}{\mbox{No rule for $\Top$-L}}

\newcommand{\LJBotL}{
  \infer[\Bot$-L$]
  {\PBSeq{\Gamma, \Bot}{C}}
  {}
}

\newcommand{\LJBotR}{\mbox{No rule for $\Bot$-R}}

\begin{figure}[H]
  \begin{center}
    {\small
      \fbox{
        \begin{tabular}{c}
          \begin{tabular}{ccc}
            \LJInit
            &
            \hspace{1cm}
            \LJTopR
            &
            \hspace{1cm}
            \LJTopL
          \end{tabular}
          \\[2em]
          \begin{tabular}{cc}
            \LJAndR
            &
            \hspace{1cm}
            \LJAndL
          \end{tabular}
          \\[2em]
          \begin{tabular}{ccc}
            \LJOrRL
            &
            \hspace{10pt}
            \LJOrRR
            &
            \hspace{10pt}
            \LJOrL
          \end{tabular}
          \\[2em]
          \begin{tabular}{cc}
            \LJBotR
            &
            \hspace{20pt}
            \LJBotL
          \end{tabular}
          \\[2em]
          \begin{tabular}{cc}
            \LJImpR
            &
            \hspace{20pt}
            \LJImpL
          \end{tabular}
        \end{tabular}
      }
    }
  \end{center}
  \caption{The backward calculus $\LJ$}
  \label{prop.lj}
\end{figure}

%%% Local Variables:
%%% mode: latex
%%% TeX-master: "../thesis"
%%% End:


Here $\vec{T}$ is a list of zero or more terms.  We call formulas $p(t_1,
\ldots, t_n)$ where $p$ is a predicate symbol \emph{atomic} formulas.
We define $\Not A$ as $A\Imp\Bot$ and $A \Iff B$ as $(A\Imp B)
\And (B\Imp A)$.

\begin{remark}
Unlike classical logic, the remaining connectives are not
inter-definable.  For instance, $A\Imp B$ is not equivalent to $\Not A\Or B$.
Nor is Skolemization a valid transformation on IL; it lacks the convenient
normal forms of classical logic.  The theorem proving problem is correspondingly
more difficult.  The propositional fragment is PSPACE-complete,
compared with the mere NP-completeness of classical propositional logic.
\end{remark}

A formula without variables is called \emph{ground}.

A variable $x$ is called
\emph{bound} if it occurs in the subtree of a $\All$ or $\Ex$.

Sequents have the form $\Seq{\Gamma}{A}$ where
$\Gamma$ is a multiset of formulas and $A$ is a formula.  The elements of
$\Gamma$ are called the \emph{antecedents} and $A$ is the \emph{consequent}.
The rules of inference are defined in Figure~\ref{fol.rules.lj}.  In a given rule,
the sequent below the line is called the \emph{conclusion} and the sequents
above the line are the \emph{hypotheses}.  We assume that antecedents of the
conclusion of a rule are always a subset of the antecedents of the hypotheses.
For example, in $\LJImpL$ the formula $A_1\Imp A_2$ is assumed to occur in
$\Gamma$.  This is essential for the completeness of the system.

The rules inductively define the proofs of $\LJ$.  A
\emph{proof} of a sequent $S$ is a tree rooted at $S$ created from the rules of
$\LJ$ such that the leaves are all instances of $\LJInit$, $\LJTopR$, or
$\LJBotL$.  Note that in the rules $\LJAllR$ and $\LJExL$,
$a$ does not occur free in $\Gamma, C$.

\begin{remark}
Figure~\ref{fol.rules.lj} is not strictly speaking Gentzen's LJ
of~\cite{Gentzen.1934.MZ}.  One difference is that Gentzen used \emph{lists} of
hypotheses rather than multisets.  One example of the consequences of this
choice is that to capture the requirement that the order of the antecedents does
not matter, he added \emph{structural} rules that allowed moving formulas around
the antecedent list.  Another difference is that his conjunction left rule was

\[
\infer[$\RuleNameSize$ \LJAndL]
{\Seq{\Gamma, A_1 \And A_2}{C}}
{\Seq{\Gamma, A_1, A_2}{C}}
\]

\noindent There are numerous choices for the details of the sequent calculus
that make no difference to the provability of a sequent.  Many of the
technical differences in such presentations are described
in~\cite{Troelstra.2000.ProofTheory}, Chapter 3.  Our system is essentially
their \emph{G3i}.  We chose this equivalent presentation to be consistent with later
chapters.
\end{remark}

\section{The Inverse Method}\label{fol.inverse}

\begin{remark}
  Say something about Decarte's proof of the existence of God here.
\end{remark}

\section{Related Work}
FT~\cite{Sahlin.1992.FT}
\cite{Sahlin.1989.SICS}

JProver~\cite{Schmitt.2001.IJCAR}
Gandalf~\cite{Tammet.1996.CADE}

Wallen~\cite{Wallen.1990.Book}, in its summary of
proof-search method in modal and intuitionistic logic, does not mention the
inverse method.
\cite{Shankar.1992.CADE}

\section{Notes}

Troelstra and Schwichtenberg discuss the proof theory of intuitionistic
logic in detail in~\cite{Troelstra.2000.ProofTheory} including the
crucial cut-elimination theorem.  Indeed, Gentzen's $\LJ$ originally
contained the cut rule as primitive, which he later showed (in the same
paper) to be redundant.

Section~\ref{fol.inverse} relies heavily on Degtyarev and Voronkov's detailed
presentation of the inverse method in~\cite{Voronkov.2001.Handbook}.  In
particular their proof of \cite{Lifschitz.1989.JAR}
  \item[Case:]
    \[ \PFContractRule \]
    \begin{tabbing}
      $\FB{\FSeq{\Gamma,A,A}{\gamma}}$ \` IH \\
      $\Seq{\Gamma, A, A}{\FB{\gamma}}$ \` Def. \\
      $\Seq{\Gamma, A}{\FB{\gamma}}$ \` Contract (Lemma~\ref{prop.thm.contract})\\
      $\FB{\FSeq{\Gamma,A}{\gamma}}$ \` Def. \\
    \end{tabbing}

%%% Local Variables:
%%% mode: latex
%%% TeX-master: "thesis"
%%% End:
