
\chapter{First-order logic}

Propositional logic allows reasoning that is parametric over the atomic formulas.
The truth of $a\And b\Imp b\And a$ is independent of the meanings of
$a$ and $b$.  We can substitute ``It is raining.'' for $a$ and ``5 is odd.c''
for $b$ and obtain a valid inference.  But the logic is not parametric
enough to capture, e.g., mathematical reasoning.  To express,
``each natural number is either odd or even``, we'd need
the infinite formula
\[ (\Even(0)\Or\Odd(0))\And(\Even(1)\Or\Odd(1))\And\ldots \]

\noindent
We would prefer to reason about natural numbers in general, given some
parametric facts about the atomic predicates $\Even$ and $\Odd$.
The above proposition can be expressed in first-order logic as
\[ \All x.~\Even(x)\Or\Odd(x). \]

\noindent
where $\All x.~A(x)$ means ``for any natural number $x$, the propositon
$A(x)$ is true''.  Moreover, we may wish to reason simultaneously about natural numbers and
real numbers, for example
\[ \All x.~\Ex n.~n\le x< n+1. \]

\noindent
where $\Ex x.~A(x)$ means ``there is a natural number $x$ where $A(x)$ is true''.
To express that the variables $x$ and $n$ should be chosen from different sets
of things ($\bR$ and $\bN$ respectively) we label the variables with
a \emph{sort}:
\[ \All x : \bR.~\Ex n : \bN.~n\le x< n+1. \]

\noindent
In this chapter we describe intuitionistic first-order logic.  We begin
by describing the syntax of sorts, terms, and formulas.  We then
define \emph{natural deduction}, the standard proof-theoretic semantics.
As in the last chapter, we then develop a polarized, focused sequent calculus
and describe an inverse method for theorem proving in the focused calculus.

\section{Syntax}

\begin{definition}[Universes and Signatures]
  \begin{itemize}
  \item[]
  \item Fix a finite set of symbols, called \emph{sorts}, denoted $\USort$.
  \item Let $\UVar$ be an infinite set of symbols, called \emph{variables}.
  \item Let $\UParam$ be an infinite set of symbols, called \emph{parameters}.
  \item Let $\UFunc$ be an infinite set of symbols, called \emph{functon symbols}.
    Each $f\in\UFunc$ has an associated \emph{signature},
    $\Sigma(f) = \ASet{\ASet{\sigma_1,\ldots,\sigma_n}, \sigma}$ where
    $\sigma_i\in\USort$.  $n$ is called the \emph{arity} of $f$, denoted
    $\Arity(f)$.  If $\Arity(f)=0$ it is sometimes called a \emph{constant}.
  \item Similarly, each predicate symbol $p\in\UPred$ has a signature,
    $\Sigma(p) = \ASet{\sigma_1,\ldots,\sigma_n}$ where
    $\sigma_i\in\USort$.  $n$ is called the \emph{arity} of $p$, denoted
    $\Arity(p)$.
  \end{itemize}
\end{definition}

\begin{definition}[Terms]
  \begin{itemize}
  \item[]
  \item
    Let $\UTerm$ be the set of \emph{terms} defined by the following grammar.

    \[
    \begin{array}{lrl}
      \mbox{Terms } &T &::= x \Sep a \Sep f(T_1,\ldots,T_n)\\
      \mbox{Sorting contexts } &\Delta &::= \cdot\Sep x:\sigma,\Delta\Sep a:\sigma,\Delta
    \end{array}
    \]
    where $x\in\UVar$, $a\in\UParam$, $f\in\UFunc$, and $\sigma\in\USort$.
    Let $\Vars(t)$ be the set of variables occurring in $t$, and
    $\Params(t)$ be the set of parameters occurring in $t$.
  \item
    A term $t$ is \emph{well-formed at sort $\sigma$ in sorting context $\Delta$}
    if a derivation of
    $\Delta\Entails t :\sigma$ exists given the following rules:

    \[
    \begin{array}{c}
      \begin{array}{cc}
        \infer{\Delta\Entails x : \sigma}{x\in\UVar & x:\sigma\in\Delta }
        &
        \hspace{2em}
        \infer{\Delta\Entails a : \sigma}{a\in\UParam & a:\sigma\in\Delta }
      \end{array}
      \\[2em]
      \begin{array}{c}
        \infer
        {\Delta \Entails f(t_1,\ldots,t_n) : \sigma}
        {\Sigma(f) = \ASet{\ASet{\sigma_1,\ldots,\sigma_n},\sigma} & \Delta\Entails t_1:\sigma_1 & \ldots & \Delta\Entails t_n:\sigma_n}
      \end{array}
    \end{array}
    \]
  \end{itemize}
\end{definition}

\begin{definition}[Formulas]
  \begin{itemize}
  \item[]
  \item Atomic formulas have the form $p(t_1,\ldots,t_n)$ where $p\in\UPred$ and $t_i\in\UTerm$.
  \item First-order formulas share the connectives propositional logic, with
    the new forms
    \[
    \mbox{Formulas } A ::= p(t_1,\ldots,t_n) \Sep \All x:\sigma.~A \Sep \Ex x:\sigma.~A \Sep\ldots
    \]
  \item A formula $A$ is well-formed in $\Delta$ if a derivation of $\Delta\Entails A$
    exists given the following rules:

    \[
    \begin{array}{c}
      \begin{array}{cc}
        \infer
        {\Delta \Entails p(t_1,\ldots,t_n)~\Ok}
        {\Sigma(p) = \ASet{\sigma_1,\ldots,\sigma_n} & \Delta\Entails t_1:\sigma_1 & \ldots & \Delta\Entails t_n:\sigma_n}
        &
        \hspace{2em}
        \infer{\Delta\Entails \Top~\Ok}{}
      \end{array}
      \\[2em]
      \begin{array}{cc}
        \Infer
        {\Delta\Entails A * B~\Ok}
        {\Delta\Entails A~\Ok & \Delta\Entails B~\Ok & *\in\Set{\And,\Or,\Imp}}
        &
        \hspace{2em}
        \infer{\Delta\Entails \Bot~\Ok}{}
      \end{array}
      \\[2em]
      \begin{array}{cc}
        \Infer
        {\Delta\Entails ?x:\sigma.~A(x)~\Ok}
        {\Delta,x:\sigma\Entails A~\Ok & \Delta\Entails B~\Ok & ?\in\Set{\All,\Ex}}
      \end{array}
    \end{array}
    \]
  \item A (multi-)set of formulas is well-formed in $\Delta$ if each of its elements
    is.

  \end{itemize}
\end{definition}

\section{Natural deduction}


\newcommand{\FNHyp}{
  \Infer[$Hyp$]
  {\FNd{\Gamma, l : A}{l : A}}
  {\Delta\Entails \Gamma,A~\Ok}
}

\newcommand{\FNUnit}{
  \Infer[\Unit$-I$]
  {\FNd{\Gamma}{\Unit : \Top}}
  {\Delta\Entails \Gamma~\Ok}
}

\newcommand{\FNAbort}{
  \Infer[\Bot$-E$]
  {\FNd{\Gamma,x:\Bot}{\Abort(x) : C}}
  {\Delta\Entails \Gamma,C~\Ok}
}

\newcommand{\FNPair}{
  \Infer[\And$-I$]
  {\FNd{\Gamma}{\ASet{M,N} : A_1\And A_2}}
  {\FNd{\Gamma}{M : A_1} & \FNd{\Gamma}{N : A_2}}
}

\newcommand{\FNFst}{
  \Infer[\And$-E$_1]
  {\FNd{\Gamma}{\Fst(M) : A_1}}
  {\FNd{\Gamma}{\ASet{M,N} : A_1\And A_2}}
}

\newcommand{\FNSnd}{
  \Infer[\And$-E$_1]
  {\FNd{\Gamma}{\Snd(M) : A_2}}
  {\FNd{\Gamma}{\ASet{M,N} : A_1\And A_2}}
}

\newcommand{\FNInl}{
  \Infer[\Or$-I$_1]
  {\FNd{\Gamma}{\Inl(M) : A_1\Or A_2}}
  {\FNd{\Gamma}{M : A_1}}
}

\newcommand{\FNInr}{
  \Infer[\Or$-I$_2]
  {\FNd{\Gamma}{\Inr(M) : A_1\Or A_2}}
  {\FNd{\Gamma}{M : A_2}}
}

\newcommand{\FNCase}{
  \Infer[\Or$-E$]
  {\FNd{\Gamma}{\Case~M~\Of~\Inr(l_1)\to M~|~\Inl(l_2)\to N : C}}
  {
    \FNd{\Gamma}{M : A_1\Or A_2}
    &
    \FNd{\Gamma, l_1:A_1}{M : C}
    &
    \FNd{\Gamma, l_2:A_2}{N : C}
  }
}

\newcommand{\FNLam}{
  \Infer[\Imp$-I$]
  {\FNd{\Gamma}{\Lam l.~M : A_1\Imp A_2}}
  {\FNd{\Gamma, l:A_1}{M : A_2}}
}

\newcommand{\FNApp}{
  \Infer[\Imp$-E$]
  {\FNd{\Gamma}{M~N : A_2}}
  {
    \FNd{\Gamma}{M : A_1\Imp A_2}
    &
    \FNd{\Gamma}{N : A_1}
  }
}

\newcommand{\FNLamt}{
  \Infer[\All$-I$^a]
  {\FNd{\Gamma}{\Lamt x:\sigma.~M : \All x.~A(x)}}
  {\FNd[\Delta,a:\sigma]{\Gamma}{M : A(a)}}
}

\newcommand{\FNAppt}{
  \Infer[\All$-E$]
  {\FNd{\Gamma}{M~t : A(t)}}
  {\FNd{\Gamma}{M : \All x:\sigma.~A(x)} & \Delta\Entails t:\sigma}
}

\newcommand{\FNPack}{
  \Infer[\Ex$-I$]
  {\FNd{\Gamma}{\ASet{M, t}\As~\Ex x:\sigma.~A(x):\Ex x:\sigma.~A(x)}}
  {\FNd{\Gamma}{M : A(t)} & \Delta\Entails t:\sigma}
}

\newcommand{\FNUnpack}{
  \Infer[\Ex$-E$]
  {\FNd{\Gamma}{\Let\ASet{x, y} = M~\In~N} : C}
  {
    \FNd{\Gamma}{M : \Ex x:\sigma.~A(x)}
    &
    \FNd[\Delta,a:\sigma]{\Gamma, l:A(a)}{N : C}
  }
}

\begin{figure}
  \begin{center}
    {\footnotesize
      \fbox{
        \begin{tabular}{c}
          \begin{tabular}{cc}
            \FNHyp
            &
            \hspace{2em}
            \FNPair
          \end{tabular}
          \\[2em]
          \begin{tabular}{ccc}
            \FNFst
            &
            \hspace{2em}
            \FNSnd
          \end{tabular}
          \\[2em]
          \begin{tabular}{c}
            \FNCase
          \end{tabular}
          \\[2em]
          \begin{tabular}{cc}
            \FNInl
            &
            \hspace{2em}
            \FNInr
          \end{tabular}
          \\[2em]
          \begin{tabular}{cc}
            \FNUnit
            &
            \hspace{2em}
            \FNAbort
          \end{tabular}
          \\[2em]
          \begin{tabular}{cc}
            \FNLam
            &
            \hspace{2em}
            \FNApp
          \end{tabular}
          \\[2em]
          \begin{tabular}{cc}
            \FNLamt
            &
            \hspace{2em}
            \FNAppt
          \end{tabular}
          \\[2em]
          \begin{tabular}{c}
            \FNPack
          \end{tabular}
          \\[2em]
          \begin{tabular}{c}
            \FNUnpack
          \end{tabular}
        \end{tabular}
      }
    }
  \end{center}
  \caption{Natural deduction.}
  \label{fol.nd}
\end{figure}

%%% Local Variables:
%%% mode: latex
%%% TeX-master: "../thesis"
%%% End:


\section{Backward sequent calculus}


\newcommand{\PBInit}{\Infer[$Init$]{\PBSeq{p}{p}}{}}

\newcommand{\PBAndR}{
  \Infer[\And$-R$]
  {\PBSeq{\cdot}{A_1 \And A_2}}
  {\PBSeq{\cdot}{A_1} & \PBSeq{\cdot}{A_2}}
}

\newcommand{\PBAndL}{
  \Infer[\And$-L$]
  {\PBSeq{A_1 \And A_2}{\cdot}}
  {\PBSeq{A_1, A_2}{\cdot}}
}

\newcommand{\PBImpR}{
  \Infer[\Imp$-R$]
  {\PBSeq{\cdot}{A_1 \Imp A_2}}
  {\PBSeq{A_1}{A_2}}
}

\newcommand{\PBImpL}{
  \Infer[\Imp$-L$]
  {\PBSeq{A_1\Imp A_2}{\cdot}}
  {\PBSeq{A_1\Imp A_2}{A_1} & \PBSeq{A_2}{\cdot}}
}

\newcommand{\PBOrRL}{
  \Infer[\Or$-R$_1]
  {\PBSeq{\cdot}{A_1 \Or A_2}}
  {\PBSeq{\cdot}{A_1}}
}

\newcommand{\PBOrRR}{
  \infer[\Or$-R$_2]
  {\PBSeq{\cdot}{A_1 \Or A_2}}
  {\PBSeq{\cdot}{A_2}}
}

\newcommand{\PBOrL}{
  \infer[\Or$-L$]
  {\PBSeq{A_1 \Or A_2}{\cdot}}
  {\PBSeq{A_1}{\cdot} & \PBSeq{\Gamma, A_2}{\cdot}}
}

\newcommand{\PBTopR}{
  \infer[\Top$-R$]
  {\PBSeq{\cdot}{\Top}}
  {}
}

\newcommand{\PBTopL}{\mbox{No rule for $\Top$-L}}

\newcommand{\PBBotL}{
  \infer[\Bot$-L$]
  {\PBSeq{\Bot}{\cdot}}
  {}
}

\newcommand{\PBBotR}{\mbox{No rule for $\Bot$-R}}

\begin{figure}
  \begin{center}
    {\small
      \fbox{
        \begin{tabular}{c}
          \begin{tabular}{ccc}
            \PBInit
            &
            \hspace{1cm}
            \PBTopR
            &
            \hspace{1cm}
            \PBTopL
          \end{tabular}
          \\[2em]
          \begin{tabular}{cc}
            \PBAndR
            &
            \hspace{1cm}
            \PBAndL
          \end{tabular}
          \\[2em]
          \begin{tabular}{ccc}
            \PBOrRL
            &
            \hspace{10pt}
            \PBOrRR
            &
            \hspace{10pt}
            \PBOrL
          \end{tabular}
          \\[2em]
          \begin{tabular}{cc}
            \PBBotR
            &
            \hspace{20pt}
            \PBBotL
          \end{tabular}
          \\[2em]
          \begin{tabular}{cc}
            \PBImpR
            &
            \hspace{20pt}
            \PBImpL
          \end{tabular}
        \end{tabular}
      }
    }
  \end{center}
  \caption{The backward calculus $\PB$}
  \label{prop.backward}
\end{figure}

%%% Local Variables:
%%% mode: latex
%%% TeX-master: "../thesis"
%%% End:


\begin{definition}[Ground]
  Terms and atomic formulas are called \emph{ground} if they have no
  variables or parameters.
  Formulas are ground if they have no free variables and no parameters.
\end{definition}

\begin{definition}[Substitutions]
  \begin{itemize}
  \item[]
  \item
    A substitution $\theta$ is a pair $\ASet{\theta_x,\theta_a}$
    of finite mappings where
    $\theta_x : \UVar\to\UTerm$ and $\theta_a : \UParam\to\UParam$.
  \item
    For historical reasons applying a substitution to a term is written
    $t\theta$.  Define
    \[
    t\theta=
    \begin{cases}
      t=x\in\UVar & \theta_x(x)\\
      t=a\in\UParam & \theta_a(a)\\
      t=f(t_1,\ldots,t_n) & f(t_1\theta,\ldots,t_n\theta)
    \end{cases}
    \]
  \item
    Define the \emph{domain} and \emph{range} of a substitution as
    \[
    \begin{array}{rl}
      \Dom(\theta) &= \Set{x \sst \theta(x)\neq x} \Union \Set{a \sst \theta(a)\neq a} \\
      \Ran(\theta) &= \bigcup_{\theta(x)\neq x}(\Vars(\theta(x))\Union\Params(\theta(x))) \Union \Set{\theta(a) \sst \theta(a)\neq a} \\
    \end{array}
    \]
  \item A substitution is called \emph{idempotent} if $t\theta\theta=t\theta$ for all $t$.
  \end{itemize}
\end{definition}

\begin{definition}[Notation]
  We systematically use the following (possibly subscripted)
  symbols to represent the new syntactic entities.
  \begin{center}
    \begin{tabular}{l@{\hspace{3em}}l}
      $c,d$ & constant function symbols\\
      $f,g$ & non-constant function symbols\\
    \end{tabular}
  \end{center}
\end{definition}

\begin{definition}[Unification]

\end{definition}

\section{Forward sequent calculus}




\section{XXX Notes XXX}

\begin{itemize}
\item Parameters are different from constants because any instantiated variable
  can depend on any constant, but variables track the parameters on which they
  can depend.
\item Define $\Free(A)$ as the free variables and params of $A$ for labeling.
\item Perfect vs. imperfect indexing
\item Freezing and thawing of parameters after first inversion
\item We use only pred symbols to determine pos/neg initial sequents.  You
can imagine using the full relation, but then you'd need to do an allPairs
unification which leads to many more initial sequents.
\item Prolog example (use Imogen as programming language interpreter.)
\item Proof reconstruction: either save substitutions, or reconstruct
  them by brute-force search.
\item Example of substitution subsumption
   val bugs = "q(a1, a2, a1, a2) =>                                \
              \ exists X1 X2 Y1 Y2.                                \
              \ q(X1, X2, Y1, Y2) & ( P(X1) & P(X2) -> P(Y1) & P(Y2) )"
Goal (1/1)
  ⊢ p24
Initial seqs: 3
  p33(x39, x40) ⊢ P(x39)
  p33(x42, x41) ⊢ P(x41)
  q(x46, x45, x44, x43) ⊢ q(x46, x45, x44, x43)
Initial rules: 1
  ⊢ q(x35, x36, x37, x38), p33(x35, x36) ⊢ P(x37), p33(x35, x36) ⊢ P(x38)
  ---------------------------{}:0
  ⊢ p24
Globals:
  q(a1, a2, a1, a2)

\item Motivation for cascading

H1  H2  H3  H4
---------------
    H

Say there's 10 ways Q1 can match H1, and ditto for Q2 and Q3.
This generates 1000 partially applied rules, even though the
final output will be reduced to a few sequents by subsumption.

\item Motivation for partial application

               q(a1, a2, a3, a1, a2, a3) =>
                       exists X1 X2 X3 Y1 Y2 Y3.
                          ( q(X1, X2, Y1, Y2) & ( ( p(X1) & p(X2) & p(X3) ) =>
                              ( p(Y1) & p(Y2) & p(X3) ) )   )"

In this problem, we generate all combinations of instantiations of variables
before finding a subsuming sequent.

\item A surprise is that generating contracts is generally a good idea!
Waiting until the end, during contraction, makes it hard to pick good
candidates for inference, as in the previous example.

\item
((∀ X1. (∃ X2. (p1(X1) ∧ p2(X2)))) => (∃ X2. (∀ X1. (p1(X1) ∧ p2(X2)))))
A nice example of why you can't maximally match a sequent to a rule hypothesis.
Maximally matching would cause proof search to fail.

\end{itemize}

%%% Local Variables:
%%% mode: latex
%%% TeX-master: "thesis"
%%% End:
