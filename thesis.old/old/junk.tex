
% ------------------------------------------------------------------------------
%  Junk
% ------------------------------------------------------------------------------
\section{Junk}

\subsection{Optimizations}

  There are obviously many improvements that can be made to the naive
algorithm.  Much of the work of this proposal will be pruning the
search space by considering a subset of possible forward derivations
and showing that any formula with a derivation in the forward calculus
has a derivation in the subset.  Focusing is an example of such an
optimization which we will describe in Section~.  In
addition to logical restrictions of the space of possible derivations,
there are numerous non-logical optimizations that play an important
part in the implementation of a practical and efficient theorem
prover.  The most significant of these are discussed briefly in this
section.

\paragraph{Redundancy Elimination.}

  An essential optimization in forward-chaining style theorem proving
is redundancy elimination.  Due to the latency of memory access, the
limiting factor in forward proof search is the size of the database of
facts~\cite{Voronkov.2001.Indexing}.  Removing any redundant
sequents from the database is essential to reasonable performance.  A
newly derived sequent that is subsumed by an existing sequent of the
database is said to be \emph{forward subsumed}.  A newly derived
sequent can also subsume existing database sequents.  Those existing
sequents are said to be \emph{backward subsumed}.  A sequent is called
\emph{redundant} if is either forward or backward subsumed.  Much
research into theorem proving implementations is directed at quickly
checking redundancy of a sequent with respect to a database.

\paragraph{Term Indexing.}

Since redundancy elimination, and therefore subsumption checking in
the database, is such an important operation, it is essential to use
data structures for which finding possible subsumption candidates are
efficient.  \emph{Term indexing} is the study of the efficient storage
and retrieval of terms satisfying certain properties.  For instance,
we may wish to retrieve all the terms that are instances, or
generalizations of another term.  Many term indexing data structures
have been studied for both forward and backward subsumption.  Good
general references
are~\cite{Graf.1996.TermIndexing,Voronkov.2001.Indexing}.
We will see some statistics for subsumption in Sections~
and~

\subsection{Example}

Here we give a concrete example of the
Algorithm~ proving the formula

\[
\xi = (\Bot \Or B) \Imp ((A \Imp B) \Imp C) \Imp (A \Or C)
\]

\paragraph{Labeling.}

First we label the signed subformulas of $\xi$.

\[
\begin{array}{c@{\Space}c@{\Space}c}
L_1^+ = A^+ \Or C^+ & L_2^+ = A^- \Imp B^+ & L_3^- = L_2^+ \Imp C^- \\
L_4^+ = L_3^- \Imp L_1^+ & L_5^- = \Bot^- \Or B^- & L_6^+ = L_5^- \Imp L_4^+
\end{array}
\]

\paragraph{Rule Specialization.}

% Now we specialize the rules of $\Inv{\cP}{}$ to the calculus
% $\Inv{\cP}{\xi}$ by generating rules for the labels $L_1, \ldots,
% L_6$.  As an implementation trick, we combine the three different
% rules for $\Imp$-R from Figure~ into one rule by
% following the convention that a given formula doesn't need to be
% present for a match to succeed.  For instance, the sequents
% $\FSeq{A}{B}$, $\FSeq{B}{B}$, and $\FSeq{A}{\cdot}$ can all match the
% rule

\[
\infer{\FSeq{\Gamma}{L_2}}{\FSeq{\Gamma, A}{B}}
\]

\noindent For brevity we can omit the non-principal antecedents
$\Gamma$ from the rule; it is implicit.  We also omit the conclusion
formula when it is not a principal formula of the rule.  We call such
$\Gamma$ and non-principal conclusions \emph{side formulas}.  Written
in this abbreviated style, the inference rules for $\xi$ are

\[
\begin{array}{c@{\Space}c@{\Space}c@{\Space}c}
\infer[R_{1a}]{\FSeq{\cdot}{L_1}}{\FSeq{\cdot}{A}} &
\infer[R_{1b}]{\FSeq{\cdot}{L_1}}{\FSeq{\cdot}{C}} &
\infer[R_2]{\FSeq{\cdot}{L_2}}{\FSeq{A}{B}} &
\infer[R_3]{\FSeq{L_3}{\cdot}}{\FSeq{\cdot}{L_2} & \FSeq{C}{\cdot}}
\\\\
\multicolumn{4}{c}{
  \begin{array}{c@{\Space}c@{\Space}c}
    \infer[R_4]{\FSeq{\cdot}{L_4}}{\FSeq{L_3}{L_1}} &
    \infer[R_5]{\FSeq{L_5}{\cdot}}{\FSeq{\Bot}{\cdot} & \FSeq{B}{\cdot}} &
    \infer[R_6]{\FSeq{\cdot}{L_6}}{\FSeq{L_5}{L_4}}
  \end{array}
}
\end{array}
\]

\noindent The initial axioms are

\[
\begin{array}{c@{\Space}c@{\Space}c@{\Space}c}
\FSeq{A}{A} & \FSeq{B}{B} & \FSeq{C}{C} & \FSeq{\Bot}{\cdot}
\end{array}
\]

We show a run of the algorithm in Figure~.
Each sequent in the database is numbered.  The non-axioms have the
rule and sequents from which they are derived in the right column.
The grayed-out sequents are either forward or backward subsumed.
Forward subsumed sequents are marked by $\FSubsumed$ and
backward by $\BSubsumed$.  (It is also easy to tell whether a sequent
is forward or backward subsumed because a forward subsumed sequent
is not give a number since it never enters the database.)  A
``stage'' corresponds to an iteration of the while loop on
line~.

  In this section we've introduced the theory of the inverse method.
In the next section we describe the generic infrastructure we've used
for implementing inverse method theorem provers.  This infrastructure
is called \emph{Imogen}.


\newcommand{\FSubsumed}{\sqsubset}
\newcommand{\BSubsumed}{\sqsupset}
\newcommand{\MarkSub}[2]{\color{gray}#1 \` \color{gray} #2}

\begin{figure}
  \begin{center}
    \fbox{
      \begin{tabular}{c@{\hspace{2cm}}c}
        \begin{minipage}[b]{0.4\linewidth}
          \begin{tabbing}
            \textbf{Goal} \\
            \\
            \FSeq{\cdot}{p_6} \\
            \\
            \textbf{Stage 0} \\
            \\
            $q_0$. \FSeq{a}{a} \\
            $q_1$. \FSeq{b}{b} \\
            $q_2$. \FSeq{c}{c} \\
            $q_3$. \FSeq{\Bot}{\cdot} \\
            \\
            \textbf{Stage 1} \\
            \\
            $q_4$. \FSeq{a}{p_1} \` ($r_1$, $q_0$) \\
            $q_5$. \FSeq{b}{p_2} \` ($r_3$, $q_1$) \\
            $q_6$. \FSeq{c}{p_1} \` ($r_2$, $q_2$) \\
            $q_7$. \FSeq{p_5}{b} \` ($r_6$, $q_3, q_1$)) \\
            \\
            \textbf{Stage 2} \\
            \\
            $q_8$. \FSeq{p_5}{p_2} \` ($r_3$, $q_7$) \\
            $q_9$. \FSeq{b, p_3}{c} \` ($r_4$, $q_5, q_2$) \\
            $q_{10}$. \FSeq{b, p_3}{p_1} \` ($r_4$, $q_5, q_6$) \\
            $q_{11}$. \FSeq{a}{p_4} \` ($r_5$, $q_4$) \\
            $q_{12}$. \FSeq{c}{p_4} \` ($r_5$, $q_6$) \\
            \MarkSub{\FSeq{p_5}{p_2}}{$(r_6, q_3, q_5) \BSubsumed q_8$} \\
          \end{tabbing}
        \end{minipage}
        &
        \begin{minipage}[b]{0.4\linewidth}
          \begin{tabbing}
            \textbf{Stage 3} \\
            \\
            \MarkSub{\FSeq{b, p_3}{p_1}}{$(r_2, q_9) \BSubsumed q_{10}$} \\
            $q_{13}$. \FSeq{p_5, p_3}{p_4} \` ($r_4$, $q_8, q_{12}$) \\
            $q_{14}$. \FSeq{p_5, p_3}{p_1} \` ($r_4$, $q_8, q_{6}$) \\
            $q_{15}$. \FSeq{p_5, p_3}{c} \` ($r_4$, $q_8, q_{2}$) \\
            \MarkSub{$q_{16}$.{\FSeq{b, p_3}{p_4}}}{($r_4$, $q_5, q_{12}) \FSubsumed q_{17}$} \\
            $q_{17}$. \FSeq{b}{p_4} \` ($r_5$, $q_{14}$) \\
            \MarkSub{{\FSeq{p_5, p_3}{c}}}{$r_6$, $q_3, q_{9}. \FSubsumed q_{15}$} \\
            \MarkSub{{\FSeq{p_5, p_3}{p_1}}}{$r_6$, $q_3, q_{10}. \FSubsumed q_{14}$} \\
            $q_{18}$. \FSeq{a}{p_6} \` ($r_7$, $q_{11}$) \\
            $q_{19}$. \FSeq{c}{p_6} \` ($r_7$, $q_{12}$) \\
            \\
            \textbf{Stage 4} \\
            \\
            \MarkSub{\FSeq{p_5, p_3}{p_1}}{$(r_2, q_{15}) \BSubsumed q_{14}$} \\
            \MarkSub{$q_{20}$. \FSeq{p_5, p_3}{p_6}}{$(r_4, q_8, q_{19}) \FSubsumed q_{22}$} \\
            $q_{21}$. \FSeq{p_5}{p_4} \` ($r_4$, $q_{14}$) \\
            \MarkSub{\FSeq{p_5}{p_4}}{$(r_6, q_{17}) \BSubsumed q_{21}$} \\
            $q_{22}$. \FSeq{p_3}{p_6} \` ($r_7$, $q_{13}$)) \\
            $q_{23}$. \FSeq{b}{p_6} \` ($r_7$, $q_{17}$)) \\
            \\
            \textbf{Stage 5} \\
            \\
            $q_{24}$. \FSeq{p_5}{p_6} \` ($r_6$, $q_{23}$)) \\
            $q_{25}$. \FSeq{\cdot}{p_6} \` ($r_7$, $q_{21}$)) \\
          \end{tabbing}
        \end{minipage}
      \end{tabular}
    }
  \end{center}
  \caption{Example}
  \label{prop.trace}
\end{figure}

%%% Local Variables:
%%% mode: latex
%%% TeX-master: "../thesis"
%%% End:


\subsection{Proof terms}

The soundness of a theorem prover is always suspect.  Why should
we trust that a complex program has no bugs that yield incorrect
answers?  This concern is allayed in different ways.  Imogen
produces a natural deduction proof term for any successful
proof search attempt.  Unlike searching for a proof, checking
one is generally a simple operation.  For instance, the proof term
printer may easily be modified to produce terms for Coq, Twelf or
some other that is generally trusted theorem prover.

There are two parts to proof term generation.  The first is to
build a skeleton of the sequent calculus rule applications.  The
second is to reconstruct the natural deduction proof from the
sequent proof skeleton.





\section{Focusing}\label{prop.sec.focusing}

\subsection{Derived rules}
\label{prop.sec.derived}

Thm: ===> A iff for any A', if |A'| = A then ===> A'

\section{The Polarized Inverse Method}\label{prop.sec.pinverse}

\section{Optimizations}\label{prop.sec.optimize}

Focusing dramatically decreases the size of the forward search space.
However, there is still much room for improvement.  In this section
we describe the other optimizations we use in Imogen to improve
performance.

\subsection{Globalization}

\subsection{Forward subsumption}

\subsection{Backward subsumption}

\subsection{Rule subsumption}

\subsection{Conflicts}

\subsection{Exactness analysis}

\subsection{Branch disabling}

\subsection{Bi-implication lengthening}

\subsection{A remark on subformula orderings}

Successful resolution systems for classical logic make use of an important
optimization called \emph{term ordering}.  One can find a total order on
formulas such that any application of the resolution rule only need consider the
greatest elements of the premeses given the ordering, while remaining
complete.  This dramatically decreases the search space.  While it is
sometimes possible to find
orderings for non-classical logics
(Voronkov found such orderings for the classical modal logic
$K$~\cite{Voronkov.2000.LICS}) it is much more difficult.  The primary
difficulty is that unlike in classical sequent calculus, some rules
are not invertible.  Thus making a choice on which formula to
decompose can make a provable sequent have unprovable

\section{Implementation}\label{prop.sec.imogen}

\section{Proof terms}\label{prop.sec.proofterms}

Traditionally, intuitionistic logic is founded
upon its natural deduction rules, and our implementation generates
proofs in that form.  Put proof search is far more convenient
in the sequent calculus.
For various technical reasons (consider, e.g., the rule $\Imp$-E), proof search
in natural deduction is difficult.

IPL is formalized by its \emph{natural deduction} rules,
shown in Figure~\ref{prop.nd}.  In the judgment $\ND{A}$,
$\Gamma$ is a set of formulas, the \emph{hypotheses}, and $A$ is
the \emph{conclusion}.  Pfenning.LectureNotes.ConstructiveLogic


\newcommand{\FNHyp}{
  \Infer[$Hyp$]
  {\FNd{\Gamma, l : A}{l : A}}
  {\Delta\Entails \Gamma,A~\Ok}
}

\newcommand{\FNUnit}{
  \Infer[\Unit$-I$]
  {\FNd{\Gamma}{\Unit : \Top}}
  {\Delta\Entails \Gamma~\Ok}
}

\newcommand{\FNAbort}{
  \Infer[\Bot$-E$]
  {\FNd{\Gamma,x:\Bot}{\Abort(x) : C}}
  {\Delta\Entails \Gamma,C~\Ok}
}

\newcommand{\FNPair}{
  \Infer[\And$-I$]
  {\FNd{\Gamma}{\ASet{M,N} : A_1\And A_2}}
  {\FNd{\Gamma}{M : A_1} & \FNd{\Gamma}{N : A_2}}
}

\newcommand{\FNFst}{
  \Infer[\And$-E$_1]
  {\FNd{\Gamma}{\Fst(M) : A_1}}
  {\FNd{\Gamma}{\ASet{M,N} : A_1\And A_2}}
}

\newcommand{\FNSnd}{
  \Infer[\And$-E$_1]
  {\FNd{\Gamma}{\Snd(M) : A_2}}
  {\FNd{\Gamma}{\ASet{M,N} : A_1\And A_2}}
}

\newcommand{\FNInl}{
  \Infer[\Or$-I$_1]
  {\FNd{\Gamma}{\Inl(M) : A_1\Or A_2}}
  {\FNd{\Gamma}{M : A_1}}
}

\newcommand{\FNInr}{
  \Infer[\Or$-I$_2]
  {\FNd{\Gamma}{\Inr(M) : A_1\Or A_2}}
  {\FNd{\Gamma}{M : A_2}}
}

\newcommand{\FNCase}{
  \Infer[\Or$-E$]
  {\FNd{\Gamma}{\Case~M~\Of~\Inr(l_1)\to M~|~\Inl(l_2)\to N : C}}
  {
    \FNd{\Gamma}{M : A_1\Or A_2}
    &
    \FNd{\Gamma, l_1:A_1}{M : C}
    &
    \FNd{\Gamma, l_2:A_2}{N : C}
  }
}

\newcommand{\FNLam}{
  \Infer[\Imp$-I$]
  {\FNd{\Gamma}{\Lam l.~M : A_1\Imp A_2}}
  {\FNd{\Gamma, l:A_1}{M : A_2}}
}

\newcommand{\FNApp}{
  \Infer[\Imp$-E$]
  {\FNd{\Gamma}{M~N : A_2}}
  {
    \FNd{\Gamma}{M : A_1\Imp A_2}
    &
    \FNd{\Gamma}{N : A_1}
  }
}

\newcommand{\FNLamt}{
  \Infer[\All$-I$^a]
  {\FNd{\Gamma}{\Lamt x:\sigma.~M : \All x.~A(x)}}
  {\FNd[\Delta,a:\sigma]{\Gamma}{M : A(a)}}
}

\newcommand{\FNAppt}{
  \Infer[\All$-E$]
  {\FNd{\Gamma}{M~t : A(t)}}
  {\FNd{\Gamma}{M : \All x:\sigma.~A(x)} & \Delta\Entails t:\sigma}
}

\newcommand{\FNPack}{
  \Infer[\Ex$-I$]
  {\FNd{\Gamma}{\ASet{M, t}\As~\Ex x:\sigma.~A(x):\Ex x:\sigma.~A(x)}}
  {\FNd{\Gamma}{M : A(t)} & \Delta\Entails t:\sigma}
}

\newcommand{\FNUnpack}{
  \Infer[\Ex$-E$]
  {\FNd{\Gamma}{\Let\ASet{x, y} = M~\In~N} : C}
  {
    \FNd{\Gamma}{M : \Ex x:\sigma.~A(x)}
    &
    \FNd[\Delta,a:\sigma]{\Gamma, l:A(a)}{N : C}
  }
}

\begin{figure}
  \begin{center}
    {\footnotesize
      \fbox{
        \begin{tabular}{c}
          \begin{tabular}{cc}
            \FNHyp
            &
            \hspace{2em}
            \FNPair
          \end{tabular}
          \\[2em]
          \begin{tabular}{ccc}
            \FNFst
            &
            \hspace{2em}
            \FNSnd
          \end{tabular}
          \\[2em]
          \begin{tabular}{c}
            \FNCase
          \end{tabular}
          \\[2em]
          \begin{tabular}{cc}
            \FNInl
            &
            \hspace{2em}
            \FNInr
          \end{tabular}
          \\[2em]
          \begin{tabular}{cc}
            \FNUnit
            &
            \hspace{2em}
            \FNAbort
          \end{tabular}
          \\[2em]
          \begin{tabular}{cc}
            \FNLam
            &
            \hspace{2em}
            \FNApp
          \end{tabular}
          \\[2em]
          \begin{tabular}{cc}
            \FNLamt
            &
            \hspace{2em}
            \FNAppt
          \end{tabular}
          \\[2em]
          \begin{tabular}{c}
            \FNPack
          \end{tabular}
          \\[2em]
          \begin{tabular}{c}
            \FNUnpack
          \end{tabular}
        \end{tabular}
      }
    }
  \end{center}
  \caption{Natural deduction.}
  \label{fol.nd}
\end{figure}

%%% Local Variables:
%%% mode: latex
%%% TeX-master: "../thesis"
%%% End:


\begin{theorem}
  $\Seq{\cdot}{A}$ iff there exists $H$ such that $\ND[]{H}{A}$
\end{theorem}

\input{prop/ndp}

\noindent Proofs are built from the grammar

\[
\begin{array}{rl}
\mbox{Terms } H &::= x \Sep \left<H, H\right> \Sep \left<\right> \Sep \Fst(H) \Sep \Snd(H) \Sep \Inl(H) \Sep \Inr(H) \Sep \Case\ H\ \Of x_1 \to H\ |\ x_2 \to H\\
                & \lambda x.\ H \Sep H H \\
\mbox{Contexts } \Gamma &::= \cdot \Sep x : A,\ \Gamma
\end{array}
\]

where $x\in\cV$, some infinite set of names.   If $\Gamma=x_1 : A_1, \ldots, x_n : A_n$,
the judgment $\ND{H}{A}$ asserts that $H$ is evidence for the truth of formula $A$, given
the truth of $A_1,\ldots,A_n$.  A proof in natural deduction is a


\section{Related Work}\label{prop.sec.related}

The inverse method was first devised by Gentzen~\cite{Gentzen.1934.MZ}
and used to prove the decidability of intuitionistic propositional
logic.  The name ``inverse method'' was coined
in~\cite{Maslov.1964.Inverse}.  For a more detailed history with
copious references, see~\cite{Voronkov.2001.Handbook} Section~9.1.

Tabling backward search

\section{Future Work}\label{prop.sec.future}

In \cite{Gentzen.1934.MZ}, Gentzen describes the decision procedure
for the propositional fragment which we described in section XXX.

Work seeking a tableau-style solution, originally published in
1952~\cite{Vorobev.1952.Doklady}, was rediscovered independently
in~\cite{Dyckhoff.1992.JSL}.  The authors make the observation that
reading the rules from consequent to antecedents, the rule

Dyckhoff/Hudelmaier don't have the subformula property

Make conflicts optimization consider rule applications, not
only subformulas.

\cite{Hudelmaier.1993.JLC}
\cite{Hudelmaier.1993.JLC}
\cite{Voronkov.1999.CADE}

\cite{Shankar.1992.CADE}


A saturated database is
evidence for the unprovability of $\xi$.

\begin{proof}
  The proof is by induction on $i$.  If $i=0$, then $Q$ is initial
  and $\unlab(Q)$ is also initial (Theorem~\ref{prop.thm.initial}).
  Otherwise there is an $R\in\irules$ and $Q_1, \ldots, Q_n\in\cD_{i-1}$
  such that $Q\in\match(R, Q_1,\ldots,Q_n)$.
\end{proof}
