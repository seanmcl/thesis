
\begin{abstract}
Non-classical logics provide a flexible means for describing and
reasoning about diverse computational phenomena.
\emph{Intuitionistic logic} provides a foundation for constructive
mathematics and is the
basis of numerous logical frameworks and proof assistants which are
used to reason about programs.  \emph{Modal logics} permit succinct
reasoning about possible worlds, which has applications to
authorization and distributed computing.  \emph{Linear logic} and
\emph{ordered logic}, with their interpretation of hypotheses as
resources, are used to study stateful and concurrent systems.

Theorem proving in these logics is typically difficult. For example,
while in classical logic the decision problem for the propositional
fragment is NP-complete, intuitionisitic logic is PSPACE-complete,
while even fragments of propositional linear logic are undecidable.
Nevertheless, it is fundamental to the adoption of such expressive
logics that we develop practical theorem proving tools.

We describe a general method for building efficient theorem provers for
non-classical logics called the \emph{polarized inverse method}.  The
\emph{generality} of our approach comes from the use of the inverse method, a
well-known proof search strategy for cut-free sequent calculi, and the use of
\emph{constraints} for delaying the results of costly computations.
\emph{Efficiency} comes from the relatively novel proof-theoretic methods of
\emph{focusing} and \emph{polarization} as well as more standard techniques
such as redundancy elimination.

Throughout the thesis we will demonstrate by example that the inverse
method, augmented with constraints, focusing, and polarization, forms
a strong foundation for efficient theorem proving in non-classical
logics.  We give support to our claims by describing an implementation of
a theorem prover called \emph{Imogen} for propositional and first-order logic,
first-order logic with equality, some first-order some standard
intuitionistic modal logics (K, T, K4, S4, S5), lax logic,
Pfenning and Davies' computational modal logic, linear and ordered
logic).
We find in empirical tests
that Imogen compares favorably with existing implementations.  For
first-order logic with equality, ordered logic and some modal logics,
Imogen is the first and only existing implementations.
\end{abstract}

%%% Local Variables:
%%% mode: latex
%%% TeX-master: "thesis"
%%% End:
